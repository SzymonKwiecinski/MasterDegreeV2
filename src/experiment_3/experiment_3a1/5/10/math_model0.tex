\documentclass{article}
\usepackage{amsmath}
\begin{document}

\title{Student Assignment to Schools Optimization Problem}
\author{}
\date{}
\maketitle

\section*{Problem Definition}

Let:
\begin{itemize}
    \item \( N \): number of neighborhoods
    \item \( S \): number of schools
    \item \( G \): number of grades
    \item \( capacity_{s,g} \): capacity of school \( s \) for grade \( g \)
    \item \( population_{n,g} \): student population for neighborhood \( n \) in grade \( g \)
    \item \( d_{n,s} \): distance from neighborhood \( n \) to school \( s \)
    \item \( x_{n,s,g} \): number of students of grade \( g \) in neighborhood \( n \) assigned to school \( s \)
\end{itemize}

\section*{Objective Function}

We want to minimize the total distance traveled by all students:

\[
\text{Minimize } Z = \sum_{n=1}^{N} \sum_{s=1}^{S} \sum_{g=1}^{G} d_{n,s} \cdot x_{n,s,g}
\]

\section*{Constraints}

1. **Capacity Constraint**:
   Each school must not exceed its capacity for each grade:
   \[
   \sum_{n=1}^{N} x_{n,s,g} \leq capacity_{s,g} \quad \forall s \in \{1, \ldots, S\}, g \in \{1, \ldots, G\}
   \]

2. **Population Constraint**:
   Each neighborhood must assign all its students for each grade:
   \[
   \sum_{s=1}^{S} x_{n,s,g} = population_{n,g} \quad \forall n \in \{1, \ldots, N\}, g \in \{1, \ldots, G\}
   \]

3. **Non-Negativity**:
   All assigned students must be non-negative:
   \[
   x_{n,s,g} \geq 0 \quad \forall n \in \{1, \ldots, N\}, s \in \{1, \ldots, S\}, g \in \{1, \ldots, G\}
   \]

\section*{Output}

The output of the optimization will include:
\begin{itemize}
    \item The assignment of students \( x_{n,s,g} \)
    \item The total distance \( total\_distance \)
\end{itemize}

\end{document}