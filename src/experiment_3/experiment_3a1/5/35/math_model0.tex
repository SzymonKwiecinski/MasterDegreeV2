\documentclass{article}
\usepackage{amsmath}
\begin{document}

\section*{Warehouse Operation Linear Programming Model}

\subsection*{Parameters}
\begin{itemize}
    \item Let \( C \) denote the warehouse capacity.
    \item Let \( h \) denote the holding cost per unit per period.
    \item Let \( p_n \) denote the selling price of the commodity in period \( n \).
    \item Let \( c_n \) denote the cost of purchasing the commodity in period \( n \).
    \item Let \( N \) denote the total number of periods.
\end{itemize}

\subsection*{Decision Variables}
\begin{itemize}
    \item Let \( b_n \) be the buying quantity in period \( n \).
    \item Let \( s_n \) be the selling quantity in period \( n \).
    \item Let \( x_n \) be the stock level at the end of period \( n \).
\end{itemize}

\subsection*{Objective Function}
The objective is to maximize the total profit over all periods, given by:
\[
\text{Maximize } Z = \sum_{n=1}^{N} (p_n s_n - c_n b_n - h x_n)
\]

\subsection*{Constraints}
\begin{align*}
1. & \quad x_n = x_{n-1} + b_n - s_n, \quad \forall n = 1, 2, \ldots, N \\
2. & \quad 0 \leq x_n \leq C, \quad \forall n = 1, 2, \ldots, N \\
3. & \quad x_0 = 0 \quad \text{(initial stock is empty)} \\
4. & \quad x_N = 0 \quad \text{(require warehouse to be empty at the end)}
\end{align*}

\subsection*{Output Variables}
The output will consist of:
\begin{itemize}
    \item Buying quantities: \( \{ b_n \}_{n=1}^N \)
    \item Selling quantities: \( \{ s_n \}_{n=1}^N \)
    \item Stock levels: \( \{ x_n \}_{n=1}^N \)
\end{itemize}

\subsection*{Output Format}
The solution will be formatted as:
\[
\{
    "buyquantity": [b_1, b_2, \ldots, b_N],
    "sellquantity": [s_1, s_2, \ldots, s_N],
    "stock": [x_1, x_2, \ldots, x_N]
\}
\]

\end{document}