\documentclass{article}
\usepackage{amsmath}
\begin{document}

\section*{Linear Programming Model for Police Officer Shift Assignment}

\subsection*{Parameters}
\begin{itemize}
    \item Let $S$ be the total number of shifts.
    \item Let $officers\_needed_{s}$ be the number of police officers needed during shift $s$ for $s = 1, \ldots, S$.
    \item Let $shift\_cost_{s}$ be the cost incurred when starting a shift at $s$ for $s = 1, \ldots, S$.
\end{itemize}

\subsection*{Decision Variables}
Let $officers\_assigned_{s}$ be the number of officers assigned to shift $s$ for $s = 1, \ldots, S$.

\subsection*{Objective Function}
Minimize the total cost:
\[
\text{Total Cost} = \sum_{s=1}^{S} shift\_cost_{s} \cdot officers\_assigned_{s}
\]

\subsection*{Constraints}
\begin{itemize}
    \item Each officer works for two consecutive shifts. Thus, for each shift $s$, the following constraints must hold:
    \[
    officers\_assigned_{s} + officers\_assigned_{s-1} \geq officers\_needed_{s} \quad \text{for } s = 2, \ldots, S
    \]
    \[
    officers\_assigned_{1} \geq officers\_needed_{1}
    \]
\end{itemize}

\subsection*{Output}
The model will yield:
\begin{itemize}
    \item $officers\_assigned$: array of assigned officers for each shift $s$.
    \item $total\_cost$: the total cost incurred by the town.
\end{itemize}

\end{document}