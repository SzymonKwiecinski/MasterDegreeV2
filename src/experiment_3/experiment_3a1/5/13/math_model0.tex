\documentclass{article}
\usepackage{amsmath}
\begin{document}

\section*{Linear Programming Model for Chebychev Center}

Given a set \( P \) described by the linear inequality constraints:

\[
P = \{ \mathbf{x} \in \mathbb{R}^{N} \mid \mathbf{a}_{i}^{T} \mathbf{x} \leq b_{i}, \; i = 1, \ldots, m \}
\]

where \( \mathbf{a}_{i} \in \mathbb{R}^{N} \) and \( b_{i} \in \mathbb{R} \), we want to find a ball with center \( \mathbf{y} \in \mathbb{R}^{N} \) and radius \( r \) that is entirely contained within the set \( P \).

The ball centered at \( \mathbf{y} \) with radius \( r \) is defined as:

\[
B(\mathbf{y}, r) = \{ \mathbf{x} \in \mathbb{R}^{N} \mid \|\mathbf{x} - \mathbf{y}\|_2 \leq r \}
\]

Our objective is to maximize the radius \( r \) while ensuring that the ball \( B(\mathbf{y}, r) \) is contained in \( P \). This leads to the following linear programming formulation:

\[
\text{Maximize } r
\]

Subject to the constraints:

\[
\|\mathbf{a}_{i}^{T} \mathbf{y} - b_{i}\| + r \leq 0, \quad \text{for } i = 1, \ldots, m
\]

This can be expressed in linear form using auxiliary variables. We reformulate the absolute value constraints:

\[
\begin{aligned}
\mathbf{a}_{i}^{T} \mathbf{y} - b_{i} + r & \leq 0, \\
-b_{i} + \mathbf{a}_{i}^{T} \mathbf{y} + r & \leq 0.
\end{aligned}
\]

Thus the complete model is:

\[
\begin{aligned}
\text{Maximize } & r \\
\text{Subject to } & \mathbf{a}_{i}^{T} \mathbf{y} - b_{i} + r \leq 0, \quad i = 1, \ldots, m, \\
& -b_{i} + \mathbf{a}_{i}^{T} \mathbf{y} + r \leq 0, \quad i = 1, \ldots, m.
\end{aligned}
\]

The output will be:

\[
\text{Output: } \{ \text{center: } [y_{j} \text{ for } j = 1, \ldots, N], \, \text{radius: } r \}
\]

\end{document}