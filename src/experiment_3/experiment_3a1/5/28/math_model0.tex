\documentclass{article}
\usepackage{amsmath}
\begin{document}

\section*{Linear Programming Model for Line Fitting}

Given the data points $(x_k, y_k)$ for \( k = 1, \ldots, K \), we want to fit a line defined by the equation: 

\[
y = bx + a
\]

where \( b \) is the slope and \( a \) is the intercept of the line. The objective is to minimize the sum of absolute deviations between the observed values \( y_k \) and the predicted values from the linear equation:

\[
\text{Minimize } \sum_{k=1}^{K} |y_k - (bx_k + a)|
\]

This can be reformulated into a linear programming form by introducing auxiliary variables \( d_k \) for the absolute deviations:

\[
d_k \geq y_k - (bx_k + a) \quad \text{for } k = 1, \ldots, K
\]
\[
d_k \geq -(y_k - (bx_k + a)) \quad \text{for } k = 1, \ldots, K
\]

The objective now becomes:

\[
\text{Minimize } \sum_{k=1}^{K} d_k
\]

subject to the constraints:

\[
d_k \geq y_k - (bx_k + a) \quad \forall k
\]
\[
d_k \geq -(y_k - (bx_k + a)) \quad \forall k
\]

The final Linear Programming formulation can be stated as:

\begin{align*}
\text{Minimize } & \sum_{k=1}^{K} d_k \\
\text{subject to } & d_k \geq y_k - bx_k - a \quad \forall k \\
                   & d_k \geq -(y_k - bx_k - a) \quad \forall k \\
                   & b, a \text{ are free variables} \\
                   & d_k \geq 0 \quad \forall k
\end{align*}

\end{document}