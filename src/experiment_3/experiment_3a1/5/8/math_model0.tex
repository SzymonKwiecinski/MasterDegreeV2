\documentclass{article}
\usepackage{amsmath}
\begin{document}

\section*{Linear Programming Model for Maximizing Profit}

\subsection*{Variables}
Let \( x_k \) be the quantity of spare part \( k \) to be produced, where \( k = 1, 2, \ldots, K \).

\subsection*{Parameters}
\begin{itemize}
    \item \( time_{k,s} \): required worker-hours for part \( k \) in shop \( s \).
    \item \( profit_k \): profit of part \( k \).
    \item \( capacity_s \): capacity of shop \( s \) in working hours.
\end{itemize}

\subsection*{Objective Function}
The objective is to maximize the total profit, which can be expressed as:
\[
\text{Maximize } Z = \sum_{k=1}^{K} profit_k \cdot x_k
\]

\subsection*{Constraints}
The production of parts is subject to the capacity constraints of each shop. The constraints can be formulated as:
\[
\sum_{k=1}^{K} time_{k,s} \cdot x_k \leq capacity_s \quad \forall s = 1, 2, \ldots, S
\]

Additionally, we must ensure that the quantity of spare parts produced is non-negative:
\[
x_k \geq 0 \quad \forall k = 1, 2, \ldots, K
\]

\subsection*{Summary of the Model}
Putting it all together, the linear programming model can be summarized as follows:

\begin{align*}
\text{Maximize } & Z = \sum_{k=1}^{K} profit_k \cdot x_k \\
\text{subject to } & \sum_{k=1}^{K} time_{k,s} \cdot x_k \leq capacity_s \quad \forall s \\
& x_k \geq 0 \quad \forall k
\end{align*}

\end{document}