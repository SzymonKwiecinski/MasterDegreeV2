\documentclass{article}
\usepackage{amsmath}
\begin{document}

\section*{Mathematical Model for the Paper Cutting Problem}

\textbf{Variables:}
\begin{itemize}
    \item Let \( x_i \) be the number of times cutting pattern \( i \) is used, where \( i = 1, \ldots, N \).
\end{itemize}

\textbf{Parameters:}
\begin{itemize}
    \item Let \( W \) be the large roll width in inches, i.e., \( W = \text{large\_roll\_width} \).
    \item Let \( M \) be the number of different small roll widths.
    \item Let \( d_j \) be the demand for rolls of width \( j \), where \( j = 1, \ldots, M \).
    \item Let \( p_{i,j} \) be the number of rolls produced of width \( j \) by cutting pattern \( i \).
\end{itemize}

\textbf{Objective Function:}
\[
\text{Minimize } Z = \sum_{i=1}^{N} x_i
\]
This objective function minimizes the total number of large rolls used.

\textbf{Constraints:}
\[
\sum_{i=1}^{N} p_{i,j} x_i \geq d_j \quad \forall j = 1, \ldots, M
\]
This constraint ensures that the demand for each type of small roll is met.

\[
\sum_{j=1}^{M} p_{i,j} x_i \leq W \cdot x_i \quad \forall i = 1, \ldots, N
\]
This constraint ensures that the total width of rolls produced by each cutting pattern does not exceed the width of the large roll.

\[
x_i \geq 0 \quad \text{and } x_i \text{ is an integer} \quad \forall i = 1, \ldots, N
\]
This ensures that the number of uses of each cutting pattern is a non-negative integer.

\end{document}