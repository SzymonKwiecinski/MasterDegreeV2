\documentclass{article}
\usepackage{amsmath}
\begin{document}

\section*{Quadratic Curve Fitting}

Let the set of observed pairs be defined as:  
\[
\{(x_k, y_k) \mid k = 1, \ldots, K\}
\]

The objective is to find the coefficients \( c \), \( b \), and \( a \) such that the quadratic function 
\[
y = c x^2 + b x + a
\]
best fits the data points by minimizing the sum of absolute deviations between the observed values \( y_k \) and the predicted values from the model. 

This leads to the optimization problem:
\[
\text{Minimize} \quad \sum_{k=1}^{K} |y_k - (c x_k^2 + b x_k + a)|
\]

Let \( d_k = y_k - (c x_k^2 + b x_k + a) \). We can rewrite the minimization of absolute deviations as:
\[
\text{Minimize} \quad \sum_{k=1}^{K} d_k
\]
subject to:
\[
d_k \geq y_k - (c x_k^2 + b x_k + a), \quad \forall k
\]
\[
d_k \geq -(y_k - (c x_k^2 + b x_k + a)), \quad \forall k
\]

Thus, the complete linear programming formulation is:
\[
\begin{aligned}
& \text{Minimize} && \sum_{k=1}^{K} d_k \\
& \text{subject to} && d_k \geq y_k - (c x_k^2 + b x_k + a), \quad \forall k \\
& & & d_k \geq -(y_k - (c x_k^2 + b x_k + a)), \quad \forall k \\
\end{aligned}
\]

The final output parameters representing the coefficients of the quadratic, linear, and constant terms are as follows:
\[
\begin{aligned}
& \text{quadratic} = c, \\
& \text{linear} = b, \\
& \text{constant} = a.
\end{aligned}
\]

\end{document}