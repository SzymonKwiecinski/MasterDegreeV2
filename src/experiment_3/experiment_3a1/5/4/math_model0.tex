\documentclass{article}
\usepackage{amsmath}
\begin{document}

\title{MILP Model for Nurse Scheduling}
\author{}
\date{}
\maketitle

\section*{Problem Definition}
Let \( n \) be the total number of nurses needed, and let \( \text{period} \) be the number of days each nurse works consecutively. Let \( d_j \) be the demand for nurses on day \( j \) for \( j = 1, \ldots, 7 \).

\section*{Variables}
Define the decision variables:
\begin{itemize}
    \item \( x_j \): the number of nurses that start their shift on day \( j \) for \( j = 1, \ldots, 7 \).
    \item \( n \): the total number of nurses hired.
\end{itemize}

\section*{Objective Function}
The goal is to minimize the total number of nurses hired:
\[
\text{Minimize } n = \sum_{j=1}^{7} x_j
\]

\section*{Constraints}
Each nurse works for \( \text{period} \) days and has \( 7 - \text{period} \) days off. We need to ensure that the demand for nurses is met on each day \( j \):

For each day \( j \):
\[
\sum_{k=0}^{\text{period}-1} x_{(j-k) \mod 7} \geq d_j \quad \text{for } j = 1, \ldots, 7
\]

Where \( x_{(j-k) \mod 7} \) represents the nurses starting their shift on day \( (j-k) \). The calculation ensures that the demand \( d_j \) is met by the nurses who started on the previous \( \text{period} \) days.

\section*{Final Model}
The full Mixed Integer Linear Programming (MILP) formulation is as follows:

\begin{align*}
\text{Minimize } & \quad n = \sum_{j=1}^{7} x_j \\
\text{subject to } & \quad \sum_{k=0}^{\text{period}-1} x_{(j-k) \mod 7} \geq d_j, & \quad j = 1, \ldots, 7 \\
& \quad x_j \geq 0 \quad \text{and integer for } j = 1, \ldots, 7
\end{align*}

\end{document}