\documentclass{article}
\usepackage{amsmath}
\begin{document}

\section*{Mathematical Model for Translator Selection}

\subsection*{Parameters}
\begin{itemize}
    \item $N$: Number of available translators.
    \item $M$: Number of required languages.
    \item $c_i$: Cost of hiring translator $i$, where $i = 1, 2, \ldots, N$.
    \item $L_i$: Set of languages that translator $i$ can translate.
    \item $R_j$: Required languages, where $j = 1, 2, \ldots, M$.
\end{itemize}

\subsection*{Decision Variables}
Let $x_i$ be a binary variable defined as:
\[
x_i = 
\begin{cases} 
1 & \text{if translator } i \text{ is selected} \\
0 & \text{otherwise} 
\end{cases}
\]

\subsection*{Objective Function}
Minimize the total cost of hiring translators:
\[
\text{Minimize } Z = \sum_{i=1}^{N} c_i x_i
\]

\subsection*{Constraints}
1. Ensure that all required languages are covered:
\[
\sum_{i=1}^{N} x_i \cdot \mathbb{I}(R_j \in L_i) \geq 1 \quad \forall j \in \{1, 2, \ldots, M\}
\]
where $\mathbb{I}(R_j \in L_i)$ is an indicator function that equals 1 if language $R_j$ can be translated by translator $i$, and 0 otherwise.

2. Binary constraints on the decision variables:
\[
x_i \in \{0, 1\} \quad \forall i \in \{1, 2, \ldots, N\}
\]

\subsection*{Output}
The output will consist of the following:
\begin{itemize}
    \item \textbf{selected\_translators}: List of translator IDs that were chosen.
    \item \textbf{total\_cost}: The total cost incurred for hiring the selected translators, calculated as $Z$.
\end{itemize}

\end{document}