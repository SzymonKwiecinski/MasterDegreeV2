\documentclass{article}
\usepackage{amsmath}
\begin{document}

\section*{Linear Programming Model}

\subsection*{Variables}
Let:
\begin{itemize}
    \item \( power_j \) be the power of the \( j \)-th lamp for \( j = 1, \ldots, M \).
    \item \( ill_i \) be the illumination of the \( i \)-th segment for \( i = 1, \ldots, N \).
    \item \( desired_i \) be the desired illumination of the \( i \)-th segment for \( i = 1, \ldots, N \).
    \item \( error_i \) be the absolute error for the \( i \)-th segment for \( i = 1, \ldots, N \).
\end{itemize}

\subsection*{Objective Function}
We want to minimize the total absolute error:
\[
\text{Minimize } Z = \sum_{i=1}^{N} error_i
\]

Where the absolute error for each segment is defined as:
\[
error_i = |ill_i - desired_i|
\]

The illumination of each segment can be expressed as:
\[
ill_i = \sum_{j=1}^{M} coeff_{i,j} \cdot power_j
\]

\subsection*{Constraints}
To handle the absolute values in the objective function, we introduce auxiliary variables \( e_i \) for \( i = 1, \ldots, N \) such that:
\[
error_i \geq ill_i - desired_i \quad \forall i
\]
\[
error_i \geq desired_i - ill_i \quad \forall i
\]

This leads to the following reformulation of the problem:
\[
\text{Minimize } Z = \sum_{i=1}^{N} e_i
\]

Subject to:
\[
\begin{align*}
ill_i &= \sum_{j=1}^{M} coeff_{i,j} \cdot power_j \quad \forall i \\
error_i &\geq ill_i - desired_i \quad \forall i \\
error_i &\geq desired_i - ill_i \quad \forall i \\
power_j &\geq 0 \quad \forall j \quad (\text{assuming non-negative lamp powers})
\end{align*}
\]

\section*{Output}
The output will provide:
\begin{itemize}
    \item \( power_j \): The optimal power of the \( j \)-th lamp.
    \item \( error \): The total absolute error between the illuminations \( ill_i \) and the desired illuminations \( desired_i \).
\end{itemize}

\end{document}