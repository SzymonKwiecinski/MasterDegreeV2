\documentclass{article}
\usepackage{amsmath}
\begin{document}

\section*{Linear Programming Model for Alloy Production}

\textbf{Parameters:}
\begin{itemize}
    \item \( alloy\_quant \): Total quantity of the alloy to be produced (in lb).
    \item \( target_{m} \): Target quantity of metal \( m \) in the alloy.
    \item \( ratio_{k,m} \): Ratio of metal \( m \) in alloy \( k \).
    \item \( price_{k} \): Price of alloy \( k \).
\end{itemize}

\textbf{Decision Variables:}
\begin{itemize}
    \item \( amount_{k} \): Amount of alloy \( k \) to be purchased.
\end{itemize}

\textbf{Objective Function:}
\begin{equation}
\text{Minimize } Z = \sum_{k=1}^{K} price_{k} \cdot amount_{k}
\end{equation}

\textbf{Constraints:}
\begin{itemize}
    \item Total weight constraint:
    \begin{equation}
    \sum_{k=1}^{K} amount_{k} = alloy\_quant
    \end{equation}
    \item Metal composition constraint for each metal \( m \):
    \begin{equation}
    \sum_{k=1}^{K} ratio_{k, m} \cdot amount_{k} = target_{m}, \quad \forall m = 1, \ldots, M
    \end{equation}
    \item Non-negativity constraints:
    \begin{equation}
    amount_{k} \geq 0, \quad \forall k = 1, \ldots, K
    \end{equation}
\end{itemize}

\textbf{Output:}
The optimal amounts of each alloy to be purchased are represented by:
\begin{equation}
amount = [amount_{1}, amount_{2}, \ldots, amount_{K}]
\end{equation}

\end{document}