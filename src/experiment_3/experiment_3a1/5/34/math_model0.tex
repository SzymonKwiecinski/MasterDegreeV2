\documentclass{article}
\usepackage{amsmath}
\begin{document}

\section*{Linear Programming Model for Food Purchase Optimization}

\subsection*{Variables}
Let \( x_k \) be the quantity of food \( k \) to purchase, where \( k = 1, 2, \ldots, K \).

\subsection*{Parameters}
\begin{itemize}
    \item \( \text{price}_k \): price per unit of food \( k \)
    \item \( \text{demand}_m \): minimum required units of nutrient \( m \) per day
    \item \( \text{nutri}_{k,m} \): units of nutrient \( m \) contained in one unit of food \( k \)
\end{itemize}

\subsection*{Objective Function}
We aim to minimize the total cost of food purchases:
\[
\text{Minimize } Z = \sum_{k=1}^{K} \text{price}_k \cdot x_k
\]

\subsection*{Constraints}
To ensure each nutritional demand is met, we have the following constraints for each nutrient \( m \):
\[
\sum_{k=1}^{K} \text{nutri}_{k,m} \cdot x_k \geq \text{demand}_m, \quad \text{for } m = 1, 2, \ldots, M
\]
Additionally, we include non-negativity constraints:
\[
x_k \geq 0, \quad \text{for } k = 1, 2, \ldots, K
\]

\subsection*{Output}
The solution will provide the optimal quantities of each food to purchase:
\[
\text{quantity} = [x_1, x_2, \ldots, x_K]
\]

\end{document}