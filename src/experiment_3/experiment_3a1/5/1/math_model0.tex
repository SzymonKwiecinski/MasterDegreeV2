\documentclass{article}
\usepackage{amsmath}
\begin{document}

\section*{Linear Programming Model}

Let:
\begin{itemize}
    \item \( M \) be the total number of goods produced.
    \item \( N \) be the total number of raw materials.
    \item \( \text{available}_i \) be the available amount of raw material \( i \) for \( i = 1, \ldots, N \).
    \item \( \text{req}_{i,j} \) be the requirement of raw material \( i \) for good \( j \) for \( j = 1, \ldots, M \).
    \item \( \text{price}_j \) be the revenue per unit produced for good \( j \) for \( j = 1, \ldots, M \).
    \item \( \text{amount}_j \) be the amount of good \( j \) produced.
\end{itemize}

\subsection*{Objective Function}

The objective is to maximize the total revenue:

\[
\text{Maximize } Z = \sum_{j=1}^{M} \text{price}_j \cdot \text{amount}_j
\]

\subsection*{Constraints}

The constraints based on the availability of raw materials are:

\[
\sum_{j=1}^{M} \text{req}_{i,j} \cdot \text{amount}_j \leq \text{available}_i \quad \forall i = 1, \ldots, N
\]

Also, we need to ensure that the amount produced is non-negative:

\[
\text{amount}_j \geq 0 \quad \forall j = 1, \ldots, M
\]

\subsection*{Conclusion}

Thus, the linear programming model can be summarized as follows:

\begin{itemize}
    \item \textbf{Maximize:} \( Z = \sum_{j=1}^{M} \text{price}_j \cdot \text{amount}_j \)
    \item \textbf{Subject to:} 
    \[
    \sum_{j=1}^{M} \text{req}_{i,j} \cdot \text{amount}_j \leq \text{available}_i \quad \forall i = 1, \ldots, N
    \]
    \item \textbf{And:} 
    \[
    \text{amount}_j \geq 0 \quad \forall j = 1, \ldots, M
    \]
\end{itemize}

\end{document}