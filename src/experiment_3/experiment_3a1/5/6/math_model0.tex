\documentclass{article}
\usepackage{amsmath}
\begin{document}

\section*{Rocket Motion Optimization Problem}

Consider a rocket that travels along a straight path. Let \( x_t \), \( v_t \), and \( a_t \) be the position, velocity, and acceleration, respectively, of the rocket at time \( t \). By discretizing time and by taking the time increment to be unity, we obtain an approximate discrete-time model of the form:

\[
x_{t+1} = x_t + v_t
\]
\[
v_{t+1} = v_t + a_t
\]

We assume that the acceleration \( a_t \) is under our control, as it is determined by the rocket thrust. In a rough model, the magnitude \( |a_t| \) of the acceleration can be assumed to be proportional to the rate of fuel consumption at time \( t \). 

Given the initial conditions \( x_0 \) and \( v_0 \) as the initial position and velocity of the rocket, we wish for the rocket to take off and reach a target position \( x_T \) with speed \( v_T \) at time \( T \). Furthermore, we want to accomplish this in an economical fashion, that is, by using as little fuel as possible.

\subsection*{Input Format}

The input is structured as follows:

\[
\{ 
    "x_0": x_0, 
    "v_0": v_0, 
    "x_T": x_T, 
    "v_T": v_T, 
    "T": T 
\}
\]

\subsection*{Objective}

The objective is to minimize the total fuel spent, given by:

\[
\text{Minimize } \sum_{t=0}^{T-1} |a_t|
\]

\subsection*{Constraints}

The following constraints must be satisfied:

\[
\begin{align*}
x_{t+1} &= x_t + v_t \quad \text{for } t = 0, 1, \ldots, T-1 \\
v_{t+1} &= v_t + a_t \quad \text{for } t = 0, 1, \ldots, T-1 \\
x_0 &= \text{initial position} \\
v_0 &= \text{initial velocity} \\
x_T &= \text{target position} \\
v_T &= \text{target velocity}
\end{align*}
\]

\subsection*{Output Format}

The expected output format is:

\[
\{ 
    "x": [x_i \, \text{for } i = 1, \ldots, T], 
    "v": [v_i \, \text{for } i = 1, \ldots, T], 
    "a": [a_i \, \text{for } i = 1, \ldots, T], 
    "fuel\_spend": \text{fuel\_spent} 
\}
\]

\end{document}