\documentclass{article}
\usepackage{amsmath}
\begin{document}

\section*{Linear Programming Model for Police Officer Shift Assignment}

\subsection*{Indices}
\begin{itemize}
    \item $s$: Shift index, where $s = 1, 2, \ldots, S$.
\end{itemize}

\subsection*{Parameters}
\begin{itemize}
    \item $S$: Total number of shifts.
    \item $officers\_needed_{s}$: Number of police officers needed during shift $s$.
    \item $shift\_cost_{s}$: Cost incurred by the town for assigning officers to shift $s$.
\end{itemize}

\subsection*{Decision Variables}
\begin{itemize}
    \item $officers\_assigned_{s}$: Number of police officers assigned to shift $s$.
\end{itemize}

\subsection*{Objective Function}
The objective is to minimize the total cost:
\[
\text{Minimize } Z = \sum_{s=1}^{S} shift\_cost_{s} \cdot \left( \frac{officers\_assigned_{s} + officers\_assigned_{s+1}}{2} \right)
\]
Note: For $s = S$, we assume $officers\_assigned_{S+1} = 0$.

\subsection*{Constraints}
The following constraints ensure that the number of officers assigned meets the required number of officers for each shift:
\[
officers\_assigned_{s} + officers\_assigned_{s-1} \geq officers\_needed_{s} \quad \text{for } s = 1, 2, \ldots, S
\]
with the additional constraint that $officers\_assigned_{0} = 0$.

\subsection*{Non-negativity Constraints}
\[
officers\_assigned_{s} \geq 0 \quad \text{for all } s
\]

\end{document}