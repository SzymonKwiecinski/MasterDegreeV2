\documentclass{article}
\usepackage{amsmath}
\begin{document}

\section*{Transportation Problem Model}

Let:
\begin{itemize}
    \item \( n \): number of cities
    \item \( A \): set of all routes
    \item \( C_{i,j} \): transport cost per metric ton along route from city \( i \) to city \( j \)
    \item \( supply_k \): soybean supply at terminal city \( k \)
    \item \( demand_l \): soybean demand at destination city \( l \)
\end{itemize}

\subsection*{Decision Variables}
Let \( amount_{i,j} \) be the amount of soybeans shipped from city \( i \) to city \( j \).

\subsection*{Objective Function}
Minimize the total transportation cost:
\[
\text{Minimize } Z = \sum_{(i,j) \in A} C_{i,j} \cdot amount_{i,j}
\]

\subsection*{Constraints}
1. Supply constraints for each terminal city \( k \):
\[
\sum_{j \text{ s.t. } (k,j) \in A} amount_{k,j} \leq supply_k \quad \forall k
\]

2. Demand constraints for each destination city \( l \):
\[
\sum_{i \text{ s.t. } (i,l) \in A} amount_{i,l} \geq demand_l \quad \forall l
\]

3. Non-negativity constraints:
\[
amount_{i,j} \geq 0 \quad \forall (i,j) \in A
\]

\subsection*{Output}
The output will consist of the amount of soybeans shipped from each route and the total transportation cost:
\begin{itemize}
    \item Distribution of soybeans:
    \[
    \text{distribution} = \left\{ \left\{ "from": i, "to": j, "amount": amount_{i,j} \right\} \text{ for } (i,j) \in A \right\}
    \]
    \item Total cost:
    \[
    \text{total\_cost} = Z
    \end{itemize}
\end{document}