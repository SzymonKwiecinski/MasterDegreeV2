\documentclass{article}
\usepackage{amsmath}
\begin{document}

\title{Linear Programming Model for Alloy Production}
\author{}
\date{}
\maketitle

\section*{Problem Definition}
A manufacturer wishes to produce \( \text{alloy\_quant} \) lb of an alloy that consists of several kinds of metals. The target quantity of metal \( m \) in the alloy is given by \( \text{target}_{m} \). The available alloys \( k \) have a ratio \( \text{ratio}_{k, m} \) of metal \( m \) and are sold at a price \( \text{price}_{k} \).

\section*{Decision Variables}
Let \( \text{amount}_{k} \) be the quantity of alloy \( k \) to be purchased.

\section*{Objective Function}
The objective is to minimize the total cost of the alloys purchased:

\[
\text{Minimize } Z = \sum_{k=1}^{K} \text{price}_{k} \cdot \text{amount}_{k}
\]

\section*{Constraints}

1. Total weight of the alloys must equal the desired alloy weight:

\[
\sum_{k=1}^{K} \text{amount}_{k} = \text{alloy\_quant}
\]

2. Ensure that the weight of metal \( m \) meets the target:

\[
\sum_{k=1}^{K} \text{ratio}_{k, m} \cdot \text{amount}_{k} = \text{target}_{m}, \quad \text{for } m = 1, \ldots, M
\]

3. Non-negativity constraints:

\[
\text{amount}_{k} \geq 0, \quad \text{for } k = 1, \ldots, K
\]

\section*{Summary of Input and Output}

\textbf{Input:}
\begin{itemize}
    \item \text{alloy\_quant} - Total quantity of the alloy to be produced.
    \item \text{target} - Target quantities of metals in the alloy \(\text{target}_{m}\) for \( m = 1, \ldots, M \).
    \item \text{ratio} - Ratios of metals in each alloy \(\text{ratio}_{k, m}\) for \( k = 1, \ldots, K \) and \( m = 1, \ldots, M \).
    \item \text{price} - Prices of each alloy \(\text{price}_{k}\) for \( k = 1, \ldots, K \).
\end{itemize}

\textbf{Output:}
\begin{itemize}
    \item \text{amount} - Quantities of each alloy to be purchased \(\text{amount}_{k}\) for \( k = 1, \ldots, K \).
\end{itemize}

\end{document}