\documentclass{article}
\usepackage{amsmath}
\begin{document}

\section*{Linear Programming Model for Chebyshev Center}

Given the set \( P \) defined by the constraints:

\[
P = \{ x \in \mathbb{R}^N \mid a_i^T x \leq b_i, \, i = 1, \ldots, m \}
\]

We want to find the center \( y \in \mathbb{R}^N \) and the radius \( r \) of the largest ball centered at \( y \) that is entirely contained within \( P \). The ball is defined as:

\[
B(y, r) = \{ x \in \mathbb{R}^N \mid \| x - y \|_2 \leq r \}
\]

To establish the linear programming model, we need to maximize the radius \( r \) subject to the constraints of the set \( P \):

\begin{align*}
\text{Maximize} \quad & r \\
\text{subject to} \quad & a_i^T y - r \cdot \| a_i \|_2 \leq b_i, \quad i = 1, \ldots, m \\
& r \geq 0
\end{align*}

Where \( a_i \) is the coefficient vector corresponding to constraint \( i \), and \( \| a_i \|_2 \) is the Euclidean norm of \( a_i \), which takes into account the maximum distance that the radius \( r \) can extend in the direction of \( a_i \).

Thus, the outputs from the model are:

\begin{itemize}
    \item Center: \( \text{center} = [y_j \text{ for } j = 1, \ldots, N] \)
    \item Radius: \( \text{radius} = r \)
\end{itemize}

\end{document}