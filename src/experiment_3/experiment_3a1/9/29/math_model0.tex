\documentclass{article}
\usepackage{amsmath}
\begin{document}

\section*{Linear Programming Model for Line Fitting}

Given a set of values \( \{x_k\}_{k=1}^K \) and \( \{y_k\}_{k=1}^K \), the goal is to fit the best straight line defined by the equation 

\[
y = bx + a
\]

such that we minimize the maximum deviation of the observed values \( y_k \) from the predicted values.

\subsection*{Variables}
Let:
- \( a \) be the intercept of the fitted line.
- \( b \) be the slope of the fitted line.
- \( d_k \) be the deviation for each observation defined as \( d_k = y_k - (bx_k + a) \).

\subsection*{Objective Function}
We want to minimize the maximum deviation, which can be formalized as:

\[
\text{Minimize} \quad t
\]

subject to:

\[
d_k \leq t \quad \forall k = 1, \ldots, K
\]
\[
-d_k \leq t \quad \forall k = 1, \ldots, K
\]

This ensures that the absolute value of the deviation \( |d_k| \) does not exceed \( t \).

\subsection*{Constraints}
The deviations can be expressed in terms of \( a \) and \( b \):

\[
d_k = y_k - (bx_k + a) \quad \forall k = 1, \ldots, K
\]

The complete linear programming model can now be structured as follows:

\subsection*{Linear Program}
\begin{align*}
\text{Minimize} & \quad t \\
\text{Subject to} & \quad y_k - (bx_k + a) \leq t \quad \forall k = 1, \ldots, K \\
                  & \quad -(y_k - (bx_k + a)) \leq t \quad \forall k = 1, \ldots, K \\
                  & \quad a \in \mathbb{R}, \quad b \in \mathbb{R}, \quad t \geq 0
\end{align*}

\subsection*{Output}
The result of this linear programming model will yield:
\begin{itemize}
    \item \texttt{intercept} = \( a \)
    \item \texttt{slope} = \( b \)
\end{itemize}

The solution of this model provides the intercept and slope of the best-fit line minimizing the maximum deviation from the observed values.

\end{document}