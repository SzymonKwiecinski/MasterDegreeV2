\documentclass{article}
\usepackage{amsmath}
\begin{document}

\section*{Linear Programming Model for Police Shift Assignment}

\textbf{Indices:} \\
Let \( s \) be the index for shifts, where \( s = 1, 2, \ldots, S \).

\textbf{Parameters:} \\
\( \text{officers\_needed}_{s} \): Number of police officers required during shift \( s \) \\
\( \text{shift\_cost}_{s} \): Cost incurred by the town for starting shift \( s \)

\textbf{Decision Variables:} \\
Let \( x_{s} \) be the number of officers assigned to shift \( s \).

\textbf{Objective Function:} \\
Minimize the total cost:
\[
\text{Total Cost} = \sum_{s=1}^{S} \text{shift\_cost}_{s} \cdot x_{s}
\]

\textbf{Constraints:} \\
Each officer works for two consecutive shifts. Therefore, the number of officers assigned to each shift needs to satisfy:
\[
x_{s} \geq \text{officers\_needed}_{s}, \quad \forall s = 1, \ldots, S
\]
\[
x_{s} = x_{s+1}, \quad \forall s = 1, \ldots, S-1
\]

Also, since the first shift requires a certain number of officers, and the last shift will have the same number of officers assigned as the previous one (if needed):
\[
x_{1} \geq \text{officers\_needed}_{1}
\]
\[
x_{S} \geq \text{officers\_needed}_{S}
\]

\textbf{Output:} \\
The number of officers assigned to each shift \( x_{s} \) for \( s = 1, \ldots, S \) and the total cost.

\end{document}