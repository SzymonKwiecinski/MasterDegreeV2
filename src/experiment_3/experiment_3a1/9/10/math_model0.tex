\documentclass{article}
\usepackage{amsmath}
\begin{document}

\section*{Linear Programming Model for School Assignment Problem}

\textbf{Sets:}
\begin{itemize}
    \item \( N \): Set of neighborhoods \( n = 1, \ldots, N \)
    \item \( S \): Set of schools \( s = 1, \ldots, S \)
    \item \( G \): Set of grades \( g = 1, \ldots, G \)
\end{itemize}

\textbf{Parameters:}
\begin{itemize}
    \item \( capacity_{s,g} \): Capacity of school \( s \) for grade \( g \)
    \item \( population_{n,g} \): Population of grade \( g \) in neighborhood \( n \)
    \item \( d_{n,s} \): Distance from neighborhood \( n \) to school \( s \)
\end{itemize}

\textbf{Decision Variables:}
\begin{itemize}
    \item \( x_{n,s,g} \): Number of students of grade \( g \) from neighborhood \( n \) assigned to school \( s \)
\end{itemize}

\textbf{Objective Function:}
\[
\text{Minimize } Z = \sum_{n=1}^{N} \sum_{s=1}^{S} \sum_{g=1}^{G} d_{n,s} \cdot x_{n,s,g}
\]

\textbf{Subject to:}

1. \textbf{Capacity Constraints:}
\[
\sum_{n=1}^{N} x_{n,s,g} \leq capacity_{s,g}, \quad \forall s \in S, \forall g \in G
\]

2. \textbf{Population Constraints:}
\[
\sum_{s=1}^{S} x_{n,s,g} = population_{n,g}, \quad \forall n \in N, \forall g \in G
\]

3. \textbf{Non-negativity Constraints:}
\[
x_{n,s,g} \geq 0, \quad \forall n \in N, \forall s \in S, \forall g \in G
\]

\textbf{Output:}
The output will provide the optimal \( x_{n,s,g} \) for all neighborhoods \( n \), schools \( s \), and grades \( g \), as well as the total distance traveled by all students, defined as:
\[
\text{total\_distance} = Z
\]

\end{document}