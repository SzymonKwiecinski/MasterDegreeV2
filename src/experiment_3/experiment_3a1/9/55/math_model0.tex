\documentclass{article}
\usepackage{amsmath}
\begin{document}

\section*{Linear Programming Model}

\subsection*{Variables}
Let \( b_p \) be the number of batches produced for part \( p \), where \( p = 1, \ldots, P \).

\subsection*{Objective Function}
The objective is to maximize the total profit:

\[
\text{Maximize } Z = \sum_{p=1}^{P} (price_p \cdot b_p) - \sum_{m=1}^{M} (cost_m \cdot \text{hours\_used}_m) - \text{labor\_cost}
\]

where \( \text{hours\_used}_m = \sum_{p=1}^{P} (time_{m,p} \cdot b_p / 100) \) for \( m = 1, \ldots, M \).

The labor cost for machine 1 (outsourced) is calculated as:

\[
\text{labor\_cost} = 
\begin{cases} 
standard\_cost \cdot \text{hours\_used}_1 & \text{if } \text{hours\_used}_1 \leq overtime\_hour \\
standard\_cost \cdot overtime\_hour + overtime\_cost \cdot (\text{hours\_used}_1 - overtime\_hour) & \text{if } \text{hours\_used}_1 > overtime\_hour 
\end{cases}
\]

\subsection*{Constraints}
1. **Batch Production Requirements**:
   \[
   b_p \geq min\_batches_p \quad \forall p \in \{1, \ldots, P\}
   \]

2. **Machine Availability**:
   \[
   \sum_{p=1}^{P} \left( \frac{time_{m,p}}{100} b_p \right) \leq available_m \quad \forall m \in \{1, \ldots, M\}
   \]

3. **Minimum Profit Requirement**:
   \[
   Z \geq min\_profit
   \]

4. **Non-negativity**:
   \[
   b_p \geq 0 \quad \forall p \in \{1, \ldots, P\}
   \]

\subsection*{Output}
The output will consist of:
- The number of batches of each part produced \( b_p \) for \( p = 1, \ldots, P \).
- The total profit \( Z \).

\end{document}