\documentclass{article}
\usepackage{amsmath}
\begin{document}

\section*{Mathematical Model for the Floppy Disk Backup Problem}

\subsection*{Parameters}
\begin{itemize}
    \item $C$: Capacity of each floppy disk (in GB).
    \item $N$: Total number of files.
    \item $size_{j}$: Size of file $j$ (for $j = 1, \ldots, N$).
\end{itemize}

\subsection*{Decision Variables}
\begin{itemize}
    \item $x_{ij}$: Binary variable that indicates whether file $j$ is placed on disk $i$ ($x_{ij} = 1$) or not ($x_{ij} = 0$).
    \item $n_{d}$: Total number of floppy disks used.
\end{itemize}

\subsection*{Objective Function}
Minimize the total number of disks used:
\[
\text{Minimize } n_{d}
\]

\subsection*{Constraints}
\begin{itemize}
    \item Each file must be assigned to exactly one disk:
    \[
    \sum_{i=1}^{n_{d}} x_{ij} = 1, \quad \forall j = 1, \ldots, N
    \]
    
    \item The total size of files on each disk cannot exceed the capacity:
    \[
    \sum_{j=1}^{N} size_{j} \cdot x_{ij} \leq C, \quad \forall i = 1, \ldots, n_{d}
    \]

    \item Number of disks used must be counted:
    \[
    n_{d} \geq x_{ij}, \quad \forall i = 1, \ldots, n_{d}, \forall j = 1, \ldots, N
    \]
\end{itemize}

\subsection*{Output Variables}
\begin{itemize}
    \item $n_{d}$: Total number of floppy disks used.
    \item $whichdisk$: A list where $whichdisk[j]$ indicates the disk on which file $j$ is stored (for $j = 1, \ldots, N$).
\end{itemize}

\end{document}