\documentclass{article}
\usepackage{amsmath}
\begin{document}

\section*{Mathematical Model}

\subsection*{Parameters}
\begin{itemize}
    \item Let \( period \) be the number of days each nurse works in a row.
    \item Let \( d_j \) be the demand for nurses on day \( j \) where \( j = 1, 2, \ldots, 7 \).
\end{itemize}

\subsection*{Decision Variables}
\begin{itemize}
    \item Let \( start_j \) be the number of nurses that start their period on day \( j \) where \( j = 1, 2, \ldots, 7 \).
    \item Let \( total \) be the total number of nurses hired.
\end{itemize}

\subsection*{Objective Function}
The goal is to minimize the total number of nurses hired:
\[
\text{Minimize } total = \sum_{j=1}^{7} start_j
\]

\subsection*{Constraints}
The number of nurses available on each day must meet the demand. For each day \( j \):
\[
\sum_{i=0}^{period-1} start_{(j-i) \mod 7} \geq d_j \quad \text{for } j = 1, 2, \ldots, 7
\]
Here, \( (j-i) \mod 7 \) ensures that the counting wraps around the week.

\subsection*{Non-negativity Constraints}
\[
start_j \geq 0 \quad \text{for } j = 1, 2, \ldots, 7
\]
\[
total \geq 0
\]

\end{document}