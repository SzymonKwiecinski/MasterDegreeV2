\documentclass{article}
\usepackage{amsmath}
\begin{document}

\section*{Linear Programming Model for Finding the Chebychev Center}

Given the set \( P \) defined by the linear inequality constraints:

\[
P = \{ x \in \mathbb{R}^N \mid a_i^T x \leq b_i, \; i = 1, \ldots, m \}
\]

where \( a_i \in \mathbb{R}^N \) and \( b_i \in \mathbb{R} \), we aim to find a ball with the largest possible radius \( r \) that is entirely contained within the set \( P \). The ball is defined with center \( y \in \mathbb{R}^N \) and radius \( r \) as the set of points \( x \) such that:

\[
\| x - y \|_2 \leq r
\]

This leads to the following optimization problem:

\subsection*{Objective Function}
Maximize the radius \( r \):

\[
\text{maximize } r
\]

\subsection*{Constraints}
The ball must be entirely contained within \( P \), which translates to the following constraints for every \( i = 1, \ldots, m \):

\[
a_i^T (y + r z) \leq b_i, \quad \forall z \in \mathbb{R}^N \text{ such that } \| z \|_2 = 1
\]

Since \( z \) is a unit vector, we can rewrite the constraints more conveniently. We can derive that the distance from the point \( y \) to the boundary defined by the constraints \( a_i^T x = b_i \) must be at least \( r \):

1. For each \( i \):

\[
\frac{b_i - a_i^T y}{\| a_i \|_2} \geq r
\]

This yields the constraints:

\[
r \leq \frac{b_i - a_i^T y}{\| a_i \|_2}, \quad \forall i = 1, \ldots, m
\]

\subsection*{Final Formulation}
Thus, the Linear Programming problem can be summarized as:

\[
\begin{align*}
\text{maximize} & \quad r \\
\text{subject to} & \quad r \leq \frac{b_i - a_i^T y}{\| a_i \|_2}, \quad \forall i = 1, \ldots, m \\
                   & \quad y \in \mathbb{R}^N
\end{align*}
\]

The output will provide the center \( y \) of the ball and the radius \( r \):

\subsection*{Output Format}
\[
\text{Output} = \{ \text{center: } [y_j \text{ for } j = 1, \ldots, N], \; \text{radius: } r \}
\]

\end{document}