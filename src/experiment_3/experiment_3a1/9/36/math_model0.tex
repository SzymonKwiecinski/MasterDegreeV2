\documentclass{article}
\usepackage{amsmath}
\begin{document}

\title{Linear Programming Model for Alloy Production}
\author{}
\date{}
\maketitle

\section*{Problem Formulation}

Let:
\begin{itemize}
    \item \( A \) be the total amount of alloy required (\( \text{alloy\_quant} \)).
    \item \( T_m \) be the target quantity of metal \( m \) (\( \text{target}_{m} \)).
    \item \( R_{k,m} \) be the ratio of metal \( m \) in alloy \( k \) (\( \text{ratio}_{k, m} \)).
    \item \( P_k \) be the price of alloy \( k \) (\( \text{price}_{k} \)).
    \item \( x_k \) be the amount of alloy \( k \) to purchase.
\end{itemize}

\section*{Objective Function}

The objective is to minimize the total cost of the alloys purchased:

\[
\text{Minimize} \quad Z = \sum_{k=1}^{K} P_k \cdot x_k
\]

\section*{Constraints}

1. The total weight of all alloys must equal the desired amount of the alloy:
\[
\sum_{k=1}^{K} x_k = A
\]

2. The total amount of each metal \( m \) in the alloys must meet the target quantity:
\[
\sum_{k=1}^{K} R_{k,m} \cdot x_k = T_m \quad \forall m = 1, \ldots, M
\]

3. Non-negativity constraints:
\[
x_k \geq 0 \quad \forall k = 1, \ldots, K
\]

\section*{Solution Variables}

The solution variable represents the amounts of each alloy to purchase:
\[
\text{amount} = [x_1, x_2, \ldots, x_K]
\]

\end{document}