\documentclass{article}
\usepackage{amsmath}
\begin{document}

\section*{Quadratic Curve Fitting Problem}

We aim to fit a quadratic curve of the form:

\[
y = c x^2 + b x + a
\]

to a set of data points \((x_k, y_k)\) for \(k = 1, \ldots, K\). The objective is to minimize the sum of absolute deviations between the observed values of \(y\) and the values predicted by the quadratic model.

\subsection*{Objective Function}

The objective function can be expressed as:

\[
\text{Minimize} \quad Z = \sum_{k=1}^{K} |y_k - (c x_k^2 + b x_k + a)|
\]

where:
- \(y_k\) is the observed value at point \(k\),
- \(x_k\) is the corresponding input value at point \(k\),
- \(c\) is the coefficient of the quadratic term,
- \(b\) is the coefficient of the linear term,
- \(a\) is the coefficient of the constant term.

\subsection*{Formulation}

To transform the absolute value into a linear problem, we introduce auxiliary variables \(u_k\) for each \(k\):

\[
u_k \geq y_k - (c x_k^2 + b x_k + a)
\]
\[
u_k \geq -(y_k - (c x_k^2 + b x_k + a))
\]

Thus, we can reformulate the objective function as:

\[
\text{Minimize} \quad Z = \sum_{k=1}^{K} u_k
\]

subject to the constraints:

\[
u_k \geq y_k - (c x_k^2 + b x_k + a) \quad \text{for } k = 1, \ldots, K
\]
\[
u_k \geq -(y_k - (c x_k^2 + b x_k + a)) \quad \text{for } k = 1, \ldots, K
\]

\subsection*{Output Variables}

The solution to the above linear programming problem will yield the coefficients:

\begin{itemize}
    \item \textbf{quadratic:} \(c\)
    \item \textbf{linear:} \(b\)
    \item \textbf{constant:} \(a\)
\end{itemize}

\end{document}