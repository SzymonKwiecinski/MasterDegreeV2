\documentclass{article}
\usepackage{amsmath}
\begin{document}

\title{Mathematical Model for Cafeteria Staff Scheduling}
\author{}
\date{}
\maketitle

\section*{Problem Definition}
To operate a cafeteria, the manager must ensure on-call duty based on the statistics on the required staff. The desired number of employees on day \( n \) is denoted as \( num_n \).

\subsection*{Parameters}
\begin{itemize}
    \item \( N \): Total number of days
    \item \( num_n \): Number of desired employees on day \( n \), for \( n = 1, 2, \ldots, N \)
    \item \( n_{working\_days} \): Number of consecutive days each employee works
    \item \( n_{resting\_days} \): Number of days off each employee has
\end{itemize}

\subsection*{Decision Variables}
\begin{itemize}
    \item \( total\_number \): Total number of employees to hire
    \item \( is\_work_{n,i} \): A binary variable that indicates if employee \( i \) is working on day \( n \) (1) or resting (0)
\end{itemize}

\subsection*{Objective Function}
Minimize the total number of employees hired:
\[
\text{Minimize } total\_number
\]

\subsection*{Constraints}
1. Ensure that the number of working employees meets the required staff for each day:
\[
\sum_{i=1}^{total\_number} is\_work_{n,i} \geq num_n, \quad \forall n = 1, 2, \ldots, N
\]
2. Each employee has a working and resting cycle:
\[
is\_work_{n,i} = 1 \text{ for } n \in [k \cdot (n_{working\_days} + n_{resting\_days}) + 1, k \cdot (n_{working\_days} + n_{resting\_days}) + n_{working\_days}] \quad \text{for some integer } k
\]
\[
is\_work_{n,i} = 0 \text{ otherwise}
\]
3. Non-negativity constraint on the decision variable:
\[
total\_number \geq 0
\]

\section*{Output}
\begin{itemize}
    \item \( total\_number \): Total number of employees to employ
    \item \( is\_work \): A matrix where \( is\_work_{n,i} \) indicates the working/resting status of employee \( i \) on day \( n \)
\end{itemize}

\end{document}