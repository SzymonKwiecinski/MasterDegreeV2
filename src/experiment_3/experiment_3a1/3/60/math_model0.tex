\documentclass{article}
\usepackage{amsmath}
\begin{document}

\title{Mathematical Model for Consultant Assignment Problem}
\author{}
\date{}
\maketitle

\section*{Problem Statement}
A company has \( I \) projects to be done by consultants. There are \( J \) available consultants, each of whom may be assigned up to \( K \) projects. Hiring consultant \( j \) incurs a fixed cost of \( f_j \), and assigning project \( i \) to consultant \( j \) incurs an additional cost of \( c_{i,j} \).

\section*{Decision Variables}
Define the binary decision variables:
\[
x_j = 
\begin{cases}
1, & \text{if consultant } j \text{ is hired} \\
0, & \text{otherwise}
\end{cases}
\]
\[
y_{i,j} = 
\begin{cases}
1, & \text{if project } i \text{ is assigned to consultant } j \\
0, & \text{otherwise}
\end{cases}
\]

\section*{Objective Function}
The objective is to minimize the total cost, which is the sum of the fixed costs of hiring consultants and the costs of assigning projects to consultants:
\[
\text{Minimize} \quad Z = \sum_{j=1}^{J} f_j x_j + \sum_{i=1}^{I} \sum_{j=1}^{J} c_{i,j} y_{i,j}
\]

\section*{Constraints}
1. Each project must be assigned to exactly one consultant:
\[
\sum_{j=1}^{J} y_{i,j} = 1, \quad \forall i = 1, \ldots, I
\]

2. A consultant can only be assigned a maximum of \( K \) projects:
\[
\sum_{i=1}^{I} y_{i,j} \leq K x_j, \quad \forall j = 1, \ldots, J
\]

3. Each assignment decision variable must be binary:
\[
y_{i,j} \in \{0, 1\}, \quad \forall i = 1, \ldots, I, \forall j = 1, \ldots, J
\]

4. Each hiring decision variable must be binary:
\[
x_j \in \{0, 1\}, \quad \forall j = 1, \ldots, J
\]

\section*{Output Variables}
The output consists of:
\begin{itemize}
    \item \( \text{assignments} = \left[ \left[y_{i,j}\right]_{i=1}^{I} \right]_{j=1}^{J} \)
    \item \( \text{total\_cost} = Z \)
\end{itemize}

\end{document}