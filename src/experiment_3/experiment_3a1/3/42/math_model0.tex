\documentclass{article}
\usepackage{amsmath}
\begin{document}

\title{Linear Programming Model for Container Transportation}
\author{}
\date{}
\maketitle

\section*{Problem Definition}

An Italian transport company must send empty containers from its depots to ports while minimizing transportation costs. 

\subsection*{Variables}
Let:
\begin{itemize}
    \item \( numdepot_i \) represent the number of containers available in depot \( i \) for \( i = 1, \ldots, I \).
    \item \( numport_j \) represent the container requirement in port \( j \) for \( j = 1, \ldots, J \).
    \item \( distance_{i,j} \) represent the distance from depot \( i \) to port \( j \).
    \item \( price \) represent the cost of transport per distance unit (Euros/km).
    \item \( number_{i,j} \) represent the number of containers sent from depot \( i \) to port \( j \).
\end{itemize}

\subsection*{Objective Function}
The objective is to minimize the total transportation cost:

\[
\text{Minimize } Z = \sum_{i=1}^{I} \sum_{j=1}^{J} \left( \frac{number_{i,j}}{2} \cdot distance_{i,j} \cdot price \right)
\]

where \( \frac{number_{i,j}}{2} \) accounts for the fact that each barge can contain two containers.

\subsection*{Constraints}

1. The total containers sent from each depot cannot exceed the available containers in the depot:
\[
\sum_{j=1}^{J} number_{i,j} \leq numdepot_i \quad \forall i = 1, \ldots, I
\]

2. The total containers received at each port must meet the demand of the port:
\[
\sum_{i=1}^{I} number_{i,j} \geq numport_j \quad \forall j = 1, \ldots, J
\]

3. The number of containers sent from each depot to each port must be non-negative:
\[
number_{i,j} \geq 0 \quad \forall i = 1, \ldots, I, \; j = 1, \ldots, J
\]

\end{document}