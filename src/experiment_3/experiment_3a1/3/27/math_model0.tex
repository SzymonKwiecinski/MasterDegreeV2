\documentclass{article}
\usepackage{amsmath}
\begin{document}

\title{Mathematical Model for Department Relocation}
\author{}
\date{}
\maketitle

\section*{Problem Definition}
A large company is considering relocating its departments from London to various cities to minimize overall yearly costs, accounting for benefits derived from relocation and communication costs between departments.

\section*{Sets and Indices}
\begin{itemize}
    \item Let \( K \) be the number of departments.
    \item Let \( L \) be the number of possible cities for relocation including London.
    \end{itemize}

\section*{Parameters}
\begin{itemize}
    \item \( \text{benefit}_{k, l} \): The benefit for department \( k \) from being relocated to city \( l \) (in thousands of pounds).
    \item \( \text{communication}_{k, j} \): The quantity of communication between department \( k \) and city \( j \).
    \item \( \text{cost}_{l, m} \): The unit cost of communication between city \( l \) and city \( m \).
\end{itemize}

\section*{Decision Variables}
\begin{itemize}
    \item \( \text{islocated}_{k, l} \): A binary variable that is 1 if department \( k \) is located in city \( l \), and 0 otherwise.
\end{itemize}

\section*{Objective Function}
We aim to minimize the overall yearly cost, which is given by:
\[
\text{Minimize} \quad Z = \sum_{k=1}^{K} \sum_{l=1}^{L} \left( \text{communication}_{k,j} \cdot \text{cost}_{l,m} \cdot \text{islocated}_{k,l} \right) - \sum_{k=1}^{K} \sum_{l=1}^{L} \text{benefit}_{k,l} \cdot \text{islocated}_{k,l}
\]

\section*{Constraints}
\begin{itemize}
    \item Each department must be located in exactly one city:
    \[
    \sum_{l=1}^{L} \text{islocated}_{k,l} = 1, \quad \forall k \in \{1, 2, \ldots, K\}
    \]
    
    \item No more than three departments may be located in any city:
    \[
    \sum_{k=1}^{K} \text{islocated}_{k,l} \leq 3, \quad \forall l \in \{1, 2, \ldots, L\}
    \]
    
    \item The binary nature of decision variables:
    \[
    \text{islocated}_{k,l} \in \{0, 1\}, \quad \forall k \in \{1, 2, \ldots, K\}, \forall l \in \{1, 2, \ldots, L\}
    \]
\end{itemize}

\section*{Output}
The output will be a matrix indicating the location of each department:
\[
\text{islocated} = \begin{bmatrix}
\text{islocated}_{1,1} & \text{islocated}_{1,2} & \cdots & \text{islocated}_{1,L} \\
\text{islocated}_{2,1} & \text{islocated}_{2,2} & \cdots & \text{islocated}_{2,L} \\
\vdots & \vdots & \ddots & \vdots \\
\text{islocated}_{K,1} & \text{islocated}_{K,2} & \cdots & \text{islocated}_{K,L} \\
\end{bmatrix}
\]

\end{document}