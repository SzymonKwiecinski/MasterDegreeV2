\documentclass{article}
\usepackage{amsmath}
\begin{document}

\section*{Mathematical Model for Police Officer Shift Assignment}

\subsection*{Parameters}
\begin{itemize}
    \item Let \( S \) = number of different shifts
    \item Let \( officers\_needed_{s} \) = number of police officers needed for shift \( s \) for \( s = 1, 2, \ldots, S \)
    \item Let \( shift\_cost_{s} \) = cost for the town when starting shift \( s \)
\end{itemize}

\subsection*{Decision Variables}
\begin{itemize}
    \item Let \( officers\_assigned_{s} \) = number of officers assigned to shift \( s \)
\end{itemize}

\subsection*{Objective Function}
The objective is to minimize the total cost:
\[
\text{Minimize} \quad \text{total\_cost} = \sum_{s=1}^{S} shift\_cost_{s} \cdot x_{s}
\]
where \( x_{s} \) is a binary variable indicating whether shift \( s \) is started (1 if started, 0 if not). 

\subsection*{Constraints}
1. Each police officer works for two consecutive shifts. Therefore:
   \[
   officers\_assigned_{s} \geq officers\_needed_{s} \quad \forall s = 1, 2, \ldots, S
   \]
2. Each shift requires a sufficient number of officers:
   \[
   officers\_assigned_{s} + officers\_assigned_{s+1} \geq officers\_needed_{s} \quad \forall s = 1, 2, \ldots, S-1
   \]
3. Non-negativity and integer constraints:
   \[
   officers\_assigned_{s} \geq 0 \quad \forall s
   \]

\subsection*{Output}
\begin{itemize}
    \item The list of assigned officers: \( officers\_assigned = [officers\_assigned_{1}, officers\_assigned_{2}, ..., officers\_assigned_{S}] \)
    \item The total cost: \( total\_cost \)
\end{itemize}

\end{document}