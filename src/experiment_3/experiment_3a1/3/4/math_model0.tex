\documentclass{article}
\usepackage{amsmath}
\begin{document}

\section*{Mathematical Model for Nurse Scheduling}

\subsection*{Parameters}
\begin{itemize}
    \item $period$: Number of consecutive days a nurse works on the night shift.
    \item $d_j$: Demand for nurses on night shift for day $j$, where $j = 1, \ldots, 7$.
\end{itemize}

\subsection*{Decision Variables}
\begin{itemize}
    \item $start_j$: Number of nurses that start their period on day $j$, where $j = 1, \ldots, 7$.
    \item $N$: Total number of nurses hired.
\end{itemize}

\subsection*{Objective Function}
Minimize the total number of nurses hired:
\[
\text{Minimize } N = \sum_{j=1}^{7} start_j
\]

\subsection*{Constraints}
The number of nurses available on each day $j$ must meet the demand $d_j$:
\[
\sum_{k=0}^{period-1} start_{(j-k) \mod 7} \geq d_j \quad \forall j \in \{1, \ldots, 7\}
\]
where $(j-k) \mod 7$ is used to wrap around the days of the week.

\subsection*{Non-negativity Constraints}
\[
start_j \geq 0 \quad \forall j \in \{1, \ldots, 7\}
\]
\[
N \geq 0
\]

\subsection*{Output}
The solution includes:
\begin{itemize}
    \item $start = [start_1, start_2, \ldots, start_7]$: Number of nurses starting on each day.
    \item $total$: Total number of nurses hired.
\end{itemize}

\end{document}