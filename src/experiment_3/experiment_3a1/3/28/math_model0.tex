\documentclass{article}
\usepackage{amsmath}
\begin{document}

\section*{Linear Programming Model for Fitting a Line}

Given the data points \( (x_k, y_k) \) for \( k = 1, \ldots, K \), we want to fit the best straight line of the form:
\[
y = bx + a
\]
Our objective is to minimize the sum of absolute deviations of each observed value of \( y \) from the value predicted by the linear relationship. This can be formulated as follows:

\subsection*{Variables}
Let:
\begin{itemize}
    \item \( a \): Intercept of the fitted line
    \item \( b \): Slope of the fitted line
    \item \( d_k \): Absolute deviation for each observation, defined as \( d_k = |y_k - (bx_k + a)| \)
\end{itemize}

\subsection*{Objective Function}
The objective is to minimize the sum of the absolute deviations:
\[
\text{Minimize } Z = \sum_{k=1}^{K} d_k
\]

\subsection*{Constraints}
For each observation \( k \):
\[
d_k \geq y_k - (bx_k + a) \quad \text{for } k = 1, \ldots, K
\]
\[
d_k \geq -(y_k - (bx_k + a)) \quad \text{for } k = 1, \ldots, K
\]

\subsection*{Output}
The output will be in the format:
\[
\{ \text{"intercept"}: a, \text{"slope"}: b \}
\]

\end{document}