\documentclass{article}
\usepackage{amsmath}
\begin{document}

\title{Linear Programming Model for Auto Parts Production}
\author{}
\date{}
\maketitle

\section*{Problem Definition}

Let:
\begin{itemize}
    \item \( P \): Number of different parts
    \item \( M \): Number of different machines
    \item \( batches_{p} \): Number of batches produced for part \( p \) (for \( p = 1, \ldots, P \))
    \item \( time_{m,p} \): Time required on machine \( m \) to produce one batch of part \( p \)
    \item \( cost_{m} \): Cost per hour for using machine \( m \)
    \item \( available_{m} \): Availability hours of machine \( m \)
    \item \( price_{p} \): Selling price per batch of part \( p \)
    \item \( min\_batches_{p} \): Minimum required batches of part \( p \)
\end{itemize}

\subsection*{Objective Function}

The objective is to maximize the total profit defined as:

\[
\text{maximize } Z = \sum_{p=1}^{P} (price_{p} \cdot batches_{p}) - \sum_{m=1}^{M} \left( cost_{m} \cdot \sum_{p=1}^{P} (time_{m,p} \cdot batches_{p}) \right)
\]

\subsection*{Constraints}

1. **Machine Time Constraints**:
   For each machine \( m \):
   \[
   \sum_{p=1}^{P} (time_{m,p} \cdot batches_{p}) \leq available_{m} \quad \forall m = 1, \ldots, M
   \]
   
2. **Minimum Production Requirements**:
   For each part \( p \):
   \[
   batches_{p} \geq min\_batches_{p} \quad \forall p = 1, \ldots, P
   \]

3. **Non-negativity**:
   \[
   batches_{p} \geq 0 \quad \forall p = 1, \ldots, P
   \]

\subsection*{Output}

The output will consist of:
\begin{itemize}
    \item \( batches \): The number of batches produced for each part \( p \) 
    \item \( total\_profit \): The total profit obtained for the month, calculated as:
    \[
    total\_profit = Z
    \end{itemize}
\end{document}