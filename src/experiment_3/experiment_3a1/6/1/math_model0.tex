\documentclass{article}
\usepackage{amsmath}
\begin{document}

\section*{Linear Programming Model}

Let:
\begin{itemize}
    \item \( M \): Number of different goods
    \item \( N \): Number of different raw materials
    \item \( \text{available}_i \): Available units of raw material \( i \) for \( i = 1, \ldots, N \)
    \item \( \text{req}_{i,j} \): Units of raw material \( i \) required to produce one unit of good \( j \) for \( j = 1, \ldots, M \)
    \item \( \text{price}_j \): Revenue generated from selling one unit of good \( j \) for \( j = 1, \ldots, M \)
    \item \( \text{amount}_j \): Amount of good \( j \) produced
\end{itemize}

\subsection*{Objective Function}
The objective is to maximize the total revenue:

\[
\text{Maximize } Z = \sum_{j=1}^{M} \text{price}_j \cdot \text{amount}_j
\]

\subsection*{Constraints}
The production of goods is limited by the availability of raw materials:

\[
\sum_{j=1}^{M} \text{req}_{i,j} \cdot \text{amount}_j \leq \text{available}_i \quad \text{for } i = 1, \ldots, N
\]

Additionally, the production amounts must be non-negative:

\[
\text{amount}_j \geq 0 \quad \text{for } j = 1, \ldots, M
\]

\subsection*{Summary}
The linear programming model can be summarized as follows:

\begin{align*}
\text{Maximize } & Z = \sum_{j=1}^{M} \text{price}_j \cdot \text{amount}_j \\
\text{Subject to: } & \sum_{j=1}^{M} \text{req}_{i,j} \cdot \text{amount}_j \leq \text{available}_i, \quad i = 1, \ldots, N \\
& \text{amount}_j \geq 0, \quad j = 1, \ldots, M
\end{align*}

\end{document}