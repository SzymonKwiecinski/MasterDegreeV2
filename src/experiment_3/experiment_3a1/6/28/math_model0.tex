\documentclass{article}
\usepackage{amsmath}

\begin{document}

\section*{Linear Programming Model for Fitting a Line}

Given a set of points \((x_k, y_k)\) for \(k = 1, \ldots, K\), we want to fit a linear model of the form:

\[
y = bx + a
\]

where \(b\) is the slope and \(a\) is the intercept. The objective is to minimize the sum of absolute deviations defined as:

\[
\min \sum_{k=1}^{K} |y_k - (bx_k + a)|
\]

To handle the absolute values, we introduce auxiliary variables \(e_k\) for each data point \(k\):

\[
e_k \geq y_k - (bx_k + a)
\]
\[
-e_k \geq -y_k + (bx_k + a)
\]

This allows us to express the absolute deviation as:

\[
e_k \geq |y_k - (bx_k + a)|
\]

The complete linear programming formulation is thus:

\begin{align*}
\text{Minimize} \quad & \sum_{k=1}^{K} e_k \\
\text{Subject to} \quad & e_k \geq y_k - (bx_k + a), \quad k = 1, \ldots, K \\
& e_k \geq -y_k + (bx_k + a), \quad k = 1, \ldots, K \\
& b, a, e_k \text{ are free variables}
\end{align*}

Upon solving this linear program, we obtain the values of the intercept \(a\) and slope \(b\) that minimize the sum of absolute deviations.

The output format is as follows:

\[
\{
    \text{"intercept"}: a,
    \text{"slope"}: b
\}
\]

\end{document}