\documentclass{article}
\usepackage{amsmath}
\begin{document}

\section*{Linear Programming Model for Oil Refinery}

\subsection*{Decision Variables}
Let \( x_l \) be the number of times process \( l \) should be executed, for \( l = 1, 2, \ldots, L \).

\subsection*{Objective Function}
The objective is to maximize the total revenue, defined as:
\[
\text{maximize } Z = \sum_{p=1}^{P} \text{price}_p \cdot \sum_{l=1}^{L} \text{output}_{l,p} \cdot x_l
\]

\subsection*{Constraints}
1. **Resource Constraints**: For each crude oil type \( i \):
\[
\sum_{l=1}^{L} \text{input}_{l,i} \cdot x_l \leq \text{allocated}_i \quad \forall i = 1, 2, \ldots, O
\]

2. **Non-negativity Constraints**: 
\[
x_l \geq 0 \quad \forall l = 1, 2, \ldots, L
\]

\subsection*{Output Variables}
- The total revenue for the month is given by the objective function value \( Z \).
- The number of times process \( l \) should be executed is given by \( x_l \).

\subsection*{Output Format}
The output will provide:
\begin{itemize}
    \item \text{revenue}: \( Z \)
    \item \text{execute}: \( [x_1, x_2, \ldots, x_L] \)
\end{itemize}

\end{document}