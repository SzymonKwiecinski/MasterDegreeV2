\documentclass{article}
\usepackage{amsmath}
\begin{document}

\section*{Linear Programming Model for Electric Utility}

\subsection*{Definitions}

\begin{itemize}
    \item Let \( P \) be the number of power plants.
    \item Let \( C \) be the number of cities.
    \item Let \( supply_p \) be the capacity of power plant \( p \) in million kWh.
    \item Let \( demand_c \) be the peak demand of city \( c \) in million kWh.
    \item Let \( transmission_{p,c} \) be the transmission cost from power plant \( p \) to city \( c \) in $/million kWh.
    \item Let \( send_{p,c} \) be the amount of electricity sent from power plant \( p \) to city \( c \) in million kWh.
\end{itemize}

\subsection*{Objective Function}

Minimize the total transmission cost:

\[
\text{Total Cost} = \sum_{p=1}^{P} \sum_{c=1}^{C} transmission_{p,c} \cdot send_{p,c}
\]

\subsection*{Constraints}

\begin{itemize}
    \item Supply constraints for each power plant:
\[
\sum_{c=1}^{C} send_{p,c} \leq supply_p \quad \forall p = 1, \ldots, P
\]

    \item Demand constraints for each city:
\[
\sum_{p=1}^{P} send_{p,c} \geq demand_c \quad \forall c = 1, \ldots, C
\]

    \item Non-negativity constraints:
\[
send_{p,c} \geq 0 \quad \forall p = 1, \ldots, P \text{ and } c = 1, \ldots, C
\end{itemize}

\subsection*{Output}

The output should provide:

\begin{itemize}
    \item The amounts \( send_{p,c} \) for each power plant \( p \) and city \( c \).
    \item The total transmission cost as \( total\_cost \).
\end{itemize}

\end{document}