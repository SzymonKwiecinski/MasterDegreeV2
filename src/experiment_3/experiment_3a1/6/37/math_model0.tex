\documentclass{article}
\usepackage{amsmath}
\begin{document}

\title{Linear Programming Model for Spare Automobile Parts Production}
\author{}
\date{}
\maketitle

\section*{Problem Definition}
A small firm specializes in making various types of spare automobile parts. Each part requires a certain number of worker-hours in multiple shops, and the goal is to maximize profit from the production of these parts.

\section*{Parameters}
\begin{itemize}
    \item Let \( K \) be the number of different spare parts.
    \item Let \( S \) be the number of shops.
    \item \( \text{time}_{k,s} \): Required worker-hours for part \( k \) in shop \( s \).
    \item \( \text{profit}_{k} \): Profit from part \( k \).
    \item \( \text{capacity}_{s} \): Capacity of shop \( s \) in working hours.
\end{itemize}

\section*{Decision Variables}
Let \( \text{quantity}_{k} \) represent the quantity of each spare part \( k \) to be produced.

\section*{Objective Function}
The objective is to maximize total profit:
\[
\text{Maximize } Z = \sum_{k=1}^{K} \text{profit}_{k} \cdot \text{quantity}_{k}
\]

\section*{Constraints}
The production of parts must adhere to the capacity constraints of each shop. For each shop \( s \):
\[
\sum_{k=1}^{K} \text{time}_{k,s} \cdot \text{quantity}_{k} \leq \text{capacity}_{s}, \quad \forall s = 1, \ldots, S
\]

Additionally, the quantities of parts produced must be non-negative:
\[
\text{quantity}_{k} \geq 0, \quad \forall k = 1, \ldots, K
\]

\section*{Model Formulation}
The complete linear programming model can be summarized as follows:

\begin{align*}
\text{Maximize} \quad & Z = \sum_{k=1}^{K} \text{profit}_{k} \cdot \text{quantity}_{k} \\
\text{Subject to} \quad & \sum_{k=1}^{K} \text{time}_{k,s} \cdot \text{quantity}_{k} \leq \text{capacity}_{s}, \quad \forall s = 1, \ldots, S \\
& \text{quantity}_{k} \geq 0, \quad \forall k = 1, \ldots, K
\end{align*}

\end{document}