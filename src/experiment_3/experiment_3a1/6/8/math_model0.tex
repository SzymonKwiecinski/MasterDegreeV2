\documentclass{article}
\usepackage{amsmath}
\begin{document}

\section*{Linear Programming Formulation}

Let \( K \) be the number of different spare parts and \( S \) be the number of shops. 

Define the following variables:

\begin{itemize}
    \item \( x_k \): quantity of spare part \( k \) produced, for \( k = 1, 2, \ldots, K \).
\end{itemize}

The objective is to maximize the total profit:

\[
\text{Maximize } Z = \sum_{k=1}^{K} profit_k \cdot x_k
\]

Subject to the constraints imposed by the capacity of each shop. For each shop \( s \):

\[
\sum_{k=1}^{K} time_{k,s} \cdot x_k \leq capacity_s \quad \text{for all } s = 1, 2, \ldots, S
\]

Additionally, we need to ensure non-negativity of the variables:

\[
x_k \geq 0 \quad \text{for all } k = 1, 2, \ldots, K
\]

In summary, the linear programming model can be stated as follows:

\begin{align*}
\text{Maximize} \quad & Z = \sum_{k=1}^{K} profit_k \cdot x_k \\
\text{Subject to} \quad & \sum_{k=1}^{K} time_{k,s} \cdot x_k \leq capacity_s, \quad s = 1, 2, \ldots, S \\
& x_k \geq 0, \quad k = 1, 2, \ldots, K
\end{align*}

\end{document}