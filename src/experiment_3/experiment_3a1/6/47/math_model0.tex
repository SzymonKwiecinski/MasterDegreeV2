\documentclass{article}
\usepackage{amsmath}
\begin{document}

\title{Police Officer Shift Assignment Problem}
\author{}
\date{}
\maketitle

\section*{Problem Definition}

Let:
\begin{itemize}
    \item \( S \) = number of shifts (1, 2, ..., \( S \))
    \item \( officers\_needed_s \) = number of police officers needed during shift \( s \)
    \item \( shift\_cost_s \) = cost incurred when starting shift \( s \)
    \item \( officers\_assigned_s \) = number of officers assigned to shift \( s \)
\end{itemize}

\section*{Objective Function}

The objective is to minimize the total cost for the town, which can be defined as:

\[
\text{Minimize } Z = \sum_{s=1}^{S} shift\_cost_s \cdot x_s
\]

where \( x_s \) is a binary variable indicating whether shift \( s \) is started (1) or not (0). 

\section*{Constraints}

Since each officer works two consecutive shifts, the following constraints must be satisfied for each shift \( s \):

\begin{align*}
officers\_assigned_{s} & \geq officers\_needed_{s}, \quad \forall s \\
officers\_assigned_{s} & = officers\_assigned_{s-1} + officers\_assigned_{s+1}, \quad \forall s \text{ with } 2 \leq s \leq S-1 \\
officers\_assigned_1 & = officers\_assigned_2 \\
officers\_assigned_S & = officers\_assigned_{S-1}
\end{align*}

\section*{Solution Variables}

The solution will consist of:
\begin{itemize}
    \item \( officers\_assigned = [officers\_assigned_s \text{ for } s = 1, ..., S] \)
    \item \( total\_cost = Z \)
\end{itemize}

\end{document}