\documentclass{article}
\usepackage{amsmath}
\begin{document}

\section*{Mathematical Model for Employee Scheduling in Cafeteria}

\subsection*{Parameters}
\begin{itemize}
    \item Let \( N \) be the total number of days.
    \item Let \( num_n \) be the number of employees required on day \( n \) where \( n = 1, 2, \ldots, N \).
    \item Let \( n_{working} \) be the number of consecutive working days for each employee.
    \item Let \( n_{resting} \) be the number of consecutive resting days for each employee.
    \item Let \( T = n_{working} + n_{resting} \) be the total cycle length for each employee.
\end{itemize}

\subsection*{Decision Variables}
\begin{itemize}
    \item Let \( total\_number \) be the total number of employees hired.
    \item Let \( is\_work_{n,i} \) be a binary variable that equals 1 if employee \( i \) is working on day \( n \) and 0 otherwise, where \( i = 1, 2, \ldots, total\_number \) and \( n = 1, 2, \ldots, N \).
\end{itemize}

\subsection*{Objective Function}
Minimize the total number of employees hired:
\[
\text{Minimize } total\_number
\]

\subsection*{Constraints}
\begin{enumerate}
    \item For each day \( n \), the total number of employees working must meet the requirement:
    \[
    \sum_{i=1}^{total\_number} is\_work_{n,i} \geq num_n \quad \forall n = 1, 2, \ldots, N
    \]
    
    \item Each employee must follow their working and resting cycle. For each employee \( i \):
    \[
    is\_work_{n,i} = 
    \begin{cases}
      1, & \text{for } n \mod T < n_{working} \\
      0, & \text{for } n \mod T \geq n_{working}
    \end{cases} \quad \forall n = 1, 2, \ldots, N
    \]
    
\end{enumerate}

\end{document}