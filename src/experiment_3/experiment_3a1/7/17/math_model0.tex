\documentclass{article}
\usepackage{amsmath}
\begin{document}

\section*{Linear Programming Model for Stock Selling}

\textbf{Variables:}

Let \( sell_i \) be the number of shares of stock \( i \) that the investor should sell.

\textbf{Parameters:}

\begin{itemize}
    \item \( N \): Total number of different stocks.
    \item \( bought_i \): Number of shares bought of stock \( i \).
    \item \( buyPrice_i \): Price at which the shares of stock \( i \) were bought.
    \item \( currentPrice_i \): Current price of one share of stock \( i \).
    \item \( futurePrice_i \): Expected price of one share of stock \( i \) in one year.
    \item \( transactionRate \): Transaction cost rate as a percentage.
    \item \( taxRate \): Tax rate on capital gains as a percentage.
    \item \( K \): Amount of money the investor needs to raise net of capital gains and transaction costs.
\end{itemize}

\textbf{Objective Function:}

Maximize the expected value of the portfolio net of sales:

\[
\text{Maximize} \quad \sum_{i=1}^{N} futurePrice_i \cdot (bought_i - sell_i)
\]

\textbf{Constraints:}

1. The total amount raised must be equal to or greater than \( K \) after accounting for transaction costs and taxes:

\[
\sum_{i=1}^{N} (currentPrice_i \cdot sell_i) \cdot (1 - \frac{transactionRate}{100}) - \sum_{i=1}^{N} ((currentPrice_i - buyPrice_i) \cdot sell_i) \cdot \frac{taxRate}{100} \geq K
\]

2. The number of shares sold cannot exceed the number of shares bought:

\[
0 \leq sell_i \leq bought_i \quad \forall i \in \{1, 2, \ldots, N\}
\]

3. Non-negativity constraint:

\[
sell_i \geq 0 \quad \forall i \in \{1, 2, \ldots, N\}
\]

\textbf{Output:}

The result will provide the values of \( sell_i \) for \( i = 1, 2, \ldots, N \):

\[
\text{Output: } \{ "sell": [sell_1, sell_2, \ldots, sell_N] \}
\]

\end{document}