\documentclass{article}
\usepackage{amsmath}
\begin{document}

\section*{Linear Programming Model for Road Illumination}

\textbf{Variable Definitions:}
\begin{itemize}
    \item Let \( N \) be the number of road segments.
    \item Let \( M \) be the number of lamps.
    \item Let \( \text{power}_j \) be the power of the \( j \)-th lamp.
    \item Let \( \text{ill}_i \) be the illumination of the \( i \)-th segment.
    \item Let \( \text{desired}_i \) be the desired illumination of the \( i \)-th segment.
    \item Let \( \text{coeff}_{i,j} \) be the known coefficient representing the effect of the \( j \)-th lamp on the \( i \)-th segment.
    \item Let \( \text{error}_i \) be the absolute error for the \( i \)-th segment, defined as \( |\text{ill}_i - \text{desired}_i| \).
\end{itemize}

\textbf{Model Formulation:}

\subsection*{Objective Function}
We want to minimize the total absolute error:
\[
\text{Minimize } Z = \sum_{i=1}^{N} |\text{ill}_i - \text{desired}_i|
\]

where the illumination for each segment is defined as:
\[
\text{ill}_i = \sum_{j=1}^{M} \text{coeff}_{i,j} \cdot \text{power}_j, \quad \forall i \in \{1, \ldots, N\}
\]

\subsection*{Constraints}
Since absolute values complicate linear programming, we introduce auxiliary variables \( \text{error}_i^+ \) and \( \text{error}_i^- \) to express the absolute errors:
\[
\text{error}_i^+ \geq \text{ill}_i - \text{desired}_i, \quad \forall i \in \{1, \ldots, N\} 
\]
\[
\text{error}_i^- \geq \text{desired}_i - \text{ill}_i, \quad \forall i \in \{1, \ldots, N\}
\]
This leads to:
\[
\text{error}_i = \text{error}_i^+ + \text{error}_i^- \quad \forall i \in \{1, \ldots, N\}
\]
Thus, we redefine our objective function:
\[
\text{Minimize } Z = \sum_{i=1}^{N} (\text{error}_i^+ + \text{error}_i^-)
\]

\subsection*{Non-negativity Constraints}
\[
\text{power}_j \geq 0, \quad \forall j \in \{1, \ldots, M\}
\]
\[
\text{error}_i^+ \geq 0, \quad \forall i \in \{1, \ldots, N\}
\]
\[
\text{error}_i^- \geq 0, \quad \forall i \in \{1, \ldots, N\}
\]

\end{document}