\documentclass{article}
\usepackage{amsmath}
\begin{document}

\section*{Linear Programming Model for Electricity Distribution}

Let:
\begin{itemize}
    \item \( P \): Number of power plants
    \item \( C \): Number of cities
    \item \( supply_{p} \): Capacity of power plant \( p \) in million kWh
    \item \( demand_{c} \): Peak demand of city \( c \) in million kWh
    \item \( transmission_{p,c} \): Transmission cost from power plant \( p \) to city \( c \) in dollars per million kWh
    \item \( send_{p,c} \): Amount of electricity sent from power plant \( p \) to city \( c \) in million kWh
\end{itemize}

\subsection*{Objective Function}
Minimize the total transmission cost:
\[
\text{Total Cost} = \sum_{p=1}^{P} \sum_{c=1}^{C} transmission_{p,c} \cdot send_{p,c}
\]

\subsection*{Constraints}
1. Supply constraints for each power plant:
\[
\sum_{c=1}^{C} send_{p,c} \leq supply_{p}, \quad \forall p \in \{1, \dots, P\}
\]

2. Demand constraints for each city:
\[
\sum_{p=1}^{P} send_{p,c} \geq demand_{c}, \quad \forall c \in \{1, \dots, C\}
\]

3. Non-negativity constraints:
\[
send_{p,c} \geq 0, \quad \forall p \in \{1, \dots, P\}, \; \forall c \in \{1, \dots, C\}
\]

\subsection*{Output}
The output will include:
\begin{itemize}
    \item The amount of electricity each power plant sends to each city: \( send_{p,c} \)
    \item The total cost: \( \text{Total Cost} \)
\end{itemize}

\end{document}