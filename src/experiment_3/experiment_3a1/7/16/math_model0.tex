\documentclass{article}
\usepackage{amsmath}
\begin{document}

\section*{Linear Programming Model for Oil Refinery Production}

\textbf{Decision Variables:}

Let \( x_l \) be the number of times process \( l \) is executed, where \( l = 1, \ldots, L \).

\textbf{Parameters:}

- \( O \): Number of crude oil types
- \( P \): Number of products
- \( L \): Number of production processes
- \( allocated_i \): Allocated crude oil type \( i \) in million barrels
- \( price_p \): Selling price of product \( p \) per barrel
- \( input_{l,i} \): Input of crude oil type \( i \) required by process \( l \) in barrels
- \( output_{l,p} \): Output of product \( p \) produced by process \( l \) in barrels
- \( cost_l \): Cost of executing process \( l \) per barrel of product produced

\textbf{Objective Function:}

We want to maximize the total revenue:

\[
\text{Maximize } Z = \sum_{p=1}^{P} \left( price_p \cdot \sum_{l=1}^{L} output_{l,p} \cdot x_l \right) - \sum_{l=1}^{L} (cost_l \cdot \sum_{p=1}^{P} output_{l,p} \cdot x_l)
\]

\textbf{Constraints:}

The constraints for the model can be formulated as follows:

1. Resource Availability Constraints:
\[
\sum_{l=1}^{L} input_{l,i} \cdot x_l \leq allocated_i \quad \forall i=1,\ldots,O
\]

2. Non-negativity Constraints:
\[
x_l \geq 0 \quad \forall l=1,\ldots,L
\]

\textbf{Output:}

The expected outputs are:
- Total revenue \( revenue \)
- Number of times each process should be executed \( execute_l \)

\end{document}