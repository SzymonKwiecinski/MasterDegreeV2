\documentclass{article}
\usepackage{amsmath}
\begin{document}

\section*{Linear Programming Model}

\textbf{Variables:}
\begin{itemize}
    \item Let \( x_k \) be the quantity of food \( k \) to purchase, where \( k = 1, 2, \ldots, K \).
\end{itemize}

\textbf{Parameters:}
\begin{itemize}
    \item \( \text{price}_k \): the price per unit of food \( k \).
    \item \( \text{demand}_m \): the minimum required units of nutrient \( m \) per day.
    \item \( \text{nutri}_{k,m} \): the units of nutrient \( m \) contained in one unit of food \( k \).
\end{itemize}

\textbf{Objective Function:}
\[
\text{Minimize} \quad Z = \sum_{k=1}^{K} \text{price}_k \cdot x_k
\]

\textbf{Constraints:}
\[
\sum_{k=1}^{K} \text{nutri}_{k,m} \cdot x_k \geq \text{demand}_m, \quad \forall m = 1, 2, \ldots, M
\]
\[
x_k \geq 0, \quad \forall k = 1, 2, \ldots, K
\]

\textbf{Output:}
\begin{itemize}
    \item The quantity of each food to purchase is given by \( x_k \).
\end{itemize}

\end{document}