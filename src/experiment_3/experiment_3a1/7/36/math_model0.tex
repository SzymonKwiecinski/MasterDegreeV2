\documentclass{article}
\usepackage{amsmath}
\begin{document}

\section*{Linear Programming Model}

\textbf{Decision Variables:}
\begin{itemize}
    \item Let \( x_k \) be the amount of alloy \( k \) to purchase, for \( k = 1, 2, \ldots, K \).
\end{itemize}

\textbf{Parameters:}
\begin{itemize}
    \item \( \text{alloy\_quant} \): Total pounds of alloy to produce.
    \item \( \text{target}_m \): Target quantity of metal \( m \).
    \item \( \text{ratio}_{k, m} \): Ratio of metal \( m \) in alloy \( k \).
    \item \( \text{price}_k \): Price of alloy \( k \).
\end{itemize}

\textbf{Objective Function:}
\[
\text{Minimize } Z = \sum_{k=1}^{K} \text{price}_k \cdot x_k
\]

\textbf{Constraints:}
\begin{itemize}
    \item Total weight of the alloys must equal the desired alloy quantity:
    \[
    \sum_{k=1}^{K} x_k = \text{alloy\_quant}
    \]
    \item The amount of each metal \( m \) in the alloy must meet the target:
    \[
    \sum_{k=1}^{K} \text{ratio}_{k, m} \cdot x_k = \text{target}_m \quad \text{for } m = 1, \ldots, M
    \]
    \item Non-negativity constraints:
    \[
    x_k \geq 0 \quad \text{for } k = 1, \ldots, K
    \]
\end{itemize}

\textbf{Output:}
\begin{itemize}
    \item Amount of each alloy purchased:
    \[
    \text{amount} = [x_1, x_2, \ldots, x_K]
    \]
\end{itemize}

\end{document}