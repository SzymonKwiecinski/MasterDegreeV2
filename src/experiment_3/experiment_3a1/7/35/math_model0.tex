\documentclass{article}
\usepackage{amsmath}
\begin{document}

\section*{Mathematical Model for Warehouse Operation Problem}

\subsection*{Parameters}
\begin{itemize}
    \item Let \( C \) be the capacity of the warehouse.
    \item Let \( h \) be the holding cost per unit for one period.
    \item Let \( p_n \) be the selling price of the commodity in period \( n \) for \( n = 1, \ldots, N \).
    \item Let \( c_n \) be the cost of purchasing the commodity in period \( n \) for \( n = 1, \ldots, N \).
\end{itemize}

\subsection*{Variables}
\begin{itemize}
    \item Let \( b_n \) be the buy quantity in period \( n \).
    \item Let \( s_n \) be the sell quantity in period \( n \).
    \item Let \( x_n \) be the stock in period \( n \) (i.e., the amount of commodity held in the warehouse).
\end{itemize}

\subsection*{Objective Function}
The objective is to maximize the profit over \( N \) periods, which can be formulated as:

\[
\text{Maximize} \quad Z = \sum_{n=1}^{N} \left( p_n s_n - c_n b_n - h x_n \right)
\]

\subsection*{Constraints}
\begin{align*}
    1. & \quad x_n = x_{n-1} + b_n - s_n, \quad \text{for } n = 1, \ldots, N \\
    2. & \quad x_n \leq C, \quad \text{for } n = 1, \ldots, N \\
    3. & \quad x_n \geq 0, \quad \text{for } n = 1, \ldots, N \\
    4. & \quad x_N = 0  \\
    5. & \quad b_n \geq 0, \quad \text{for } n = 1, \ldots, N \\
    6. & \quad s_n \geq 0, \quad \text{for } n = 1, \ldots, N
\end{align*}

\subsection*{Final Output Format}
\begin{itemize}
    \item The output should provide:
    \begin{itemize}
        \item \( \text{"buyquantity": } [b_n \text{ for } n = 1, \ldots, N] \)
        \item \( \text{"sellquantity": } [s_n \text{ for } n = 1, \ldots, N] \)
        \item \( \text{"stock": } [x_n \text{ for } n = 1, \ldots, N] \)
    \end{itemize}
\end{itemize}

\end{document}