\documentclass{article}
\usepackage{amsmath}
\begin{document}

\section*{Linear Programming Model for Spare Automobile Parts Production}

\subsection*{Sets}
\begin{itemize}
    \item Let \( K \) be the total number of types of spare automobile parts.
    \item Let \( S \) be the total number of shops.
\end{itemize}

\subsection*{Parameters}
\begin{itemize}
    \item \( time_{k,s} \): Required worker-hours for part \( k \) in shop \( s \).
    \item \( profit_{k} \): Profit earned from producing part \( k \).
    \item \( capacity_{s} \): Capacity of shop \( s \) in working hours.
\end{itemize}

\subsection*{Decision Variables}
\begin{itemize}
    \item \( quantity_{k} \): The quantity of spare part \( k \) to be produced.
\end{itemize}

\subsection*{Objective Function}
Maximize the total profit:
\[
\text{Maximize} \quad Z = \sum_{k=1}^{K} profit_{k} \cdot quantity_{k}
\]

\subsection*{Constraints}
\begin{itemize}
    \item For each shop \( s \):
    \[
    \sum_{k=1}^{K} time_{k,s} \cdot quantity_{k} \leq capacity_{s} \quad \forall s \in \{1, \ldots, S\}
    \]
    \item Non-negativity constraints:
    \[
    quantity_{k} \geq 0 \quad \forall k \in \{1, \ldots, K\}
    \end{itemize}
\]

\subsection*{Output}
The final output will be:
\[
\{ "quantity": [quantity_{k} \text{ for } k = 1, \ldots, K] \}
\]

\end{document}