\documentclass{article}
\usepackage{amsmath}
\begin{document}

\title{MILP Model for Night Shift Nurse Scheduling}
\author{}
\date{}
\maketitle

\section*{Problem Definition}

Let:
\begin{itemize}
    \item \( \text{period} \): The number of consecutive days a nurse works on the night shift.
    \item \( d_j \): The demand for nurses on day \( j \) where \( j = 1, \ldots, 7 \).
    \item \( n \): Total number of nurses hired.
    \item \( x_j \): Number of nurses starting their shift on day \( j \), where \( j = 1, \ldots, 7 \).
\end{itemize}

\section*{Objective Function}

The objective is to minimize the total number of nurses hired:
\[
\text{Minimize } n = \sum_{j=1}^7 x_j
\]

\section*{Constraints}

For each day \( j \) (where \( j = 1, \ldots, 7 \)), the demand must be satisfied:
\[
\sum_{i=0}^{\text{period}-1} x_{(j-i) \mod 7} \geq d_j \quad \forall j = 1, \ldots, 7
\]

Where \( x_{(j-i) \mod 7} \) represents the number of nurses starting their shift on day \( (j-i) \) modulo 7, ensuring that the schedule wraps around the week.

\section*{Variable Definition}

The variables \( x_j \) are non-negative integers:
\[
x_j \geq 0 \quad \forall j = 1, \ldots, 7
\]

\section*{Output Format}

The output will consist of:
\begin{itemize}
    \item A list \( \text{start} = [x_1, x_2, x_3, x_4, x_5, x_6, x_7] \) indicating how many nurses start their shifts on each day.
    \item The total number of nurses hired, \( n \).
\end{itemize}

\end{document}