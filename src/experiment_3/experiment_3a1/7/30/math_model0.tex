\documentclass{article}
\usepackage{amsmath}
\begin{document}

\title{Quadratic Curve Fitting using Linear Programming}
\author{}
\date{}
\maketitle

\section*{Problem Definition}

Given a set of observed values for \( y \) and corresponding values for \( x \):

\[
y = [y_{k} \text{ for } k = 1,\ldots,K], \quad x = [x_{k} \text{ for } k = 1,\ldots,K]
\]

We want to fit the best quadratic curve of the form:

\[
y = c \cdot x^2 + b \cdot x + a
\]

The objective is to minimize the sum of absolute deviations of each observed value of \( y \) from the value predicted by the quadratic relationship. This can be mathematically formulated as:

\[
\text{Minimize} \quad Z = \sum_{k=1}^{K} |y_{k} - (c \cdot x_{k}^2 + b \cdot x_{k} + a)|
\]

\section*{Mathematical Model}

To handle the absolute value, we introduce auxiliary variables \( d_{k} \) for each \( k \) such that:

\[
d_{k} \geq y_{k} - (c \cdot x_{k}^2 + b \cdot x_{k} + a)
\]
\[
d_{k} \geq -(y_{k} - (c \cdot x_{k}^2 + b \cdot x_{k} + a))
\]

Thus, the linear programming formulation becomes:

\[
\text{Minimize} \quad Z = \sum_{k=1}^{K} d_{k}
\]

Subject to the constraints:

\[
d_{k} \geq y_{k} - (c \cdot x_{k}^2 + b \cdot x_{k} + a), \quad \forall k
\]
\[
d_{k} \geq -(y_{k} - (c \cdot x_{k}^2 + b \cdot x_{k} + a)), \quad \forall k
\]
\[
d_{k} \geq 0, \quad \forall k
\]

\section*{Output}

The output will consist of the coefficients obtained from the optimization:

\[
\text{Output: } \{ 
\text{"quadratic"} : c, 
\text{"linear"} : b, 
\text{"constant"} : a 
\}
\]

\end{document}