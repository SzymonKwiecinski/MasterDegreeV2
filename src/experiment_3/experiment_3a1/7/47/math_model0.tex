\documentclass{article}
\usepackage{amsmath}
\begin{document}

\section*{Linear Programming Model for Police Shift Assignment}

\subsection*{Parameters}
\begin{itemize}
    \item Let \( S \) be the number of different shifts.
    \item Let \( officers\_needed_s \) be the number of police officers needed for shift \( s \) for \( s = 1, \ldots, S \).
    \item Let \( shift\_cost_s \) be the cost incurred for starting shift \( s \).
\end{itemize}

\subsection*{Decision Variables}
\begin{itemize}
    \item Let \( officers\_assigned_s \) be the number of police officers assigned to shift \( s \).
\end{itemize}

\subsection*{Objective Function}
Minimize the total cost:
\[
\text{Total Cost} = \sum_{s=1}^{S} shift\_cost_s \cdot x_s
\]
where \( x_s \) is a binary variable indicating if shift \( s \) is started (1) or not (0).

\subsection*{Constraints}
\begin{align}
    & officers\_assigned_s \geq officers\_needed_s \quad \forall s \in \{1, \ldots, S\} \\
    & officers\_assigned_s = officers\_assigned_{s-1} + officers\_assigned_s \quad \forall s \in \{2, \ldots, S\} \\
    & officers\_assigned_s \geq 0 \quad \forall s \in \{1, \ldots, S\}
\end{align}

\subsection*{Output}
The solution will return:
\begin{itemize}
    \item \( officers\_assigned \): A list of the number of officers assigned to each shift.
    \item \( total\_cost \): The total cost incurred by the town.
\end{itemize}

\end{document}