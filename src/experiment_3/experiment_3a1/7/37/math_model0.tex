\documentclass{article}
\usepackage{amsmath}
\begin{document}

\section*{Linear Programming Model for Spare Automobile Parts Production}

\subsection*{Decision Variables}
Let \( x_k \) be the quantity of spare part \( k \) produced, where \( k = 1, 2, \ldots, K \).

\subsection*{Parameters}
\begin{itemize}
    \item \( time_{k,s} \): the required worker-hours for part \( k \) in shop \( s \)
    \item \( profit_k \): the profit of part \( k \)
    \item \( capacity_s \): the capacity of shop \( s \) in working hours
\end{itemize}

\subsection*{Objective Function}
The objective is to maximize the total profit, which can be expressed as:
\[
\text{Maximize } Z = \sum_{k=1}^{K} profit_k \cdot x_k
\]

\subsection*{Constraints}
For each shop \( s \):
\[
\sum_{k=1}^{K} time_{k,s} \cdot x_k \leq capacity_s \quad \text{for } s = 1, 2, \ldots, S
\]

\subsection*{Non-negativity Constraints}
\[
x_k \geq 0 \quad \text{for } k = 1, 2, \ldots, K
\]

\subsection*{Summary}
The complete linear programming formulation can be summarized as follows:

\begin{align*}
\text{Maximize } & Z = \sum_{k=1}^{K} profit_k \cdot x_k \\
\text{Subject to } & \sum_{k=1}^{K} time_{k,s} \cdot x_k \leq capacity_s, \quad s = 1, 2, \ldots, S \\
                   & x_k \geq 0, \quad k = 1, 2, \ldots, K
\end{align*}

\end{document}