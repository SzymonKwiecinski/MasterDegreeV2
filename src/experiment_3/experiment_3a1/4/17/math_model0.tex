\documentclass{article}
\usepackage{amsmath}
\begin{document}

\section*{Linear Programming Model for Stock Selling}

\subsection*{Definitions}

Let:
\begin{itemize}
    \item \( N \) : Number of different stocks
    \item \( bought_i \) : Number of shares bought of stock \( i \)
    \item \( buyPrice_i \) : Price at which each share of stock \( i \) was bought
    \item \( currentPrice_i \) : Current price of one share of stock \( i \)
    \item \( futurePrice_i \) : Expected price of one share of stock \( i \) in one year
    \item \( transactionRate \) : Transaction cost rate (in percent)
    \item \( taxRate \) : Tax rate on capital gains (in percent)
    \item \( K \) : Amount of money the investor wants to raise net of capital gains and transaction costs
    \item \( sell_i \) : Number of shares of stock \( i \) that the investor should sell
\end{itemize}

\subsection*{Objective}

Maximize the expected value of the portfolio next year, subject to the constraints of raising the desired amount of money.

\subsection*{Mathematical Model}

The objective function can be written as:

\[
\text{Maximize } Z = \sum_{i=1}^{N} futurePrice_i (bought_i - sell_i)
\]

Subject to:

\[
\sum_{i=1}^{N} \left( currentPrice_i \cdot sell_i \cdot \left(1 - \frac{transactionRate}{100}\right) - (sell_i \cdot bought_i \cdot (currentPrice_i - buyPrice_i) \cdot \frac{taxRate}{100}) \right) \geq K
\]

\[
0 \leq sell_i \leq bought_i \quad \text{for } i = 1, \ldots, N
\]

\subsection*{Output}

The output should be:

\[
\text{sell} = [sell_1, sell_2, \ldots, sell_N]
\]

Where \( sell_i \) is the number of shares of stock \( i \) that the investor should sell.

\end{document}