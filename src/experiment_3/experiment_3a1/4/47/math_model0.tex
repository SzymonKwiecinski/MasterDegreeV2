\documentclass{article}
\usepackage{amsmath}
\begin{document}

\section*{Mathematical Model for Police Shift Assignment}

Given the problem description, we define the following variables and parameters:

\subsection*{Parameters}
\begin{itemize}
    \item Let \( S \) be the number of different shifts.
    \item Let \( officers\_needed_s \) denote the number of police officers required during shift \( s \) for \( s = 1, 2, \ldots, S \).
    \item Let \( shift\_cost_s \) denote the cost incurred when starting shift \( s \).
    \item Let \( x_s \) be the number of officers assigned to shift \( s \).
\end{itemize}

\subsection*{Objective Function}
The objective is to minimize the total cost incurred by assigning officers to shifts:

\[
\text{Minimize } Z = \sum_{s=1}^{S} shift\_cost_s \cdot x_s
\]

\subsection*{Constraints}
Each officer works for two consecutive shifts, so the assignment must satisfy the following constraints for each shift:

1. For each shift \( s \):
   \[
   x_s \geq officers\_needed_s \quad \text{for } s = 1, 2, \ldots, S
   \]

2. Each officer assigned to shift \( s \) also covers shift \( s+1 \) (for \( s < S \)). Therefore, the number of officers assigned to shift \( s \) must ensure that the requirements for the next shift are met:
   \[
   x_s + x_{s-1} \geq officers\_needed_s \quad \text{for } s = 2, 3, \ldots, S
   \]

3. Non-negativity constraint:
   \[
   x_s \geq 0 \quad \text{for } s = 1, 2, \ldots, S
   \]

\subsection*{Output}
The output should provide:
\begin{itemize}
    \item \( officers\_assigned = [x_1, x_2, \ldots, x_S] \)
    \item The total cost \( total\_cost = Z \).
\end{itemize}

\end{document}