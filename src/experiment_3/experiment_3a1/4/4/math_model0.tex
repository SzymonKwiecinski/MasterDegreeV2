\documentclass{article}
\usepackage{amsmath}
\begin{document}

\section*{Mathematical Model for Nurse Scheduling}

\subsection*{Parameters}
\begin{itemize}
    \item Let \( P \) be the period of consecutive working days of a nurse.
    \item Let \( d_j \) be the demand for nurses on day \( j \), where \( j = 1, 2, \ldots, 7 \).
\end{itemize}

\subsection*{Decision Variables}
\begin{itemize}
    \item Let \( start_j \) be the number of nurses who start their working period on day \( j \), where \( j = 1, 2, \ldots, 7 \).
    \item Let \( N \) be the total number of nurses hired.
\end{itemize}

\subsection*{Objective Function}
The objective is to minimize the total number of nurses hired:
\[
\text{Minimize } N = \sum_{j=1}^{7} start_j
\]

\subsection*{Constraints}
For each day \( j \), the total number of nurses available must meet the demand:
\[
\sum_{k=0}^{P-1} start_{(j-k \mod 7)} \geq d_j \quad \forall j = 1, 2, \ldots, 7
\]
This ensures that on each day \( j \), the total number of nurses working meets the required demand \( d_j \).

\subsection*{Non-negativity Constraints}
\[
start_j \geq 0 \quad \forall j = 1, 2, \ldots, 7
\]

\end{document}