\documentclass{article}
\usepackage{amsmath}
\begin{document}

\section*{Rocket Trajectory Optimization Problem}

Let \( x_t \), \( v_t \), and \( a_t \) be the position, velocity, and acceleration of the rocket at time \( t \), respectively. The discrete-time model can be formulated as follows:

\begin{align}
    x_{t+1} &= x_t + v_t \\
    v_{t+1} &= v_t + a_t
\end{align}

We are given the initial conditions:

\begin{align*}
    x_0 & = \text{initial position} \\
    v_0 & = \text{initial velocity} \\
\end{align*}

The objective is to reach a target position \( x_T \) with a target velocity \( v_T \) at time \( T \). The goal is to minimize the maximum thrust required, which can be expressed as:

\[
\text{Minimize} \quad \max_{t=0}^{T} |a_t|
\]

\text{Subject to:}

\begin{align}
    x_T &= x_0 + \sum_{t=0}^{T-1} v_t \\
    v_T &= v_0 + \sum_{t=0}^{T-1} a_t
\end{align}

The variables are defined as follows:

- \( a_t \) is controlled acceleration, reflecting fuel consumption.
- We aim to minimize the fuel spent given by the maximum of the absolute values of acceleration over time.

The output of the model will be summarized as follows:

\begin{align*}
    \text{Output:} \\
    & \text{"x"}: [x_i \text{ for } i = 0, \ldots, T] \\
    & \text{"v"}: [v_i \text{ for } i = 0, \ldots, T] \\
    & \text{"a"}: [a_i \text{ for } i = 0, \ldots, T] \\
    & \text{"fuel\_spend"}: \text{total fuel spent by the rocket}
\end{align*}

\end{document}