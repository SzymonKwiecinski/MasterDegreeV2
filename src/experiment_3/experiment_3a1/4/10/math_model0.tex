\documentclass{article}
\usepackage{amsmath}
\begin{document}

\section*{Linear Programming Model for School Assignment Problem}

\subsection*{Parameters}
\begin{itemize}
    \item Let \( N \) be the number of neighborhoods.
    \item Let \( S \) be the number of schools.
    \item Let \( G \) be the number of grades at each school.
    \item \( capacity_{s,g} \): Capacity of school \( s \) for grade \( g \).
    \item \( population_{n,g} \): Student population in neighborhood \( n \) for grade \( g \).
    \item \( d_{n,s} \): Distance from neighborhood \( n \) to school \( s \).
\end{itemize}

\subsection*{Decision Variables}
Define \( x_{n,s,g} \) as the number of students of grade \( g \) in neighborhood \( n \) assigned to school \( s \).

\subsection*{Objective Function}
We want to minimize the total distance traveled by all students:
\[
\text{Minimize} \quad Z = \sum_{n=1}^{N} \sum_{s=1}^{S} \sum_{g=1}^{G} d_{n,s} \cdot x_{n,s,g}
\]

\subsection*{Constraints}

1. **Capacity Constraints**: The total number of students assigned to a school for each grade cannot exceed its capacity:
\[
\sum_{n=1}^{N} x_{n,s,g} \leq capacity_{s,g} \quad \forall s \in \{1, \ldots, S\}, \forall g \in \{1, \ldots, G\}
\]

2. **Demand Constraints**: All students in each neighborhood for each grade must be assigned to schools:
\[
\sum_{s=1}^{S} x_{n,s,g} = population_{n,g} \quad \forall n \in \{1, \ldots, N\}, \forall g \in \{1, \ldots, G\}
\]

3. **Non-negativity Constraint**: The number of students assigned must be non-negative:
\[
x_{n,s,g} \geq 0 \quad \forall n \in \{1, \ldots, N\}, \forall s \in \{1, \ldots, S\}, \forall g \in \{1, \ldots, G\}
\]

\subsection*{Output}
The output should return:
\begin{itemize}
    \item An assignment structure \( x_{n,s,g} \) that represents the number of students of grade \( g \) from neighborhood \( n \) assigned to school \( s \).
    \item The total distance traveled \( total\_distance \).
\end{itemize}

\end{document}