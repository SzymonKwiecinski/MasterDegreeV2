\documentclass{article}
\usepackage{amsmath}
\begin{document}

\section*{Linear Programming Model for Soybean Transportation}

\subsection*{Sets}
\begin{itemize}
    \item Let \( A \) be the set of all routes where each route is defined as \((i, j)\).
    \item Let \( K \) be the set of source terminal cities.
    \item Let \( P \) be the set of port cities.
    \item Let \( L \) be the set of destination cities.
\end{itemize}

\subsection*{Parameters}
\begin{itemize}
    \item \( C_{i,j} \): Transportation cost per metric ton along route from city \( i \) to city \( j \).
    \item \( supply_k \): Soybean supply at terminal city \( k \).
    \item \( demand_l \): Soybean demand at destination city \( l \).
\end{itemize}

\subsection*{Decision Variables}
\begin{itemize}
    \item \( amount_{i,j} \): Amount of soybeans shipped from city \( i \) to city \( j \).
\end{itemize}

\subsection*{Objective Function}
Minimize total transportation cost:
\[
\text{Minimize } Z = \sum_{(i,j) \in A} C_{i,j} \cdot amount_{i,j}
\]

\subsection*{Subject to Constraints}
\begin{itemize}
    \item Supply constraints for each terminal city \( k \):
    \[
    \sum_{j \in P} amount_{k,j} \leq supply_k \quad \forall k \in K
    \]

    \item Demand constraints for each destination city \( l \):
    \[
    \sum_{i \in P} amount_{i,l} = demand_l \quad \forall l \in L
    \]

    \item Non-negativity constraints:
    \[
    amount_{i,j} \geq 0 \quad \forall (i,j) \in A
    \]
\end{itemize}

\subsection*{Output Format}
The output will be in the following format:
\begin{verbatim}
{
    "distribution": [
        {"from": i,
         "to": j,
         "amount": amount_{i,j}
        }
        for id in 1, ..., m
    ],
    "total_cost": total_cost
}
\end{verbatim}

\end{document}