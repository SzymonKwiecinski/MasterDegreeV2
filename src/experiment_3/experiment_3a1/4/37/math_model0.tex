\documentclass{article}
\usepackage{amsmath}
\begin{document}

\title{Linear Programming Model for Maximizing Profit in Spare Automobile Parts Production}
\author{}
\date{}
\maketitle

\section*{Problem Formulation}

Let:
\begin{itemize}
    \item \( K \): Number of spare parts
    \item \( S \): Number of shops
    \item \( time_{k,s} \): Worker-hours required for part \( k \) in shop \( s \)
    \item \( profit_{k} \): Profit from part \( k \)
    \item \( capacity_{s} \): Capacity (working hours) of shop \( s \)
    \item \( quantity_{k} \): Quantity of part \( k \) to be produced
\end{itemize}

\subsection*{Objective Function}

The objective is to maximize the total profit from producing the parts, given by:

\[
\text{Maximize } Z = \sum_{k=1}^{K} profit_{k} \cdot quantity_{k}
\]

\subsection*{Constraints}

The constraints are based on the shop capacities. The total worker-hours used in each shop must not exceed its capacity:

\[
\sum_{k=1}^{K} time_{k,s} \cdot quantity_{k} \leq capacity_{s}, \quad \forall s = 1, 2, \ldots, S
\]

Additionally, we have non-negativity constraints:

\[
quantity_{k} \geq 0, \quad \forall k = 1, 2, \ldots, K
\]

\subsection*{Summary of the Model}

The complete linear programming model can be summarized as follows:

\[
\begin{align*}
\text{Maximize } & Z = \sum_{k=1}^{K} profit_{k} \cdot quantity_{k} \\
\text{subject to } & \sum_{k=1}^{K} time_{k,s} \cdot quantity_{k} \leq capacity_{s}, \quad \forall s \\
& quantity_{k} \geq 0, \quad \forall k
\end{align*}
\]

\end{document}