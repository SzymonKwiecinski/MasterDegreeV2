\documentclass{article}
\usepackage{amsmath}
\begin{document}

\title{Mathematical Model for Roll Cutting Problem}
\author{}
\date{}
\maketitle

\section*{Problem Definition}
Let:
\begin{itemize}
    \item \( W \) be the width of the large roll, where \( W = \text{large\_roll\_width} \).
    \item \( M \) be the number of smaller roll types.
    \item \( d_j \) be the demand for rolls of width \( j \) where \( j \in \{1, 2, \ldots, M\} \).
    \item \( N \) be the number of cutting patterns available.
    \item \( p_{i,j} \) be the number of rolls produced of width \( j \) using cutting pattern \( i \).
    \item \( x_i \) be the number of times cutting pattern \( i \) is used.
\end{itemize}

\section*{Objective Function}
The objective is to minimize the total number of large rolls used:
\[
\text{Minimize} \quad Z = \sum_{i=1}^{N} x_i
\]

\section*{Constraints}
The constraints ensuring that the demand for each roll width is fulfilled can be expressed as:
\[
\sum_{i=1}^{N} p_{i,j} x_i \geq d_j \quad \forall j \in \{1, 2, \ldots, M\}
\]

Additionally, each pattern must respect the width of the large roll:
\[
\sum_{j=1}^{M} p_{i,j} \leq W \quad \forall i \in \{1, 2, \ldots, N\}
\]

\section*{Variable Definitions}
\begin{itemize}
    \item \( x_i \geq 0 \) and integer for all \( i \in \{1, 2, \ldots, N\} \).
\end{itemize}

\section*{Output Format}
The output will include the details of the cutting patterns applied and the total number of large rolls used:
\begin{verbatim}
{
    "patterns": [
        {
         "pattern": [pattern_j for j in 1, ..., M],
         "amount": amount_i
        }
        for i in 1, ..., N
    ],
    "total_large_rolls_used": total_large_rolls
}
\end{verbatim}

\end{document}