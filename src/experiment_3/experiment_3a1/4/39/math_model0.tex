\documentclass{article}
\usepackage{amsmath}
\begin{document}

\section*{Mathematical Model for Cafeteria Staff Scheduling}

\subsection*{Indices}
\begin{itemize}
    \item \( n \): Day index, where \( n = 1, 2, \ldots, N \)
    \item \( i \): Employee index, where \( i = 1, 2, \ldots, \text{total\_number} \)
\end{itemize}

\subsection*{Parameters}
\begin{itemize}
    \item \( \text{num}_n \): Number of desired employees on day \( n \)
    \item \( n_{\text{working}} \): Number of consecutive working days for each employee
    \item \( n_{\text{resting}} \): Number of consecutive resting days for each employee
    \item \( T = n_{\text{working}} + n_{\text{resting}} \): Total cycle length of work and rest
\end{itemize}

\subsection*{Decision Variables}
\begin{itemize}
    \item \( x_i \in \{0, 1\} \): 1 if employee \( i \) is working on day \( n \), 0 otherwise
\end{itemize}

\subsection*{Objective Function}
Minimize the total number of employees:
\[
\text{Minimize } \text{total\_number} = \sum_{i=1}^{\text{total\_number}} 1
\]

\subsection*{Constraints}
1. Each day's required staff must be met:
\[
\sum_{i=1}^{\text{total\_number}} x_i \geq \text{num}_n \quad \forall n = 1, \ldots, N
\]

2. Employee work/rest cycles:
\[
\text{For each employee } i, \text{ define the working days according to the cycle:}
\]
\[
x_{i, n} = 
\begin{cases} 
1 & \text{if employee } i \text{ works on day } n \\
0 & \text{if employee } i \text{ is resting on day } n 
\end{cases}
\]
Where:
\[
x_{i, n} = x_{i, n-1} \text{ for } n \in [k \cdot T + 1, k \cdot T + n_{\text{working}}] \text{ (for all } k \text{ being integers)}
\]
\[
x_{i, n} = 0 \text{ for } n \in [k \cdot T + n_{\text{working}} + 1, (k+1) \cdot T] 
\]

\subsection*{Output Variables}
- \( \text{total\_number} \): Total number of employees to employ
- \( \text{is\_work}_{n, i} \): A binary matrix indicating whether employee \( i \) is working on day \( n \)

\end{document}