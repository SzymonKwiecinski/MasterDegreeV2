\documentclass{article}
\usepackage{amsmath}
\begin{document}

\section*{Linear Programming Model for Best Fit Line}

Given a set of observations with corresponding values \( y_k \) and \( x_k \) for \( k = 1, \ldots, K \), we aim to fit a linear model of the form:

\[
y = bx + a
\]

where \( a \) is the intercept and \( b \) is the slope of the fitted line. 

The objective is to minimize the sum of the absolute deviations of the observed values of \( y \) from the values predicted by this linear relationship. This can be expressed mathematically as:

\[
\text{Minimize } Z = \sum_{k=1}^{K} |y_k - (bx_k + a)|
\]

To facilitate the optimization process, we introduce auxiliary variables \( d_k \) to represent the absolute deviations:

\[
d_k \geq y_k - (bx_k + a) \quad \forall k
\]
\[
d_k \geq -(y_k - (bx_k + a)) \quad \forall k
\]

This leads to the following linear programming formulation:

\[
\text{Minimize } Z = \sum_{k=1}^{K} d_k
\]

subject to:

\[
d_k \geq y_k - (bx_k + a) \quad \forall k
\]
\[
d_k \geq -(y_k - (bx_k + a)) \quad \forall k
\]

The output will be:

\[
\{ \text{intercept} = a, \text{slope} = b \}
\]

\end{document}