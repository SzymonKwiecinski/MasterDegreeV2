\documentclass{article}
\usepackage{amsmath}
\begin{document}

\title{MILP Model for Floppy Disk Backup}
\author{}
\date{}
\maketitle

\section*{Problem Definition}
Given a set of files with corresponding sizes and a set of floppy disks with a specified capacity, our objective is to minimize the number of disks used while ensuring that no disk exceeds its capacity.

\section*{Variables}
\begin{itemize}
    \item Let \( N \) be the total number of files.
    \item Let \( C \) be the capacity of each floppy disk.
    \item Let \( \text{size}_j \) represent the size of file \( j \).
    \item Let \( n_{\text{disks}} \) be the total number of floppy disks used.
    \item Let \( x_{ij} \) be a binary decision variable, where:
    \[
    x_{ij} = 
    \begin{cases} 
    1 & \text{if file } j \text{ is placed on disk } i \\
    0 & \text{otherwise}
    \end{cases}
    \]
\end{itemize}

\section*{Objective Function}
Minimize the number of disks used:
\[
\text{Minimize } n_{\text{disks}} = \sum_{i=1}^{M} y_i
\]
where \( y_i \) is a binary variable indicating whether disk \( i \) is used (1) or not (0).

\section*{Constraints}
\begin{itemize}
    \item Each file must be assigned to exactly one disk:
    \[
    \sum_{i=1}^{M} x_{ij} = 1, \quad \forall j = 1, \ldots, N
    \]
    
    \item The total size of files assigned to each disk cannot exceed the capacity:
    \[
    \sum_{j=1}^{N} \text{size}_j \cdot x_{ij} \leq C \cdot y_i, \quad \forall i = 1, \ldots, M
    \]
    
    \item The number of disks used should be at least as many as the number of files:
    \[
    n_{\text{disks}} \geq \max_{j} (x_{ij}), \quad \forall i = 1, \ldots, M
    \]
\end{itemize}

\section*{Output}
The output should include:
\begin{itemize}
    \item \( n_{\text{disks}} \): the total number of floppy disks used.
    \item \( \text{whichdisk} \): an array indicating on which disk each file is placed, where \( \text{whichdisk}_j \) indicates the disk assigned to file \( j \).
\end{itemize}

\end{document}