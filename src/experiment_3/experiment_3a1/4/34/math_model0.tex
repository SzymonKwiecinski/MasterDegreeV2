\documentclass{article}
\usepackage{amsmath}
\begin{document}

\section*{Linear Programming Model}

\subsection*{Definitions}

Let:
\begin{itemize}
    \item \( K \) be the number of different foods,
    \item \( M \) be the number of nutritional ingredients,
    \item \( \text{price}_k \) be the price per unit of food \( k \),
    \item \( \text{demand}_m \) be the minimum daily requirement of nutrient \( m \),
    \item \( \text{nutri}_{k, m} \) be the amount of nutrient \( m \) in one unit of food \( k \),
    \item \( \text{quantity}_k \) be the quantity of food \( k \) to purchase.
\end{itemize}

\subsection*{Objective Function}

The objective is to minimize the total cost of the foods purchased:

\[
\text{Minimize } Z = \sum_{k=1}^{K} \text{price}_k \cdot \text{quantity}_k
\]

\subsection*{Constraints}

To ensure that the dietary requirements are met for each nutrient \( m \), we have the following constraints:

\[
\sum_{k=1}^{K} \text{nutri}_{k, m} \cdot \text{quantity}_k \geq \text{demand}_m \quad \forall m \in \{1, 2, \ldots, M\}
\]

Additionally, we have non-negativity constraints:

\[
\text{quantity}_k \geq 0 \quad \forall k \in \{1, 2, \ldots, K\}
\]

\subsection*{Summary}

The linear programming model can be summarized as follows:

\begin{align*}
\text{Minimize } & Z = \sum_{k=1}^{K} \text{price}_k \cdot \text{quantity}_k \\
\text{subject to } & \sum_{k=1}^{K} \text{nutri}_{k, m} \cdot \text{quantity}_k \geq \text{demand}_m \quad \forall m \\
                   & \text{quantity}_k \geq 0 \quad \forall k
\end{align*}

\end{document}