\documentclass{article}
\usepackage{amsmath}
\begin{document}

\section*{Quadratic Curve Fitting Using Linear Programming}

Let us define the problem mathematically. We are given a set of values for \( y \) and \( x \) such that:

\[
y = [y_1, y_2, \ldots, y_K]
\]
\[
x = [x_1, x_2, \ldots, x_K]
\]

We aim to fit a quadratic equation of the form:

\[
y = c \cdot x^2 + b \cdot x + a
\]

Where \( c \), \( b \), and \( a \) are the coefficients we want to determine.

To achieve the best fit, we will minimize the sum of absolute deviations of observed values of \( y \) from the predicted values given by the quadratic equation. The absolute deviation for each observation can be expressed as:

\[
d_k = |y_k - (c \cdot x_k^2 + b \cdot x_k + a)|
\]

Our objective is to minimize:

\[
\sum_{k=1}^{K} d_k
\]

To convert this into a linear programming format, we introduce auxiliary variables \( z_k \) for each \( k \):

\[
z_k \geq y_k - (c \cdot x_k^2 + b \cdot x_k + a)
\]
\[
z_k \geq -(y_k - (c \cdot x_k^2 + b \cdot x_k + a))
\]

This leads to the following linear programming formulation:

\textbf{Objective Function:}
\[
\text{Minimize} \quad \sum_{k=1}^{K} z_k
\]

\textbf{Subject to:}
\[
z_k \geq y_k - (c \cdot x_k^2 + b \cdot x_k + a), \quad k = 1, 2, \ldots, K
\]
\[
z_k \geq -(y_k - (c \cdot x_k^2 + b \cdot x_k + a)), \quad k = 1, 2, \ldots, K
\]

The solution of this linear programming problem will provide us with the coefficients \( c \), \( b \), and \( a \) which represent:

\[
\text{Output:} \quad
\begin{align*}
\text{quadratic} & = c \\
\text{linear} & = b \\
\text{constant} & = a
\end{align*}
\]

\end{document}