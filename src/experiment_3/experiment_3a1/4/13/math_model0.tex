\documentclass{article}
\usepackage{amsmath}
\begin{document}

\section*{Linear Programming Model for the Chebyshev Center}

Given a set \( P \) defined by the constraints:
\[
P = \{ \mathbf{x} \in \mathbb{R}^N \mid \mathbf{a_i}^T \mathbf{x} \leq b_i, \; i = 1, \ldots, m \}
\]
where \( \mathbf{a_i} \in \mathbb{R}^N \) and \( b_i \in \mathbb{R} \), we aim to find the Chebyshev center of the set \( P \), which is a ball centered at \( \mathbf{y} \in \mathbb{R}^N \) with the largest possible radius \( r \).

The model can be formulated as follows:

\subsection*{Variables}
\begin{itemize}
    \item \( \mathbf{y} \in \mathbb{R}^N \): The center of the ball.
    \item \( r \in \mathbb{R} \): The radius of the ball.
\end{itemize}

\subsection*{Objective}
Maximize the radius \( r \):
\[
\text{Maximize } r
\]

\subsection*{Constraints}
The ball must be contained within the set \( P \). This leads to the following constraints for all \( i = 1, \ldots, m \):
\[
\mathbf{a_i}^T \mathbf{y} + r \cdot \|\mathbf{a_i}\| \leq b_i
\]
\[
\mathbf{a_i}^T \mathbf{y} - r \cdot \|\mathbf{a_i}\| \leq b_i
\]

This can be simplified to:
\[
\mathbf{a_i}^T \mathbf{y} \leq b_i - r \cdot \|\mathbf{a_i}\| \quad \text{and} \quad \mathbf{a_i}^T \mathbf{y} \geq b_i + r \cdot \|\mathbf{a_i}\| 
\]

Hence, we define the full set of constraints as:
\[
b_i - r \cdot \|\mathbf{a_i}\| \geq \mathbf{a_i}^T \mathbf{y} \geq b_i + r \cdot \|\mathbf{a_i}\|
\]

\subsection*{Final Model}
The final linear programming model is:
\[
\begin{align*}
\text{Maximize} & \quad r \\
\text{subject to} & \quad \mathbf{a_i}^T \mathbf{y} + r \cdot \|\mathbf{a_i}\| \leq b_i, \; i = 1, \ldots, m \\
& \quad \mathbf{a_i}^T \mathbf{y} - r \cdot \|\mathbf{a_i}\| \leq b_i, \; i = 1, \ldots, m \\
& \quad r \geq 0
\end{align*}
\]

\subsection*{Output}
After the resolution of this optimization problem, the output shall provide:
\begin{itemize}
    \item \textbf{center} (list): The center of the ball as a list of floats of length \( N \).
    \item \textbf{radius} (float): The radius of the ball as a float.
\end{itemize}

\end{document}