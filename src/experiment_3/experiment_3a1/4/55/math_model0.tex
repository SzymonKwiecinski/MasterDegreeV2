\documentclass{article}
\usepackage{amsmath}
\begin{document}

\section*{Linear Programming Model for Auto Parts Manufacturer}

\subsection*{Variables}
Let \( b_p \) represent the number of batches produced for part \( p \) where \( p = 1, \ldots, P \).

\subsection*{Parameters}
\begin{itemize}
    \item \( time_{m,p} \): Time required on machine \( m \) to produce one batch of part \( p \) (in hours).
    \item \( cost_m \): Cost per hour for using machine \( m \).
    \item \( available_m \): Availability of machine \( m \) (in hours).
    \item \( price_p \): Selling price per batch of part \( p \).
    \item \( min\_batches_p \): Minimum batches of part \( p \) required to fulfill contract.
    \item \( standard\_cost \): Standard labor cost per hour.
    \item \( overtime\_cost \): Overtime labor cost per hour.
    \item \( overtime\_hour \): Hours for which standard cost applies before overtime kicks in.
    \item \( min\_profit \): Minimum desired profit.
\end{itemize}

\subsection*{Objective Function}
Maximize total profit:

\[
\text{Maximize } Z = \sum_{p=1}^{P} (price_p \cdot b_p) - \sum_{m=1}^{M} (cost_m \cdot \sum_{p=1}^{P} (time_{m,p} \cdot b_p))
\]

Where labor costs for machine 1 are given by:

\[
\text{Labor Cost} = 
\begin{cases} 
standard\_cost \cdot \min(\text{Hours Used}, overtime\_hour) + overtime\_cost \cdot \max(0, \text{Hours Used} - overtime\_hour) & \text{if } m = 1 \\
0 & \text{else}
\end{cases}
\]

\subsection*{Constraints}
1. Time availability for each machine:
   \[
   \sum_{p=1}^{P} (time_{m,p} \cdot b_p) \leq available_m, \quad \forall m = 1, \ldots, M
   \]

2. Minimum batch requirements:
   \[
   b_p \geq min\_batches_p, \quad \forall p = 1, \ldots, P
   \]

3. Profit constraint:
   \[
   Z \geq min\_profit
   \]

4. Non-negativity:
   \[
   b_p \geq 0, \quad \forall p = 1, \ldots, P
   \]

\end{document}