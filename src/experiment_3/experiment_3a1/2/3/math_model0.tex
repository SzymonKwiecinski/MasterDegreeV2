\documentclass{article}
\usepackage{amsmath}
\begin{document}

\section*{Linear Programming Model for Electricity Capacity Planning}

\subsection*{Parameters}
\begin{itemize}
    \item $T$: Number of years
    \item $demand_t$: Demand for electricity in year $t$
    \item $oil_t$: Existing oil capacity in year $t$
    \item $coal\_cost$: Capital cost per megawatt of coal-fired capacity
    \item $nuke\_cost$: Capital cost per megawatt of nuclear power capacity
    \item $max\_nuke$: Maximum percentage of total capacity that can be nuclear
    \item $coal\_life$: Lifespan of coal plants
    \item $nuke\_life$: Lifespan of nuclear plants
\end{itemize}

\subsection*{Variables}
Let:
\begin{itemize}
    \item $coal_t$: Capacity added for coal in year $t$
    \item $nuke_t$: Capacity added for nuclear in year $t$
    \item $total\_cost$: Total cost of the capacity expansion
\end{itemize}

\subsection*{Objective Function}
Minimize the total cost:
\[
\text{Minimize } Z = \sum_{t=1}^{T} (coal\_cost \cdot coal_t + nuke\_cost \cdot nuke_t)
\]

\subsection*{Constraints}
1. Capacity must meet demand:
\[
oil_t + \sum_{s=1}^{t} coal_s \cdot I_{s,t} + \sum_{s=1}^{t} nuke_s \cdot I_{s,t} \geq demand_t \quad \forall t = 1, \ldots, T
\]
where \(I_{s,t}\) is an indicator function that is 1 if the capacity added in year \(s\) is still operational in year \(t\).

2. Nuclear capacity constraint:
\[
\sum_{t=1}^{T} nuke_t \leq \frac{max\_nuke}{100} \cdot \left( oil_t + \sum_{s=1}^{T} coal_s + \sum_{s=1}^{T} nuke_s \right) \quad \forall t
\]

3. Capacity lifespan:
\[
coal_t = 0 \quad \text{for } t < coal\_life
\]
\[
nuke_t = 0 \quad \text{for } t < nuke\_life
\]

4. Non-negativity constraints:
\[
coal_t \geq 0, \quad nuke_t \geq 0 \quad \forall t
\]

\end{document}