\documentclass{article}
\usepackage{amsmath}
\begin{document}

\section*{Mathematical Model for Nurse Scheduling Problem}

\subsection*{Parameters}
\begin{itemize}
    \item $d_j$: Demand for nurses on day $j$, where $j = 1, \ldots, 7$.
    \item $period$: Number of consecutive days a nurse works the night shift.
\end{itemize}

\subsection*{Variables}
\begin{itemize}
    \item $start_j$: Number of nurses starting their shifts on day $j$, where $j = 1, \ldots, 7$.
    \item $N$: Total number of nurses hired.
\end{itemize}

\subsection*{Objective}
Minimize the total number of nurses hired:
\[
\text{Minimize } N = \sum_{j=1}^{7} start_j
\]

\subsection*{Constraints}
Ensure that the demand for nurses is met for each day:
\[
\sum_{j=1}^{7} start_{j - k} \geq d_j \quad \forall j = 1, \ldots, 7, \quad \text{where } k = 0, 1, \ldots, period - 1
\]
Where $start_{j - k}$ is understood to be zero when $j - k < 1$ or $j - k > 7$.

\subsection*{Non-negativity Constraints}
\[
start_j \geq 0 \quad \forall j = 1, \ldots, 7
\]
\[
N \geq 0
\]

\subsection*{Output}
The output will consist of:
\begin{itemize}
    \item An array of $start = [start_1, start_2, \ldots, start_7]$.
    \item Total number of nurses hired $N$.
\end{itemize}

\end{document}