\documentclass{article}
\usepackage{amsmath}
\begin{document}

\section*{Linear Programming Model for Investor's Portfolio}

Let:
\begin{itemize}
    \item \( N \): Number of different stocks.
    \item \( bought_i \): Number of shares bought of stock \( i \).
    \item \( buyPrice_i \): Price per share at which stock \( i \) was bought.
    \item \( currentPrice_i \): Current price per share of stock \( i \).
    \item \( futurePrice_i \): Expected future price per share of stock \( i \).
    \item \( transactionRate \): Transaction cost rate (in percentage).
    \item \( taxRate \): Tax rate on capital gains (in percentage).
    \item \( K \): Desired net amount of money to raise.
    \item \( sell_i \): Number of shares of stock \( i \) to sell.
\end{itemize}

The objective of the investor is to determine the number of shares to sell such that the net amount raised, after considering transaction costs and taxes, meets or exceeds \( K \).

\subsection*{Objective Function}
Maximize the expected future value of the portfolio:

\[
\text{Maximize } Z = \sum_{i=1}^{N} (futurePrice_i \cdot (bought_i - sell_i)) 
\]

\subsection*{Constraints}
To ensure that the net money raised is at least \( K \):
\[
\sum_{i=1}^{N} \left( (currentPrice_i \cdot sell_i) - transactionRate \cdot (currentPrice_i \cdot sell_i) - taxRate \cdot \left( (currentPrice_i \cdot sell_i - buyPrice_i \cdot sell_i) \right) \right) \geq K
\]

Breaking this down, we simplify the net amount raised to:
\[
\sum_{i=1}^{N} \left( (1 - transactionRate - taxRate)(currentPrice_i \cdot sell_i) + (taxRate \cdot buyPrice_i \cdot sell_i) \right) \geq K
\]

Also, ensuring that the number of shares sold does not exceed the number owned:
\[
0 \leq sell_i \leq bought_i \quad \text{for all } i \in \{1, 2, \ldots, N\}
\]

\subsection*{Variables}
Define \( sell_i \) as the decision variable for the number of shares to sell for stock \( i \).

\subsection*{Output}
The solution will yield:
\[
\text{Output: } \{ "sell": [sell_1, sell_2, \ldots, sell_N] \}
\]

\end{document}