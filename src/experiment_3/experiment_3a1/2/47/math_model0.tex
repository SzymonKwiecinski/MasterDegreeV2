\documentclass{article}
\usepackage{amsmath}
\begin{document}

\section*{Linear Programming Model for Police Shift Assignment}

\subsection*{Definitions}
Let:
\begin{itemize}
    \item \( S \) be the total number of shifts.
    \item \( officers\_needed_{s} \) be the number of police officers needed during shift \( s \) for \( s = 1, 2, \ldots, S \).
    \item \( shift\_cost_{s} \) be the cost incurred by the town when starting a shift at \( s \).
    \item \( officers\_assigned_{s} \) be the number of police officers assigned to shift \( s \).
\end{itemize}

\subsection*{Objective}
Minimize the total cost:
\[
\text{Total Cost} = \sum_{s=1}^{S} shift\_cost_{s} \cdot x_{s}
\]

where \( x_{s} \) is a binary variable indicating whether shift \( s \) is assigned (1) or not assigned (0).

\subsection*{Constraints}
1. Each officer works for two consecutive shifts:
   \[
   officers\_assigned_{s} \geq officers\_needed_{s} \quad \forall s = 1, 2, \ldots, S
   \]

2. The number of officers assigned to a shift is dependent on the officers assigned to the previous shift:
   \[
   officers\_assigned_{s} \leq officers\_assigned_{s-1} + 2 \cdot x_{s} \quad \forall s = 2, 3, \ldots, S
   \]
   \[
   officers\_assigned_{1} \leq 2 \cdot x_{1} 
   \]

3. Non-negativity constraint:
   \[
   officers\_assigned_{s} \geq 0 \quad \forall s
   \]

4. Binary decision variable:
   \[
   x_{s} \in \{0, 1\} \quad \forall s
   \]

\subsection*{Output}
The solution will provide:
\begin{itemize}
    \item \( officers\_assigned \): The list of officers assigned to each shift.
    \item \( total\_cost \): The total cost incurred by the town for the assigned shifts.
\end{itemize}

\end{document}