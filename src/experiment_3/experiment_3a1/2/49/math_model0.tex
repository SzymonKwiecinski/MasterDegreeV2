\documentclass{article}
\usepackage{amsmath}
\begin{document}

\section*{Linear Programming Model for Auto Parts Manufacturer}

\subsection*{Variables}
Let \(batches_{p}\) be the number of batches produced for part \(p\), where \(p = 1, \ldots, P\).

\subsection*{Objective Function}
The objective is to maximize total profit, which can be expressed as: 
\[
\text{Maximize } Z = \sum_{p=1}^{P} \left( price_{p} \cdot batches_{p} - \sum_{m=1}^{M} (cost_{m} \cdot time_{m,p} \cdot batches_{p}) \right)
\]

\subsection*{Constraints}
1. **Machine Time Availability**: Ensure that the total time required on each machine does not exceed its availability.
   \[
   \sum_{p=1}^{P} time_{m,p} \cdot batches_{p} \leq available_{m} \quad \forall m = 1, \ldots, M
   \]
   Note that Machine \(M\) and Machine \(M-1\) can share availability.

2. **Minimum Production Requirement**: Each part must meet its minimum batch requirement.
   \[
   batches_{p} \geq min\_batches_{p} \quad \forall p = 1, \ldots, P
   \]

\subsection*{Non-negativity Constraints}
\[
batches_{p} \geq 0 \quad \forall p = 1, \ldots, P
\]

\subsection*{Output}
The result will provide the number of batches produced for each part:
\[
\text{Output: } \{ batches = [batches_{1}, batches_{2}, \ldots, batches_{P}], total\_profit = Z \}
\]

\end{document}