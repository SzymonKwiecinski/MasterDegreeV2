\documentclass{article}
\usepackage{amsmath}
\begin{document}

\section*{Linear Programming Model}

\subsection*{Parameters}
\begin{itemize}
    \item Let \( A \) be the total weight of the alloy required: \( A = \text{alloy\_quant} \)
    \item Let \( M \) be the number of metals.
    \item Let \( K \) be the number of available alloys.
    \item Let \( \text{target}_{m} \) be the target quantity of metal \( m \).
    \item Let \( \text{ratio}_{k, m} \) be the ratio of metal \( m \) in alloy \( k \).
    \item Let \( \text{price}_{k} \) be the price of alloy \( k \).
\end{itemize}

\subsection*{Decision Variables}
Let \( x_k \) be the amount of alloy \( k \) to purchase for \( k = 1, \ldots, K \).

\subsection*{Objective Function}
Minimize the total cost:
\[
\text{Minimize} \quad Z = \sum_{k=1}^{K} \text{price}_{k} \cdot x_k
\]

\subsection*{Constraints}
1. The total weight of the alloys must equal the desired alloy weight:
\[
\sum_{k=1}^{K} x_k = A
\]

2. The quantity of each metal \( m \) in the mixture must satisfy the target:
\[
\sum_{k=1}^{K} \text{ratio}_{k,m} \cdot x_k = \text{target}_{m}, \quad \forall m = 1, \ldots, M
\]

3. Non-negativity constraints:
\[
x_k \geq 0, \quad \forall k = 1, \ldots, K
\]

\subsection*{Output}
The solution will provide the amounts of each alloy:
\[
\text{amount} = [x_1, x_2, \ldots, x_K]
\]

\end{document}