\documentclass{article}
\usepackage{amsmath}
\begin{document}

\section*{Quadratic Curve Fitting Problem}

Given a set of known pairs of values \( (x_k, y_k) \) for \( k = 1, \ldots, K \), we aim to fit a quadratic curve of the form:

\[
y = c \cdot x^2 + b \cdot x + a
\]

where:
- \( c \) is the coefficient of the quadratic term,
- \( b \) is the coefficient of the linear term,
- \( a \) is the constant term.

The objective is to minimize the sum of absolute deviations between the observed values \( y_k \) and the predicted values from the quadratic relationship:

\[
\text{Minimize} \quad \sum_{k=1}^{K} |y_k - (c \cdot x_k^2 + b \cdot x_k + a)|
\]

This formulation represents a Linear Programming (LP) problem where:

1. We introduce auxiliary variables \( d_k \) to handle the absolute values:
   
   \[
   d_k \geq y_k - (c \cdot x_k^2 + b \cdot x_k + a), \quad \forall k
   \]
   \[
   d_k \geq -(y_k - (c \cdot x_k^2 + b \cdot x_k + a)), \quad \forall k
   \]

2. The objective function can be rewritten as:

\[
\text{Minimize} \quad \sum_{k=1}^{K} d_k
\]

3. Subject to the constraints defined above.

The solution will result in values for \( c \), \( b \), and \( a \) corresponding to the best fit quadratic curve.

\subsection*{Output Format}

The output will provide the coefficients of the fitted quadratic curve in the following format:

\[
\{
    "quadratic": c,
    "linear": b,
    "constant": a
\}
\]

Where \( c \) is the coefficient of the quadratic term, \( b \) is the coefficient of the linear term, and \( a \) is the constant.

\end{document}