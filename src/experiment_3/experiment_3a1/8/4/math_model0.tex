\documentclass{article}
\usepackage{amsmath}
\begin{document}

\section*{Nurse Scheduling Problem: MILP Formulation}

\textbf{Definitions:}

Let:
\begin{itemize}
    \item $N$: total number of nurses hired.
    \item $d_j$: demand for nurses on day $j$, where $j = 1, \ldots, 7$.
    \item $period$: number of consecutive days a nurse works the night shift.
    \item $start_j$: number of nurses starting their shift on day $j$.
\end{itemize}

\textbf{Objective:}

Minimize the total number of nurses hired:

\[
\text{Minimize } N = \sum_{j=1}^{7} start_j
\]

\textbf{Constraints:}

For each day \( j \) (where \( j = 1, \ldots, 7 \)):
\[
\sum_{i=0}^{period-1} start_{(j-i) \mod 7} \geq d_j \quad \forall j \in \{1,\ldots,7\}
\]

Where \( start_{(j-i) \mod 7} \) represents the nurses starting on day \( j-i \) and ensures that the demand \( d_j \) is met for each day.

\textbf{Non-negativity constraints:}
\[
start_j \geq 0 \quad \forall j \in \{1,\ldots,7\}
\]

\textbf{Binary Constraints:}
Since \( start_j \) must be an integer, we can define:
\[
start_j \in \mathbb{Z}^{+} \quad \forall j \in \{1,\ldots,7\}
\]

\textbf{Output:}

Finally, the output will be:
\begin{itemize}
    \item $start = [start_1, start_2, \ldots, start_7]$
    \item $total = N$
\end{itemize}

\end{document}