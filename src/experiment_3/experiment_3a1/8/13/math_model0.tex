\documentclass{article}
\usepackage{amsmath}
\begin{document}

\section*{Linear Programming Model for the Chebyshev Center}

Given a set \( P \) described by the linear inequalities:

\[
P = \{ \mathbf{x} \in \mathbb{R}^N \mid \mathbf{a}_i^T \mathbf{x} \leq b_i, \; i = 1, \ldots, m \}
\]

where \( \mathbf{a}_i \in \mathbb{R}^N \) and \( b_i \in \mathbb{R} \).

We aim to find the largest ball with center \( \mathbf{y} \) and radius \( r \) that is entirely contained within the set \( P \). The ball is defined as:

\[
B(\mathbf{y}, r) = \{ \mathbf{x} \in \mathbb{R}^N \mid \|\mathbf{x} - \mathbf{y}\| \leq r \}
\]

The problem can be formulated as a linear programming problem as follows:

\begin{align*}
\text{Maximize} \quad & r \\
\text{subject to} \quad & \mathbf{a}_i^T \mathbf{y} + r \|\mathbf{a}_i\|_2 \leq b_i, \; i = 1, \ldots, m \\
& r \geq 0 \\
\end{align*}

where \( \|\mathbf{a}_i\|_2 \) denotes the Euclidean norm of the vector \( \mathbf{a}_i \).

\subsection*{Input Format}

The input consists of:
\begin{itemize}
    \item \( A \): A matrix where each row corresponds to \( \mathbf{a}_i \).
    \item \( b \): A vector where each element corresponds to \( b_i \).
\end{itemize}

\subsection*{Output Format}

The output is given as:
\begin{verbatim}
{
    "center": [y_j for j in 1, ..., n],
    "radius": r
}
\end{verbatim}

\end{document}