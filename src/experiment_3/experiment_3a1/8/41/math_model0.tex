\documentclass{article}
\usepackage{amsmath}
\begin{document}

\section*{Mathematical Model for File Distribution on Floppy Disks}

\textbf{Parameters:}
\begin{itemize}
    \item $C$: Capacity of each floppy disk (in GB).
    \item $N$: Number of files.
    \item $size_j$: Size of file $j$, for $j = 1, \ldots, N$.
\end{itemize}

\textbf{Variables:}
\begin{itemize}
    \item $n\_disks$: Total number of floppy disks used.
    \item $whichdisk_j$: The disk on which file $j$ is stored, for $j = 1, \ldots, N$.
\end{itemize}

\textbf{Objective:}
Minimize the number of floppy disks used:
\[
\text{Minimize } n\_disks
\]

\textbf{Constraints:}
1. Each file must be assigned to exactly one disk:
\[
\sum_{d=1}^{n\_disks} x_{jd} = 1, \quad \forall j \in \{1, 2, \ldots, N\}
\]
where \( x_{jd} \) is a binary variable that is 1 if file $j$ is stored on disk $d$, and 0 otherwise.

2. The total size of files on each disk must not exceed its capacity:
\[
\sum_{j=1}^{N} size_j \cdot x_{jd} \leq C, \quad \forall d \in \{1, 2, \ldots, n\_disks\}
\]

3. Define the number of disks used:
\[
n\_disks = \sum_{d=1}^{M} y_d
\]
where \( y_d \) is a binary variable that is 1 if disk $d$ is used, and 0 otherwise, and \( M \) is an upper bound on the number of disks.

4. Ensure that a disk is only counted if at least one file is assigned to it:
\[
y_d \geq x_{jd}, \quad \forall j \in \{1, 2, \ldots, N\}, \quad \forall d \in \{1, 2, \ldots, n\_disks\}
\]

\textbf{Output:}
\begin{itemize}
    \item $n\_disks$: Total number of floppy disks used.
    \item $whichdisk$: An array where each entry $disk_j$ indicates the disk on which file $j$ is stored.
\end{itemize}

\end{document}