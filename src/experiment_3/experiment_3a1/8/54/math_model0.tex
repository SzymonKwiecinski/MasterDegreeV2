\documentclass{article}
\usepackage{amsmath}
\begin{document}

\section*{Linear Programming Model for Auto Parts Manufacturer}

\subsection*{Variables}
Let:
\begin{itemize}
    \item $batches_p$ be the number of batches produced of part $p$, for $p = 1, 2, \ldots, P$.
\end{itemize}

\subsection*{Parameters}
Given:
\begin{itemize}
    \item $time_{m,p}$: time required in hours on machine $m$ to produce one batch of part $p$.
    \item $cost_m$: cost per hour for using machine $m$.
    \item $available_m$: availability in hours per month for machine $m$.
    \item $price_p$: selling price per batch of part $p$.
    \item $min\_batches_p$: minimum number of batches of part $p$ that must be produced.
    \item $standard\_cost$: cost per hour for standard labor.
    \item $overtime\_cost$: cost per hour for overtime labor.
    \item $overtime\_hour$: maximum hours of standard labor before overtime rates apply.
\end{itemize}

\subsection*{Objective Function}
Maximize the total profit:

\[
\text{Maximize } Z = \sum_{p=1}^{P} \left( price_p \cdot batches_p - \sum_{m=1}^{M} (time_{m,p} \cdot batches_p \cdot cost_m) \right) - \text{Labor Costs}
\]

Where Labor Costs for machine 1 can be calculated as:

\[
\text{Labor Costs} = 
\begin{cases}
standard\_cost \cdot hours & \text{if } hours \leq overtime\_hour \\
standard\_cost \cdot overtime\_hour + overtime\_cost \cdot (hours - overtime\_hour) & \text{if } hours > overtime\_hour
\end{cases}
\]

with \( hours = time_{1,p} \cdot batches_p \).

\subsection*{Constraints}
1. Capacity constraints for each machine \( m \):

\[
\sum_{p=1}^{P} (time_{m,p} \cdot batches_p) \leq available_m, \quad \forall m = 1, \ldots, M
\]

2. Minimum production requirements for each part:

\[
batches_p \geq min\_batches_p, \quad \forall p = 1, \ldots, P
\]

3. Non-negativity constraints:

\[
batches_p \geq 0, \quad \forall p = 1, \ldots, P
\]

\subsection*{Output}
The output will be:

\begin{itemize}
    \item \( batches_p \): number of batches produced for each part \( p \).
    \item \( total\_profit = Z \): total profit obtained for the month.
\end{itemize}

\end{document}