\documentclass{article}
\usepackage{amsmath}
\begin{document}

\section*{Linear Programming Model for Stock Selling Problem}

\subsection*{Parameters}
\begin{align*}
N & : \text{Number of different stocks} \\
bought_i & : \text{Number of shares bought of stock } i \\
buyPrice_i & : \text{Price per share bought of stock } i \\
currentPrice_i & : \text{Current price per share of stock } i \\
futurePrice_i & : \text{Expected future price per share of stock } i \\
transactionRate & : \text{Transaction cost rate (percentage)} \\
taxRate & : \text{Tax rate on capital gains (percentage)} \\
K & : \text{Desired net amount to raise}
\end{align*}

\subsection*{Decision Variables}
\begin{align*}
sell_i & : \text{Number of shares of stock } i \text{ to sell}
\end{align*}

\subsection*{Objective Function}
We need to maximize the expected value of the portfolio next year, given by:

\begin{align*}
\text{Maximize } & \sum_{i=1}^{N} (futurePrice_i \cdot (bought_i - sell_i)) - \sum_{i=1}^{N} (currentPrice_i \cdot sell_i \cdot (1 - \frac{transactionRate}{100}) - (sell_i \cdot (currentPrice_i - buyPrice_i) \cdot \frac{taxRate}{100}))
\end{align*}

\subsection*{Constraints}
To ensure that the investor raises at least the desired amount \( K \), we have the following constraint:

\begin{align*}
\sum_{i=1}^{N} (currentPrice_i \cdot sell_i \cdot (1 - \frac{transactionRate}{100}) - (sell_i \cdot (currentPrice_i - buyPrice_i) \cdot \frac{taxRate}{100})) & \geq K
\end{align*}

Additionally, the number of shares sold must not exceed the number of shares bought:

\begin{align*}
sell_i & \leq bought_i, \quad \text{for } i = 1, \ldots, N
\end{align*}

And the number of shares sold must be non-negative:

\begin{align*}
sell_i & \geq 0, \quad \text{for } i = 1, \ldots, N
\end{align*}

\subsection*{Output Format}
The output will consist of the number of shares to sell for each stock:

\begin{align*}
\text{Output: } & \{ "sell": [sell_1, sell_2, \ldots, sell_N] \}
\end{align*}

\end{document}