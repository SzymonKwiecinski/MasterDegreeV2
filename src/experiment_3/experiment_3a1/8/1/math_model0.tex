\documentclass{article}
\usepackage{amsmath}
\begin{document}

\section*{Linear Programming Model}

\subsection*{Decision Variables}
Let \( x_j \) be the amount of good \( j \) produced, for \( j = 1, 2, \ldots, M \).

\subsection*{Parameters}
\begin{itemize}
    \item \( available_i \): The amount of raw material \( i \) available, for \( i = 1, 2, \ldots, N \).
    \item \( req_{i,j} \): The amount of raw material \( i \) required to produce one unit of good \( j \).
    \item \( price_j \): The revenue generated from producing one unit of good \( j \).
\end{itemize}

\subsection*{Objective Function}
The goal is to maximize the total revenue:
\[
\text{Maximize } Z = \sum_{j=1}^{M} price_j \cdot x_j
\]

\subsection*{Constraints}
The production of goods is limited by the availability of raw materials. For each raw material \( i \):
\[
\sum_{j=1}^{M} req_{i,j} \cdot x_j \leq available_i, \quad \text{for } i = 1, 2, \ldots, N
\]
Additionally, the production quantities must be non-negative:
\[
x_j \geq 0, \quad \text{for } j = 1, 2, \ldots, M
\]

\subsection*{Output}
The output will be the amounts of each good produced:
\[
\text{Output: } \{ amount_j \}_{j=1}^M \text{ where } amount_j = x_j
\]

\end{document}