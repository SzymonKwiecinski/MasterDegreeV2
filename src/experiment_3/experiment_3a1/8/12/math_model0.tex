\documentclass{article}
\usepackage{amsmath}
\begin{document}

\section*{Linear Programming Formulation}

Let:
\begin{itemize}
    \item \(N\) be the number of available currencies,
    \item \(x_{i,j}\) be the amount of currency \(i\) exchanged for currency \(j\),
    \item \(start_i\) be the starting amount of currency \(i\),
    \item \(limit_i\) be the limit on the amount of currency \(i\) that can be exchanged,
    \item \(rate_{i,j}\) be the exchange rate from currency \(i\) to currency \(j\).
\end{itemize}

We aim to maximize the total amount of currency \(N\) at the end of the day.

\subsection*{Objective Function}

The objective function to maximize is:
\[
\text{Maximize } z = x_{N, N} + \sum_{j=1}^{N} x_{N,j}
\]
where \(x_{N, N}\) represents the amount of currency \(N\) retained after transactions and \(\sum_{j=1}^{N} x_{N,j}\) is the total amount of currency \(N\) obtained through exchanges with other currencies.

\subsection*{Subject to the following constraints:}

1. \textbf{Exchange Amount Constraints}:
   For each currency \(i\):
   \[
   \sum_{j=1}^{N} x_{i,j} \leq limit_i, \quad \forall i = 1, 2, \ldots, N
   \]

2. \textbf{Starting Amount Constraints}:
   The amount exchanged from currency \(i\) must not exceed the starting amount:
   \[
   x_{i,j} \leq start_i, \quad \forall i = 1, 2, \ldots, N, \forall j = 1, 2, \ldots, N
   \]

3. \textbf{Exchange Rate Constraints}:
   The converted amounts must respect the exchange rates:
   \[
   x_{i,j} \cdot rate_{i,j} \leq x_{j,i}, \quad \forall i,j \text{ such that } i \neq j
   \]

4. \textbf{Non-negativity Constraints}:
   All exchanged amounts must be non-negative:
   \[
   x_{i,j} \geq 0, \quad \forall i, j
   \]

\subsection*{Final Amount of Currency N}

At the end of the day, the total amount of currency \(N\) will be calculated as:
\[
final\_amount\_of\_currency\_N = x_{N,N} + \sum_{j=1}^{N} (x_{N,j} \cdot rate_{N,j})
\]

This formulation captures the optimization problem of maximizing the total amount of currency \(N\) through a series of transactions while adhering to the specified constraints.

\end{document}