\documentclass{article}
\usepackage{amsmath}
\begin{document}

\section*{Mathematical Model for Police Officer Shift Assignment}

\subsection*{Parameters}
\begin{align*}
S & : \text{Number of shifts} \\
officers\_needed_{s} & : \text{Number of police officers needed during shift } s, \quad s = 1, \ldots, S \\
shift\_cost_{s} & : \text{Cost incurred when starting shift } s, \quad s = 1, \ldots, S \\
\end{align*}

\subsection*{Decision Variables}
\begin{align*}
officers\_assigned_{s} & : \text{Number of police officers assigned to shift } s, \quad s = 1, \ldots, S \\
\end{align*}

\subsection*{Objective Function}
\begin{align*}
\text{Minimize } Z & = \sum_{s=1}^{S} shift\_cost_{s} \cdot \left\lfloor \frac{officers\_assigned_{s} + officers\_assigned_{s+1}}{2} \right\rfloor 
\end{align*}

\subsection*{Constraints}
\begin{align*}
officers\_assigned_{s} + officers\_assigned_{s+1} & \geq officers\_needed_{s}, \quad s = 1, \ldots, S-1 \\
officers\_assigned_{s} & \geq 0 \quad \forall s \\
\end{align*}

\end{document}