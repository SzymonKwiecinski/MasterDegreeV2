\documentclass{article}
\usepackage{amsmath}
\begin{document}

\section*{Mathematical Model for the Paper Cutting Problem}

\textbf{Parameters:}
\begin{itemize}
    \item \( W \): Large roll width (inches) 
    \item \( M \): Number of different types of smaller rolls 
    \item \( d_j \): Demand for rolls of width \( j \) (for \( j = 1, \ldots, M \)) 
    \item \( N \): Number of different cutting patterns 
    \item \( p_{ij} \): Number of rolls of width \( j \) produced by cutting pattern \( i \) (for \( i = 1, \ldots, N \) and \( j = 1, \ldots, M \))
\end{itemize}

\textbf{Decision Variables:}
\begin{itemize}
    \item \( x_i \): Number of large rolls cut according to pattern \( i \) (for \( i = 1, \ldots, N \))
\end{itemize}

\textbf{Objective Function:}
\begin{align*}
    \text{Minimize} \quad & \sum_{i=1}^{N} x_i
\end{align*}

\textbf{Constraints:}
\begin{align*}
    \sum_{i=1}^{N} p_{ij} x_i & \geq d_j \quad \text{for } j = 1, \ldots, M \\
    x_i & \geq 0 \quad \text{and integer} \quad \text{for } i = 1, \ldots, N 
\end{align*}

\textbf{Output:}
\begin{itemize}
    \item The cutting patterns applied to the large rolls along with their amounts used 
    \item Total number of large rolls used: \( \sum_{i=1}^{N} x_i \)
\end{itemize}

\end{document}