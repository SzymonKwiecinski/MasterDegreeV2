\documentclass{article}
\usepackage{amsmath}
\begin{document}

\title{Linear Programming Model for Alloy Production}
\author{}
\date{}
\maketitle

\section*{Problem Formulation}

Let:
\begin{itemize}
    \item \( A \) be the total quantity of alloy to be produced, where \( A = \text{alloy\_quant} \).
    \item \( M \) be the number of metals in the alloy.
    \item \( K \) be the number of available alloys.
    \item \( t_m \) be the target quantity of metal \( m \), where \( t_m = \text{target}[m] \) for \( m = 1, \ldots, M \).
    \item \( r_{k,m} \) be the ratio of metal \( m \) in alloy \( k \), where \( r_{k,m} = \text{ratio}[k][m] \) for \( k = 1, \ldots, K \) and \( m = 1, \ldots, M \).
    \item \( p_k \) be the price of alloy \( k \), where \( p_k = \text{price}[k] \).
    \item \( x_k \) be the amount of alloy \( k \) to purchase, where \( k = 1, \ldots, K \).
\end{itemize}

\subsection*{Objective Function}

The objective is to minimize the total cost of the alloys used:

\[
\text{Minimize } Z = \sum_{k=1}^{K} p_k x_k
\]

\subsection*{Constraints}

1. The total weight of the purchased alloys must equal the desired weight of the alloy:

\[
\sum_{k=1}^{K} x_k = A
\]

2. The amount of each metal in the purchased alloys must meet the target requirements:

\[
\sum_{k=1}^{K} r_{k,m} x_k = t_m \quad \text{for } m = 1, \ldots, M
\]

3. Non-negativity constraints:

\[
x_k \geq 0 \quad \text{for } k = 1, \ldots, K
\]

\subsection*{Output}

The output will be the amounts of each alloy that need to be purchased:

\[
\text{Output: } \{ \text{amount} : [x_k \text{ for } k = 1, \ldots, K] \}
\]

\end{document}