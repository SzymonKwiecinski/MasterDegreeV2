\documentclass{article}
\usepackage{amsmath}
\begin{document}

\section*{Linear Programming Formulation}

Consider the following linear programming formulation for the currency exchange problem. 

Define the variables:

\begin{align*}
x_{i,j} & : \text{ the amount of currency } i \text{ exchanged to currency } j \\
y_i & : \text{ the final amount of currency } i \text{ we have at the end of the day}
\end{align*}

The objective is to maximize the total amount of currency \( N \):

\[
\text{Maximize } y_N
\]

Subject to the following constraints:

1. **Initial Amount Constraint**:
   \[
   y_i = \text{start}_i + \sum_{j=1}^{N} x_{j,i} - \sum_{j=1}^{N} x_{i,j} \quad \forall i = 1, \ldots, N
   \]

   This represents the final amount of each currency after considering the amounts exchanged in and out.

2. **Limit Constraints**:
   \[
   \sum_{j=1}^{N} x_{i,j} + \sum_{j=1}^{N} x_{j,i} \leq \text{limit}_i \quad \forall i = 1, \ldots, N
   \]

   This ensures that the total amount of currency \( i \) exchanged does not exceed the given limits.

3. **Exchange Rate Constraints**:
   \[
   x_{i,j} \leq \text{start}_i \cdot \text{rate}_{i,j} \quad \forall i, j = 1, \ldots, N
   \]

   This ensures that the amount exchanged does not exceed the amount of currency available and adheres to the exchange rates.

4. **Non-negativity Constraints**:
   \[
   x_{i,j} \geq 0 \quad \forall i, j = 1, \ldots, N
   \]

   The exchanged amounts cannot be negative.

This linear programming model provides a structured way to analyze the best currency exchanges to maximize the amount of currency \( N \) at the end of the day under the given constraints.

\end{document}