\documentclass{article}
\usepackage{amsmath}
\begin{document}

\section*{Linear Programming Model for Production Optimization}

\subsection*{Variables}
Let \( x_j \) be the amount of good \( j \) produced, where \( j = 1, 2, \ldots, M \).

\subsection*{Objective Function}
Maximize the total revenue:
\[
\text{Maximize } Z = \sum_{j=1}^{M} \text{price}_j \cdot x_j
\]

\subsection*{Constraints}
The production of goods is limited by the availability of raw materials. For each raw material \( i \) (where \( i = 1, 2, \ldots, N \)), we have the following constraint:
\[
\sum_{j=1}^{M} \text{req}_{i,j} \cdot x_j \leq \text{available}_i \quad \forall i
\]

\subsection*{Non-negativity Constraints}
The amount produced for each good must be non-negative:
\[
x_j \geq 0 \quad \forall j
\]

\subsection*{Model Summary}
The complete linear programming model is summarized as follows:

\begin{align*}
\text{Maximize} \quad & Z = \sum_{j=1}^{M} \text{price}_j \cdot x_j \\
\text{subject to} \quad & \sum_{j=1}^{M} \text{req}_{i,j} \cdot x_j \leq \text{available}_i \quad \forall i \\
& x_j \geq 0 \quad \forall j
\end{align*}

\end{document}