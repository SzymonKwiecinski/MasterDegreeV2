\documentclass{article}
\usepackage{amsmath}
\begin{document}

\title{Linear Programming Model for Spare Automobile Parts Production}
\author{}
\date{}
\maketitle

\section*{Problem Formulation}

Let \( K \) be the total number of spare parts and \( S \) be the number of shops. Define the following variables:

\begin{itemize}
    \item \( \text{time}_{k, s} \): Required worker-hours for part \( k \) in shop \( s \).
    \item \( \text{profit}_{k} \): Profit from producing part \( k \).
    \item \( \text{capacity}_{s} \): Capacity of shop \( s \) in working hours.
    \item \( \text{quantity}_{k} \): Quantity of part \( k \) to be produced.
\end{itemize}

\subsection*{Objective Function}
The objective is to maximize the total profit:

\[
\text{Maximize } Z = \sum_{k=1}^{K} \text{profit}_{k} \cdot \text{quantity}_{k}
\]

\subsection*{Constraints}
The constraints for the problem are imposed by the capacities of each shop. For each shop \( s \):

\[
\sum_{k=1}^{K} \text{time}_{k, s} \cdot \text{quantity}_{k} \leq \text{capacity}_{s} \quad \text{for } s = 1, 2, \ldots, S
\]

Additionally, we have non-negativity constraints for the quantities:

\[
\text{quantity}_{k} \geq 0 \quad \text{for } k = 1, 2, \ldots, K
\]

\section*{Summary}
The linear programming model can be summarized as:

\[
\begin{align*}
\text{Maximize } & Z = \sum_{k=1}^{K} \text{profit}_{k} \cdot \text{quantity}_{k} \\
\text{subject to } & \sum_{k=1}^{K} \text{time}_{k, s} \cdot \text{quantity}_{k} \leq \text{capacity}_{s}, \quad s = 1, 2, \ldots, S \\
& \text{quantity}_{k} \geq 0, \quad k = 1, 2, \ldots, K
\end{align*}
\]

\end{document}