\documentclass{article}
\usepackage{amsmath}
\begin{document}

\section*{Linear Programming Model for Police Shift Assignment}

\subsection*{Parameters}
\begin{itemize}
    \item \( S \): Number of different shifts
    \item \( \text{officers\_needed}_{s} \): Number of police officers needed during shift \( s \) (for \( s = 1, 2, \ldots, S \))
    \item \( \text{shift\_cost}_{s} \): Cost incurred by starting shift \( s \) (for \( s = 1, 2, \ldots, S \))
\end{itemize}

\subsection*{Decision Variables}
\begin{itemize}
    \item \( \text{officers\_assigned}_{s} \): Number of officers assigned to shift \( s \) (for \( s = 1, 2, \ldots, S \))
\end{itemize}

\subsection*{Objective Function}
Minimize the total cost:

\[
\text{Total Cost} = \sum_{s=1}^{S} \text{shift\_cost}_{s} \cdot y_s
\]

where \( y_s \) is a binary variable that equals 1 if shift \( s \) is staffed (i.e., \( \text{officers\_assigned}_{s} > 0 \)) and 0 otherwise.

\subsection*{Constraints}
\begin{align*}
\text{officers\_assigned}_{s} & \geq \text{officers\_needed}_{s} && \text{for } s = 1, 2, \ldots, S \\
\text{officers\_assigned}_{s} & = 0 && \text{if } y_s = 0 \text{ (non-staffed shifts)} \\
\text{officers\_assigned}_{s} & + \text{officers\_assigned}_{s+1} \geq \text{officers\_needed}_{s} && \text{for } s = 1, 2, \ldots, S-1 \\
\end{align*}

\subsection*{Output}
The solution will provide:
\begin{itemize}
    \item \( \text{officers\_assigned} = [\text{officers\_assigned}_{1}, \text{officers\_assigned}_{2}, \ldots, \text{officers\_assigned}_{S}] \)
    \item \( \text{total\_cost} \)
\end{itemize}

\end{document}