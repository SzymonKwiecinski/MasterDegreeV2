\documentclass{article}
\usepackage{amsmath}
\begin{document}

\section*{Quadratic Curve Fitting using Linear Programming}

We want to fit a quadratic curve of the form 

\[
y = c \cdot x^2 + b \cdot x + a
\]

to a set of data points \((x_k, y_k)\) for \(k = 1, 2, \ldots, K\). The objective is to minimize the sum of absolute deviations between the observed values of \(y\) and the predicted values based on the quadratic relationship.

Let \(y_k\) be the observed value and \(\hat{y_k} = c \cdot x_k^2 + b \cdot x_k + a\) be the predicted value. The absolute deviation can be expressed as 

\[
d_k = |y_k - \hat{y_k}|
\]

We aim to minimize the total absolute deviation:

\[
\text{Minimize} \quad \sum_{k=1}^{K} d_k
\]

We introduce auxiliary variables \(u_k\) and \(v_k\) to represent the positive and negative deviations:

\[
y_k - \hat{y_k} = u_k - v_k
\]

with the constraints:

\[
d_k = u_k + v_k
\]

Thus, we can rewrite our objective function as:

\[
\text{Minimize} \quad \sum_{k=1}^{K} (u_k + v_k)
\]

The complete optimization problem can be formulated as:

\[
\begin{align*}
\text{Minimize} \quad & \sum_{k=1}^{K} (u_k + v_k) \\
\text{Subject to} \quad & u_k - v_k = y_k - (c \cdot x_k^2 + b \cdot x_k + a), \quad k = 1, 2, \ldots, K \\
& u_k \geq 0, \quad v_k \geq 0, \quad k = 1, 2, \ldots, K
\end{align*}
\]

After solving the above linear programming problem, we can extract the coefficients:

\[
\begin{align*}
\text{quadratic} & : c \\
\text{linear} & : b \\
\text{constant} & : a
\end{align*}
\]

The final output format will be:

\[
\{
    "quadratic": c,
    "linear": b,
    "constant": a
\}
\]

\end{document}