\documentclass{article}
\usepackage{amsmath}
\begin{document}

\section*{Mathematical Model for Cafeteria Staffing}

\subsection*{Parameters}
\begin{itemize}
    \item $N$: Number of days
    \item $num_n$: Number of employees required on day $n$, for $n = 1, 2, \ldots, N$
    \item $n\_working\_days$: Number of consecutive working days for each employee
    \item $n\_resting\_days$: Number of consecutive days off for each employee
\end{itemize}

\subsection*{Decision Variables}
\begin{itemize}
    \item $total\_number$: Total number of employees to hire
    \item $is\_work_{n,i}$: Binary variable indicating whether employee $i$ is working on day $n$ ($is\_work_{n,i} = 1$ if employee $i$ works on day $n$, 0 otherwise), for $n = 1, \ldots, N$ and $i = 1, \ldots, total\_number$
\end{itemize}

\subsection*{Objective Function}
Minimize the total number of employees:
\[
\text{Minimize } total\_number
\]

\subsection*{Constraints}
1. Ensure that the number of employees working meets the requirement for each day:
\[
\sum_{i=1}^{total\_number} is\_work_{n,i} \geq num_n, \quad \forall n = 1, \ldots, N
\]
2. Define the working and resting schedule for each employee:
If an employee starts working on day $d$, they will work for $n\_working\_days$ and rest for $n\_resting\_days$. Thus:
\[
is\_work_{n,i} = 
\begin{cases}
1 & \text{if } (n - d) \mod (n\_working\_days + n\_resting\_days) < n\_working\_days \\
0 & \text{otherwise}
\end{cases}, \quad \forall i = 1, \ldots, total\_number, \forall n
\]

\subsection*{Output Information}
The output should provide:
\begin{itemize}
    \item $total\_number$: Total number of employees hired.
    \item $is\_work$: A matrix where element $is\_work_{n,i}$ indicates the working status of employee $i$ on day $n$.
\end{itemize}

\end{document}