\documentclass{article}
\usepackage{amsmath}
\begin{document}

\section*{Electricity Capacity Expansion Model}

\subsection*{Parameters}
\begin{itemize}
    \item $T$: Total number of years for planning.
    \item $demand_t$: Forecasted electricity demand in megawatts for year $t$.
    \item $oil_t$: Existing oil-fired capacity available in year $t$.
    \item $coal\_cost$: Capital cost per megawatt for coal-fired capacity.
    \item $nuke\_cost$: Capital cost per megawatt for nuclear capacity.
    \item $max\_nuke$: Maximum percentage of total capacity that can be nuclear.
    \item $coal\_life$: Lifespan of coal plants in years.
    \item $nuke\_life$: Lifespan of nuclear plants in years.
\end{itemize}

\subsection*{Decision Variables}
\begin{itemize}
    \item $x_t^{coal}$: Megawatts of coal capacity added in year $t$.
    \item $x_t^{nuke}$: Megawatts of nuclear capacity added in year $t$.
\end{itemize}

\subsection*{Objective Function}
Minimize the total cost:
\[
\text{Minimize} \quad \sum_{t=1}^{T} (coal\_cost \times x_t^{coal} + nuke\_cost \times x_t^{nuke})
\]

\subsection*{Constraints}
\begin{itemize}
    \item \textbf{Demand Satisfaction:} For each year $t$, the total available capacity should meet or exceed the demand.
    \[
    oil_t + \sum_{j=\max(1, t-coal\_life+1)}^{t} x_j^{coal} + \sum_{k=\max(1, t-nuke\_life+1)}^{t} x_k^{nuke} \geq demand_t, \quad \forall t = 1, \ldots, T
    \]

    \item \textbf{Nuclear Capacity Limit:} The total nuclear capacity should not exceed the maximum allowed percentage of total capacity.
    \[
    \sum_{k=\max(1, t-nuke\_life+1)}^{t} x_k^{nuke} \leq max\_nuke \times \left( oil_t + \sum_{j=\max(1, t-coal\_life+1)}^{t} x_j^{coal} + \sum_{k=\max(1, t-nuke\_life+1)}^{t} x_k^{nuke} \right) / 100, \quad \forall t = 1, \ldots, T
    \]

    \item \textbf{Non-negativity:} The added capacities must be non-negative.
    \[
    x_t^{coal} \geq 0, \quad x_t^{nuke} \geq 0, \quad \forall t = 1, \ldots, T
    \]
\end{itemize}

\end{document}