\documentclass{article}
\usepackage{amsmath}
\begin{document}

\section*{Mathematical Model}

\subsection*{Indices}
\begin{itemize}
    \item \( k \): index for source terminals.
    \item \( p \): index for port cities.
    \item \( l \): index for destination cities.
    \item \( i, j \): indices for the cities in the transportation route network.
\end{itemize}

\subsection*{Parameters}
\begin{itemize}
    \item \( C_{i,j} \): transportation cost per metric ton from city \( i \) to city \( j \).
    \item \( supply_{k} \): supply of soybeans available at terminal city \( k \).
    \item \( demand_{l} \): demand for soybeans at destination city \( l \).
    \item \( A \): set of all possible routes (i, j).
\end{itemize}

\subsection*{Decision Variables}
\begin{itemize}
    \item \( amount_{i,j} \): quantity of soybeans shipped from city \( i \) to city \( j \).
\end{itemize}

\subsection*{Objective Function}
Minimize the total transportation cost:
\[
\text{Minimize } \quad \sum_{(i,j) \in A} C_{i,j} \times amount_{i,j}
\]

\subsection*{Constraints}
\begin{itemize}
    \item Supply constraints for each terminal \( k \):
    \[
    \sum_{j \mid (k,j) \in A} amount_{k,j} \leq supply_{k} \quad \forall k
    \]
    \item Demand constraints for each destination \( l \):
    \[
    \sum_{i \mid (i,l) \in A} amount_{i,l} \geq demand_{l} \quad \forall l
    \]
    \item Non-negativity constraints:
    \[
    amount_{i,j} \geq 0 \quad \forall (i,j) \in A
    \]
\end{itemize}

This linear programming model can be solved using standard optimization software or solvers like CPLEX, Gurobi, or open-source options such as PuLP in Python to obtain the optimal distribution of soybeans from source to destination while minimizing costs.

\end{document}