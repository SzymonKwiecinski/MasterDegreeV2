\documentclass{article}
\usepackage{amsmath}
\begin{document}

\title{Linear Programming Formulation for Fitting a Line}
\author{}
\date{}
\maketitle

Given a set of data points \((x_k, y_k)\) for \(k=1,\ldots,K\), the objective is to fit a line \(y = bx + a\) to these points by minimizing the sum of absolute deviations between observed and predicted values of \(y\).

\section*{Mathematical Model}

\textbf{Objective:}

\[
\min \sum_{k=1}^{K} z_k
\]

\textbf{Subject to:}

For each \(k = 1, \ldots, K\):
\[
z_k \geq y_k - (bx_k + a)
\]
\[
z_k \geq -(y_k - (bx_k + a))
\]

Where:
- \( z_k \) is the auxiliary variable representing the absolute deviation for the \(k\)-th data point.
- \( y_k \) is the observed value.
- \( x_k \) is the corresponding \(x\) value.
- \( a \) is the intercept of the line.
- \( b \) is the slope of the line.

The decision variables are:
- The intercept \(a\)
- The slope \(b\)
- The auxiliary variables \(z_k\) for each \(k\).

\end{document}