\documentclass{article}
\usepackage{amsmath}
\begin{document}

\section*{Mathematical Model}

\subsection*{Indices and Sets}
\begin{itemize}
    \item \( S \): Number of shifts in a day.
    \item \( s \): Index for shifts, \( s = 1, 2, \ldots, S \).
\end{itemize}

\subsection*{Parameters}
\begin{itemize}
    \item \texttt{officers\_needed}$_s$: Number of officers required for shift \( s \).
    \item \texttt{shift\_cost}$_s$: Cost of starting a shift at \( s \).
\end{itemize}

\subsection*{Decision Variables}
\begin{itemize}
    \item \( x_s \): Number of officers starting at shift \( s \).
\end{itemize}

\subsection*{Objective}
Minimize the total cost for assigning officers to shifts:
\[
\text{Minimize } Z = \sum_{s=1}^{S} \texttt{shift\_cost}_s \cdot x_s
\]

\subsection*{Constraints}
Each shift \( s \) must have the required number of officers, considering that each officer works for two consecutive shifts:
\[
x_s + x_{s-1} \geq \texttt{officers\_needed}_s, \quad \forall s = 1, 2, \ldots, S
\]
*Note: For the first shift, assume \( x_0 = x_S \) to account for the cyclic nature of shifts.*

\subsection*{Non-negativity}
\[
x_s \geq 0, \quad \forall s = 1, 2, \ldots, S
\]

\end{document}