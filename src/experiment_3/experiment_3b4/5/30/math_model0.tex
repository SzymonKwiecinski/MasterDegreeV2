\documentclass{article}
\usepackage{amsmath}
\begin{document}

\section*{Quadratic Curve Fitting via Linear Programming}

Given data points \((x_k, y_k)\) for \(k = 1, \ldots, K\), we aim to fit a quadratic curve \( y = c \cdot x^2 + b \cdot x + a \) by minimizing the sum of absolute deviations between observed and predicted values of \( y \).

\subsection*{Objective}

Minimize:
\[
\sum_{k=1}^K |y_k - (c \cdot x_k^2 + b \cdot x_k + a)|
\]

\subsection*{Linear Programming Formulation}

Introduce non-negative variables \( u_k \) and \( v_k \) to represent the positive and negative deviations, respectively:

For each data point \( k = 1, \ldots, K \):
\[
y_k - (c \cdot x_k^2 + b \cdot x_k + a) = u_k - v_k
\]
\[ 
u_k, v_k \geq 0
\]

The linear programming objective is to minimize:
\[
\sum_{k=1}^K (u_k + v_k)
\]

Subject to the constraints:
\[
y_k - c \cdot x_k^2 - b \cdot x_k - a = u_k - v_k \quad \forall \, k = 1, \ldots, K
\]
\[
u_k, v_k \geq 0 \quad \forall \, k = 1, \ldots, K
\]

\subsection*{Variables}

- \( c \): Coefficient of the quadratic term.
- \( b \): Coefficient of the linear term.
- \( a \): Constant term.
- \( u_k, v_k \): Non-negative variables representing deviations for each data point \( k \).

\end{document}