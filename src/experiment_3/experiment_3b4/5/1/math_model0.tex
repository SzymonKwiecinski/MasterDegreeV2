\documentclass{article}
\usepackage{amsmath}
\begin{document}

\section*{Linear Programming Model for Maximizing Revenue}

\subsection*{Problem Description}

A firm produces \( M \) different goods using \( N \) different raw materials. The objective is to determine the amount of each good to produce to maximize the firm's total revenue, subject to the availability of raw materials.

\subsection*{Data}

\begin{itemize}
    \item Let \( \text{available}_i \) be the amount of raw material \( i \) available for \( i = 1, \ldots, N \).
    \item Let \( \text{req}_{i,j} \) be the units of material \( i \) required per unit of good \( j \) produced for \( i = 1, \ldots, N \) and \( j = 1, \ldots, M \).
    \item Let \( \text{price}_j \) be the revenue per unit produced of good \( j \) for \( j = 1, \ldots, M \).
\end{itemize}

\subsection*{Variables}

\begin{itemize}
    \item Let \( \text{amount}_j \) be the amount of good \( j \) produced for \( j = 1, \ldots, M \).
\end{itemize}

\subsection*{Objective Function}

Maximize the total revenue:

\[
\text{Maximize} \quad Z = \sum_{j=1}^{M} \text{price}_j \times \text{amount}_j
\]

\subsection*{Constraints}

\begin{itemize}
    \item Material constraints: Ensure that the production does not exceed available raw materials.
    \[
    \sum_{j=1}^{M} \text{req}_{i,j} \times \text{amount}_j \leq \text{available}_i, \quad \forall i = 1, \ldots, N
    \]
    \item Non-negativity constraints: Production amount for each good cannot be negative.
    \[
    \text{amount}_j \geq 0, \quad \forall j = 1, \ldots, M
    \]
\end{itemize}

\end{document}