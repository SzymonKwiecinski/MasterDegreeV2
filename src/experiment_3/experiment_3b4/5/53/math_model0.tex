\documentclass{article}
\usepackage{amsmath}

\begin{document}

\section*{Mathematical Model}

Given the problem, we need to formulate a linear programming model that minimizes transportation costs in a network of cities.

\subsection*{Parameters and Variables}
- Let \( n \) be the number of cities.
- Let \( A \) be the set of possible routes for transporting soybeans.
- Let \( C_{i,j} \) be the cost per metric ton of transporting soybeans from city \( i \) to city \( j \).
- Let \( supply_k \) be the supply of soybeans available at terminal city \( k \).
- Let \( demand_l \) be the demand for soybeans at destination city \( l \).

The decision variable is:
- \( amount_{i,j} \) - the quantity of soybeans transported from city \( i \) to city \( j \).

\subsection*{Objective Function}
The objective is to minimize the total transportation cost:

\[
\text{Minimize} \quad Z = \sum_{(i,j) \in A} C_{i,j} \cdot amount_{i,j}
\]

\subsection*{Constraints}

1. **Supply Constraints**: Ensure that the shipment from each terminal city \( k \) does not exceed its supply:
   \[
   \sum_{(k,j) \in A} amount_{k,j} \leq supply_k, \quad \forall k
   \]

2. **Demand Constraints**: Ensure that the destination city \( l \) receives at least its demand:
   \[
   \sum_{(i,l) \in A} amount_{i,l} \geq demand_l, \quad \forall l
   \]

3. **Flow Conservation**: For each port city \( p \), the incoming shipment should equal outgoing shipment:
   \[
   \sum_{(i,p) \in A} amount_{i,p} = \sum_{(p,j) \in A} amount_{p,j}, \quad \forall p
   \]

4. **Non-negativity Constraints**: The shipped quantity cannot be negative:
   \[
   amount_{i,j} \geq 0, \quad \forall (i,j) \in A
   \]

\end{document}