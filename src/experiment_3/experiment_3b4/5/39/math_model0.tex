\documentclass{article}
\usepackage{amsmath}

\begin{document}

\section*{Problem Formulation}

We are given the following:

\begin{itemize}
    \item \(\text{num}_n\): Required number of employees for each day \(n\), where \(n = 1, \ldots, N\).
    \item \(\text{n\_working\_days}\): Number of consecutive days an employee works.
    \item \(\text{n\_resting\_days}\): Number of consecutive days an employee rests after working.
\end{itemize}

Define the variables:

\begin{itemize}
    \item \(x_i\): Binary variable indicating if employee \(i\) is employed.
    \item \(y_{n,i}\): Binary variable indicating if employee \(i\) is working on day \(n\).
\end{itemize}

The objective is to minimize the total number of employees hired:

\[
\text{Minimize} \quad \sum_{i=1}^{M} x_i
\]

Where \(M\) is a sufficiently large number representing a potential pool of employees to consider.

Subject to the constraints:

1. Each day \(n\) must have the required number of employees:

\[
\sum_{i=1}^{M} y_{n,i} \geq \text{num}_n, \quad \forall n \in \{1, \ldots, N\}
\]

2. Ensure that an employee works for \(\text{n\_working\_days}\) followed by \(\text{n\_resting\_days}\). This can be expressed with the following logical constraints:

For each employee \(i\), if they start working on day \(n\), then:

\[
y_{n+k,i} = 1, \quad \forall k \in \{0, \ldots, \text{n\_working\_days} - 1\}, \quad \text{if}\; x_i = 1
\]
\[
y_{n+k,i} = 0, \quad \forall k \in \{\text{n\_working\_days}, \ldots, \text{n\_working\_days} + \text{n\_resting\_days} - 1\}, \quad \text{if}\; x_i = 1
\]

3. The working constraint for \(y_{n,i}\) should depend on \(x_i\) (i.e., an employee cannot work if not hired):

\[
y_{n,i} \leq x_i, \quad \forall n \in \{1, \ldots, N\}, \forall i \in \{1, \ldots, M\}
\]

Given these constraints, solving the MILP will yield the optimal number of employees to hire along with their schedules.

\end{document}