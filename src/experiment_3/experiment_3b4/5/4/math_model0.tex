\documentclass{article}
\usepackage{amsmath}
\begin{document}

\textbf{MILP Formulation for Nurse Scheduling Problem}

\textbf{Objective:}
Minimize the total number of nurses hired.

\[
\text{Minimize } Z = \sum_{j=1}^{7} \text{start}_j
\]

\textbf{Constraints:}

1. Ensure that the demand on each day \( j \) is met:

For each day \( j \in \{1,...,7\} \), the total number of nurses covering the night shift should be greater than or equal to the demand \( d_j \).

\[
\sum_{i=0}^{\text{period}-1} \text{start}_{(j-i-1) \mod 7 + 1} \geq d_j, \quad \forall j = 1, \ldots, 7
\]

Where \((j-i-1) \mod 7 + 1\) ensures a cyclic consideration of days leading up to day \( j \).

2. Non-negativity constraints:

\[
\text{start}_j \geq 0 \quad \text{and integer}, \quad \forall j = 1, \ldots, 7
\]

\textbf{Variables:}

- \(\text{start}_j\): The number of nurses starting their period on day \( j \).

\textbf{Parameters:}

- \(d_j\): Demand for nurses on day \( j \).
- \(\text{period}\): Number of consecutive days each nurse works.

\end{document}