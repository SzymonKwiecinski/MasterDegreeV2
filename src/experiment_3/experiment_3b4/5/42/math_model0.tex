\documentclass{article}
\usepackage{amsmath}
\begin{document}

\section*{Mathematical Model}

\subsection*{Decision Variables}
Let \( x_{i,j} \) be the number of containers transported from depot \( i \) to port \( j \).

\subsection*{Objective Function}
Minimize the total transportation cost:
\[
\text{Minimize} \quad Z = \sum_{i=1}^{I} \sum_{j=1}^{J} \left( \frac{x_{i,j}}{2} \right) \cdot \text{price} \cdot \text{distance}_{i,j}
\]
Note that \( \frac{x_{i,j}}{2} \) represents the number of barges since each barge carries two containers.

\subsection*{Constraints}

1. Supply constraints: The number of containers sent from each depot \( i \) cannot exceed the available number of containers.
\[
\sum_{j=1}^{J} x_{i,j} \leq \text{numdepot}_i, \quad \forall i = 1, \ldots, I
\]

2. Demand constraints: Each port \( j \) must receive the required number of containers.
\[
\sum_{i=1}^{I} x_{i,j} = \text{numport}_j, \quad \forall j = 1, \ldots, J
\]

3. Non-negativity constraints: The number of containers sent must be non-negative.
\[
x_{i,j} \geq 0, \quad \forall i = 1, \ldots, I, \quad \forall j = 1, \ldots, J
\]

\end{document}