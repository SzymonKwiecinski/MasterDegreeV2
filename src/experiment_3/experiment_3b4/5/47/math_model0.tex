\documentclass{article}
\usepackage{amsmath}
\begin{document}

\section*{Mathematical Model}

\subsection*{Sets and Indices}
\begin{itemize}
    \item \( S \): Total number of shifts.
    \item \( s \): Index for shifts, \( s = 1, 2, \ldots, S \).
\end{itemize}

\subsection*{Parameters}
\begin{itemize}
    \item \( \text{officers\_needed}_s \): Number of officers needed for shift \( s \).
    \item \( \text{shift\_cost}_s \): Cost for starting officers at shift \( s \).
\end{itemize}

\subsection*{Decision Variables}
\begin{itemize}
    \item \( x_s \): Number of officers starting their shift at \( s \) (non-negative integers).
\end{itemize}

\subsection*{Objective Function}
Minimize the total cost of assigning officers:
\[
\text{Minimize} \quad \sum_{s=1}^{S} \text{shift\_cost}_s \cdot x_s
\]

\subsection*{Constraints}

\begin{itemize}
    \item \textbf{Coverage Constraints:}
    \begin{align*}
        x_1 + x_S &\geq \text{officers\_needed}_1 \\
        x_2 + x_1 &\geq \text{officers\_needed}_2 \\
        &\vdots \\
        x_S + x_{S-1} &\geq \text{officers\_needed}_S \\
    \end{align*}
    Each constraint ensures that the number of officers covering a shift meets the required number.

    \item \textbf{Non-negativity Constraints:}
    \[
    x_s \geq 0 \quad \forall s = 1, 2, \ldots, S
    \]
\end{itemize}

\subsection*{Variables are Integers}
Since the number of officers is discrete, \( x_s \) should be integer values. However, if this is an LP relaxation, integer constraints might be omitted.

\end{document}