\documentclass{article}
\usepackage{amsmath}
\begin{document}

\section*{Mathematical Model}

\subsection*{Indices:}
\begin{align*}
s & : \text{Index for shifts, } s = 1, 2, \ldots, S
\end{align*}

\subsection*{Parameters:}
\begin{align*}
\text{officers\_needed}_s & : \text{Number of officers needed for shift } s \\
\text{shift\_cost}_s & : \text{Cost of starting a shift at } s
\end{align*}

\subsection*{Decision Variables:}
\begin{align*}
x_s & : \text{Number of officers starting at shift } s \quad (x_s \geq 0 \text{ and integer})
\end{align*}

\subsection*{Objective:}
Minimize the total cost:
\begin{align*}
\text{Minimize } & \sum_{s=1}^{S} \text{shift\_cost}_s \cdot x_s
\end{align*}

\subsection*{Constraints:}
Each shift has to be covered by the required number of officers:
\begin{align*}
x_s + x_{s-1} & \geq \text{officers\_needed}_s, \quad \forall s = 1, 2, \ldots, S
\end{align*}

Note: The indices are cyclic, meaning:
\begin{align*}
x_{0} & \equiv x_S \quad (\text{Officer on last shift rolls over to first shift})
\end{align*}

\end{document}