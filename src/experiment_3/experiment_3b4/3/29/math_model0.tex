\documentclass{article}
\usepackage{amsmath}
\begin{document}

\textbf{Objective:}

\[
\min \delta
\]

\textbf{Subject to constraints:}

For each data point \( k = 1, 2, \ldots, K \):

\[
y_k - (bx_k + a) \leq \delta
\]

\[
(bx_k + a) - y_k \leq \delta
\]

These constraints represent that the absolute deviation between the observed value \( y_k \) and the predicted value \( bx_k + a \) is less than or equal to \( \delta \).

This is a linear programming problem with the decision variables:
- \( a \): Intercept of the fitted line.
- \( b \): Slope of the fitted line.
- \( \delta \): Maximum deviation.

\textbf{Solution:}

By solving this linear programming problem, we will obtain the values for \( a \) (intercept) and \( b \) (slope), which will be used to define the best-fit line. The objective value will give the minimized maximum deviation.

\end{document}