\documentclass{article}
\usepackage{amsmath}
\begin{document}

To find the Chebyshev center of the set \(\var{P}\), we need to formulate a linear programming problem. The Chebyshev center is the center of the largest ball that can be inscribed within the set \(\var{P}\). Let's define the problem mathematically.

Given:
\[
\var{P} = \{\var{x} \in \mathbb{R}^N \mid \var{a_i}^T \var{x} \leq \var{b_i}, \, i = 1, \ldots, m\}
\]
where \(\var{a_i} \in \mathbb{R}^N\) and \(\var{b_i} \in \mathbb{R}\).

We seek to find the center \(\var{y} \in \mathbb{R}^N\) and the largest possible radius \(r\) such that a ball centered at \(\var{y}\) with radius \(r\) is completely contained in \(\var{P}\).

The LP formulation is as follows:

Minimize the negative radius, \(-r\), (or equivalently, maximize \(r\)):
\[
\max \, r
\]

Subject to:
\[
\var{a_i}^T \var{y} + \| \var{a_i} \|_2 \cdot r \leq \var{b_i}, \quad i = 1, \ldots, m
\]

Here, \(\| \var{a_i} \|_2\) is the Euclidean norm of \(\var{a_i}\), \(\| \var{a_i} \|_2 = \sqrt{\sum_{j=1}^{N} (a_{i,j})^2}\).

The constraint ensures that the ball of radius \(r\), centered at \(\var{y}\), does not exceed the boundary defined by each linear constraint. The objective function maximizes the radius \(r\).

To implement, solve this linear programming problem using a suitable solver which handles such constraints.

\end{document}