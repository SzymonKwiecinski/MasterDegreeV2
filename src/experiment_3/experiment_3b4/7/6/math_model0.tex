\documentclass{article}
\usepackage{amsmath}
\begin{document}

\section*{Rocket Optimization Problem}

\subsection*{Problem Formulation}

Given:
\begin{align*}
    & x_0 \quad \text{(initial position)} \\
    & v_0 \quad \text{(initial velocity)} \\
    & x_T \quad \text{(target position at time } T\text{)} \\
    & v_T \quad \text{(target velocity at time } T\text{)} \\
    & T \quad \text{(final time)}
\end{align*}

We are tasked with determining the sequence of accelerations \( a_t \) that minimizes the total fuel consumption:

\[
\min \sum_{t=0}^{T-1} |a_t|
\]

Subject to the discrete-time dynamics:
\begin{align*}
    x_{t+1} &= x_t + v_t \quad \forall t = 0, 1, \ldots, T-1 \\
    v_{t+1} &= v_t + a_t \quad \forall t = 0, 1, \ldots, T-1
\end{align*}

And the boundary conditions:
\begin{align*}
    x_0 &= \text{given} \\
    v_0 &= \text{given} \\
    x_T &= \text{given} \\
    v_T &= \text{given}
\end{align*}

To incorporate the absolute value into the linear program, we represent each \( |a_t| \) using an auxiliary variable \( u_t \):

\begin{align*}
    a_t &\leq u_t \quad \forall t = 0, 1, \ldots, T-1 \\
    -a_t &\leq u_t \quad \forall t = 0, 1, \ldots, T-1
\end{align*}

Thus, the objective function becomes:
\[
\min \sum_{t=0}^{T-1} u_t
\]

\subsection*{Solution Format}

The output will be represented as:
\begin{itemize}
    \item \texttt{x}: List of positions at each time step from 0 to \( T \).
    \item \texttt{v}: List of velocities at each time step from 0 to \( T \).
    \item \texttt{a}: List of accelerations at each time step from 0 to \( T \).
    \item \texttt{fuel\_spend}: Total fuel spent.
\end{itemize}

\end{document}