\documentclass{article}
\usepackage{amsmath}
\begin{document}

\section*{Mathematical Model}

\textbf{Sets:}
\begin{align*}
  & T = \{1, 2, \ldots, N\} \quad \text{(set of translators)} \\
  & L = \{1, 2, \ldots, M\} \quad \text{(set of required languages)}
\end{align*}

\textbf{Parameters:}
\begin{align*}
  & \text{cost}_i \quad \text{Cost of hiring translator } i, \quad \forall i \in T \\
  & A_{ij} = 
  \begin{cases} 
    1, & \text{if translator } i \text{ can translate language } j \\ 
    0, & \text{otherwise}
  \end{cases}, \quad \forall i \in T, j \in L
\end{align*}

\textbf{Decision Variables:}
\begin{align*}
  & x_i \in \{0, 1\}, \quad \text{Binary variable indicating whether translator } i \text{ is selected}, \quad \forall i \in T
\end{align*}

\textbf{Objective Function:}
\begin{align*}
  & \min \sum_{i \in T} \text{cost}_i \cdot x_i
\end{align*}

\textbf{Constraints:}
\begin{align*}
  & \sum_{i \in T} A_{ij} \cdot x_i \geq 1, \quad \forall j \in L \quad \text{(ensure each required language is covered)}
\end{align*}

The above formulation ensures that we pick the minimum cost set of translators such that all required languages are covered.

\end{document}