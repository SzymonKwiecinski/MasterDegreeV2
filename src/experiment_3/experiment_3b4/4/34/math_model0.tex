\documentclass{article}
\usepackage{amsmath}
\begin{document}

\section*{Linear Programming Model for Minimizing Food Cost}

\textbf{Objective:}
Minimize the total purchase cost of foods:

\[
\min \sum_{k=1}^{K} \text{price}_k \cdot \text{quantity}_k
\]

\textbf{Subject to:}

For each nutrient \( m \), ensure that the nutritional demand is met:

\[
\sum_{k=1}^{K} \text{nutri}_{k, m} \cdot \text{quantity}_k \geq \text{demand}_m \quad \text{for } m = 1, \ldots, M
\]

\textbf{Decision Variables:}

\[
\text{quantity}_k \geq 0 \quad \text{for } k = 1, \ldots, K
\]

In this model:
\begin{itemize}
    \item \(\text{price}_k\) is the cost of one unit of food \( k \).
    \item \(\text{demand}_m\) is the required amount of nutrient \( m \).
    \item \(\text{nutri}_{k, m}\) is the amount of nutrient \( m \) in one unit of food \( k \).
    \item \(\text{quantity}_k\) is the amount of food \( k \) to purchase.
\end{itemize}

This linear programming model can be solved using optimization software to determine the optimal quantities of each food to purchase, thus minimizing the total cost while meeting all nutritional demands.

\end{document}