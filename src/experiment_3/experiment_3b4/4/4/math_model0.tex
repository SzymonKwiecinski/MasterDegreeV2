\documentclass{article}
\usepackage{amsmath}
\begin{document}

\section*{Mathematical Model}

\subsection*{Parameters}
\begin{itemize}
    \item \( \text{period} \): Number of consecutive days a nurse works.
    \item \( d_j \): Demand for nurses on day \( j \), \( j = 1, \dots, 7 \).
\end{itemize}

\subsection*{Variables}
\begin{itemize}
    \item \( \text{start}_j \): Number of nurses starting their shift on day \( j \), \( j = 1, \dots, 7 \).
\end{itemize}

\subsection*{Objective}
Minimize the total number of nurses hired:
\[
\text{Minimize} \quad \sum_{j=1}^{7} \text{start}_j
\]

\subsection*{Constraints}
For each day \( j \), ensure the demand is met:
\[
\begin{aligned}
    n_j \geq d_j \quad & \forall j = 1, \dots, 7
\end{aligned}
\]

Where \( n_j \) is calculated as:
\[
n_j = \sum_{i=0}^{\text{period}-1} \text{start}_{((j-i-1) \mod 7) + 1}
\]

This expression accounts for the cyclic nature of the week, ensuring that for day \( j \), we consider nurses starting from the previous \text{period} days.

\subsection*{Integer Constraint}
\[
\text{start}_j \in \mathbb{Z}^+, \quad \forall j = 1, \dots, 7
\]

\end{document}