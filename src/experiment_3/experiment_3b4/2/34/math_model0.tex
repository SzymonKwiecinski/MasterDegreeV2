\documentclass{article}
\usepackage{amsmath}

\begin{document}

\section*{Diet Optimization Problem}

\subsection*{Objective Function}
We aim to minimize the total cost of purchasing the foods. The objective function can be formulated as:

\[
\text{Minimize } Z = \sum_{k=1}^{K} \text{price}_k \cdot \text{quantity}_k
\]

where \(\text{price}_k\) is the price per unit of food \(k\) and \(\text{quantity}_k\) is the quantity of food \(k\) to purchase.

\subsection*{Constraints}
The dietary constraints ensure that each nutritional requirement is met:

\[
\sum_{k=1}^{K} \text{nutri}_{k, m} \cdot \text{quantity}_k \geq \text{demand}_m, \quad \forall m = 1, \ldots, M
\]

where \(\text{nutri}_{k, m}\) is the amount of nutrient \(m\) in food \(k\) and \(\text{demand}_m\) is the required amount of nutrient \(m\).

\subsection*{Non-Negativity Constraints}
All food quantities must be non-negative:

\[
\text{quantity}_k \geq 0, \quad \forall k = 1, \ldots, K
\]

\subsection*{Solution Representation}
The solution will provide the optimal quantity of each food to purchase in order to minimize cost while meeting all nutritional requirements:

\[
\text{Output: } \text{quantity} = [\text{quantity}_1, \ldots, \text{quantity}_K]
\]

\end{document}