\documentclass{article}
\usepackage{amsmath}

\begin{document}

\section*{MILP Formulation for Nurse Scheduling}

\subsection*{Parameters}
\begin{itemize}
    \item $period$: Number of consecutive days a nurse works before taking days off.
    \item $d_j$: Demand for nurses on day $j$, for $j = 1, \ldots, 7$.
\end{itemize}

\subsection*{Decision Variables}
\begin{itemize}
    \item $start_j$: The number of nurses starting their working period on day $j$, for $j = 1, \ldots, 7$.
    \item $total$: Total number of nurses hired.
\end{itemize}

\subsection*{Objective Function}
Minimize the total number of nurses hired:
\[
\text{Minimize } total = \sum_{j=1}^{7} start_j
\]

\subsection*{Constraints}
For each day $j = 1, \ldots, 7$, the constraint ensuring sufficient nurses are working is:
\[
\sum_{k=0}^{period-1} start_{(j-k-1) \bmod 7 + 1} \geq d_j
\]
where $(j-k-1) \bmod 7 + 1$ ensures cyclic counting of days from 1 to 7.

\subsection*{Additional Constraints}
All decision variables must take non-negative integer values:
\[
start_j \geq 0, \quad \text{integer, for } j = 1, \ldots, 7
\]

\subsection*{Solution Output}
The solution will provide:
\begin{itemize}
    \item $start = [start_1, \ldots, start_7]$: Number of nurses starting on each day.
    \item $total$: Minimum number of nurses hired.
\end{itemize}

\end{document}