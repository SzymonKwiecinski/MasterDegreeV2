\documentclass{article}
\usepackage{amsmath}
\begin{document}

\textbf{Objective:} Minimize the total number of nurses hired.

\textbf{Decision Variables:}
\begin{itemize}
    \item $x_j$: Number of nurses who start their shift on day $j$, for $j = 1, \ldots, 7$.
\end{itemize}

\textbf{Parameters:}
\begin{itemize}
    \item $d_j$: Demand for nurses on day $j$, for $j = 1, \ldots, 7$.
    \item \text{period}: Number of consecutive days a nurse works.
\end{itemize}

\textbf{Constraints:}
\begin{align}
    &\sum_{i=0}^{\text{period}-1} x_{(j-i-1) \bmod 7 + 1} \geq d_j, \quad \forall j = 1, \ldots, 7 \\
    &x_j \geq 0 \quad \text{and integer}, \quad \forall j = 1, \ldots, 7
\end{align}

\textbf{Objective Function:}
\begin{align}
    \min \sum_{j=1}^{7} x_j
\end{align}

\textbf{Explanation:} 
\begin{itemize}
    \item Constraint (1) ensures that the total number of nurses working on any given day $j$ meets the demand $d_j$. The summation $\sum_{i=0}^{\text{period}-1} x_{(j-i-1) \bmod 7 + 1}$ represents the number of nurses available each day, accounting for the period they work starting from day $j-i$.
    \item Constraint (2) ensures that the number of nurses starting each day is non-negative and integer.
    \item The objective (3) is to minimize the total number of nurses hired, which is the sum of all nurses starting their shifts on each day.
\end{itemize}

\end{document}