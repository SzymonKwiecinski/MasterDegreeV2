\documentclass{article}
\usepackage{amsmath}
\begin{document}

\section*{Linear Programming Model for Rocket Fuel Minimization}

Given the problem, we want to minimize the total fuel consumption over the time period from $t = 0$ to $t = T-1$. 

Let:
- $x_t$ represent the position of the rocket at time $t$,
- $v_t$ represent the velocity of the rocket at time $t$,
- $a_t$ represent the acceleration of the rocket at time $t$.

The discrete-time equations of motion are given by:
\begin{align}
    x_{t+1} &= x_t + v_t \quad \forall t = 0, 1, \dots, T-1, \\
    v_{t+1} &= v_t + a_t \quad \forall t = 0, 1, \dots, T-1.
\end{align}

The objective is to minimize the total fuel consumption, which is proportional to the sum of the absolute values of the accelerations:
\[
\text{Minimize } \sum_{t=0}^{T-1} |a_t|
\]

Subject to the constraints:
\begin{align}
    x_0 &= \text{given initial position}, \\
    v_0 &= \text{given initial velocity}, \\
    x_T &= \text{target position}, \\
    v_T &= \text{target velocity}.
\end{align}

Additionally, we can linearize the absolute value function by introducing auxiliary variables $a^+_t$ and $a^-_t$ (where $a_t = a^+_t - a^-_t$ and $|a_t| = a^+_t + a^-_t$):
\begin{align}
    a_t &= a^+_t - a^-_t \quad \forall t = 0, 1, \dots, T-1, \\
    a^+_t, a^-_t &\geq 0 \quad \forall t = 0, 1, \dots, T-1.
\end{align}

The linear programming problem in standard form becomes:
\begin{align}
    \text{Minimize } & \sum_{t=0}^{T-1} (a^+_t + a^-_t) \\
    \text{Subject to:} \\
    x_{t+1} &= x_t + v_t \quad \forall t = 0, 1, \dots, T-1, \\
    v_{t+1} &= v_t + a^+_t - a^-_t \quad \forall t = 0, 1, \dots, T-1, \\
    x_0 &= \text{given initial position}, \\
    v_0 &= \text{given initial velocity}, \\
    x_T &= \text{target position}, \\
    v_T &= \text{target velocity}, \\
    a^+_t, a^-_t &\geq 0 \quad \forall t = 0, 1, \dots, T-1.
\end{align}

This formulation captures the essence of the problem constraints and objectives.

\end{document}