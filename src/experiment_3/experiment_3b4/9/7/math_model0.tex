\documentclass{article}
\usepackage{amsmath}
\begin{document}

\section*{Linear Programming Model for Rocket Trajectory Optimization}

\subsection*{Objective}
Minimize the maximum thrust required, which is given by:

\[
\text{minimize} \quad M = \max_{t} |a_t|
\]

\subsection*{Constraints}

1. **Position and Velocity Constraints**:
   \begin{align*}
   x_{t+1} &= x_t + v_t & \forall t = 0, \ldots, T-1 \\
   v_{t+1} &= v_t + a_t & \forall t = 0, \ldots, T-1
   \end{align*}

2. **Initial Conditions**:
   \begin{align*}
   x_0 &= \text{given initial position (input)} \\
   v_0 &= \text{given initial velocity (input)}
   \end{align*}

3. **Target Conditions**:
   \begin{align*}
   x_T &= \text{target position (input)} \\
   v_T &= \text{target velocity (input)}
   \end{align*}

4. **Thrust (Acceleration) Constraints**:
   The magnitude of each acceleration must be bounded by the maximum thrust:
   \[
   -M \leq a_t \leq M \quad \forall t = 0, \ldots, T-1
   \]

5. **Fuel Usage**:
   Total fuel spend can be represented as the sum of absolute values of accelerations:
   \[
   \text{Fuel Spend} = \sum_{t=0}^{T-1} |a_t|
   \]

\subsection*{Decision Variables}

- \(x_t\): Position of the rocket at time \(t\).
- \(v_t\): Velocity of the rocket at time \(t\).
- \(a_t\): Acceleration of the rocket at time \(t\).
- \(M\): Maximum absolute acceleration, to be minimized.

\subsection*{Model Formulation}

To solve the above problem, we'll use linear programming techniques. The main objective is to minimize \(M\), subject to the position, velocity, and boundary constraints. 

The optimization problem can be implemented and solved using any LP solver by defining it with the given sets of constraints and conditions.

\end{document}