\documentclass{article}
\usepackage{amsmath}
\begin{document}

\textbf{Mathematical Model for Nurse Scheduling Problem}

\textbf{Parameters:}
\begin{itemize}
    \item \( \text{period} \): Number of consecutive days a nurse works.
    \item \( d_j \): Demand for nurses on day \( j \), for \( j = 1, \ldots, 7 \).
\end{itemize}

\textbf{Decision Variables:}
\begin{itemize}
    \item \( \text{start}_j \): Integer variable representing the number of nurses starting their shift on day \( j \), for \( j = 1, \ldots, 7 \).
\end{itemize}

\textbf{Objective:}
\[
\text{Minimize } \sum_{j=1}^{7} \text{start}_j
\]

\textbf{Constraints:}
\begin{align*}
    \sum_{i=0}^{\text{period}-1} \text{start}_{(j-i-1) \mod 7 + 1} &\geq d_j, \quad \text{for } j = 1, \ldots, 7
\end{align*}

\textbf{Cyclicity:}
\begin{itemize}
    \item Use modular arithmetic to ensure week cycle: \( \text{start}_{(j-i-1) \mod 7 + 1} \).
\end{itemize}

\end{document}