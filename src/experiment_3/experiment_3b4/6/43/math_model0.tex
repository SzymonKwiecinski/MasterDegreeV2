\documentclass{article}
\usepackage{amsmath}
\begin{document}

\section*{Linear Programming Model for Maximizing Profit}

\subsection*{Decision Variables}
Let \( x_j \) be the number of units of product \( j \) to produce, where \( j = 1, \ldots, M \).

\subsection*{Objective Function}
Maximize the total profit:
\[
\text{Total Profit} = \sum_{j=1}^{M} ( \text{price}_j - \text{cost}_j ) \times x_j
\]

\subsection*{Constraints}

1. Material Constraints: For each raw material \( i = 1, \ldots, N \),
   \[
   \sum_{j=1}^{M} \text{req}_{i,j} \times x_j \leq \text{available}_i
   \]

2. Demand Constraints: For each product \( j = 1, \ldots, M \),
   \[
   x_j \leq \text{demand}_j
   \]

3. Non-negativity Constraints:
   \[
   x_j \geq 0 \quad \text{for all } j = 1, \ldots, M
   \]

\subsection*{Constraints Description}
- The material constraints ensure that the production does not use more raw materials than are available.
- The demand constraints ensure that the production does not exceed the forecasted maximum demand for each product.
- The non-negativity constraints ensure that negative production quantities are not feasible.

\end{document}