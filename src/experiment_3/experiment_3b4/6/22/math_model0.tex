\documentclass{article}
\usepackage{amsmath}
\begin{document}

\section*{Mathematical Model for Manpower Planning}

We are given a situation where a company needs to plan its manpower requirements over multiple years, taking into consideration recruitment, retraining, redundancy, and short-time working. To address this problem, we will formulate a Linear Programming (LP) model.

\subsection*{Indices and Parameters}

\begin{align*}
    & k \quad \text{index for manpower categories (e.g., unskilled, semi-skilled, skilled)}, \\
    & i \quad \text{index for years (e.g., 1, 2, \ldots, I)}, \\
    & \text{requirement}_{k, i} \quad \text{manpower requirement of category } k \text{ in year } i, \\
    & \text{strength}_{k} \quad \text{current manpower strength in category } k, \\
    & \text{lessonewaste}_{k} \quad \text{wastage rate for category } k \text{ with less than one year of service}, \\
    & \text{moreonewaste}_{k} \quad \text{wastage rate for category } k \text{ with more than one year of service}, \\
    & \text{recruit}_{k} \quad \text{maximum number of recruits allowed for category } k \text{ per year}, \\
    & \text{costredundancy}_{k} \quad \text{redundancy cost for category } k, \\
    & \text{num\_overman} \quad \text{maximum allowable overmanning across all categories}, \\
    & \text{costoverman}_{k} \quad \text{cost of overmanning for category } k, \\
    & \text{num\_shortwork} \quad \text{maximum number of short-time workers allowed per category per year}, \\
    & \text{costshort}_{k} \quad \text{cost for short-time working for category } k.
\end{align*}

\subsection*{Decision Variables}

\begin{align*}
    & \text{recruit}_{k, i} \quad \text{number of recruits in category } k \text{ in year } i, \\
    & \text{overmanning}_{k, i} \quad \text{number of overmanning workers in category } k \text{ in year } i, \\
    & \text{short}_{k, i} \quad \text{number of short-time working employees in category } k \text{ in year } i.
\end{align*}

\subsection*{Objective Function}

The objective is to minimize redundancy throughout the planning period:

\[
\text{Minimize} \quad \sum_{k} \sum_{i} \text{redundancy}_{k, i}
\]

where

\[
\text{redundancy}_{k, i} = \max\left(0, \text{strength}_{k} - \text{requirement}_{k, i} + \text{overmanning}_{k, i} - \text{short}_{k, i} \right)
\]

\subsection*{Constraints}

1. **Workforce Balance:**
   \[
   \text{strength}_{k, i+1} = \underbrace{\text{strength}_{k, i} \times (1 - \text{moreonewaste}_{k})}_{\text{existing\ workforce}} + \text{recruit}_{k, i} \times (1 - \text{lessonewaste}_{k}) - \text{redundancy}_{k, i} 
   \]

2. **Recruitment Limits:**
   \[
   0 \leq \text{recruit}_{k, i} \leq \text{recruit}_{k}
   \]

3. **Overmanning Limits:**
   \[
   0 \leq \sum_{k} \text{overmanning}_{k, i} \leq \text{num\_overman}
   \]

4. **Short-time Work Limit:**
   \[
   0 \leq \text{short}_{k, i} \leq \text{num\_shortwork}
   \]

5. **Production Requirement:**
   \[
   \text{requirement}_{k, i} \leq \text{strength}_{k, i} + \text{overmanning}_{k, i} - 0.5 \times \text{short}_{k, i}
   \]

6. **Non-negativity:**
   \[
   \text{recruit}_{k, i}, \text{overmanning}_{k, i}, \text{short}_{k, i}, \text{redundancy}_{k, i} \geq 0
   \]

\end{document}