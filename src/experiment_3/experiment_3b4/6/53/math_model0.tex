\documentclass{article}
\usepackage{amsmath}
\begin{document}

\section*{Mathematical Model for the Soybean Transportation Problem}

\subsection*{Sets and Indices}
- Let \( n \) be the number of cities involved in the transportation network.
- Let \( A \) be the set of all routes available in the network.
- Index \( i \) denotes the start of a route and \( j \) denotes the end of a route.

\subsection*{Parameters}
- \( C_{i,j} \): Cost of transportation per metric ton from city \( i \) to city \( j \).
- \( supply_k \): Supply of soybeans available at terminal city \( k \).
- \( demand_l \): Demand for soybeans at destination city \( l \).

\subsection*{Decision Variables}
- \( amount_{i,j} \): Amount of soybeans to be transported from city \( i \) to city \( j \).

\subsection*{Objective Function}
The objective is to minimize the total transportation cost:
\[
\text{Minimize} \quad \sum_{(i,j) \in A} C_{i,j} \times amount_{i,j}
\]

\subsection*{Constraints}
1. \textbf{Supply Constraints:} 
   For each source terminal \( k \):
   \[
   \sum_{j: (k,j) \in A} amount_{k,j} \leq supply_k
   \]

2. \textbf{Demand Constraints:} 
   For each destination city \( l \):
   \[
   \sum_{i: (i,l) \in A} amount_{i,l} \geq demand_l
   \]

3. \textbf{Flow Conservation Constraints:} 
   For each intermediate city \( p \), ensure that the inflow equals the outflow:
   \[
   \sum_{i: (i,p) \in A} amount_{i,p} = \sum_{j: (p,j) \in A} amount_{p,j}
   \]

4. \textbf{Non-Negativity Constraints:}
   \[
   amount_{i,j} \geq 0 \quad \forall (i,j) \in A
   \]

\subsection*{Summary}
The above linear programming model determines the optimal transportation amount \( amount_{i,j} \) for each route in order to minimize the total transportation cost while satisfying all demand and supply constraints.

\end{document}