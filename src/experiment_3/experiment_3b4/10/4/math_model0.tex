\documentclass{article}
\usepackage{amsmath}
\begin{document}

\section*{MILP Model for Nurse Scheduling}

\subsection*{Parameters}
\begin{itemize}
    \item \( \text{period} \): Number of consecutive days a nurse works.
    \item \( d_j \): Demand for nurses on day \( j \), \( j = 1, \ldots, 7 \).
\end{itemize}

\subsection*{Decision Variables}
\begin{itemize}
    \item \( \text{start}_j \): Number of nurses starting their period on day \( j \), \( j = 1, \ldots, 7 \).
\end{itemize}

\subsection*{Objective Function}
Minimize the total number of nurses hired:
\[
\min \sum_{j=1}^{7} \text{start}_j
\]

\subsection*{Constraints}
\begin{align*}
\forall j = 1,\ldots,7, \quad \sum_{i=j-\text{period}+1}^{j} \text{start}_i & \geq d_j \\
\text{start}_i & \geq 0 \quad \text{and integer} \quad \forall i = 1,\ldots,7
\end{align*}

\subsection*{Explanation}
\begin{itemize}
    \item The constraint ensures that for each day \( j \), the sum of nurses who have started within the last \( \text{period} \) days is at least the demand \( d_j \).
    \item Indices in the constraints are calculated modulo 7 to account for the weekly cycle, i.e., after day 7, the next day is day 1.
\end{itemize}

\end{document}