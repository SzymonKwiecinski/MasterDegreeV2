\documentclass{article}
\usepackage{amsmath}
\usepackage{amssymb}
\usepackage{amsfonts}
\usepackage{algorithm}
\usepackage{algorithmic}
\begin{document}

\textbf{Objective:} Minimize the total number of employees to hire.
\begin{equation}
\min \sum_{i=1}^{M} x_i
\end{equation}

\textbf{Subject to:}

1. \textbf{Demand Satisfaction:}
\begin{equation}
\sum_{i=1}^{M} is\_work_{n,i} \geq num_n, \quad \forall n \in [1, N]
\end{equation}

2. \textbf{Work and Rest Cycle:} 
\begin{align}
is\_work_{n, i} + is\_work_{n+1, i} + \ldots + is\_work_{n + n\_working\_days - 1, i} & \leq n\_working\_days \cdot x_i, \\
is\_work_{n + n\_working\_days, i} = 0, & \quad \forall i \in [1, M], \\
is\_work_{n, i} = 0, & \quad \forall i \in [1, M], \quad \forall n \in [n + n\_working\_days, N]
\end{align}

3. \textbf{Binary Constraints:}
\begin{align}
x_i & \in \{0, 1\}, \quad \forall i \in [1, M] \\
is\_work_{n, i} & \in \{0, 1\}, \quad \forall n \in [1, N], \forall i \in [1, M]
\end{align}

Here, \( M \) is an upper bound on the total number of employees, potentially the sum of all \( num_n \) or some larger number sufficient for the problem's constraints.

The solution to this MILP will give you the minimum number of employees required and their working schedule.

\end{document}