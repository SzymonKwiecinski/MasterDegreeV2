\documentclass{article}
\usepackage{amsmath}

\begin{document}

\section*{Linear Programming Model for Rocket Trajectory Optimization}

\subsection*{Objective:}
Minimize the total fuel consumption:
\[
\min \sum_{t=0}^{T-1} |a_t|
\]

\subsection*{Constraints:}

1. Initial conditions:
   \[
   x_0 = \text{given}, \quad v_0 = \text{given}
   \]

2. Motion equations for \( t = 0, 1, 2, \ldots, T-1 \):
   \[
   x_{t+1} = x_t + v_t 
   \]
   \[
   v_{t+1} = v_t + a_t
   \]

3. Final conditions:
   \[
   x_T = \text{given}, \quad v_T = \text{given}
   \]

\subsection*{Variables:}
- \( x_t \): Position of the rocket at time \( t \)
- \( v_t \): Velocity of the rocket at time \( t \)
- \( a_t \): Acceleration of the rocket at time \( t \)

\subsection*{Note:}
The absolute value constraint on the acceleration \(|a_t|\) can be linearized by introducing auxiliary variables and constraints:
Let \( a_t^+ \geq 0 \) and \( a_t^- \geq 0 \).

Then, express \( a_t = a_t^+ - a_t^- \) and add the constraint:
\[
a_t^+ + a_t^- = |a_t|
\]

The new objective becomes:
\[
\min \sum_{t=0}^{T-1} (a_t^+ + a_t^-)
\]

The LP model now includes these auxiliary variables for each time step to handle the absolute value constraint linearly.

\end{document}