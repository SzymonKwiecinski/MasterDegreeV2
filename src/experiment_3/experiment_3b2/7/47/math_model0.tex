\documentclass{article}
\usepackage{amsmath}
\begin{document}

\section*{Linear Programming Model}

\subsection*{Parameters}
\begin{itemize}
    \item \( S \): Number of shifts.
    \item \( \text{officers\_needed}_s \): Number of officers needed for shift \( s \), for \( s = 1, \ldots, S \).
    \item \( \text{shift\_cost}_s \): Cost of assigning officers to start at shift \( s \), for \( s = 1, \ldots, S \).
\end{itemize}

\subsection*{Decision Variables}
\begin{itemize}
    \item \( x_s \): Number of officers starting their two-shift work at shift \( s \).
\end{itemize}

\subsection*{Objective Function}
Minimize the total cost of assigning officers:
\[
\text{Minimize} \quad Z = \sum_{s=1}^{S} \text{shift\_cost}_s \cdot x_s
\]

\subsection*{Constraints}
\begin{itemize}
    \item Demand constraints for each shift:
    \[
    \begin{aligned}
        x_s + x_{s-1} & \geq \text{officers\_needed}_s, & \quad s = 1, \\
        x_2 + x_1 & \geq \text{officers\_needed}_2, \\
        & \vdots \\
        x_S + x_{S-1} & \geq \text{officers\_needed}_S, \\
        x_1 + x_S & \geq \text{officers\_needed}_1. 
    \end{aligned}
    \]
    (Assuming the schedule wraps around circularly, i.e., after shift \( S \), it returns to shift 1)
    
    \item Non-negativity constraints:
    \[
    x_s \geq 0, \quad s = 1, \ldots, S.
    \]
\end{itemize}

\end{document}