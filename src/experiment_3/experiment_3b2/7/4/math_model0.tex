\documentclass{article}
\usepackage{amsmath}
\begin{document}

\textbf{MILP Model for Nurse Scheduling}

\textbf{Parameters:}
\begin{align*}
\text{period} & : \text{ Number of consecutive days a nurse works on night shifts.} \\
d_j & : \text{ Demand for nurses on night shift for day } j \in \{1, \ldots, 7\}.
\end{align*}

\textbf{Decision Variables:}
\begin{align*}
\text{Let } start_j & \text{ be the number of nurses starting their period on day } j \in \{1, \ldots, 7\}. \\
\text{Let } total & \text{ be the total number of nurses hired.}
\end{align*}

\textbf{Objective:}
\[
\min \; total
\]

\textbf{Constraints:}
\begin{align*}
\text{Total Nurses Hired: } \\
total & = \sum_{j=1}^{7} start_j \\

\text{Demand Satisfaction: } \\
\sum_{k=0}^{\text{period}-1} start_{(j-k-1) \mod 7 + 1} & \geq d_j \quad \forall j \in \{1, \ldots, 7\} \\

\text{Non-negativity and Integer Constraints: } \\
start_j & \geq 0 \quad \text{and integer} \quad \forall j \in \{1, \ldots, 7\}
\end{align*}

\textbf{Explanation:}
\begin{itemize}
    \item The objective is to minimize the total number of nurses hired.
    \item The constraint ensures that for each day \( j \), the number of nurses currently on shift meets the demand \( d_j \). 
    \item Nurses work for a consecutive number of \text{period} days, hence for each day \( j \), the supply of nurses comes from those who started their shift up to \text{period} days prior.
    \item The modulo operation accommodates the weekly cycle, ensuring that day indices wrap around from 7 back to 1.
\end{itemize}

\end{document}