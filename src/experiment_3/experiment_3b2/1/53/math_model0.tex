\documentclass{article}
\usepackage{amsmath}
\begin{document}

\section*{Mathematical Model}

\textbf{Decision Variables:}

\[
x_{i,j} = \text{Amount of soybeans transported from node } i \text{ to node } j
\]

\textbf{Objective Function:}

Minimize the total cost of transportation:

\[
\text{Minimize } Z = \sum_{(i,j) \in A} C_{i,j} \cdot x_{i,j}
\]

\textbf{Constraints:}

1. Supply constraints: The total amount shipped from each terminal \(k\) should not exceed the supply at that terminal.

\[
\sum_{j : (k,j) \in A} x_{k,j} \leq \text{supply}_k, \quad \forall k
\]

2. Demand constraints: The total amount received at each destination \(l\) should meet its demand.

\[
\sum_{i : (i,l) \in A} x_{i,l} \geq \text{demand}_l, \quad \forall l
\]

3. Flow balance for intermediate nodes (ports): The total inflow should equal the total outflow.

\[
\sum_{i : (i,p) \in A} x_{i,p} = \sum_{j : (p,j) \in A} x_{p,j}, \quad \forall \text{port } p
\]

4. Non-negativity constraints: All transported amounts should be non-negative.

\[
x_{i,j} \geq 0, \quad \forall (i,j) \in A
\]

\end{document}