\documentclass{article}
\usepackage{amsmath}
\begin{document}

\section*{Linear Programming Model for Maximizing Revenue at an Oil Refinery}

\subsection*{Decision Variables}
Let \( x_l \) be the number of times process \( l \) is executed, for \( l = 1, 2, \ldots, L \).

\subsection*{Objective Function}
Maximize the total revenue:

\[
\text{Maximize } Z = \sum_{p=1}^{P} \left( \left( \sum_{l=1}^{L} \text{output}_{l,p} \cdot x_l \right) \cdot \text{price}_p \right) - \sum_{l=1}^{L} \left( \text{cost}_l \cdot \left( \sum_{p=1}^{P} \text{output}_{l,p} \cdot x_l \right) \right)
\]

\subsection*{Constraints}

1. Crude oil constraints: For each crude oil type \( i \):

\[
\sum_{l=1}^{L} \text{input}_{l,i} \cdot x_l \leq \text{allocated}_i, \quad \forall i = 1, \ldots, O
\]

2. Non-negativity constraints:

\[
x_l \geq 0, \quad \forall l = 1, \ldots, L
\]

\end{document}