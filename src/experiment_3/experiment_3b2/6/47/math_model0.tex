\documentclass{article}
\usepackage{amsmath}
\begin{document}

\section*{Linear Programming Model for Police Shift Scheduling}

\subsection*{Sets and Indices}
\begin{itemize}
    \item \( S \): Number of shifts in the day.
    \item \( s \): Index for shifts, \( s = 1, 2, \ldots, S \).
\end{itemize}

\subsection*{Parameters}
\begin{itemize}
    \item \(\text{officers\_needed}_s\): Number of officers needed for shift \( s \).
    \item \(\text{shift\_cost}_s\): Cost of starting a shift at \( s \).
\end{itemize}

\subsection*{Decision Variables}
\begin{itemize}
    \item \( x_s \): Number of officers starting their shift at \( s \), \( x_s \geq 0 \).
\end{itemize}

\subsection*{Objective Function}
Minimize the total cost:
\[
\text{Minimize } \sum_{s=1}^{S} \text{shift\_cost}_s \times x_s
\]

\subsection*{Constraints}
\begin{align*}
    & \text{For each shift } s: \\
    & x_{s} + x_{s-1} \geq \text{officers\_needed}_s, \quad \text{where } x_{0} = x_{S} \\
    & x_s \geq 0, \quad \forall s \in \{1, 2, \ldots, S\}
\end{align*}

\end{document}