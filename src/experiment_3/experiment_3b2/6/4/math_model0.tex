\documentclass{article}
\usepackage{amsmath}
\begin{document}

\section*{MILP Formulation for Nurse Scheduling Problem}

\subsection*{Indices}
\begin{itemize}
    \item $j \in \{1, 2, \ldots, 7\}$: Day of the week
\end{itemize}

\subsection*{Parameters}
\begin{itemize}
    \item $d_j$: Demand for nurses on day $j$ during the night shift
    \item \text{period}: Number of consecutive days a nurse works followed by $(7 - \text{period})$ days off
\end{itemize}

\subsection*{Decision Variables}
\begin{itemize}
    \item $x_j$: Number of nurses who start their shift period on day $j$, $x_j \in \mathbb{Z}_{\geq 0}$ for $j = 1, 2, \ldots, 7$
    \item $N$: Total number of nurses hired
\end{itemize}

\subsection*{Objective}
Minimize the total number of nurses hired:
\[
\min N
\]

\subsection*{Constraints}
\begin{itemize}
    \item Link the total number of nurses with the starting schedule:
    \[
    N = \sum_{j=1}^{7} x_j
    \]
    \item Meet the demand on each day:
    \[
    \sum_{k=0}^{\text{period} - 1} x_{(j-k-1) \mod 7 + 1} \geq d_j, \quad \text{for } j = 1, 2, \ldots, 7
    \]
    The expression $(j-k-1) \mod 7 + 1$ is used to handle cyclic indexing for days.
\end{itemize}

\subsection*{Explanation}
Each nurse works for a continuous period of "period" days and is off for the remaining days of the week. The constraint ensures that for each day, the cumulative number of nurses who start within the last "period" days meets the demand for that day. The objective function seeks to minimize the total number of nurses hired, which corresponds to minimizing the sum of nurses starting their period on each day.

\end{document}