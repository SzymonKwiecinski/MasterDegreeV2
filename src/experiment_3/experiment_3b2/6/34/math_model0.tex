\documentclass{article}
\usepackage{amsmath}

\begin{document}

\section*{Mathematical Model for Minimizing the Cost of a Balanced Diet}

\subsection*{Sets and Indices}
\begin{itemize}
    \item \( k = 1, \ldots, K \): Index for types of food
    \item \( m = 1, \ldots, M \): Index for types of nutrients
\end{itemize}

\subsection*{Parameters}
\begin{itemize}
    \item \(\text{price}_k\): Price per unit of food \( k \)
    \item \(\text{demand}_m\): Required units of nutrient \( m \)
    \item \(\text{nutri}_{k,m}\): Units of nutrient \( m \) in food \( k \)
\end{itemize}

\subsection*{Decision Variables}
\begin{itemize}
    \item \( x_k \): Quantity of food \( k \) to purchase
\end{itemize}

\subsection*{Objective Function}
Minimize the total cost of purchasing the foods:
\[
\text{Minimize } \sum_{k=1}^{K} \text{price}_k \cdot x_k
\]

\subsection*{Constraints}
\begin{itemize}
    \item Nutritional Requirements:
    \[
    \sum_{k=1}^{K} \text{nutri}_{k, m} \cdot x_k \geq \text{demand}_m \quad \text{for } m = 1, \ldots, M
    \]
    \item Non-negativity Constraints:
    \[
    x_k \geq 0 \quad \text{for } k = 1, \ldots, K
    \]
\end{itemize}

\end{document}