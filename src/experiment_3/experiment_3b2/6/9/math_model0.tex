\documentclass{article}
\usepackage{amsmath}
\begin{document}

\section*{Linear Programming Model for Optimal Lamp Power}

Let \( N \) be the number of road segments and \( M \) be the number of lamps.

\subsection*{Variables}
- \( \text{power}_j \) for \( j = 1, \ldots, M \): Power of the \( j \)-th lamp.
- \( \text{error}_i \) for \( i = 1, \ldots, N \): Absolute error between the actual and desired illumination of the \( i \)-th segment.

\subsection*{Parameters}
- \( \text{coeff}_{i,j} \): Coefficient representing the effect of the \( j \)-th lamp on the \( i \)-th segment.
- \( \text{desired}_i \): Desired illumination level for the \( i \)-th segment.

\subsection*{Objective Function}
The objective is to minimize the total absolute error over all segments:

\[
\min \sum_{i=1}^{N} \text{error}_i
\]

\subsection*{Constraints}
For each road segment \( i \), the illumination is calculated as:

\[
\text{ill}_i = \sum_{j=1}^{M} \text{coeff}_{i,j} \cdot \text{power}_j
\]

We define the absolute error with the following constraints:

\[
\text{ill}_i - \text{desired}_i \leq \text{error}_i, \quad \forall i = 1, \ldots, N
\]

\[
\text{desired}_i - \text{ill}_i \leq \text{error}_i, \quad \forall i = 1, \ldots, N
\]

Where:
- \( \text{ill}_i \) is the calculated illumination of the \( i \)-th segment.
- \( \text{error}_i \) is a non-negative variable representing the absolute error.

\subsection*{Non-negativity Constraints}
- \( \text{power}_j \geq 0, \quad \forall j = 1, \ldots, M \)
- \( \text{error}_i \geq 0, \quad \forall i = 1, \ldots, N \)

\end{document}