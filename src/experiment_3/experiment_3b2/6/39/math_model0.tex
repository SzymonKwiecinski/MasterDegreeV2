\documentclass{article}
\usepackage{amsmath}
\begin{document}

\section*{Mathematical Model}

\subsection*{Decision Variables}
\begin{align*}
x_i & : \text{Binary variable indicating if employee } i \text{ is hired}, \quad \forall i = 1, \ldots, I\\
y_{n,i} & : \text{Binary variable indicating if employee } i \text{ works on day } n, \quad \forall n, \forall i
\end{align*}

\subsection*{Objective Function}
Minimize the total number of employees:
\[
\text{minimize} \quad \sum_{i=1}^{I} x_i
\]

\subsection*{Constraints}
\begin{align*}
& \text{1. Employee requirement per day:}\\
& \sum_{i=1}^{I} y_{n,i} \geq \text{num}_n, \quad \forall n\\
& \text{2. Employee working days:}\\
& y_{n,i} \leq x_i, \quad \forall n, \forall i\\
\end{align*}

\subsection*{Additional Notes}
- Each employee works for \text{n\_working\_days} followed by \text{n\_resting\_days}.
- The planning must ensure that these cycles are respected.
- The decision variables \( x_i \) and \( y_{n,i} \) are binary.

\end{document}