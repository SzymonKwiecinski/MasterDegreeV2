\documentclass{article}
\usepackage{amsmath}
\begin{document}

\section*{Linear Programming Model for Maximizing Profit}

Let \( P \) be the number of different parts, and \( M \) be the number of different machines. The decision variables are:
\[
x_p \quad \text{for } p = 1, 2, \ldots, P
\]
where \( x_p \) represents the number of batches of part \( p \) produced each month.

The objective is to maximize the total profit, which can be expressed as:
\[
\text{Maximize} \quad Z = \sum_{p=1}^{P} \left( \text{price}_p \cdot x_p \right) - \sum_{m=1}^{M} \sum_{p=1}^{P} \left( \text{cost}_m \cdot \text{time}_{m,p} \cdot x_p \right)
\]

Subject to the following constraints:

1. **Minimum Production Requirement:**
   \[
   x_p \geq \text{min\_batches}_p \quad \text{for } p = 1, 2, \ldots, P
   \]

2. **Machine Availability (for \( m = 2, \ldots, M \)):**
   \[
   \sum_{p=1}^{P} \left( \text{time}_{m,p} \cdot x_p \right) \leq \text{available}_m \quad \text{for } m = 2, \ldots, M
   \]

3. **Outsourced Machine 1 Costs:**
   \[
   \sum_{p=1}^{P} \left( \text{time}_{1,p} \cdot x_p \right) \leq \text{overtime\_hour} + \frac{\text{(Available Time for Machine 1)}}{\text{Standard Work Hour}}
   \]

   The cost for Machine 1 is calculated separately:
   - If used hours \( \leq \text{overtime\_hour} \), cost is \(\text{standard\_cost} \cdot \sum_{p=1}^{P} \left( \text{time}_{1,p} \cdot x_p \right)\).
   - If used hours \( > \text{overtime\_hour} \), additional cost is \(\text{overtime\_cost} \cdot \left( \sum_{p=1}^{P} \left( \text{time}_{1,p} \cdot x_p \right) - \text{overtime\_hour} \right)\).

4. **Non-negativity Constraints:**
   \[
   x_p \geq 0 \quad \text{for } p = 1, 2, \ldots, P
   \]

This model determines the optimal number of batches \( x_p \) of each part \( p \) that should be manufactured to maximize the profit.

\end{document}