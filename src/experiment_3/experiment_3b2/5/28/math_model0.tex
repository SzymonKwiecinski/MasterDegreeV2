\documentclass{article}
\usepackage{amsmath}
\begin{document}

\section*{Linear Programming Model for Fitting a Line}

Given the data points \((x_k, y_k)\) for \(k = 1, 2, \ldots, K\), we want to find the slope \(b\) and intercept \(a\) that minimize the sum of absolute deviations of observed values \(y_k\) from the values predicted by the line \( y = bx + a \).

\subsection*{Objective Function}
The objective is to minimize the sum of absolute deviations:
\[
\min \sum_{k=1}^{K} |y_k - (bx_k + a)|
\]

\subsection*{Transformation to Linear Programming}
Introduce auxiliary variables \(d_k^+\) and \(d_k^-\) for each deviation, such that:
\[
y_k - (bx_k + a) = d_k^+ - d_k^-
\]
where \(d_k^+ \geq 0\) and \(d_k^- \geq 0\).

Then, the absolute deviation can be expressed as \( |y_k - (bx_k + a)| = d_k^+ + d_k^- \).

The LP objective function becomes:
\[
\min \sum_{k=1}^{K} (d_k^+ + d_k^-)
\]

\subsection*{Constraints}
For each data point, the following constraints must hold:
\[
y_k - bx_k - a = d_k^+ - d_k^- 
\]
\[
d_k^+ \geq 0, \quad d_k^- \geq 0
\]

\subsection*{Variables}
- \( a \): intercept of the fitted line
- \( b \): slope of the fitted line
- \( d_k^+, d_k^- \): auxiliary variables for absolute deviations

\end{document}