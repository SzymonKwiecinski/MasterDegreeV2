\documentclass{article}
\usepackage{amsmath}
\begin{document}

\section*{Linear Programming Formulation for Currency Exchange}

\subsection*{Decision Variables}
Let \( x_{i,j} \) be the amount of currency \( i \) exchanged to currency \( j \).

\subsection*{Objective Function}
Maximize the final amount of currency \( N \), which is:

\[
\text{Maximize } \sum_{i=1}^{N} x_{i,N} \times \text{rate}_{i,N}
\]

\subsection*{Constraints}

1. Currency Exchange Limit Constraints:
   \[
   \sum_{j=1}^{N} x_{i,j} \leq \text{limit}_i, \quad \forall i = 1, \ldots, N
   \]

2. Starting Currency Amount Constraints:
   \[
   \sum_{j=1}^{N} x_{j,i} \times \text{rate}_{j,i} \leq \text{start}_i, \quad \forall i = 1, \ldots, N
   \]

3. Non-negativity Constraints:
   \[
   x_{i,j} \geq 0, \quad \forall i, j = 1, \ldots, N
   \]

\subsection*{Interpretation}

- The objective function seeks to maximize the total units of currency \( N \) obtained through exchanges.
- The first set of constraints ensures that the total currency exchanged from each currency \( i \) does not exceed the regulatory limit.
- The second set of constraints ensures that the net currency exchange involving each currency does not exceed the available starting amount.
- The non-negativity constraints ensure that all exchanges involve non-negative amounts of currency.

\end{document}