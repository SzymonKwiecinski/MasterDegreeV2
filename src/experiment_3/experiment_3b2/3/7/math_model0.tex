\documentclass{article}
\usepackage{amsmath}
\begin{document}

We are given the following data:
\begin{align*}
x_0 &= \text{initial position of the rocket}, \\
v_0 &= \text{initial velocity of the rocket}, \\
x_T &= \text{target position at time } T, \\
v_T &= \text{target velocity at time } T, \\
T &= \text{time horizon}.
\end{align*}

We aim to minimize the maximum thrust, denoted by $\max_t |a_t|$, subjected to the discrete-time rocket equations:
\begin{align}
x_{t+1} &= x_t + v_t, \quad \forall t = 0, 1, \ldots, T-1, \\
v_{t+1} &= v_t + a_t, \quad \forall t = 0, 1, \ldots, T-1,
\end{align}
with the initial conditions:
\begin{align}
x_0 &= x_0, \\
v_0 &= v_0,
\end{align}
and the final conditions:
\begin{align}
x_T &= x_T, \\
v_T &= v_T.
\end{align}

Additionally, let us introduce a variable $M$ to represent the maximum absolute acceleration. Our objective function becomes:
\[
\min M
\]

Subject to the constraints:
\begin{align}
-M &\leq a_t \leq M, \quad \forall t = 0, 1, \ldots, T-1, \\
x_{t+1} &= x_t + v_t, \\
v_{t+1} &= v_t + a_t, \\
x_0 &= x_0, \quad v_0 = v_0, \\
x_T &= x_T, \quad v_T = v_T.
\end{align}

The fuel consumption is proportional to the absolute acceleration. Thus, the total fuel spent, $F$, can be calculated as:
\[
F = \sum_{t=0}^{T-1} |a_t|
\]

This completes the linear programming formulation of the rocket problem.

\end{document}