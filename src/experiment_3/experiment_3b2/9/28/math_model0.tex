\documentclass{article}
\usepackage{amsmath}
\begin{document}

\textbf{Objective:} Minimize the sum of absolute deviations:
\[
\min \sum_{k=1}^{K} |y_k - (bx_k + a)|
\]

To handle absolute values in linear programming, we introduce additional variables \( e_k \) (for errors), and rewrite the objective function using these variables:

\[
\min \sum_{k=1}^{K} e_k
\]

\textbf{Subject to:}
\[
y_k - (bx_k + a) \leq e_k, \quad \text{for } k = 1, 2, \ldots, K
\]
\[
-(y_k - (bx_k + a)) \leq e_k, \quad \text{for } k = 1, 2, \ldots, K
\]
\[
e_k \geq 0, \quad \text{for } k = 1, 2, \ldots, K
\]

Where:
- \( a \) is the intercept of the line.
- \( b \) is the slope of the line.
- \( e_k \) are the error variables representing the absolute deviations.

This linear programming model will yield the values for the intercept \( a \) and the slope \( b \) that minimize the total absolute deviation.

\end{document}