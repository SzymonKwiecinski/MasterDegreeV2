\documentclass{article}
\usepackage{amsmath}
\begin{document}

\textbf{Linear Programming Model}

\textbf{Decision Variables:}  
Let \( x_k \) be the amount of alloy \( k \) to purchase for \( k = 1, \ldots, K \).

\textbf{Objective Function:}  
Minimize the total cost of purchasing the alloys:
\[
\min \sum_{k=1}^{K} \text{price}_k \times x_k
\]

\textbf{Constraints:}  
1. The total amount of alloy produced must be equal to the required amount:
\[
\sum_{k=1}^{K} x_k = \text{alloy\_quant}
\]

2. For each metal \( m \), ensure the correct proportion in the final alloy:
\[
\sum_{k=1}^{K} \text{ratio}_{k,m} \times x_k = \text{target}_{m} \times \text{alloy\_quant}, \quad \forall m = 1, \ldots, M
\]

3. Non-negativity constraints:
\[
x_k \geq 0, \quad \forall k = 1, \ldots, K
\]

\end{document}