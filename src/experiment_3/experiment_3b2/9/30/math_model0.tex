\documentclass{article}
\usepackage{amsmath}

\begin{document}

\section*{Quadratic Curve Fitting Using Linear Programming}

Given a set of data points \((x_k, y_k)\) for \( k = 1, 2, \ldots, K\), we aim to fit the quadratic curve \( y = c \cdot x^2 + b \cdot x + a \) by minimizing the sum of absolute deviations from the observed values.

\subsection*{Variables}
\begin{itemize}
    \item \( a \): Constant coefficient.
    \item \( b \): Linear coefficient.
    \item \( c \): Quadratic coefficient.
    \item \( d_k \): Absolute deviation for the \( k \)-th data point.
\end{itemize}

\subsection*{Objective Function}
Minimize the sum of absolute deviations:
\[
\min \sum_{k=1}^{K} d_k
\]

\subsection*{Constraints}
For each data point \( (x_k, y_k) \):
\[
d_k \geq y_k - (c \cdot x_k^2 + b \cdot x_k + a), \quad \text{for } k = 1, 2, \ldots, K
\]
\[
d_k \geq -(y_k - (c \cdot x_k^2 + b \cdot x_k + a)), \quad \text{for } k = 1, 2, \ldots, K
\]

These constraints ensure that \( d_k \) is the absolute deviation between the observed value \( y_k \) and the predicted value given by the quadratic model.

\subsection*{Solution}
After formulating this linear programming problem, a suitable solver can be used to find the optimal values for \( a \), \( b \), and \( c \) such that the objective function is minimized.

\end{document}