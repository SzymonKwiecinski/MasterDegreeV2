\documentclass{article}
\usepackage{amsmath}
\begin{document}

\section*{Linear Programming Model}

\textbf{Given:}
\begin{itemize}
    \item \(N\) different raw materials.
    \item \(M\) different goods.
    \item \( \text{available}_i \) is the amount of raw material \(i\) available.
    \item \( \text{req}_{i,j} \) is the amount of material \(i\) required to produce one unit of good \(j\).
    \item \(\text{price}_j\) is the revenue per unit produced of good \(j\).
\end{itemize}

\textbf{Decision Variables:}
\[
x_j = \text{amount of good } j \text{ produced, where } j = 1, \ldots, M
\]

\textbf{Objective:}
Maximize the total revenue:
\[
\text{maximize } \sum_{j=1}^{M} \text{price}_j \cdot x_j
\]

\textbf{Constraints:}
\begin{align*}
\sum_{j=1}^{M} \text{req}_{i,j} \cdot x_j & \leq \text{available}_i, & \text{for } i = 1, \ldots, N \\
x_j & \geq 0, & \text{for } j = 1, \ldots, M
\end{align*}

\textbf{Where:}
\begin{itemize}
    \item The first constraint ensures that the total usage of each raw material does not exceed its available amount.
    \item The non-negativity constraint ensures that production amounts are non-negative.
\end{itemize}

\end{document}