\documentclass{article}
\usepackage{amsmath}
\begin{document}

\textbf{Objective:}
Minimize the total cost of assigning officers to shifts.

\textbf{Decision Variables:}
\( x_s \): Number of officers starting at shift \( s \).

\textbf{Parameters:}
\begin{itemize}
    \item \( S \): Total number of shifts.
    \item \( \text{officers\_needed}_s \): Number of officers needed at shift \( s \).
    \item \( \text{shift\_cost}_s \): Cost of starting one officer at shift \( s \).
\end{itemize}

\textbf{Mathematical Formulation:}

\[
\text{Minimize} \quad Z = \sum_{s=1}^{S} \text{shift\_cost}_s \cdot x_s
\]

\textbf{Subject to:}

\[
x_s + x_{s-1 \mod S} \geq \text{officers\_needed}_s, \quad \forall s = 1, \ldots, S
\]

\[
x_s \geq 0, \quad \forall s = 1, \ldots, S
\]

Here, the constraint \( x_s + x_{s-1 \mod S} \) ensures that the number of officers on shift \( s \) is the sum of those starting shift \( s \) and those who started the previous shift (in a cyclic manner, hence the modulo operation).

\end{document}