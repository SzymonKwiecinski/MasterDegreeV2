\documentclass{article}
\usepackage{amsmath}
\begin{document}

\section*{Mathematical Model}

\subsection*{Decision Variables}
Let \( x_l \) be the number of times process \( l \) is executed for \( l = 1, \ldots, L \).

\subsection*{Objective Function}
Maximize the revenue:
\[
\text{Revenue} = \sum_{p=1}^{P} \text{price}_p \left( \sum_{l=1}^{L} \text{output}_{l,p} \cdot x_l \right) - \sum_{l=1}^{L} \text{cost}_l \cdot \sum_{p=1}^{P} \text{output}_{l,p} \cdot x_l
\]
This can be simplified to:
\[
\max \sum_{l=1}^{L} \left( \sum_{p=1}^{P} (\text{price}_p \cdot \text{output}_{l,p} - \text{cost}_l \cdot \text{output}_{l,p}) \right) x_l
\]

\subsection*{Constraints}
1. Crude Oil Availability:
   For each crude \( i \), the consumption cannot exceed its allocation:
   \[
   \sum_{l=1}^{L} \text{input}_{l,i} \cdot x_l \leq \text{allocated}_i, \quad i = 1, \ldots, O
   \]

2. Non-negativity:
   \[
   x_l \geq 0, \quad l = 1, \ldots, L
   \]

\subsection*{Summary}
The problem is to maximize the objective function subject to the above constraints. The decision variables \( x_l \) represent the number of times each process is executed, which are continuous and non-negative.

\end{document}