\documentclass{article}
\usepackage{amsmath}
\begin{document}

\title{Nurse Scheduling MILP Model}
\author{}
\date{}
\maketitle

\section*{Model}

\subsection*{Parameters}
\begin{itemize}
    \item \( \text{period} \): Number of consecutive days a nurse works.
    \item \( d_j \): Demand for nurses on day \( j \) for \( j = 1, 2, \ldots, 7 \).
\end{itemize}

\subsection*{Decision Variables}
\begin{itemize}
    \item \( x_j \): Number of nurses starting their shift on day \( j \), \( j = 1, 2, \ldots, 7 \).
    \item \( \text{total} \): Total number of nurses hired.
\end{itemize}

\subsection*{Objective Function}
\[
\min \text{ total} = \sum_{j=1}^{7} x_j
\]

\subsection*{Constraints}

For each day \( j \), the demand \( d_j \) must be satisfied. Nurses starting on day \( j \) are available to work for \( \text{period} \) days. Thus, the constraint for day \( j \) is:

\begin{align*}
\text{For } j = 1, \ldots, 7, \quad & \sum_{k=0}^{\text{period}-1} x_{(j-k-1) \bmod 7 + 1} \geq d_j
\end{align*}

\subsection*{Variables Constraints}
\begin{align*}
x_j & \geq 0, \quad \text{and integer} \quad \forall j = 1, \ldots, 7
\end{align*}

\end{document}