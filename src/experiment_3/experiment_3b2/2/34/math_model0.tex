\documentclass{article}
\usepackage{amsmath}
\begin{document}

\title{Diet Optimization Problem}
\author{}
\date{}
\maketitle

\section*{Problem Formulation}

\textbf{Given:}
\begin{itemize}
    \item Let \( K \) be the number of different foods.
    \item Let \( M \) be the number of different nutritional requirements.
    \item \( \text{price}_k \) is the price per unit of food \( k \) for \( k = 1, 2, \ldots, K \).
    \item \( \text{demand}_m \) is the demand of nutrient \( m \) for \( m = 1, 2, \ldots, M \).
    \item \( \text{nutri}_{k,m} \) is the amount of nutrient \( m \) contained in one unit of food \( k \).
\end{itemize}

\textbf{Objective:}
\[
\text{Minimize} \quad Z = \sum_{k=1}^{K} \text{price}_k \cdot \text{quantity}_k
\]

\textbf{Subject to:}
\[
\sum_{k=1}^{K} \text{nutri}_{k,m} \cdot \text{quantity}_k \geq \text{demand}_m \quad \text{for each } m = 1, 2, \ldots, M
\]
\[
\text{quantity}_k \geq 0 \quad \text{for each } k = 1, 2, \ldots, K
\]

\textbf{Decision Variables:}
\begin{itemize}
    \item \( \text{quantity}_k \) represents the quantity of food \( k \) to purchase.
\end{itemize}

\end{document}