\documentclass{article}
\usepackage{amsmath}
\begin{document}

\section*{Linear Programming Model}

\paragraph{Decision Variables:}
\begin{align*}
x_t &\quad \text{Position of the rocket at time } t, \quad t = 0, 1, \ldots, T \\
v_t &\quad \text{Velocity of the rocket at time } t, \quad t = 0, 1, \ldots, T \\
a_t &\quad \text{Acceleration of the rocket at time } t, \quad t = 0, 1, \ldots, T-1 \\
M &\quad \text{Maximum absolute value of the acceleration, } M \geq |a_t|, \quad t = 0, 1, \ldots, T-1
\end{align*}

\paragraph{Objective:}
\[
\min M
\]

\paragraph{Subject to:}
\begin{align*}
x_{t+1} &= x_t + v_t, \quad t = 0, 1, \ldots, T-1 \\
v_{t+1} &= v_t + a_t, \quad t = 0, 1, \ldots, T-1 \\
x_0 &= x_0, \\
v_0 &= v_0, \\
x_T &= x_T, \\
v_T &= v_T, \\
-M &\leq a_t \leq M, \quad t = 0, 1, \ldots, T-1
\end{align*}

\paragraph{Fuel Consumption:}
To calculate total fuel spent, we assume it is proportional to the sum of the absolute values of the accelerations:
\[
\text{Fuel Spend} = \sum_{t=0}^{T-1} |a_t|
\]

\end{document}