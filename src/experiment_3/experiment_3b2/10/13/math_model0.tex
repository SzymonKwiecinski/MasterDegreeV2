\documentclass{article}
\usepackage{amsmath}

\begin{document}

\textbf{Objective:} Maximize the radius of the ball, \( r \).

\textbf{Decision Variables:}
\begin{itemize}
    \item \( y \in \mathbb{R}^N \): The center of the ball.
    \item \( r \in \mathbb{R} \): The radius of the ball.
\end{itemize}

\textbf{Optimization Problem:}

\begin{align*}
    \max \quad & r \\
    \text{subject to} \quad & a_i^T y + \| a_i \| r \leq b_i, \quad i = 1, \ldots, m \\
    & r \geq 0
\end{align*}

where \( a_i \) are the rows of the matrix \( A \) and \( b_i \) are the elements of vector \( b \).

\textbf{Explanation:}

The constraints \( a_i^T y + \| a_i \| r \leq b_i \) ensure that the ball with center \( y \) and radius \( r \) does not violate any of the linear inequalities that define the polyhedral set \( P \). Here, \( \| a_i \| \) denotes the Euclidean norm of the vector \( a_i \).

The variable \( r \) must be non-negative since it represents a radius.

\end{document}