\documentclass{article}
\usepackage{amsmath}
\begin{document}

\section*{Linear Programming Model for Rocket Trajectory Optimization}

\textbf{Objective:} Minimize the total fuel consumption,
\[
\min \sum_{t=0}^{T-1} |a_t|
\]

\textbf{Subject to the constraints:}
\begin{align*}
x_{t+1} &= x_t + v_t, \quad \text{for } t = 0, 1, \ldots, T-1, \\
v_{t+1} &= v_t + a_t, \quad \text{for } t = 0, 1, \ldots, T-1, \\
x_0 &= \text{given initial position}, \\
v_0 &= \text{given initial velocity}, \\
x_T &= \text{target position at time } T, \\
v_T &= \text{target velocity at time } T.
\end{align*}

\textbf{Decision variables:}
- \( x_t \) for \( t = 0, 1, \ldots, T \)
- \( v_t \) for \( t = 0, 1, \ldots, T \)
- \( a_t \) for \( t = 0, 1, \ldots, T-1 \)

\textbf{Remarks:}
- \( |a_t| \) represents the absolute value of acceleration, relevant for fuel consumption.
- The absolute value can be handled in linear programming by introducing additional variables and constraints, e.g., \( a_t = a_t^+ - a_t^- \) with \( a_t^+, a_t^- \geq 0 \) and minimizing \( a_t^+ + a_t^- \).

\end{document}