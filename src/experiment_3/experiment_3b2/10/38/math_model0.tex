\documentclass{article}
\usepackage{amsmath}
\begin{document}

\section*{Linear Programming Model}

\subsection*{Decision Variables}
\begin{itemize}
    \item \( x_n \): Number of units produced using regular production in month \( n \).
    \item \( y_n \): Number of units produced using overtime production in month \( n \).
    \item \( s_n \): Number of units stored at the end of month \( n \).
\end{itemize}

\subsection*{Objective Function}
Minimize the total cost:
\[
\text{Minimize} \quad \sum_{n=1}^{N} \left( \text{cost\_regular} \cdot x_n + \text{cost\_overtime} \cdot y_n + \text{store\_cost} \cdot s_n \right)
\]

\subsection*{Constraints}
\begin{align*}
    & x_n + y_n + s_{n-1} = \text{demand}_n + s_n, \quad \forall n = 1, \ldots, N \\
    & x_n \leq \text{max\_regular\_amount}, \quad \forall n = 1, \ldots, N \\
    & x_n \geq 0, \, y_n \geq 0, \, s_n \geq 0, \quad \forall n = 1, \ldots, N \\
    & s_0 = 0
\end{align*}

\subsection*{Explanation}
\begin{itemize}
    \item The first constraint ensures that the production plus the inventory from the previous month meets the demand in the current month while updating the inventory for the next month.
    \item The second constraint limits the regular production to the maximum allowed quantity.
    \item The non-negativity constraints ensure that production and storage quantities are not negative.
\end{itemize}

\end{document}