\documentclass{article}
\usepackage{amsmath}
\begin{document}

\section*{Mathematical Model for Paper Cutting Problem}

Given the problem, our goal is to determine the minimum number of large rolls needed to meet the demand for smaller rolls using available cutting patterns. This is formulated as a Mixed Integer Linear Programming (MILP) problem as follows:

\subsection*{Sets and Indices}
\begin{itemize}
    \item \( j \): Index for small roll types, \( j = 1, \ldots, M \).
    \item \( i \): Index for cutting patterns, \( i = 1, \ldots, N \).
\end{itemize}

\subsection*{Parameters}
\begin{itemize}
    \item \( \text{large\_roll\_width} \): Width of a large roll.
    \item \( \text{inch}_j \): Width of small roll type \( j \).
    \item \( \text{demand}_j \): Demand for small roll type \( j \).
    \item \( \text{pattern}_{i,j} \): Number of rolls of type \( j \) produced by pattern \( i \).
\end{itemize}

\subsection*{Decision Variables}
\begin{itemize}
    \item \( x_i \): Integer variable representing the number of times pattern \( i \) is used.
\end{itemize}

\subsection*{Objective Function}

Minimize the total number of large rolls used:

\[
\text{minimize} \quad \sum_{i=1}^{N} x_i
\]

\subsection*{Constraints}

\begin{itemize}
    \item Demand fulfillment constraint for each small roll type \( j \):
    \[
    \sum_{i=1}^{N} \text{pattern}_{i,j} \cdot x_i \geq \text{demand}_j, \quad \forall j = 1, \ldots, M
    \]
    This ensures that the demand for each type of small roll is met.

    \item Non-negativity and integer constraint:
    \[
    x_i \geq 0 \quad \text{and integer}, \quad \forall i = 1, \ldots, N
    \]
\end{itemize}

This MILP model aims to optimize the cutting of large rolls into small rolls by leveraging the available cutting patterns to satisfy the demands while minimizing the number of large rolls used.

\end{document}