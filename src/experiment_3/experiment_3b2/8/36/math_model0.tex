\documentclass{article}
\usepackage{amsmath}
\begin{document}

\section*{Mathematical Model for Alloy Production}

\subsection*{Parameters:}
\begin{itemize}
    \item $alloy\_quant$: Total quantity of the alloy to be produced.
    \item $M$: Number of metals in the target alloy composition.
    \item $K$: Number of available alloys.
    \item $target_m$: Target weight of metal $m$ in the desired alloy.
    \item $ratio_{k,m}$: Ratio of metal $m$ in alloy $k$.
    \item $price_k$: Price of alloy $k$.
\end{itemize}

\subsection*{Decision Variables:}
\begin{itemize}
    \item $amount_k$: Quantity of alloy $k$ to purchase.
\end{itemize}

\subsection*{Objective Function:}

Minimize the total cost of purchasing the alloys:
\[
\text{Minimize } \sum_{k=1}^{K} price_k \times amount_k
\]

\subsection*{Constraints:}

1. Total weight constraint:
   \[
   \sum_{k=1}^{K} amount_k = alloy\_quant
   \]

2. Metal composition constraints for each metal $m$, where $m=1,...,M$:
   \[
   \sum_{k=1}^{K} ratio_{k,m} \times amount_k \geq target_m
   \]

3. Non-negativity constraints:
   \[
   amount_k \geq 0 \quad \text{for } k=1,...,K
   \]

\subsection*{Conclusion:}

The solution to this linear programming problem will provide the optimal quantities of each alloy to purchase in order to achieve the desired alloy composition at the minimum cost.

\end{document}