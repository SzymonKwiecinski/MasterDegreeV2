\documentclass{article}
\usepackage{amsmath}
\begin{document}

\section*{Mathematical Model}

We are given a division of an auto parts manufacturer producing \( P \) different parts using \( M \) different machines. The goal is to determine the number of batches of each part to produce in order to maximize the profit.

\subsection*{Decision Variables}

Let:
\[
x_p \geq 0 \quad \text{for } p = 1, 2, \dots, P
\]
represent the number of batches of part \( p \) produced each month. 

\subsection*{Objective Function}

The objective is to maximize the total profit. The profit for selling one batch of part \( p \) is given by the price minus the cost incurred on machines. Hence, the objective function is:
\[
\text{Maximize } Z = \sum_{p=1}^{P} \left( \text{price}_{p} - \sum_{m=1}^{M} \text{time}_{m,p} \times \text{cost}_{m} \right) \times x_p
\]

\subsection*{Constraints}

We have the following constraints:

1. **Machine Availability Constraints:**

   The total time used on each machine \( m \) should not exceed the available hours, except for machines \( M \) and \( M-1 \) which can share availability. This results in:
   \[
   \sum_{p=1}^{P} \text{time}_{m,p} \times x_p \leq \text{available}_{m} \quad \text{for } m = 1, 2, \ldots, M-2
   \]

   For machines \( M \) and \( M-1 \), we have:
   \[
   \sum_{p=1}^{P} \text{time}_{M,p} \times x_p + \sum_{p=1}^{P} \text{time}_{M-1,p} \times x_p \leq \text{available}_{M} + \text{available}_{M-1}
   \]

2. **Minimum Batch Requirements:**

   Each part must meet a minimum batch production requirement to fulfill a contract:
   \[
   x_p \geq \text{min\_batches}_{p} \quad \text{for } p = 1, 2, \ldots, P
   \]

\end{document}