\documentclass{article}
\usepackage{amsmath}
\begin{document}

\section*{Mathematical Model for Manpower Planning}

Let \( K \) represent the number of skill categories and \( I \) represent the number of years for planning.

\subsection*{Decision Variables:}

\begin{align*}
x_{k,i} & : \text{Number of manpower recruited in skill category } k \text{ in year } i. \\
o_{k,i} & : \text{Number of overmanning manpower in skill category } k \text{ in year } i. \\
s_{k,i} & : \text{Number of manpower on short-time working in skill category } k \text{ in year } i.
\end{align*}

\subsection*{Parameters:}

\begin{align*}
\text{requirement}_{k,i} & : \text{Manpower requirement for category } k \text{ in year } i. \\
\text{strength}_k & : \text{Current strength of manpower for category } k. \\
\text{lessonewaste}_k & : \text{Wastage rate of manpower } k \text{ with less than one year's service.} \\
\text{moreonewaste}_k & : \text{Wastage rate of manpower } k \text{ with more than one year's service.} \\
\text{recruit}_k & : \text{Maximum number of workers that can be recruited for } k \text{ each year.} \\
\text{costredundancy}_k & : \text{Redundancy payment cost for manpower } k. \\
\text{num\_overman} & : \text{Maximum number of overmanning allowed over the company.} \\
\text{costoverman}_k & : \text{Cost per overmanning employee per year for manpower } k. \\
\text{num\_shortwork} & : \text{Maximum short-time working allowed per category.} \\
\text{costshort}_k & : \text{Cost per short-time employee per year for manpower } k.
\end{align*}

\subsection*{Objective:}

Minimize the total redundancy payment over all categories and years:

\[
\min \sum_{k=1}^{K} \sum_{i=1}^{I} \text{costredundancy}_k \cdot \left( \text{strength}_k + x_{k,i} - \text{requirement}_{k,i} - o_{k,i} - 0.5 \cdot s_{k,i} \right)
\]

\subsection*{Constraints:}

1. \textbf{Labour Balance:}
   Ensure that the manpower meets the requirement considering recruitment, wastage, overmanning and short-time work:
   \[
   \text{strength}_k + x_{k,i} \cdot (1 - \text{lessonewaste}_k)^{i \leq 1} \cdot (1 - \text{moreonewaste}_k)^{i > 1} + o_{k,i} + 0.5 \cdot s_{k,i} \geq \text{requirement}_{k,i}, \quad \forall k, \forall i
   \]

2. \textbf{Recruitment Limits:}
   \[
   0 \leq x_{k,i} \leq \text{recruit}_k, \quad \forall k, \forall i
   \]

3. \textbf{Overmanning Limits:}
   \[
   \sum_{k=1}^{K} o_{k,i} \leq \text{num\_overman}, \quad \forall i
   \]

4. \textbf{Short-time Working Limits:}
   \[
   0 \leq s_{k,i} \leq \text{num\_shortwork}, \quad \forall k, \forall i
   \]

5. \textbf{Non-negativity:}
   \[
   x_{k,i}, o_{k,i}, s_{k,i} \geq 0, \quad \forall k, \forall i
   \]

\noindent
This linear programming model aims to minimize redundancy by optimizing recruitment, overmanning, and short-time working while satisfying manpower requirements for each skill category over the planning horizon.

\end{document}