\documentclass{article}
\usepackage{amsmath}
\begin{document}

\section*{Mathematical Model for Nurse Scheduling}

\subsection*{Parameters}
\begin{itemize}
    \item $period$: Number of consecutive days a nurse works, followed by $(7 - period)$ days off.
    \item $d_j$: Demand for nurses on day $j$, for $j = 1, \ldots, 7$.
\end{itemize}

\subsection*{Variables}
\begin{itemize}
    \item $start_j$: Integer variable representing the number of nurses starting their period on day $j$, for $j = 1, \ldots, 7$.
    \item $total$: Integer variable representing the total number of nurses hired.
\end{itemize}

\subsection*{Objective Function}
Minimize the total number of nurses hired:
\[
\text{Minimize} \quad total = \sum_{j=1}^{7} start_j
\]

\subsection*{Constraints}
For each day $j$, the number of nurses available should meet the demand:
\[
\begin{align*}
& \sum_{i=j-period+1}^{j} start_i \geq d_j, \quad \forall j = 1, \ldots, 7
\end{align*}
\]
where the indices are considered modulo 7, i.e., $start_{j} = start_{j+7}$ for $j < 1$ or $j > 7$. 

\subsection*{Explanation}
- Each nurse works continuously for \texttt{period} days starting from their assigned start day, so for any day $j$, the number of nurses who start from the last \texttt{period} days should be at least the demand $d_j$.

\end{document}