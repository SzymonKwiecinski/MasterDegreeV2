\documentclass{article}
\usepackage{amsmath}
\begin{document}

\section*{Linear Programming Model for Wild Sports Production}

\subsection*{Variables}
Let \( x_j \) be the amount produced of product \( j \) for \( j = 1, \ldots, M \).

\subsection*{Objective Function}
The objective is to maximize the total profit, which can be expressed as:
\[
\text{Maximize } Z = \sum_{j=1}^{M} (price_j - cost_j) \cdot x_j
\]

\subsection*{Constraints}
1. **Raw Material Constraints**:
   For each raw material \( i \) (where \( i = 1, \ldots, N \)):
   \[
   \sum_{j=1}^{M} req_{i,j} \cdot x_j \leq available_i
   \]

2. **Demand Constraints**:
   For each product \( j \):
   \[
   x_j \leq demand_j
   \]

3. **Non-negativity Constraints**:
   For each product \( j \):
   \[
   x_j \geq 0
   \]

\subsection*{Input Format}
The input format for the model is as follows:
\begin{verbatim}
{
    "available": [available_{i} for i in 1, ..., N],
    "requirements": [[req_{i,j} for i in 1, ..., N] for j in 1, ..., M],
    "prices": [price_{j} for j in 1, ..., M],
    "costs": [cost_{j} for j in 1, ..., M],
    "demands": [demand_{j} for j in 1, ..., M]
}
\end{verbatim}

\subsection*{Output Format}
The output format of the solution will be:
\begin{verbatim}
{
    "amount": [amount_{j} for j in 1, ..., M],
    "total_profit": total_profit
}
\end{verbatim}

\end{document}