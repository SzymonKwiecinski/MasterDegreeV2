\documentclass{article}
\usepackage{amsmath}
\begin{document}

\title{Linear Programming Model for Maximizing Revenue}
\author{}
\date{}
\maketitle

\section*{Problem Definition}

Let:
\begin{itemize}
    \item \( M \): Number of different goods
    \item \( N \): Number of different raw materials
    \item \( \text{available}_{i} \): Amount of raw material \( i \) available, for \( i = 1, \ldots, N \)
    \item \( \text{req}_{i,j} \): Units of raw material \( i \) required to produce one unit of good \( j \), for \( j = 1, \ldots, M \)
    \item \( \text{price}_{j} \): Revenue generated from producing one unit of good \( j \), for \( j = 1, \ldots, M \)
    \item \( \text{amount}_{j} \): Amount of good \( j \) produced, for \( j = 1, \ldots, M \)
\end{itemize}

\section*{Objective Function}

The objective is to maximize total revenue \( R \):
\[
\text{Maximize } R = \sum_{j=1}^{M} \text{price}_{j} \times \text{amount}_{j}
\]

\section*{Constraints}

The constraints based on the availability of raw materials are given by:
\[
\sum_{j=1}^{M} \text{req}_{i,j} \times \text{amount}_{j} \leq \text{available}_{i} \quad \forall i = 1, \ldots, N
\]

Additionally, we have non-negativity constraints:
\[
\text{amount}_{j} \geq 0 \quad \forall j = 1, \ldots, M
\]

\section*{Complete Model}

The complete linear programming model can be expressed as follows:

\begin{align*}
\text{Maximize} \quad & R = \sum_{j=1}^{M} \text{price}_{j} \times \text{amount}_{j} \\
\text{subject to} \quad & \sum_{j=1}^{M} \text{req}_{i,j} \times \text{amount}_{j} \leq \text{available}_{i} \quad \forall i = 1, \ldots, N \\
& \text{amount}_{j} \geq 0 \quad \forall j = 1, \ldots, M
\end{align*}

\section*{Output}

The solution will provide the optimal amount of each good to be produced:
\[
\text{Output: } \{\text{amount}: [\text{amount}_{j} \text{ for } j = 1, \ldots, M]\}
\]

\end{document}