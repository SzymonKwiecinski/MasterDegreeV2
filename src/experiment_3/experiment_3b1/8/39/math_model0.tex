\documentclass{article}
\usepackage{amsmath}
\begin{document}

\section*{Cafeteria Employee Scheduling Model}

\subsection*{Parameters}
Let:
\begin{itemize}
    \item \( num_n \): the number of desired employees on day \( n \) for \( n = 1, \ldots, N \)
    \item \( n_{\text{working}} \): the number of consecutive working days for each employee 
    \item \( n_{\text{resting}} \): the number of consecutive days off for each employee
\end{itemize}

\subsection*{Decision Variables}
Define \( x \): the total number of employees to hire, and let \( is\_work_{n,i} \) be a binary variable that indicates whether employee \( i \) is working on day \( n \):
\[
is\_work_{n,i} = 
\begin{cases} 
1 & \text{if employee } i \text{ is working on day } n \\
0 & \text{otherwise} 
\end{cases}
\]

\subsection*{Objective Function}
The objective is to minimize the total number of employees hired:
\[
\text{Minimize } x
\]

\subsection*{Constraints}
For each day \( n \) from \( 1 \) to \( N \), the following constraint ensures that the number of employees working on that day meets the required number:
\[
\sum_{i=1}^{x} is\_work_{n,i} \geq num_n \quad \forall n \in \{1, \ldots, N\}
\]

Additionally, to account for the working and resting days of each employee, we need to ensure that employees adhere to their schedule. For each employee \( i \):
\[
is\_work_{n,i} = 1 \quad \text{for } n \in \{j, j+1, \ldots, j+n_{\text{working}}-1\} \text{ and } 0 \text{ for } n \in \{j+n_{\text{working}}, j+n_{\text{working}}+1, \ldots, j+n_{\text{working}}+n_{\text{resting}}-1\}
\]
for suitable \( j \).

\subsection*{Output Variables}
The output variables will be:
\begin{itemize}
    \item \( total\_number = x \): the total number of employees hired.
    \item \( is\_work \): a matrix where \( is\_work[n][i] \) indicates whether employee \( i \) works on day \( n \).
\end{itemize}

\end{document}