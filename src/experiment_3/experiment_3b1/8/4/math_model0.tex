\documentclass{article}
\usepackage{amsmath}
\begin{document}

\section*{Mathematical Model for Nurse Scheduling}

\subsection*{Parameters}
\begin{itemize}
    \item \( d_j \): Demand for nurses on day \( j \) where \( j = 1, \ldots, 7 \).
    \item \( \text{period} \): Number of consecutive days a nurse works.
\end{itemize}

\subsection*{Decision Variables}
\begin{itemize}
    \item \( x_j \): Number of nurses that start their shift on day \( j \) where \( j = 1, \ldots, 7 \).
    \item \( N \): Total number of nurses hired.
\end{itemize}

\subsection*{Objective}
\[
\text{Minimize } N = \sum_{j=1}^{7} x_j
\]

\subsection*{Constraints}
The demand for nurses must be satisfied for each day \( j \):
\[
\sum_{k=1}^{7} x_{(j-k+1) \text{ mod } 7 + 1} \geq d_j \quad \text{for } j = 1, \ldots, 7
\]

Where \( x_{(j-k+1) \text{ mod } 7 + 1} \) accounts for the \( \text{period} \) of consecutive shifts.

\subsection*{Non-negativity Constraints}
\[
x_j \geq 0 \quad \text{for } j = 1, \ldots, 7
\]

\end{document}