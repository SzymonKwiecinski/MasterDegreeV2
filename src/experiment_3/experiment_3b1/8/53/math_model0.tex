\documentclass{article}
\usepackage{amsmath}
\begin{document}

\title{Transportation Problem: Soybean Distribution}
\author{}
\date{}
\maketitle

\section*{Problem Definition}
Consider a transportation network consisting of \( n \) cities where soybeans are shipped from a terminal city \( k \) to a port city \( p \), and subsequently to a destination city \( l \). 
Let \( A \) be the set of all routes between the cities. 
The transport cost per metric ton along the route from city \( i \) to city \( j \) is denoted as \( C_{i,j} \). 
Each source terminal \( k \) has a soybean supply denoted by \( supply_k \), and each destination city \( l \) has a demand denoted by \( demand_l \).

\subsection*{Objective}
We aim to determine the quantity of soybeans to ship from each source to each destination in order to minimize the total transportation costs.

\subsection*{Mathematical Formulation}

\textbf{Decision Variables:}
\begin{itemize}
    \item Let \( amount_{i,j} \) be the amount of soybeans shipped from city \( i \) to city \( j \).
\end{itemize}

\textbf{Objective Function:}
\[
\text{Minimize } Z = \sum_{(i,j) \in A} C_{i,j} \cdot amount_{i,j}
\]

\textbf{Subject to:}
\begin{itemize}
    \item Supply constraints:
    \[
    \sum_{j} amount_{k,j} \leq supply_k \quad \forall k
    \]
    
    \item Demand constraints:
    \[
    \sum_{i} amount_{i,l} \geq demand_l \quad \forall l
    \]
    
    \item Non-negativity constraints:
    \[
    amount_{i,j} \geq 0 \quad \forall (i,j) \in A
    \]
\end{itemize}

\subsection*{Output}
The solution will provide:
\begin{itemize}
    \item The amount of soybeans shipped from route \( (i,j) \) represented as \( amount_{i,j} \).
    \item The total transportation cost represented as \( total\_cost \).
\end{itemize}

\end{document}