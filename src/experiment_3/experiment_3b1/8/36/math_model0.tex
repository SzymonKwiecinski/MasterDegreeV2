\documentclass{article}
\usepackage{amsmath}
\begin{document}

\title{Linear Programming Model for Alloy Production}
\author{}
\date{}
\maketitle

\section*{Problem Definition}

A manufacturer wishes to produce \( \text{alloy\_quant} \) lb of an alloy that consists of several kinds of metals. The target quantity of metal \( m \) in the alloy is denoted by \( \text{target}_{m} \). Several alloys are available for purchase, each with specific properties and prices.

\section*{Variables}

Let:
\begin{itemize}
    \item \( x_k \) = amount of alloy \( k \) to purchase, for \( k = 1, \ldots, K \)
\end{itemize}

\section*{Parameters}

Given:
\begin{itemize}
    \item \( \text{alloy\_quant} \) = total weight of the alloy to be produced
    \item \( \text{target}_{m} \) = target weight of metal \( m \) in the alloy, for \( m = 1, \ldots, M \)
    \item \( \text{ratio}_{k, m} \) = ratio of metal \( m \) in alloy \( k \)
    \item \( \text{price}_{k} \) = price per pound of alloy \( k \)
\end{itemize}

\section*{Objective Function}

The objective is to minimize the total cost of alloys purchased:
\[
\text{Minimize } Z = \sum_{k=1}^{K} \text{price}_{k} x_k
\]

\section*{Constraints}

1. The total weight of the alloys must equal \( \text{alloy\_quant} \):
\[
\sum_{k=1}^{K} x_k = \text{alloy\_quant}
\]

2. The total weight of each metal \( m \) in the alloy must meet the target:
\[
\sum_{k=1}^{K} \text{ratio}_{k, m} x_k = \text{target}_{m}, \quad \forall m = 1, \ldots, M
\]

3. Non-negativity constraints:
\[
x_k \geq 0, \quad \forall k = 1, \ldots, K
\]

\section*{Output}

The output will include the amount of each alloy to purchase:
\[
\text{amount} = [x_{1}, x_{2}, \ldots, x_{K}]
\]

\end{document}