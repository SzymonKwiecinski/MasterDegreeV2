\documentclass{article}
\usepackage{amsmath}
\begin{document}

\section*{Quadratic Curve Fitting Problem}

Given a set of $K$ observed values of $x$ and $y$, where the $k$-th observed value of $y$ is denoted as $y_k$ corresponding to the $k$-th observed value of $x$ denoted as $x_k$, the objective is to fit the best quadratic curve defined by the equation:

\[
y = c x^2 + b x + a
\]

where $c$, $b$, and $a$ are the coefficients we aim to determine.

The goal is to minimize the sum of absolute deviations of the observed values of $y$ from the predicted values given by the quadratic relationship. This can be formulated as:

\[
\text{minimize} \quad \sum_{k=1}^{K} |y_k - (c x_k^2 + b x_k + a)|
\]

We introduce auxiliary variables $u_k$ for the absolute deviations, leading to the transformation:

\[
u_k \geq y_k - (c x_k^2 + b x_k + a) \quad \forall k
\]
\[
u_k \geq -(y_k - (c x_k^2 + b x_k + a)) \quad \forall k
\]

This provides the linearization needed for our linear programming formulation. Consequently, the complete linear programming model can be expressed as follows:

\[
\text{minimize} \quad \sum_{k=1}^{K} u_k
\]

subject to:

\[
u_k \geq y_k - (c x_k^2 + b x_k + a) \quad \forall k
\]
\[
u_k \geq -(y_k - (c x_k^2 + b x_k + a)) \quad \forall k
\]

where $u_k$ are the auxiliary variables representing the absolute deviations.

The solution to this linear programming problem will yield the optimal values of the coefficients:

\begin{itemize}
    \item quadratic term coefficient: $c$
    \item linear term coefficient: $b$
    \item constant term coefficient: $a$
\end{itemize}

The final output will be structured as:

\[
\{ "quadratic": c, "linear": b, "constant": a \}
\]

\end{document}