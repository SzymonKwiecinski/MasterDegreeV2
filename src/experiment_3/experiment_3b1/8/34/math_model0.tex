\documentclass{article}
\usepackage{amsmath}
\begin{document}

\section*{Linear Programming Model for Food Purchase Optimization}

\textbf{Sets:}
\begin{itemize}
    \item Let \( K \) be the set of food items, indexed by \( k = 1, \ldots, K \).
    \item Let \( M \) be the set of nutrients, indexed by \( m = 1, \ldots, M \).
\end{itemize}

\textbf{Parameters:}
\begin{itemize}
    \item \( \text{price}_{k} \): Price per unit of food \( k \).
    \item \( \text{demand}_{m} \): Minimum units of nutrient \( m \) required per day.
    \item \( \text{nutri}_{k, m} \): Units of nutrient \( m \) contained in one unit of food \( k \).
\end{itemize}

\textbf{Variables:}
\begin{itemize}
    \item \( \text{quantity}_{k} \): Quantity of food \( k \) to purchase.
\end{itemize}

\textbf{Objective Function:}
We want to minimize the total cost of food purchased:

\[
\text{Minimize } Z = \sum_{k=1}^{K} \text{price}_{k} \cdot \text{quantity}_{k}
\]

\textbf{Constraints:}
Each nutrient \( m \) must meet its respective daily demand:

\[
\sum_{k=1}^{K} \text{nutri}_{k, m} \cdot \text{quantity}_{k} \geq \text{demand}_{m}, \quad \forall m \in M
\]

\textbf{Non-negativity Constraints:}
\[
\text{quantity}_{k} \geq 0, \quad \forall k \in K
\]

\end{document}