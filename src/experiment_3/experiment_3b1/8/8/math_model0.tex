\documentclass{article}
\usepackage{amsmath}
\begin{document}

\title{Linear Programming Model for Maximizing Profit in Spare Parts Production}
\author{}
\date{}
\maketitle

\section*{Problem Formulation}

Let \( K \) be the number of different spare parts, and \( S \) be the number of shops. We define the following variables and parameters:

\begin{itemize}
    \item \( time_{k,s} \): Required worker-hours for part \( k \) in shop \( s \).
    \item \( profit_{k} \): Profit from producing one unit of part \( k \).
    \item \( capacity_{s} \): Capacity of shop \( s \) in working hours.
    \item \( quantity_{k} \): Quantity of part \( k \) to be produced.
\end{itemize}

\subsection*{Objective Function}
The objective is to maximize the total profit from the production of all spare parts:

\[
\text{Maximize} \quad Z = \sum_{k=1}^{K} profit_{k} \cdot quantity_{k}
\]

\subsection*{Constraints}
The production of spare parts is constrained by the capacity of the shops. The total worker-hours used in each shop cannot exceed its capacity:

\[
\sum_{k=1}^{K} time_{k,s} \cdot quantity_{k} \leq capacity_{s} \quad \text{for all } s = 1, 2, \ldots, S
\]

Additionally, the quantity of each spare part produced must be non-negative:

\[
quantity_{k} \geq 0 \quad \text{for all } k = 1, 2, \ldots, K
\]

\subsection*{Complete Model}
The complete linear programming model is as follows:

\[
\begin{align*}
\text{Maximize} \quad & Z = \sum_{k=1}^{K} profit_{k} \cdot quantity_{k} \\
\text{subject to} \quad & \sum_{k=1}^{K} time_{k,s} \cdot quantity_{k} \leq capacity_{s} \quad \text{for all } s = 1, 2, \ldots, S \\
& quantity_{k} \geq 0 \quad \text{for all } k = 1, 2, \ldots, K
\end{align*}
\]

\end{document}