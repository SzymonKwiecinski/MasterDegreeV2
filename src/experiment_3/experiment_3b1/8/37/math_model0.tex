\documentclass{article}
\usepackage{amsmath}
\begin{document}

\section*{Linear Programming Formulation of Spare Automobile Parts Production}

\subsection*{Indices}
\begin{itemize}
    \item $k$: index for spare parts, where $k = 1, 2, \ldots, K$
    \item $s$: index for shops, where $s = 1, 2, \ldots, S$
\end{itemize}

\subsection*{Parameters}
\begin{itemize}
    \item $time_{k, s}$: required worker-hours for part $k$ in shop $s$
    \item $profit_{k}$: profit from producing part $k$
    \item $capacity_{s}$: capacity of shop $s$ in working hours
\end{itemize}

\subsection*{Decision Variables}
\begin{itemize}
    \item $quantity_{k}$: quantity of spare part $k$ to be produced
\end{itemize}

\subsection*{Objective Function}
We aim to maximize the total profit from producing the spare parts:
\[
\text{Maximize } Z = \sum_{k=1}^{K} profit_{k} \cdot quantity_{k}
\]

\subsection*{Constraints}
1. Capacity constraints for each shop:
\[
\sum_{k=1}^{K} time_{k, s} \cdot quantity_{k} \leq capacity_{s}, \quad \forall s = 1, 2, \ldots, S
\]

2. Non-negativity constraints:
\[
quantity_{k} \geq 0, \quad \forall k = 1, 2, \ldots, K
\]

\subsection*{Complete Model}
Putting it all together, we can summarize the linear programming problem as follows:

\begin{align*}
\text{Maximize} \quad & Z = \sum_{k=1}^{K} profit_{k} \cdot quantity_{k} \\
\text{subject to} \quad & \sum_{k=1}^{K} time_{k, s} \cdot quantity_{k} \leq capacity_{s}, \quad \forall s = 1, 2, \ldots, S \\
& quantity_{k} \geq 0, \quad \forall k = 1, 2, \ldots, K
\end{align*}

\end{document}