\documentclass{article}
\usepackage{amsmath}
\begin{document}

\section*{Linear Programming Formulation}

Let \( N \) be the number of available currencies. We define the variables, parameters, and constraints for the linear programming model as follows:

\subsection*{Variables}
Let \( x_{i,j} \) be the amount of currency \( i \) exchanged for currency \( j \).

\subsection*{Parameters}
\begin{itemize}
    \item \( \text{start}_i \): the starting amount of currency \( i \).
    \item \( \text{limit}_i \): the limit on the total amount of currency \( i \) that can be exchanged.
    \item \( \text{rate}_{i,j} \): the rate at which currency \( i \) can be exchanged for currency \( j \).
\end{itemize}

\subsection*{Objective Function}
The objective is to maximize the amount of currency \( N \) at the end of the day:
\[
\text{Maximize } Z = y_N
\]
where \( y_N \) represents the final amount of currency \( N \).

\subsection*{Constraints}
The constraints for the model are as follows:

1. For each currency \( i \):
\[
\sum_{j=1}^{N} x_{i,j} \leq \text{limit}_i \quad \forall i \in \{1, 2, \ldots, N\}
\]

2. The amount of currency \( N \) at the end of the day can be expressed using the exchanged amounts:
\[
y_N = \text{start}_N + \sum_{j=1}^{N} x_{j,N} - \sum_{k=1}^{N} x_{N,k}
\]

3. The outflow of currency \( i \) cannot exceed the amount available after considering the limits:
\[
\sum_{j=1}^{N} x_{i,j} - \sum_{k=1}^{N} x_{k,i} \leq \text{start}_i \quad \forall i \in \{1, 2, \ldots, N\}
\]

4. The rate constraints define the conversion limits:
\[
x_{i,j} \leq \text{rate}_{i,j} \cdot \text{start}_i \quad \forall i,j \in \{1, 2, \ldots, N\}
\]

\subsection*{Non-negativity Constraints}
All exchanged amounts must be non-negative:
\[
x_{i,j} \geq 0 \quad \forall i,j \in \{1, 2, \ldots, N\}
\]

\end{document}