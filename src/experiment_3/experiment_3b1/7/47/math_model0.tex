\documentclass{article}
\usepackage{amsmath}
\begin{document}

\section*{Mathematical Model for Police Officer Shift Assignment}

\subsection*{Parameters}
\begin{itemize}
    \item Let \( S \) be the total number of shifts.
    \item Let \( officers\_needed_s \) be the number of police officers needed during shift \( s \) for \( s = 1, \ldots, S \).
    \item Let \( shift\_cost_s \) be the cost for the town when starting a shift at \( s \) for \( s = 1, \ldots, S \).
\end{itemize}

\subsection*{Decision Variables}
\begin{itemize}
    \item Let \( officers\_assigned_s \) be the number of police officers assigned to shift \( s \).
\end{itemize}

\subsection*{Objective Function}
We aim to minimize the total cost:
\[
\text{Minimize} \quad total\_cost = \sum_{s=1}^{S} shift\_cost_s \cdot x_s
\]
where \( x_s \) is a binary variable that equals 1 if at least one officer is assigned to shift \( s \) and 0 otherwise.

\subsection*{Constraints}
\begin{align}
    \text{(1) Officer Assignment Constraint:} & \quad officers\_assigned_s \geq officers\_needed_s \quad \forall s = 1, \ldots, S \\
    \text{(2) Shift Coverage Constraint:} & \quad officers\_assigned_s = officers\_assigned_{s-1} \quad \forall s = 2, \ldots, S \\
    \text{(3) Non-negativity:} & \quad officers\_assigned_s \geq 0 \quad \forall s = 1, \ldots, S \\
    \text{(4) Binary Shift Indicator:} & \quad x_s \in \{0, 1\} \quad \forall s = 1, \ldots, S
\end{align}

\subsection*{Output}
The expected output consists of:
\begin{itemize}
    \item The number of officers assigned to each shift: \( officers\_assigned = [officers\_assigned_s \text{ for } s = 1, \ldots, S] \)
    \item The total cost: \( total\_cost \)
\end{itemize}

\end{document}