\documentclass{article}
\usepackage{amsmath}
\begin{document}

\section*{Linear Programming Model for Oil Refinery Production}

\subsection*{Parameters}
\begin{itemize}
    \item Let \( O \) be the number of crude oil types.
    \item Let \( P \) be the number of products.
    \item Let \( L \) be the number of production processes.
    \item Let \( \text{allocated}_i \) be the allocated million barrels of crude oil type \( i \).
    \item Let \( \text{price}_p \) be the selling price per barrel of product \( p \).
    \item Let \( \text{input}_{l,i} \) be the number of barrels of crude oil type \( i \) required for process \( l \).
    \item Let \( \text{output}_{l,p} \) be the number of barrels of product \( p \) produced by process \( l \).
    \item Let \( \text{cost}_l \) be the cost per barrel of product produced by process \( l \).
\end{itemize}

\subsection*{Decision Variables}
\begin{itemize}
    \item Let \( x_l \) be the number of times process \( l \) is executed.
\end{itemize}

\subsection*{Objective Function}
Maximize the total revenue:
\[
\text{Revenue} = \sum_{p=1}^{P} \text{price}_p \cdot \sum_{l=1}^{L} x_l \cdot \text{output}_{l,p} - \sum_{l=1}^{L} x_l \cdot \text{cost}_l
\]

\subsection*{Constraints}
1. Crude oil constraints:
\[
\sum_{l=1}^{L} x_l \cdot \text{input}_{l,i} \leq \text{allocated}_i \quad \forall i = 1, \ldots, O
\]
2. Non-negativity constraints:
\[
x_l \geq 0 \quad \forall l = 1, \ldots, L
\]

\subsection*{Output Variables}
The total revenue and the execution quantities for each process:
\begin{itemize}
    \item \( \text{revenue} \) is the total revenue for the month.
    \item \( \text{execute}_l \) is the number of times that process \( l \) should be executed.
\end{itemize}

\end{document}