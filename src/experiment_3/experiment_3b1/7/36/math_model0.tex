\documentclass{article}
\usepackage{amsmath}
\begin{document}

\section*{Linear Programming Model for Alloy Production}

\subsection*{Parameters}
\begin{itemize}
    \item Let \( \text{alloy\_quant} \) be the total quantity of the alloy to be produced (in lb).
    \item Let \( \text{target}_{m} \) be the target quantity of metal \( m \) in the alloy.
    \item Let \( \text{ratio}_{k, m} \) be the ratio of metal \( m \) in alloy \( k \).
    \item Let \( \text{price}_{k} \) be the price of alloy \( k \).
    \item Let \( K \) be the total number of available alloys.
    \item Let \( M \) be the total number of metals involved in the alloy.
\end{itemize}

\subsection*{Decision Variables}
\begin{itemize}
    \item Let \( \text{amount}_{k} \) be the quantity of alloy \( k \) to purchase (in lb).
\end{itemize}

\subsection*{Objective Function}
The objective is to minimize the total cost of alloys purchased:
\[
\text{Minimize} \quad Z = \sum_{k=1}^{K} \text{price}_{k} \cdot \text{amount}_{k}
\]

\subsection*{Constraints}
1. The total quantity of the produced alloy must be equal to the desired quantity:
\[
\sum_{k=1}^{K} \text{amount}_{k} = \text{alloy\_quant}
\]

2. The weight of each metal \( m \) in the final alloy must meet the target requirements:
\[
\sum_{k=1}^{K} \text{ratio}_{k, m} \cdot \text{amount}_{k} = \text{target}_{m} \quad \text{for } m = 1, \ldots, M
\]

3. Non-negativity constraints:
\[
\text{amount}_{k} \geq 0 \quad \text{for } k = 1, \ldots, K
\]

\subsection*{Output}
The output will provide the quantities of each alloy to purchase:
\[
\text{amount} = [\text{amount}_{k} \text{ for } k = 1, \ldots, K]
\]

\end{document}