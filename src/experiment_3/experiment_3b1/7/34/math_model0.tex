\documentclass{article}
\usepackage{amsmath}
\begin{document}

\title{Linear Programming Model for Nutritional Diet}
\author{}
\date{}
\maketitle

\section*{Problem Formulation}

Let:
\begin{itemize}
    \item \( K \) be the number of different types of food available in the market.
    \item \( M \) be the number of different nutritional ingredients.
    \item \( price_k \) be the price per unit of food \( k \) for \( k = 1, \ldots, K \).
    \item \( demand_m \) be the minimum required units of nutrient \( m \) per day for \( m = 1, \ldots, M \).
    \item \( nutri_{k,m} \) be the amount of nutrient \( m \) contained in one unit of food \( k \).
    \item \( quantity_k \) be the quantity of food \( k \) to purchase.
\end{itemize}

\subsection*{Objective Function}

We aim to minimize the total cost of purchasing the foods:

\[
\text{Minimize } Z = \sum_{k=1}^{K} price_k \cdot quantity_k
\]

\subsection*{Constraints}

To ensure that the diet meets the nutritional demands, we have the following constraints:

\[
\sum_{k=1}^{K} nutri_{k, m} \cdot quantity_k \geq demand_m, \quad \forall m = 1, \ldots, M
\]

Additionally, we have non-negativity constraints on the quantities:

\[
quantity_k \geq 0, \quad \forall k = 1, \ldots, K
\]

\subsection*{Input Format}

The inputs to this linear programming model are given in the following format:

\begin{verbatim}
{
    "price": [price_{k} for k in 1,...,K],
    "demand": [demand_{m} for m in 1,...,M],
    "nutrition": [[nutri_{k, m} for m in 1,...,M] for k in 1,...,K]
}
\end{verbatim}

\subsection*{Output Format}

The output, representing the quantity of each food to purchase, is structured as follows:

\begin{verbatim}
{
    "quantity": [quantity_{k} for k in 1,...,K]
}
\end{verbatim}

\end{document}