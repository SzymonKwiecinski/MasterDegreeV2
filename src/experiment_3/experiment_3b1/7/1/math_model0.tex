\documentclass{article}
\usepackage{amsmath}
\begin{document}

\section*{Linear Programming Model}

\subsection*{Parameters}
\begin{itemize}
    \item Let \( M \) be the number of different goods.
    \item Let \( N \) be the number of different raw materials.
    \item Let \( \text{available}_i \) be the available units of raw material \( i \) for \( i = 1, \ldots, N \).
    \item Let \( \text{req}_{i,j} \) be the units of raw material \( i \) required to produce one unit of good \( j \) for \( j = 1, \ldots, M \).
    \item Let \( \text{price}_j \) be the revenue generated from producing one unit of good \( j \) for \( j = 1, \ldots, M \).
\end{itemize}

\subsection*{Decision Variables}
Let \( x_j \) be the amount of good \( j \) produced for \( j = 1, \ldots, M \).

\subsection*{Objective Function}
We want to maximize the total revenue:
\[
\text{Maximize} \quad Z = \sum_{j=1}^{M} \text{price}_j \cdot x_j
\]

\subsection*{Constraints}
The production of goods is limited by the availability of raw materials:
\[
\sum_{j=1}^{M} \text{req}_{i,j} \cdot x_j \leq \text{available}_i \quad \text{for } i = 1, \ldots, N
\]

Additionally, we have non-negativity constraints for our decision variables:
\[
x_j \geq 0 \quad \text{for } j = 1, \ldots, M
\]

\subsection*{Summary}
The complete linear programming model can be summarized as follows:

\[
\begin{aligned}
& \text{Maximize} && Z = \sum_{j=1}^{M} \text{price}_j \cdot x_j \\
& \text{subject to} \\
& && \sum_{j=1}^{M} \text{req}_{i,j} \cdot x_j \leq \text{available}_i \quad \forall i = 1, \ldots, N \\
& && x_j \geq 0 \quad \forall j = 1, \ldots, M
\end{aligned}
\]

\end{document}