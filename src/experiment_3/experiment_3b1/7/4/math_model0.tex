\documentclass{article}
\usepackage{amsmath}
\begin{document}

\section*{Mixed Integer Linear Programming Model for Nurse Scheduling}

\subsection*{Sets and Parameters}
\begin{itemize}
    \item Let \( J = \{1, 2, \ldots, 7\} \) be the set of days in a week.
    \item Let \( d_j \) be the demand for nurses on day \( j \in J \).
    \item Let \( \text{period} \) be the number of consecutive days a nurse works.
\end{itemize}

\subsection*{Decision Variables}
\begin{itemize}
    \item Let \( x_j \) be the number of nurses starting their shift on day \( j \in J \).
    \item Let \( N \) be the total number of nurses hired.
\end{itemize}

\subsection*{Objective Function}
The objective is to minimize the total number of nurses hired:
\[
\text{Minimize } N = \sum_{j=1}^{7} x_j
\]

\subsection*{Constraints}
\begin{itemize}
    \item For each day \( j \in J \), the demand must be met:
    \[
    \sum_{k=0}^{\text{period}-1} x_{(j-k-1) \mod 7} \geq d_j, \quad \forall j \in J
    \]
    where \( (j-k-1) \mod 7 \) corresponds to the effective day considering the cyclic nature of the schedule.
    
    \item The total number of nurses hired is the sum of the nurses starting on each day:
    \[
    N = \sum_{j=1}^{7} x_j
    \]
\end{itemize}

\subsection*{Solution Format}
The solution will provide:
\begin{itemize}
    \item A list \( \text{start} = [x_1, x_2, \ldots, x_7] \) representing the number of nurses starting on each day.
    \item The total number of nurses hired \( N \).
\end{itemize}

\end{document}