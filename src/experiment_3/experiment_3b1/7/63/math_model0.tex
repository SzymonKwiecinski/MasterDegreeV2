\documentclass{article}
\usepackage{amsmath}
\begin{document}

\section*{Mathematical Model for the Paper Cutting Problem}

\textbf{Parameters:}
\begin{itemize}
    \item $W$: Width of the large roll (inches)
    \item $M$: Number of different types of smaller rolls
    \item $d_j$: Demand for rolls of width $j$ (units), for $j = 1, \ldots, M$
    \item $N$: Number of different cutting patterns
    \item $p_{i,j}$: Number of rolls of width $j$ produced using cutting pattern $i$
\end{itemize}

\textbf{Decision Variables:}
\begin{itemize}
    \item $x_i$: Number of times cutting pattern $i$ is used, for $i = 1, \ldots, N$
\end{itemize}

\textbf{Objective Function:}
\begin{equation}
\text{Minimize } Z = \sum_{i=1}^{N} x_i
\end{equation}
This objective function aims to minimize the total number of large rolls used.

\textbf{Constraints:}
\begin{equation}
\sum_{i=1}^{N} p_{i,j} x_i \geq d_j \quad \text{for } j = 1, \ldots, M
\end{equation}
This constraint ensures that the total number of smaller rolls produced meets or exceeds the demands for each roll type.

\begin{equation}
x_i \geq 0 \quad \text{and integer for all } i = 1, \ldots, N
\end{equation}
This condition ensures that the number of times each cutting pattern is used is non-negative and integral.

\end{document}