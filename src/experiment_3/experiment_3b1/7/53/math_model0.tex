\documentclass{article}
\usepackage{amsmath}
\begin{document}

\section*{Transportation Problem Model}

Let \( n \) represent the number of cities in the transportation network. The objective is to determine the optimal quantity of soybeans to ship from source terminal cities \( k \) to destination cities \( l \) while minimizing the transportation costs.

\subsection*{Sets}
\begin{itemize}
    \item \( A \): Set of all routes, where each route is defined by a start city \( i \) and an end city \( j \).
    \item \( K \): Set of terminal cities \( k \).
    \item \( L \): Set of destination cities \( l \).
\end{itemize}

\subsection*{Parameters}
\begin{itemize}
    \item \( C_{i,j} \): Cost per metric ton of shipping soybeans along the route from city \( i \) to city \( j \).
    \item \( supply_{k} \): Supply of soybeans available at terminal city \( k \).
    \item \( demand_{l} \): Demand for soybeans at destination city \( l \).
\end{itemize}

\subsection*{Variables}
Let \( amount_{i,j} \) be the amount of soybeans shipped from city \( i \) to city \( j \).

\subsection*{Objective Function}
The objective is to minimize the total transportation cost:
\[
\text{Minimize } Z = \sum_{(i,j) \in A} C_{i,j} \cdot amount_{i,j}
\]

\subsection*{Constraints}
1. Supply constraints for each terminal city:
\[
\sum_{j | (k,j) \in A} amount_{k,j} \leq supply_{k}, \quad \forall k \in K
\]

2. Demand constraints for each destination city:
\[
\sum_{i | (i,l) \in A} amount_{i,l} \geq demand_{l}, \quad \forall l \in L
\]

3. Non-negativity constraints:
\[
amount_{i,j} \geq 0, \quad \forall (i,j) \in A
\]

\subsection*{Output Format}
The solution will provide the distribution of soybeans along with the total cost:
\begin{itemize}
    \item \textbf{Distribution:} A list containing amounts shipped from \( i \) to \( j \):
    \[
    \text{distribution} = \{(i, j, amount_{i,j}) \, | \, \text{for } i,j \text{ in routes}\}
    \]
    \item \textbf{Total Cost:} 
    \[
    total\_cost = Z
    \]
\end{itemize}

\end{document}