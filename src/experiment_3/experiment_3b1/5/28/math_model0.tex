\documentclass{article}
\usepackage{amsmath}
\begin{document}

\section*{Linear Programming Model for Fitting a Straight Line}

Given a set of data points \((x_k, y_k)\) for \(k = 1, \ldots, K\), we aim to find the best fit line defined by the equation:
\[
y = bx + a
\]
where \(b\) is the slope and \(a\) is the intercept.

\subsection*{Objective}
The objective is to minimize the sum of absolute deviations between the observed values \(y_k\) and the predicted values \(bx_k + a\). This can be formulated as:

\[
\text{Minimize } Z = \sum_{k=1}^{K} |y_k - (bx_k + a)|
\]

\subsection*{Variables}
Let:
\[
d_k = y_k - (bx_k + a) \quad \text{for } k = 1, \ldots, K
\]
Then, we can define the following constraints for absolute deviations:
\[
d_k \geq y_k - (bx_k + a) \quad \forall k
\]
\[
-d_k \geq -(y_k - (bx_k + a)) \quad \forall k
\]

This leads us to the modified form:
\[
\text{Minimize } Z = \sum_{k=1}^{K} d_k
\]

\subsection*{Constraints}
The constraints can be written as:
\[
d_k \geq y_k - (bx_k + a) \quad \forall k
\]
\[
d_k \geq -(y_k - (bx_k + a)) \quad \forall k
\]

\subsection*{Summary of the Model}
Thus, the complete linear programming model can be summarized as follows:

\begin{align*}
\text{Minimize } & Z = \sum_{k=1}^{K} d_k \\
\text{subject to } & d_k \geq y_k - (bx_k + a), \quad \forall k \\
                   & d_k \geq -(y_k - (bx_k + a)), \quad \forall k \\
                   & b, a \text{ are real numbers} \\
                   & d_k \text{ are non-negative}, \quad \forall k
\end{align*}

\subsection*{Output}
The results of the linear programming model will give us the values for:
\begin{itemize}
    \item \textbf{intercept} \(a\)
    \item \textbf{slope} \(b\)
\end{itemize}

\end{document}