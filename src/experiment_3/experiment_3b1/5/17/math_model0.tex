\documentclass{article}
\usepackage{amsmath}
\begin{document}

\section*{Linear Programming Model for Stock Selling}

\subsection*{Parameters}
\begin{align*}
N & : \text{Number of different stocks} \\
bought_i & : \text{Number of shares bought of stock } i \\
buyPrice_i & : \text{Price at which the shares of stock } i \text{ were bought} \\
currentPrice_i & : \text{Current price of one share of stock } i \\
futurePrice_i & : \text{Expected price of one share of stock } i \text{ in one year} \\
transactionRate & : \text{Transaction cost rate (\%)} \\
taxRate & : \text{Tax rate on capital gains (\%)} \\
K & : \text{Target amount of money to raise net of capital gains and transaction costs}
\end{align*}

\subsection*{Decision Variables}
\begin{align*}
sell_i & : \text{Number of shares to sell of stock } i 
\end{align*}

\subsection*{Objective Function}
Maximize the expected value of the portfolio next year, which can be formulated as:
\begin{align}
\text{Maximize } Z = \sum_{i=1}^{N} (futurePrice_i - currentPrice_i) \cdot (bought_i - sell_i)
\end{align}

\subsection*{Constraints}
1. The amount raised from selling shares, net of transaction costs and taxes, must meet or exceed the target amount \( K \):
\begin{align}
\sum_{i=1}^{N} \left( (currentPrice_i \cdot sell_i) - \left( transactionRate \cdot \frac{currentPrice_i \cdot sell_i}{100} \right) - \left( taxRate \cdot \frac{(currentPrice_i - buyPrice_i) \cdot sell_i}{100} \right) \right) \geq K
\end{align}

2. The number of shares sold cannot exceed the number of shares bought:
\begin{align}
sell_i \leq bought_i \quad \forall i = 1, ..., N
\end{align}

3. Non-negativity constraints:
\begin{align}
sell_i \geq 0 \quad \forall i = 1, ..., N
\end{align}

\subsection*{Output}
The output will consist of the number of shares to sell for each stock:
\begin{align*}
\text{Output} &: \{ sell : [sell_1, sell_2, ..., sell_N] \}
\end{align*}

\end{document}