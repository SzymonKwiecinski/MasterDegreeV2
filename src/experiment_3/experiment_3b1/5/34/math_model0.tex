\documentclass{article}
\usepackage{amsmath}
\begin{document}

\section*{Linear Programming Model for Food Selection}

\subsection*{Parameters}
\begin{itemize}
    \item Let \( K \) be the number of different foods available in the market.
    \item Let \( M \) be the number of nutritional ingredients.
    \item Let \( \text{price}_{k} \) be the price per unit of food \( k \).
    \item Let \( \text{demand}_{m} \) be the minimum required units of nutrient \( m \) per day.
    \item Let \( \text{nutri}_{k, m} \) be the units of nutrient \( m \) contained in food \( k \).
\end{itemize}

\subsection*{Decision Variables}
Let \( \text{quantity}_{k} \) be the quantity of food \( k \) to purchase.

\subsection*{Objective Function}
We want to minimize the total cost of purchasing the foods:
\[
\text{Minimize} \quad Z = \sum_{k=1}^{K} \text{price}_{k} \cdot \text{quantity}_{k}
\]

\subsection*{Constraints}
To ensure that the nutritional demands are met, we have the following constraints:
\[
\sum_{k=1}^{K} \text{nutri}_{k, m} \cdot \text{quantity}_{k} \geq \text{demand}_{m}, \quad \forall m = 1, \ldots, M
\]

\subsection*{Non-negativity Constraints}
\[
\text{quantity}_{k} \geq 0, \quad \forall k = 1, \ldots, K
\]

\subsection*{Summary}
The complete linear programming model can be summarized as follows:
\begin{align*}
\text{Minimize} & \quad Z = \sum_{k=1}^{K} \text{price}_{k} \cdot \text{quantity}_{k} \\
\text{subject to} & \quad \sum_{k=1}^{K} \text{nutri}_{k, m} \cdot \text{quantity}_{k} \geq \text{demand}_{m}, \quad \forall m = 1, \ldots, M \\
& \quad \text{quantity}_{k} \geq 0, \quad \forall k = 1, \ldots, K
\end{align*}

\end{document}