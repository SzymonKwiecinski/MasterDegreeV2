\documentclass{article}
\usepackage{amsmath}
\begin{document}

\title{Mathematical Model for Floppy Disk Backup Problem}
\author{}
\date{}
\maketitle

\section*{Problem Description}
We want to distribute files across a set of floppy disks to minimize the number of disks used, given their capacity and the sizes of the files.

\section*{Parameters}
\begin{itemize}
    \item Let \( C \) be the capacity of each floppy disk (in GB).
    \item Let \( N \) be the number of files.
    \item Let \( size_j \) be the size of file \( j \) for \( j = 1, \ldots, N \).
\end{itemize}

\section*{Decision Variables}
\begin{itemize}
    \item Let \( n \) be the number of floppy disks used.
    \item Let \( x_{ij} \) be a binary variable such that:
    \[
    x_{ij} = 
    \begin{cases} 
    1 & \text{if file } j \text{ is placed on disk } i \\ 
    0 & \text{otherwise} 
    \end{cases}
    \]
\end{itemize}

\section*{Objective Function}
The objective is to minimize the total number of disks used:
\[
\text{Minimize } n
\]

\section*{Constraints}
1. Each file must be assigned to exactly one disk:
\[
\sum_{i=1}^{n} x_{ij} = 1 \quad \forall j = 1, \ldots, N
\]

2. The total size of files on each disk must not exceed its capacity:
\[
\sum_{j=1}^{N} size_j \cdot x_{ij} \leq C \quad \forall i = 1, \ldots, n
\]

3. The number of disks used should be represented as:
\[
n \geq 1
\]

\section*{Output}
The solution to this model will provide:
\begin{itemize}
    \item \( n_{disks} \): The total number of floppy disks used.
    \item \( whichdisk \): An array indicating the disk number for each file.
\end{itemize}

\end{document}