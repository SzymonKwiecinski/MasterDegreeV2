\documentclass{article}
\usepackage{amsmath}
\begin{document}

\section*{Linear Programming Model for Cafeteria Staffing}

\subsection*{Parameters}
\begin{itemize}
    \item Let \( N \) be the total number of days considered.
    \item Let \( num_n \) be the number of employees required on day \( n \) where \( n = 1, 2, \ldots, N \).
    \item Let \( n_{\text{working}} \) be the number of consecutive working days for each employee.
    \item Let \( n_{\text{resting}} \) be the number of consecutive resting days for each employee.
    \item Let \( T = n_{\text{working}} + n_{\text{resting}} \) be the total cycle length for each employee.
\end{itemize}

\subsection*{Decision Variables}
\begin{itemize}
    \item Let \( x \) be the total number of employees hired.
    \item Let \( is\_work_{n,i} \) be a binary variable that indicates whether employee \( i \) is working on day \( n \) (1 if working, 0 if resting).
\end{itemize}

\subsection*{Objective Function}
Minimize the total number of employees to hire:
\[
\text{Minimize } x
\]

\subsection*{Constraints}
1. Staffing requirements:
\[
\sum_{i=1}^{x} is\_work_{n,i} \geq num_n, \quad \forall n = 1, 2, \ldots, N
\]

2. Employee working pattern:
Each employee works \( n_{\text{working}} \) days and then rests for \( n_{\text{resting}} \) days. Thus, for each employee \( i \):
\[
is\_work_{n,i} = 1 \quad \text{for } n = k(T) + 1, \ldots, k(T + n_{\text{working}}) \text{ for } k \in \mathbb{Z}
\]
\[
is\_work_{n,i} = 0 \quad \text{for } n = k(T + n_{\text{working}} + 1), \ldots, k(T + n_{\text{working}} + n_{\text{resting}}) \text{ for } k \in \mathbb{Z}
\]

3. Binary constraints:
\[
is\_work_{n,i} \in \{0, 1\}, \quad \forall n = 1, 2, \ldots, N, \, i = 1, 2, \ldots, x
\]

\subsection*{Output}
The output should provide:
\begin{itemize}
    \item \( total\_number \): the total number of employees to employ.
    \item \( is\_work \): a matrix of size \( N \times total\_number \) indicating the working status of each employee on each day.
\end{itemize}

\end{document}