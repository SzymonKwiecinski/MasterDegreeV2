\documentclass{article}
\usepackage{amsmath}
\begin{document}

\section*{Mathematical Model for Police Shift Assignment Problem}

\textbf{Parameters:}
\begin{itemize}
    \item \( S \): Number of different shifts
    \item \( officers\_needed_{s} \): Number of police officers needed during shift \( s \) for \( s = 1, \dots, S \)
    \item \( shift\_cost_{s} \): Cost incurred for starting shift \( s \) for \( s = 1, \dots, S \)
\end{itemize}

\textbf{Decision Variables:}
\begin{itemize}
    \item \( officers\_assigned_{s} \): Number of police officers assigned to shift \( s \) for \( s = 1, \dots, S \)
\end{itemize}

\textbf{Objective Function:}
\[
\text{Minimize } Z = \sum_{s=1}^{S} shift\_cost_{s} \cdot y_{s}
\]
where \( y_{s} \) is a binary variable that indicates whether shift \( s \) is started (1) or not (0).

\textbf{Constraints:}
\begin{align*}
    officers\_assigned_{s} & \geq officers\_needed_{s} \quad \forall s \in \{1, \ldots, S\} \\
    officers\_assigned_{s} & = officers\_assigned_{s-1} + officers\_assigned_{s-2} \quad \forall s \in \{3, \ldots, S\} \\
    officers\_assigned_{1} & = officers\_assigned_{1} \\
    officers\_assigned_{2} & = officers\_assigned_{2}
\end{align*}

\textbf{Output:}
\begin{itemize}
    \item \( officers\_assigned_{s} \) for \( s = 1, \ldots, S \)
    \item \( total\_cost = Z \)
\end{itemize}

\end{document}