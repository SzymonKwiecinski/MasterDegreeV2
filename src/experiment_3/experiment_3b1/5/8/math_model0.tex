\documentclass{article}
\usepackage{amsmath}
\begin{document}

\section*{Linear Programming Formulation}

Let \( K \) be the number of spare parts and \( S \) be the number of shops. We denote the following variables:

- \( x_k \): quantity of spare part \( k \) produced, for \( k = 1, 2, \ldots, K \).

Given the inputs:
\begin{itemize}
    \item \( time_{k,s} \): required worker-hours for part \( k \) in shop \( s \)
    \item \( profit_k \): profit of part \( k \)
    \item \( capacity_s \): capacity of shop \( s \) in working hours
\end{itemize}

The objective is to maximize total profit:

\[
\text{Maximize } Z = \sum_{k=1}^{K} profit_k \cdot x_k
\]

Subject to the constraints related to the capacity of each shop:

\[
\sum_{k=1}^{K} time_{k,s} \cdot x_k \leq capacity_s, \quad \text{for } s = 1, 2, \ldots, S
\]

Additionally, we include the non-negativity constraints:

\[
x_k \geq 0, \quad \text{for } k = 1, 2, \ldots, K
\]

Thus, the complete linear programming model can be summarized as follows:

\begin{align*}
\text{Maximize } & Z = \sum_{k=1}^{K} profit_k \cdot x_k \\
\text{subject to } & \sum_{k=1}^{K} time_{k,s} \cdot x_k \leq capacity_s, \quad s = 1, 2, \ldots, S \\
& x_k \geq 0, \quad k = 1, 2, \ldots, K
\end{align*}

\end{document}