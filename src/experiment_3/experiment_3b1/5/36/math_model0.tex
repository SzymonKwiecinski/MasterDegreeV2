\documentclass{article}
\usepackage{amsmath}
\begin{document}

\section*{Linear Programming Model for Alloy Production}

\subsection*{Parameters}
\begin{itemize}
    \item Let \( A \) be the total quantity of alloy to be produced: \( A = \text{alloy\_quant} \)
    \item Let \( M \) be the number of metals.
    \item Let \( K \) be the number of available alloys.
    \item Let \( T_m \) be the target quantity of metal \( m \): \( T_m = \text{target}_{m} \)
    \item Let \( R_{k,m} \) be the ratio of metal \( m \) in alloy \( k \): \( R_{k,m} = \text{ratio}_{k,m} \)
    \item Let \( P_k \) be the price of alloy \( k \): \( P_k = \text{price}_{k} \)
    \item Let \( x_k \) be the amount of alloy \( k \) purchased.
\end{itemize}

\subsection*{Objective Function}
The objective is to minimize the total cost of alloys purchased:
\[
\text{Minimize } Z = \sum_{k=1}^{K} P_k x_k
\]

\subsection*{Constraints}
1. The total weight of the produced alloy must be equal to the desired alloy quantity:
\[
\sum_{k=1}^{K} x_k = A
\]

2. The target quantity of each metal \( m \) must be met:
\[
\sum_{k=1}^{K} R_{k,m} x_k = T_m \quad \text{for } m = 1, 2, \ldots, M
\]

3. Non-negativity constraints:
\[
x_k \geq 0 \quad \text{for } k = 1, 2, \ldots, K
\]

\subsection*{Decision Variables}
\[
x_k \text{ is the amount of alloy } k \text{ purchased.}
\]

\subsection*{Output}
The resulting output will be:
\[
\text{amount} = [x_k \text{ for } k = 1, \ldots, K]
\]

\end{document}