\documentclass{article}
\usepackage{amsmath}
\begin{document}

\section*{Linear Programming Model for Spare Automobile Parts Production}

\textbf{Decision Variables:}
\begin{itemize}
    \item Let \( x_k \) be the quantity of spare part \( k \) produced, where \( k = 1, 2, \ldots, K \).
\end{itemize}

\textbf{Parameters:}
\begin{itemize}
    \item \( time_{k, s} \): Required worker-hours for part \( k \) in shop \( s \).
    \item \( profit_k \): Profit of part \( k \).
    \item \( capacity_s \): Capacity of shop \( s \) in working hours.
\end{itemize}

\textbf{Objective Function:}
Maximize total profit:
\[
\text{Maximize} \quad Z = \sum_{k=1}^{K} profit_k \cdot x_k
\]

\textbf{Constraints:}
\begin{itemize}
    \item For each shop \( s \), the total worker-hours used cannot exceed the shop's capacity:
    \[
    \sum_{k=1}^{K} time_{k, s} \cdot x_k \leq capacity_s \quad \text{for } s = 1, 2, \ldots, S
    \]
    \item Non-negativity constraints:
    \[
    x_k \geq 0 \quad \text{for } k = 1, 2, \ldots, K
    \]
\end{itemize}

\textbf{Output:}
The optimal quantities of each spare part to be made are represented as:
\[
\text{quantity} = [x_k \text{ for } k = 1, \ldots, K]
\]

\end{document}