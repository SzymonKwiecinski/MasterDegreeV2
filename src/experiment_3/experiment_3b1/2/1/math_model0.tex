\documentclass{article}
\usepackage{amsmath}
\begin{document}

\section*{Linear Programming Model}

\subsection*{Variables}
Let \( x_j \) be the amount of good \( j \) produced, where \( j = 1, 2, \ldots, M \).

\subsection*{Parameters}
\begin{itemize}
    \item \( \text{available}_i \): Available amount of raw material \( i \), where \( i = 1, 2, \ldots, N \).
    \item \( \text{req}_{i,j} \): Amount of raw material \( i \) required to produce one unit of good \( j \).
    \item \( \text{price}_j \): Revenue generated from producing one unit of good \( j \).
\end{itemize}

\subsection*{Objective Function}
Maximize the total revenue:
\[
\text{Maximize } Z = \sum_{j=1}^{M} \text{price}_j \cdot x_j
\]

\subsection*{Constraints}
The constraints for raw materials are given as:
\[
\sum_{j=1}^{M} \text{req}_{i,j} \cdot x_j \leq \text{available}_i, \quad \forall i = 1, 2, \ldots, N
\]

Additionally, we have non-negativity constraints:
\[
x_j \geq 0, \quad \forall j = 1, 2, \ldots, M
\]

\subsection*{Output}
The output will consist of the amounts produced for each good \( j \):
\[
\text{Output: } \{ \text{amount}_j = x_j \text{ for } j = 1, 2, \ldots, M \}
\]

\end{document}