\documentclass{article}
\usepackage{amsmath}
\begin{document}

\section*{Linear Programming Formulation}

Given the problem, we define the following variables and constants:

\begin{itemize}
    \item Let \( x_{i,j} \) be the amount of currency \( i \) exchanged for currency \( j \).
    \item Let \( start_i \) be the initial amount of currency \( i \).
    \item Let \( limit_i \) be the limit on the total amount of currency \( i \) that can be exchanged.
    \item Let \( rate_{i,j} \) be the exchange rate from currency \( i \) to currency \( j \).
    \item Let \( N \) be the total number of currencies.
\end{itemize}

\subsection*{Objective Function}
The objective is to maximize the amount of currency \( N \) we end up with at the end of the day. Thus, the objective function can be formulated as:

\[
\text{Maximize } Z = \sum_{j=1}^{N} x_{N,j}
\]

where \( x_{N,j} \) is the amount of currency \( N \) exchanged for currency \( j \).

\subsection*{Constraints}

1. **Currency Exchange Limits**:
   For each currency \( i \), the total amount of currency exchanged cannot exceed the limit:

   \[
   \sum_{j=1}^{N} x_{i,j} \leq limit_i, \quad \forall i \in \{1, 2, \ldots, N\}
   \]

2. **Initial Amounts**:
   The amount of each currency after transactions must not exceed the initially available amount. For currency \( i \):

   \[
   start_i - \sum_{j=1}^{N} x_{i,j} + \sum_{j=1}^{N} x_{j,i} \geq 0, \quad \forall i \in \{1, 2, \ldots, N\}
   \]

3. **Exchange Rates**:
    The amount exchanged must respect the exchange rates:

   \[
   x_{i,j} \leq start_i \cdot rate_{i,j}, \quad \forall i,j \in \{1, 2, \ldots, N\}
   \]

4. **Non-negativity**:
   All exchange amounts must be non-negative:

   \[
   x_{i,j} \geq 0, \quad \forall i,j \in \{1, 2, \ldots, N\}
   \]

\end{document}