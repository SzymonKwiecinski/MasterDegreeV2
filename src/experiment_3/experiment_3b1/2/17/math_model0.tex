\documentclass{article}
\usepackage{amsmath}
\begin{document}

\section*{Linear Programming Model}

Let:
\begin{itemize}
    \item $N$: Number of different stocks.
    \item $bought_i$: Number of shares bought for stock $i$.
    \item $buyPrice_i$: Purchase price per share of stock $i$.
    \item $currentPrice_i$: Current price per share of stock $i$.
    \item $futurePrice_i$: Expected future price per share of stock $i$.
    \item $transactionRate$: Transaction cost rate as a percentage.
    \item $taxRate$: Tax rate on capital gains as a percentage.
    \item $K$: Desired net amount of money to be raised.
    \item $sell_i$: Number of shares of stock $i$ that the investor should sell.
\end{itemize}

\subsection*{Objective Function}
We want to maximize the expected value of the portfolio next year after selling stocks, thus we need to maximize:
\[
\text{Maximize } Z = \sum_{i=1}^{N} \left( futurePrice_i \cdot (bought_i - sell_i) \right)
\]

\subsection*{Constraints}
1. The total amount raised from selling stocks must be equal to the amount needed ($K$) net of transaction costs and taxes:

\[
\sum_{i=1}^{N} \left( currentPrice_i \cdot sell_i \cdot (1 - transactionRate/100) - (currentPrice_i - buyPrice_i) \cdot sell_i \cdot (taxRate/100) \right) \geq K
\]

2. The number of shares sold must not exceed the number of shares bought for each stock:

\[
0 \leq sell_i \leq bought_i, \quad \forall i = 1, \ldots, N
\]

3. Non-negativity constraint:

\[
sell_i \geq 0, \quad \forall i = 1, \ldots, N
\]

\subsection*{Output Format}
The output will consist of the number of shares that should be sold for each stock:

\[
\text{Output: } \{ "sell": [sell_1, sell_2, \ldots, sell_N] \}
\]

\end{document}