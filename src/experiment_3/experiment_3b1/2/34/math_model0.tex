\documentclass{article}
\usepackage{amsmath}
\begin{document}

\section*{Linear Programming Model for Nutritional Optimization}

\subsection*{Variables}
Let \( x_k \) be the quantity of food \( k \) to purchase, where \( k = 1, 2, \ldots, K \).

\subsection*{Parameters}
\begin{itemize}
    \item \( \text{price}_k \): price per unit of food \( k \)
    \item \( \text{demand}_m \): minimum required units of nutrient \( m \)
    \item \( \text{nutri}_{k,m} \): units of nutrient \( m \) in food \( k \)
\end{itemize}

\subsection*{Objective Function}
We want to minimize the total cost of the foods purchased:
\[
\text{Minimize } Z = \sum_{k=1}^K \text{price}_k \cdot x_k
\]

\subsection*{Constraints}
To satisfy the nutritional requirements, we have the following constraints for each nutrient \( m \):
\[
\sum_{k=1}^K \text{nutri}_{k,m} \cdot x_k \geq \text{demand}_m \quad \text{for each } m = 1, 2, \ldots, M
\]

Additionally, we assume that the quantities purchased cannot be negative:
\[
x_k \geq 0 \quad \text{for each } k = 1, 2, \ldots, K
\]

\subsection*{Summary}
The complete linear programming model can be summarized as follows:

\[
\begin{align*}
\text{Minimize } & Z = \sum_{k=1}^K \text{price}_k \cdot x_k \\
\text{subject to } & \sum_{k=1}^K \text{nutri}_{k,m} \cdot x_k \geq \text{demand}_m, \quad m = 1, 2, \ldots, M \\
& x_k \geq 0, \quad k = 1, 2, \ldots, K
\end{align*}
\]

\end{document}