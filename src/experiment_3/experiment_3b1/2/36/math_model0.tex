\documentclass{article}
\usepackage{amsmath}
\begin{document}

\section*{Linear Programming Model for Alloy Production}

\textbf{Parameters:}
\begin{itemize}
    \item \( alloy\_quant \): Total weight of the alloy to produce (in lb).
    \item \( target_{m} \): Target quantity of metal \( m \) (for \( m = 1, \ldots, M \)).
    \item \( ratio_{k, m} \): Ratio of metal \( m \) in alloy \( k \) (for \( k = 1, \ldots, K \)).
    \item \( price_{k} \): Price of alloy \( k \) (for \( k = 1, \ldots, K \)).
\end{itemize}

\textbf{Decision Variables:}
\begin{itemize}
    \item \( amount_{k} \): Quantity of alloy \( k \) to use (for \( k = 1, \ldots, K \)).
\end{itemize}

\textbf{Objective Function:}
\[
\text{Minimize} \quad Z = \sum_{k=1}^{K} price_{k} \cdot amount_{k}
\]

\textbf{Constraints:}
\begin{itemize}
    \item Total weight of the alloys must equal the desired alloy quantity:
    \[
    \sum_{k=1}^{K} amount_{k} = alloy\_quant
    \]
    
    \item For each metal \( m \), the total amount must meet the target:
    \[
    \sum_{k=1}^{K} ratio_{k, m} \cdot amount_{k} = target_{m} \quad \text{for } m = 1, \ldots, M
    \]

    \item Non-negativity constraints for the amounts:
    \[
    amount_{k} \geq 0 \quad \text{for } k = 1, \ldots, K
    \]
\end{itemize}

\textbf{Output:}
The solution will provide the values of \( amount_{k} \) for \( k = 1, \ldots, K \).

\end{document}