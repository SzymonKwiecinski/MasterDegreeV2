\documentclass{article}
\usepackage{amsmath}
\begin{document}

\section*{Mathematical Model for Cafeteria Staffing Problem}

\subsection*{Parameters}
\begin{itemize}
    \item Let \( N \) be the total number of days.
    \item Let \( num_n \) be the number of employees required on day \( n \) for \( n = 1, 2, \ldots, N \).
    \item Let \( n_{\text{working}} \) be the number of consecutive working days for each employee.
    \item Let \( n_{\text{resting}} \) be the number of consecutive resting days for each employee.
    \item Let \( T = n_{\text{working}} + n_{\text{resting}} \) be the total cycle length for each employee.
\end{itemize}

\subsection*{Decision Variables}
\begin{itemize}
    \item Let \( total\_number \) be the total number of employees to hire.
    \item Let \( is\_work_{n, i} \) be a binary variable that indicates whether employee \( i \) is working on day \( n \):
    \[
    is\_work_{n, i} =
    \begin{cases}
      1 & \text{if employee } i \text{ works on day } n \\
      0 & \text{otherwise}
    \end{cases}
    \end{equation*}
\end{itemize}

\subsection*{Objective Function}
Minimize the total number of employees hired:
\[
\text{Minimize } total\_number
\]

\subsection*{Constraints}
For each day \( n = 1, 2, \ldots, N \):
\[
\sum_{i=1}^{total\_number} is\_work_{n, i} \geq num_n
\]

For each employee \( i \):
\begin{equation}
is\_work_{n, i} \text{ must satisfy the working/resting schedule}
\end{equation}
This implies that if an employee works on day \( n \), they must work for \( n_{\text{working}} \) days and then rest for \( n_{\text{resting}} \) days, creating the following constraints:
\[
is\_work_{n, i} + is\_work_{n+1, i} + \ldots + is\_work_{n+n_{\text{working}}-1, i} = n_{\text{working}} \quad \text{for } n \text{ such that } n + n_{\text{working}} - 1 \leq N
\]
\[
is\_work_{n+n_{\text{working}}, i} + is\_work_{n+n_{\text{working}}+1, i} + \ldots + is\_work_{n+n_{\text{working}}+n_{\text{resting}}-1, i} = 0 \quad \text{for } n \text{ such that } n+n_{\text{working}} + n_{\text{resting}} - 1 \leq N
\]

\subsection*{Output}
The output will consist of:
\begin{itemize}
    \item \( total\_number \): Total number of employees hired.
    \item \( is\_work \): A matrix where \( is\_work[n][i] \) indicates whether employee \( i \) is working on day \( n \).
\end{itemize}

\end{document}