\documentclass{article}
\usepackage{amsmath}
\begin{document}

\section*{Linear Programming Model for Oil Refinery Production}

\subsection*{Parameters}
\begin{itemize}
    \item Let \( O \) be the number of crude oil types.
    \item Let \( P \) be the number of products.
    \item Let \( L \) be the number of production processes.
    \item Let \( \text{allocated}_i \) represent the amount of crude oil type \( i \) allocated for production (in million barrels).
    \item Let \( \text{price}_p \) be the selling price per barrel of product \( p \).
    \item Let \( \text{input}_{l,i} \) be the amount of crude oil type \( i \) required for process \( l \) (in barrels).
    \item Let \( \text{output}_{l,p} \) be the amount of product \( p \) produced by process \( l \) (in barrels).
    \item Let \( \text{cost}_l \) be the cost per barrel for process \( l \).
\end{itemize}

\subsection*{Decision Variables}
\begin{itemize}
    \item Let \( \text{execute}_l \) be the number of times process \( l \) is executed.
\end{itemize}

\subsection*{Objective Function}
The objective is to maximize the total revenue:
\[
\text{Maximize } Z = \sum_{p=1}^{P} \text{price}_p \cdot \left( \sum_{l=1}^{L} \text{output}_{l,p} \cdot \text{execute}_l \right)
\]

\subsection*{Constraints}
\begin{itemize}
    \item \textbf{Crude Oil Allocation Constraint:}
    \[
    \sum_{l=1}^{L} \text{input}_{l,i} \cdot \text{execute}_l \leq \text{allocated}_i \quad \forall i = 1, \ldots, O
    \]
    
    \item \textbf{Non-negativity Constraint:}
    \[
    \text{execute}_l \geq 0 \quad \forall l = 1, \ldots, L
    \]
\end{itemize}

\subsection*{Output Variables}
The output of the model will include:
\begin{itemize}
    \item Total revenue for the month: \( \text{revenue} = Z \)
    \item Number of times to execute each process: \( \text{execute} = [\text{execute}_l \quad \forall l = 1, \ldots, L] \)
\end{itemize}

\end{document}