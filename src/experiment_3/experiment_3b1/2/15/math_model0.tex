\documentclass{article}
\usepackage{amsmath}
\begin{document}

\section*{Mathematical Model for the Production Problem}

\textbf{Parameters:}
\begin{itemize}
    \item \( N \): Number of product types
    \item \( assemblyHour_i \): Assembly hours required for product \( i \)
    \item \( testingHour_i \): Testing hours required for product \( i \)
    \item \( materialCost_i \): Cost of raw materials for product \( i \) (in dollars)
    \item \( maxAssembly \): Maximum assembly hours available per day
    \item \( maxTesting \): Maximum testing hours available per day
    \item \( price_i \): Selling price of product \( i \) (in dollars)
    \item \( maxOvertimeAssembly \): Maximum overtime assembly hours allowed
    \item \( overtimeAssemblyCost \): Cost per hour of overtime assembly
    \item \( materialDiscount \): Discount percentage on materials if the daily bill exceeds the threshold
    \item \( discountThreshold \): Threshold for receiving material discount (in dollars)
\end{itemize}

\textbf{Decision Variables:}
\begin{itemize}
    \item \( x_i \): Number of units produced of product \( i \)
    \item \( overtimeAssembly \): Hours of overtime assembly labor scheduled
\end{itemize}

\textbf{Objective Function:}
Maximize daily profit:
\[
\text{maximize } Z = \sum_{i=1}^{N} (price_i \cdot x_i) - \left( \sum_{i=1}^{N} (materialCost_i \cdot x_i) - \left( \text{if } \sum_{i=1}^{N} (materialCost_i \cdot x_i) > discountThreshold \text{ then } \frac{materialDiscount}{100} \cdot \sum_{i=1}^{N} (materialCost_i \cdot x_i) \right) \right) - (overtimeAssembly \cdot overtimeAssemblyCost)
\]

\textbf{Subject to:}
\begin{align*}
\sum_{i=1}^{N} (assemblyHour_i \cdot x_i) + overtimeAssembly & \leq maxAssembly + maxOvertimeAssembly \\
\sum_{i=1}^{N} (testingHour_i \cdot x_i) & \leq maxTesting \\
x_i & \geq 0 \quad \text{(for all } i = 1, \ldots, N\text{)} \\
overtimeAssembly & \geq 0
\end{align*}

\textbf{Output:}
The following outputs are to be calculated:
\begin{itemize}
    \item \( dailyProfit \): Amount of daily profit
    \item \( unitsProduced_i \): Number of units produced of product \( i \)
    \item \( overtimeAssembly \): Hours of overtime assembly labor scheduled
    \item \( materialBought \): Total amount of raw materials bought
\end{itemize}

\end{document}