\documentclass{article}
\usepackage{amsmath}
\begin{document}

\section*{Quadratic Curve Fitting via Linear Programming}

Given a set of data points $(x_k, y_k)$ for $k = 1, 2, \ldots, K$, we aim to fit a quadratic curve of the form:

\[
y = c \cdot x^2 + b \cdot x + a
\]

where \( c \), \( b \), and \( a \) are coefficients that we need to determine.

The objective is to minimize the sum of absolute deviations between the observed values \( y_k \) and the predicted values \( \hat{y}_k = c \cdot x_k^2 + b \cdot x_k + a \). We define the absolute deviations as follows:

\[
d_k = y_k - (c \cdot x_k^2 + b \cdot x_k + a)
\]

To handle the absolute values, we introduce auxiliary variables \( t_k \) such that:

\[
t_k \geq d_k \quad \text{and} \quad t_k \geq -d_k
\]

This leads to the following Linear Programming formulation:

\begin{align*}
\text{Minimize} \quad & \sum_{k=1}^{K} t_k \\
\text{subject to} \quad & t_k \geq y_k - (c \cdot x_k^2 + b \cdot x_k + a), \quad \forall k \\
& t_k \geq -(y_k - (c \cdot x_k^2 + b \cdot x_k + a)), \quad \forall k \\
& c, b, a \text{ are free variables}
\end{align*}

After solving this linear program, the coefficients can be extracted as follows:

\begin{align*}
\text{Output:} \quad & \{ \text{"quadratic"}: c, \text{"linear"}: b, \text{"constant"}: a \}
\end{align*}

\end{document}