\documentclass{article}
\usepackage{amsmath}
\begin{document}

\section*{Mathematical Model for Floppy Disk Backup Problem}

\subsection*{Parameters}
\begin{itemize}
    \item Let \( C \) be the capacity of each floppy disk (in GB).
    \item Let \( N \) be the number of files.
    \item Let \( size_j \) be the size of file \( j \) where \( j = 1, 2, \ldots, N \).
\end{itemize}

\subsection*{Decision Variables}
\begin{itemize}
    \item Let \( n \) be the number of floppy disks used.
    \item Let \( x_{ij} \) be a binary variable where:
        \[
        x_{ij} = 
        \begin{cases} 
        1 & \text{if file } j \text{ is assigned to disk } i \\
        0 & \text{otherwise} 
        \end{cases}
        \]
    \item Let \( y_i \) be a binary variable where:
        \[
        y_i = 
        \begin{cases} 
        1 & \text{if disk } i \text{ is used} \\
        0 & \text{otherwise} 
        \end{cases}
        \]
\end{itemize}

\subsection*{Objective Function}
Minimize the total number of disks used:
\[
\text{Minimize } n = \sum_{i=1}^{N} y_i
\]

\subsection*{Constraints}
1. Each file must be assigned to exactly one disk:
   \[
   \sum_{i=1}^{n} x_{ij} = 1, \quad \forall j = 1, 2, \ldots, N
   \]
   
2. The total size of files on each disk cannot exceed its capacity:
   \[
   \sum_{j=1}^{N} size_j \cdot x_{ij} \leq C \cdot y_i, \quad \forall i = 1, 2, \ldots, n
   \]
   
3. The number of disks used should be equal to the sum of used disks:
   \[
   n = \sum_{i=1}^{N} y_i
   \]

\subsection*{Output}
The output consists of:
\begin{itemize}
    \item \( n\_disks \) representing the total number of floppy disks used.
    \item \( whichdisk \) representing on which disk each file is distributed.
\end{itemize}

\end{document}