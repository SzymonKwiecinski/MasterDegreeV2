\documentclass{article}
\usepackage{amsmath}
\begin{document}

\section*{Linear Programming Model for School Assignment Problem}

\subsection*{Variables}
Let \( x_{n,s,g} \) denote the number of students of grade \( g \) in neighborhood \( n \) that are assigned to school \( s \).

\subsection*{Parameters}
\begin{itemize}
    \item \( N \): Number of neighborhoods
    \item \( S \): Number of schools
    \item \( G \): Number of grades
    \item \( capacity_{s,g} \): Capacity of school \( s \) for grade \( g \)
    \item \( population_{n,g} \): Student population in neighborhood \( n \) for grade \( g \)
    \item \( d_{n,s} \): Distance from neighborhood \( n \) to school \( s \)
\end{itemize}

\subsection*{Objective Function}
Minimize the total distance traveled by all students:
\[
\text{Minimize } Z = \sum_{n=1}^{N} \sum_{s=1}^{S} \sum_{g=1}^{G} d_{n,s} \cdot x_{n,s,g}
\]

\subsection*{Constraints}
\begin{itemize}
    \item \textbf{Supply Constraints:} For each neighborhood and grade, the total number of students assigned must equal the population:
    \[
    \sum_{s=1}^{S} x_{n,s,g} = population_{n,g} \quad \forall n = 1, \ldots, N, \; g = 1, \ldots, G
    \]
    
    \item \textbf{Demand Constraints:} For each school and grade, the number of students assigned must not exceed the school's capacity:
    \[
    \sum_{n=1}^{N} x_{n,s,g} \leq capacity_{s,g} \quad \forall s = 1, \ldots, S, \; g = 1, \ldots, G
    \]
    
    \item \textbf{Non-negativity Constraints:} The variables must be non-negative:
    \[
    x_{n,s,g} \geq 0 \quad \forall n = 1, \ldots, N, \; s = 1, \ldots, S, \; g = 1, \ldots, G
    \]
\end{itemize}

\subsection*{Output}
The output will consist of:
\begin{itemize}
    \item An assignment matrix \( x_{n,s,g} \) showing the number of students assigned from each neighborhood to each school for each grade.
    \item The total distance \( total\_distance \) calculated from the objective function.
\end{itemize}

\end{document}