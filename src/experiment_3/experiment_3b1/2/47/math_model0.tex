\documentclass{article}
\usepackage{amsmath}
\begin{document}

\section*{Problem Formulation}

We want to minimize the total cost of assigning police officers to shifts while ensuring that the required number of officers during each shift is met. 

\subsection*{Variables}
Let:
\begin{itemize}
    \item $S$: Total number of shifts.
    \item $officers\_needed_{s}$: Number of officers needed during shift $s$, for $s = 1, 2, \ldots, S$.
    \item $shift\_cost_{s}$: Cost incurred when starting a shift at $s$.
    \item $officers\_assigned_{s}$: Number of officers assigned to shift $s$.
\end{itemize}

\subsection*{Objective Function}
We aim to minimize the total cost:
\[
\text{Minimize } Z = \sum_{s=1}^{S} shift\_cost_{s} \cdot \left\lceil \frac{officers\_needed_{s}}{2} \right\rceil
\]

\subsection*{Constraints}
Each officer works for two consecutive shifts. Hence, we set up the constraints for each shift:

\[
officers\_assigned_{s} + officers\_assigned_{s-1} \geq officers\_needed_{s}, \quad \forall s = 2, 3, \ldots, S
\]

For the first shift:
\[
officers\_assigned_{1} \geq officers\_needed_{1}
\]

And for the last shift, we need to ensure we account for the wrap-around:
\[
officers\_assigned_{S} + officers\_assigned_{S-1} \geq officers\_needed_{S}
\]

\subsection*{Non-negativity Constraints}
\[
officers\_assigned_{s} \geq 0, \quad \forall s = 1, 2, \ldots, S
\]

\subsection*{Output}
The final output should consist of:
\begin{itemize}
    \item $officers\_assigned$: List of officers assigned to each shift.
    \item $total\_cost$: The total cost incurred by the town.
\end{itemize}

\end{document}