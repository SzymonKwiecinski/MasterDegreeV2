\documentclass{article}
\usepackage{amsmath}
\begin{document}

\title{Mathematical Model for Translator Selection Problem}
\author{}
\date{}
\maketitle

\section*{Problem Definition}

Given a set of translators and a requirement to translate a list of languages, the objective is to select a combination of translators that ensures all required languages can be translated into, at the minimum possible cost.

\section*{Parameters}

\begin{itemize}
    \item Let \( N \) be the total number of translators.
    \item Let \( M \) be the total number of required languages.
    \item Let \( \text{cost}_i \) be the cost associated with translator \( i \).
    \item Let \( L_i \) be the set of languages that translator \( i \) can translate.
    \item Let \( \text{languages}_m \) be the required languages for \( m = 1, \ldots, M \).
\end{itemize}

\section*{Decision Variables}

Define the binary decision variable:
\[
x_i = 
\begin{cases} 
1 & \text{if translator } i \text{ is selected} \\
0 & \text{otherwise}
\end{cases}
\]

\section*{Objective Function}

Minimize the total cost of the selected translators:
\[
\text{Minimize} \quad Z = \sum_{i=1}^{N} \text{cost}_i \cdot x_i
\]

\section*{Constraints}

1. Coverage of required languages:
\[
\sum_{i=1}^{N} x_i \cdot \mathbf{1}_{\{language_m \in L_i\}} \geq 1 \quad \forall m = 1, \ldots, M
\]
where \( \mathbf{1}_{\{language_m \in L_i\}} \) is an indicator function that equals 1 if language \( m \) is included in the languages that translator \( i \) can translate.

2. Non-negativity and binary constraints:
\[
x_i \in \{0, 1\} \quad \forall i = 1, \ldots, N
\]

\section*{Outputs}

The outputs of the model will yield:
\begin{itemize}
    \item \texttt{selected\_translators}: the list of ids of the chosen translators.
    \item \texttt{total\_cost}: the total cost for hiring the selected translators.
\end{itemize}

\end{document}