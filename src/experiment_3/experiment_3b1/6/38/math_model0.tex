\documentclass{article}
\usepackage{amsmath}
\begin{document}

\section*{Linear Programming Formulation}

Let:
\begin{itemize}
    \item \( N \) be the total number of months.
    \item \( demand_n \) be the demand in month \( n \) for \( n = 1, \ldots, N \).
    \item \( max\_regular\_amount \) be the maximum regular production capacity in any month.
    \item \( cost\_regular \) be the cost per unit for regular production.
    \item \( cost\_overtime \) be the cost per unit for overtime production.
    \item \( store\_cost \) be the cost per unit per month for storing inventory.
    \item \( reg\_quant_n \) be the quantity produced using regular production in month \( n \).
    \item \( over\_quant_n \) be the quantity produced using overtime production in month \( n \).
    \item \( inventory_n \) be the inventory at the end of month \( n \).
\end{itemize}

\subsection*{Objective Function}

The objective is to minimize the total cost, which includes regular production costs, overtime production costs, and storage costs:

\[
\min \sum_{n=1}^{N} \left( cost\_regular \cdot reg\_quant_n + cost\_overtime \cdot over\_quant_n + store\_cost \cdot inventory_n \right)
\]

\subsection*{Constraints}

1. **Production Capacity Constraints**:
   \[
   reg\_quant_n + over\_quant_n \leq max\_regular\_amount + inventory_{n-1} \quad \text{for } n = 1, \ldots, N
   \]
   (For \( n = 1 \), \( inventory_0 = 0 \))

2. **Demand Satisfaction Constraints**:
   \[
   inventory_n \geq demand_n \quad \text{for } n = 1, \ldots, N
   \]

3. **Inventory Balance Constraints**:
   \[
   inventory_n = inventory_{n-1} + reg\_quant_n + over\_quant_n - demand_n \quad \text{for } n = 1, \ldots, N
   \]

4. **Non-negativity Constraints**:
   \[
   reg\_quant_n \geq 0, \quad over\_quant_n \geq 0, \quad inventory_n \geq 0 \quad \text{for } n = 1, \ldots, N
   \]

\subsection*{Output Format}

The solution will provide:
\begin{itemize}
    \item \( reg\_quant \) which represents the quantity of products produced using regular production in each future month.
    \item \( over\_quant \) which represents the quantity of products produced using overtime production in each future month.
\end{itemize}

\[
\text{Output: } 
\begin{cases}
    reg\_quant = [reg\_quant_1, reg\_quant_2, \ldots, reg\_quant_N] \\
    over\_quant = [over\_quant_1, over\_quant_2, \ldots, over\_quant_N]
\end{cases}
\]

\end{document}