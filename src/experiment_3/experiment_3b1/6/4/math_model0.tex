\documentclass{article}
\usepackage{amsmath}
\begin{document}

\section*{Mathematical Model for Nursing Shift Scheduling}

\subsection*{Parameters}
\begin{itemize}
    \item $period$: Number of consecutive days a nurse works the night shift.
    \item $d_j$: Demand for nurses on day $j$ for $j = 1, \ldots, 7$.
    \item $N$: Total number of nurses hired.
\end{itemize}

\subsection*{Decision Variables}
\begin{itemize}
    \item $start_j$: Number of nurses starting their shift on day $j$ for $j = 1, \ldots, 7$.
\end{itemize}

\subsection*{Objective Function}
Minimize the total number of nurses hired:

\[
\text{Minimize } N = \sum_{j=1}^{7} start_j
\]

\subsection*{Constraints}
For each day $j$, the demand must be satisfied by the nurses starting their shifts that day and the ones from the previous $period - 1$ days:

\[
\sum_{i=0}^{\min(period-1, j-1)} start_{j-i} \geq d_j \quad \forall j \in \{1, 2, \ldots, 7\}
\]

This ensures that the number of nurses available on each day meets the required demand.

\subsection*{Output}
The model will output:
\begin{itemize}
    \item $start_j$ for each day $j = 1, \ldots, 7$.
    \item The total number of nurses hired $N$.
\end{itemize}

\end{document}