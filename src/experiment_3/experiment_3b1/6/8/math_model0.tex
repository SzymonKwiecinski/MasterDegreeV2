\documentclass{article}
\usepackage{amsmath}
\begin{document}

\section*{Linear Programming Formulation for Maximizing Profit from Spare Automobile Parts}

\textbf{Variables:}
\begin{itemize}
    \item Let \( x_k \) be the quantity of spare part \( k \) to be produced for \( k = 1, \ldots, K \).
\end{itemize}

\textbf{Parameters:}
\begin{itemize}
    \item Let \( time_{k,s} \) be the required worker-hours for part \( k \) in shop \( s \).
    \item Let \( profit_k \) be the profit from part \( k \).
    \item Let \( capacity_s \) be the available working hours in shop \( s \).
\end{itemize}

\textbf{Objective Function:}
We aim to maximize the total profit from the production of spare parts:
\[
\text{Maximize } Z = \sum_{k=1}^{K} profit_k \cdot x_k
\]

\textbf{Constraints:}
The production must adhere to the capacity limitations of each shop. For each shop \( s \), the total worker-hours used cannot exceed the shop's capacity:
\[
\sum_{k=1}^{K} time_{k,s} \cdot x_k \leq capacity_s \quad \forall s = 1, \ldots, S
\]

\textbf{Non-negativity Constraints:}
\[
x_k \geq 0 \quad \forall k = 1, \ldots, K
\]

\textbf{Complete Model:}
The linear programming model can be summarized as follows:

\[
\begin{aligned}
\text{Maximize } & Z = \sum_{k=1}^{K} profit_k \cdot x_k \\
\text{subject to } & \sum_{k=1}^{K} time_{k,s} \cdot x_k \leq capacity_s, \quad \forall s = 1, \ldots, S \\
& x_k \geq 0, \quad \forall k = 1, \ldots, K
\end{aligned}
\]

\end{document}