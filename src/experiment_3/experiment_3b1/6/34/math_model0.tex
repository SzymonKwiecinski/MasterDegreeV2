\documentclass{article}
\usepackage{amsmath}
\begin{document}

\section*{Linear Programming Model for Food Purchase Optimization}

\subsection*{Parameters}
\begin{itemize}
    \item Let $K$ be the number of food types available in the market.
    \item Let $M$ be the number of nutritional ingredients.
    \item Let $price_k$ be the price per unit of food $k$.
    \item Let $demand_m$ be the minimum required units of nutrient $m$ per day.
    \item Let $nutri_{k,m}$ be the units of nutrient $m$ contained in food $k$.
\end{itemize}

\subsection*{Decision Variables}
Let $quantity_k$ be the quantity of food $k$ to purchase.

\subsection*{Objective Function}
We want to minimize the total cost of the foods purchased:
\[
\text{Minimize } Z = \sum_{k=1}^{K} price_k \cdot quantity_k
\]

\subsection*{Constraints}
To ensure that each nutrient requirement is met, we must satisfy the following constraints for all nutrients $m$:
\[
\sum_{k=1}^{K} nutri_{k,m} \cdot quantity_k \geq demand_m \quad \forall m = 1, \ldots, M
\]

\subsection*{Non-negativity Constraints}
\[
quantity_k \geq 0 \quad \forall k = 1, \ldots, K
\]

\subsection*{Summary}
The linear programming model can be summarized as follows:

\begin{align*}
\text{Minimize } & Z = \sum_{k=1}^{K} price_k \cdot quantity_k \\
\text{Subject to } & \sum_{k=1}^{K} nutri_{k,m} \cdot quantity_k \geq demand_m, \quad \forall m = 1, \ldots, M \\
                   & quantity_k \geq 0, \quad \forall k = 1, \ldots, K
\end{align*}

\end{document}