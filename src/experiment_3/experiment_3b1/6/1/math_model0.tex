\documentclass{article}
\usepackage{amsmath}
\begin{document}

\title{Linear Programming Model for Maximizing Revenue}
\author{}
\date{}
\maketitle

\section*{Problem Description}

A firm produces \( M \) different goods using \( N \) different raw materials. The firm has \( \text{available}_i \) of raw material \( i \) available. Good \( j \) requires \( \text{req}_{i,j} \) units of material \( i \) per unit produced. Good \( j \) results in a revenue of \( \text{price}_j \) per unit produced.

\section*{Input Format}

Let:
\begin{itemize}
    \item \( \text{available} = [\text{available}_1, \text{available}_2, \ldots, \text{available}_N] \)
    \item \( \text{requirements} = \left[ \left[ \text{req}_{1,j}, \text{req}_{2,j}, \ldots, \text{req}_{N,j} \right] \text{ for } j = 1, \ldots, M \right] \)
    \item \( \text{prices} = [\text{price}_1, \text{price}_2, \ldots, \text{price}_M] \)
\end{itemize}

\section*{Objective}

Maximize the total revenue, which can be mathematically formulated as follows:

\[
\text{Maximize } Z = \sum_{j=1}^{M} \text{price}_j \cdot \text{amount}_j
\]

\section*{Subject to}

The constraints based on the availability of raw materials can be expressed as:

\[
\sum_{j=1}^{M} \text{req}_{i,j} \cdot \text{amount}_j \leq \text{available}_i \quad \forall i = 1, 2, \ldots, N
\]

\section*{Non-negativity Constraint}

Additionally, the production amounts must be non-negative:

\[
\text{amount}_j \geq 0 \quad \forall j = 1, 2, \ldots, M
\]

\section*{Output Format}

The solution to the optimization problem will yield the amount of each good to produce:

\[
\text{Output} = \{ \text{amount} : [\text{amount}_1, \text{amount}_2, \ldots, \text{amount}_M] \}
\]

\end{document}