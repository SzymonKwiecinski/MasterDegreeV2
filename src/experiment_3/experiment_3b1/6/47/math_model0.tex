\documentclass{article}
\usepackage{amsmath}
\begin{document}

\section*{Police Officer Shift Assignment Problem}

\textbf{Sets:}
\begin{itemize}
    \item Let \( S \) be the number of different shifts.
    \item Let \( s \in \{1, 2, \ldots, S\} \) represent the index of the shift.
\end{itemize}

\textbf{Parameters:}
\begin{itemize}
    \item \( \text{officers\_needed}_{s} \): Number of police officers needed during shift \( s \).
    \item \( \text{shift\_cost}_{s} \): Cost for the town when starting a shift at \( s \).
\end{itemize}

\textbf{Decision Variables:}
\begin{itemize}
    \item Let \( x_{s} \) be the number of police officers assigned to shift \( s \).
\end{itemize}

\textbf{Objective:}
Minimize the total cost:
\[
\text{Minimize } Z = \sum_{s=1}^{S} \text{shift\_cost}_{s} \cdot x_{s}
\]

\textbf{Constraints:}
\begin{itemize}
    \item Each police officer works for two consecutive shifts:
    \[
    x_{s} + x_{s+1} \geq \text{officers\_needed}_{s}, \quad s = 1, 2, \ldots, S-1
    \]
    \item For the last shift:
    \[
    x_{S} \geq \text{officers\_needed}_{S}
    \]
\end{itemize}

\textbf{Non-negativity:}
\[
x_{s} \geq 0, \quad s = 1, 2, \ldots, S
\]

\end{document}