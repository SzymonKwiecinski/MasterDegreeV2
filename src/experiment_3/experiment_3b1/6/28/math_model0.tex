\documentclass{article}
\usepackage{amsmath}
\begin{document}

\section*{Linear Programming Model for Best Fit Line}

Given a set of corresponding values for \(y\) and \(x\):

\[
y = [y_k \text{ for } k = 1, \ldots, K]
\]
\[
x = [x_k \text{ for } k = 1, \ldots, K]
\]

we aim to fit the best straight line of the form:

\[
y = bx + a
\]

The objective is to minimize the sum of absolute deviations of each observed value of \(y\) from the value predicted by the linear relationship. This can be expressed mathematically as:

\[
\text{Minimize } Z = \sum_{k=1}^{K} |y_k - (bx_k + a)|
\]

To solve this using linear programming, we introduce auxiliary variables \(d_k\) to represent the absolute deviations:

\[
d_k \geq y_k - (bx_k + a) \quad \text{for } k = 1, \ldots, K
\]
\[
d_k \geq -(y_k - (bx_k + a)) \quad \text{for } k = 1, \ldots, K
\]

Thus, the problem can be reformulated as follows:

\[
\text{Minimize } Z = \sum_{k=1}^{K} d_k
\]

Subject to:

\[
d_k \geq y_k - (bx_k + a) \quad \forall k
\]
\[
d_k \geq -(y_k - (bx_k + a)) \quad \forall k
\]

The variables are:

- \(a\): intercept of the fitted line
- \(b\): slope of the fitted line
- \(d_k\): deviation for each observation

The final output consists of the computed values of the intercept and slope, which are represented as follows:

\[
\text{Output: } \{ \text{"intercept": } a, \text{"slope": } b \}
\]

\end{document}