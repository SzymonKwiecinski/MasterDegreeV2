\documentclass{article}
\usepackage{amsmath}
\begin{document}

\section*{Linear Programming Model}

\subsection*{Variables}
Let:
\begin{itemize}
    \item \( N \) = number of different stocks
    \item \( bought_i \) = number of shares bought of stock \( i \)
    \item \( buyPrice_i \) = price per share of stock \( i \) when bought
    \item \( currentPrice_i \) = current price per share of stock \( i \)
    \item \( futurePrice_i \) = expected price per share of stock \( i \) in one year
    \item \( transactionRate \) = transaction cost rate (as a percentage)
    \item \( taxRate \) = tax rate on capital gains (as a percentage)
    \item \( K \) = amount of money the investor wants to raise net of capital gains and transaction costs
    \item \( sell_i \) = number of shares of stock \( i \) that the investor should sell
\end{itemize}

\subsection*{Objective}
Maximize the expected value of the portfolio next year:
\[
\text{Maximize } Z = \sum_{i=1}^{N} \left( futurePrice_i \cdot (bought_i - sell_i) \right)
\]

\subsection*{Constraints}
1. The amount received from selling shares, net of transaction costs and taxes, should be at least \( K \):
\[
\sum_{i=1}^{N} \left( currentPrice_i \cdot sell_i \cdot (1 - \frac{transactionRate}{100}) - (currentPrice_i - buyPrice_i) \cdot sell_i \cdot \frac{taxRate}{100} \right) \geq K
\]

2. The number of shares sold cannot exceed the number of shares owned:
\[
sell_i \leq bought_i, \quad \forall i \in \{1, \ldots, N\}
\]

3. The number of shares sold must be non-negative:
\[
sell_i \geq 0, \quad \forall i \in \{1, \ldots, N\}
\]

\subsection*{Output}
The solution will yield the optimal number of shares to sell for each stock \( i \):
\[
\text{Output: } \{ "sell": [sell_1, sell_2, \ldots, sell_N] \}
\]

\end{document}