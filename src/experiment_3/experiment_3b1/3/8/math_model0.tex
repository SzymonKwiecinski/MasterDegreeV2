\documentclass{article}
\usepackage{amsmath}
\begin{document}

\title{Linear Programming Model for Spare Automobile Parts Production}
\author{}
\date{}
\maketitle

\section*{Problem Formulation}

Let:
\begin{itemize}
    \item \( K \): Number of different spare parts
    \item \( S \): Number of shops
    \item \( \text{time}_{k, s} \): Worker-hours required to make part \( k \) in shop \( s \)
    \item \( \text{profit}_{k} \): Profit from part \( k \)
    \item \( \text{capacity}_{s} \): Capacity in working hours of shop \( s \)
    \item \( x_{k} \): Quantity of part \( k \) to be produced
\end{itemize}

\subsection*{Objective Function}

We aim to maximize the total profit:

\[
\text{Maximize } Z = \sum_{k=1}^{K} \text{profit}_{k} \cdot x_{k}
\]

\subsection*{Constraints}

The production of parts is limited by the capacity of each shop. For each shop \( s \), the constraint is given by:

\[
\sum_{k=1}^{K} \text{time}_{k,s} \cdot x_{k} \leq \text{capacity}_{s}  \quad \forall s = 1, 2, \ldots, S
\]

In addition, we need to ensure that the quantities produced are non-negative:

\[
x_{k} \geq 0  \quad \forall k = 1, 2, \ldots, K
\]

\subsection*{Decision Variables}

The decision variables for this model are the quantities of each spare part to be made:

\[
\text{quantity} = [x_{k} \text{ for } k = 1, \ldots, K]
\]

\end{document}