\documentclass{article}
\usepackage{amsmath}
\begin{document}

\section*{Linear Programming Formulation}

\subsection*{Variables}
Let \( x_k \) be the amount of alloy \( k \) to be purchased, for \( k = 1, \ldots, K \).

\subsection*{Parameters}
\begin{itemize}
    \item \( \text{alloy\_quant} \): Total weight of alloy to be produced (in lbs).
    \item \( \text{target}_{m} \): Target weight of metal \( m \) in the alloy (for \( m = 1, \ldots, M \)).
    \item \( \text{ratio}_{k, m} \): Ratio of metal \( m \) in alloy \( k \).
    \item \( \text{price}_{k} \): Price of alloy \( k \) per pound.
\end{itemize}

\subsection*{Objective Function}
Minimize the total cost:
\[
\text{Minimize } Z = \sum_{k=1}^{K} \text{price}_{k} \cdot x_k
\]

\subsection*{Constraints}
We need to ensure that the total weight of the alloy equals the desired weight and that we have the right amounts of each metal:
\begin{align*}
\sum_{k=1}^{K} x_k &= \text{alloy\_quant} \quad \text{(Total amount of alloys)} \\
\sum_{k=1}^{K} \text{ratio}_{k, m} \cdot x_k &\geq \text{target}_{m} \quad \text{for } m = 1, \ldots, M \quad \text{(Metal requirements)}
\end{align*}

\subsection*{Non-negativity Constraints}
\[
x_k \geq 0 \quad \text{for } k = 1, \ldots, K
\]

\end{document}