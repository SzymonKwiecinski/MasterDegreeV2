\documentclass{article}
\usepackage{amsmath}
\begin{document}

\section*{Linear Programming Model for Oil Refinery Production}

\subsection*{Parameters}
\begin{align*}
\text{Let } O & \text{ be the number of crude oil types.} \\
\text{Let } P & \text{ be the number of products.} \\
\text{Let } L & \text{ be the number of production processes.} \\
\text{Let } \text{allocated}_i & \text{ be the quantity of crude oil type } i \text{ (in million barrels).} \\
\text{Let } \text{price}_p & \text{ be the selling price of product } p \text{ (in dollars per barrel).} \\
\text{Let } \text{input}_{l,i} & \text{ be the quantity of crude oil type } i \text{ needed for process } l. \\
\text{Let } \text{output}_{l,p} & \text{ be the quantity of product } p \text{ produced by process } l. \\
\text{Let } \text{cost}_l & \text{ be the cost per barrel for process } l.
\end{align*}

\subsection*{Decision Variables}
\begin{align*}
\text{Let } x_l & \text{ be the number of times process } l \text{ is executed.}
\end{align*}

\subsection*{Objective Function}
Maximize total revenue:
\[
\text{revenue} = \sum_{p=1}^{P} \text{price}_p \cdot \sum_{l=1}^{L} \text{output}_{l,p} \cdot x_l
\]

\subsection*{Constraints}
1. Crude oil availability:
\[
\sum_{l=1}^{L} \text{input}_{l,i} \cdot x_l \leq \text{allocated}_i, \quad \forall i \in \{1, \ldots, O\}
\]

2. Non-negativity:
\[
x_l \geq 0, \quad \forall l \in \{1, \ldots, L\}
\]

\subsection*{Output}
The output of the model will provide:
\begin{align*}
\text{revenue} & \text{ is the total revenue for the month.} \\
\text{execute}_l & \text{ is the number of times that process } l \text{ should be executed.}
\end{align*}

\end{document}