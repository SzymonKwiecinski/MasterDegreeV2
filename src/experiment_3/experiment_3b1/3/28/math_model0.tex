\documentclass{article}
\usepackage{amsmath}
\begin{document}

\section*{Linear Programming Model for Fitting a Line}

Given the data points \( (x_k, y_k) \) for \( k = 1, \ldots, K \), we want to fit the best straight line of the form:

\[
y = bx + a
\]

The objective is to minimize the sum of absolute deviations of each observed value of \( y \) from the predicted values:

\[
\text{Minimize} \quad Z = \sum_{k=1}^{K} |y_k - (bx_k + a)|
\]

To handle the absolute values, we can introduce non-negative slack variables \( t_k \) for each deviation:

\[
t_k \geq y_k - (bx_k + a) \quad \forall k
\]

\[
t_k \geq -(y_k - (bx_k + a)) \quad \forall k
\]

This leads us to the following linear programming formulation:

\textbf{Objective:}
\[
\text{Minimize} \quad Z = \sum_{k=1}^{K} t_k
\]

\textbf{Subject to:}
\[
t_k \geq y_k - (bx_k + a) \quad \forall k
\]
\[
t_k \geq -(y_k - (bx_k + a)) \quad \forall k
\]
\[
t_k \geq 0 \quad \forall k
\]

\textbf{Variables:}
- \( a \): Intercept of the fitted line
- \( b \): Slope of the fitted line
- \( t_k \): Non-negative slack variables for each \( k \)

Once the linear program is solved, the fitted line parameters can be extracted as:

\[
\text{intercept} = a
\]
\[
\text{slope} = b
\]

\end{document}