\documentclass{article}
\usepackage{amsmath}
\begin{document}

\section*{Linear Programming Model}

\subsection*{Parameters}
\begin{itemize}
    \item Let \( K \) be the number of different foods.
    \item Let \( M \) be the number of different nutritional ingredients.
    \item Let \( \text{price}_{k} \) be the price per unit of food \( k \).
    \item Let \( \text{demand}_{m} \) be the minimum requirement of nutrient \( m \) per day.
    \item Let \( \text{nutri}_{k, m} \) be the amount of nutrient \( m \) in food \( k \).
\end{itemize}

\subsection*{Decision Variables}
\begin{itemize}
    \item Let \( x_{k} \) be the quantity of food \( k \) to purchase.
\end{itemize}

\subsection*{Objective Function}
We aim to minimize the total cost of food purchased:
\[
\text{Minimize } Z = \sum_{k=1}^{K} \text{price}_{k} \cdot x_{k}
\]

\subsection*{Constraints}
Each individual must receive at least the required amount of each nutrient:
\[
\sum_{k=1}^{K} \text{nutri}_{k, m} \cdot x_{k} \geq \text{demand}_{m}, \quad m = 1, \ldots, M
\]

\subsection*{Non-negativity Constraints}
\[
x_{k} \geq 0, \quad k = 1, \ldots, K
\]

\subsection*{Summary}
The complete linear programming model can be summarized as:
\[
\begin{align*}
\text{Minimize } & Z = \sum_{k=1}^{K} \text{price}_{k} \cdot x_{k} \\
\text{subject to } & \sum_{k=1}^{K} \text{nutri}_{k, m} \cdot x_{k} \geq \text{demand}_{m}, \quad m = 1, \ldots, M \\
& x_{k} \geq 0, \quad k = 1, \ldots, K
\end{align*}
\]

\end{document}