\documentclass{article}
\usepackage{amsmath}
\begin{document}

\section*{Mathematical Model for Electricity Capacity Expansion}

\subsection*{Parameters}
\begin{itemize}
    \item \( T \): Number of years.
    \item \( d_t \): Demand for electricity in year \( t \), for \( t = 1, \ldots, T \).
    \item \( oil_t \): Existing oil-fired capacity available in year \( t \).
    \item \( coal\_cost \): Capital cost per megawatt of coal-fired capacity.
    \item \( nuke\_cost \): Capital cost per megawatt of nuclear power capacity.
    \item \( max\_nuke \): Maximum percentage of total capacity that can be nuclear.
    \item \( coal\_life \): Lifespan of coal plants (in years).
    \item \( nuke\_life \): Lifespan of nuclear plants (in years).
\end{itemize}

\subsection*{Decision Variables}
\begin{itemize}
    \item \( coal_t \): Capacity of coal added in year \( t \).
    \item \( nuke_t \): Capacity of nuclear added in year \( t \).
\end{itemize}

\subsection*{Objective Function}
Minimize the total cost of capacity expansion:
\[
\text{Minimize } Z = \sum_{t=1}^{T} (coal\_cost \cdot coal_t + nuke\_cost \cdot nuke_t)
\]

\subsection*{Constraints}
1. Capacity demand constraints:
\[
oil_t + \sum_{j=1}^{\min(t, coal\_life)} coal_{t-j} + \sum_{j=1}^{\min(t, nuke\_life)} nuke_{t-j} \geq d_t, \quad \forall t = 1, \ldots, T
\]
2. Maximum nuclear capacity constraint:
\[
\sum_{j=1}^{\min(t, coal\_life)} coal_{t-j} + \sum_{j=1}^{\min(t, nuke\_life)} nuke_{t-j} \cdot \frac{nuke\_cost}{coal\_cost} \leq \frac{max\_nuke}{100} \cdot (oil_t + \sum_{j=1}^{\min(t, coal\_life)} coal_{t-j} + \sum_{j=1}^{\min(t, nuke\_life)} nuke_{t-j}), \quad \forall t = 1, \ldots, T
\]

\subsection*{Output Format}
The output should provide the following information:
\begin{itemize}
    \item \texttt{coal\_cap\_added}: List of coal capacity added each year.
    \item \texttt{nuke\_cap\_added}: List of nuclear capacity added each year.
    \item \texttt{total\_cost}: Total cost of the capacity expansion.
\end{itemize}

\end{document}