\documentclass{article}
\usepackage{amsmath}
\begin{document}

\section*{Linear Programming Model}

\subsection*{Indices}
\begin{itemize}
    \item Let \( j \) be the index for goods, where \( j = 1, 2, \ldots, M \).
    \item Let \( i \) be the index for raw materials, where \( i = 1, 2, \ldots, N \).
\end{itemize}

\subsection*{Parameters}
\begin{itemize}
    \item \( \text{available}_i \): Available amount of raw material \( i \).
    \item \( \text{req}_{i,j} \): Amount of raw material \( i \) required to produce one unit of good \( j \).
    \item \( \text{price}_j \): Revenue from selling one unit of good \( j \).
\end{itemize}

\subsection*{Decision Variables}
\begin{itemize}
    \item \( x_j \): Amount of good \( j \) to produce.
\end{itemize}

\subsection*{Objective Function}
Maximize the total revenue:
\[
\text{Maximize} \quad Z = \sum_{j=1}^{M} \text{price}_j \cdot x_j
\]

\subsection*{Constraints}
The constraints are based on the available raw materials:
\[
\sum_{j=1}^{M} \text{req}_{i,j} \cdot x_j \leq \text{available}_i \quad \forall i = 1, 2, \ldots, N
\]

Additionally, the decision variables must be non-negative:
\[
x_j \geq 0 \quad \forall j = 1, 2, \ldots, M
\]

\subsection*{Output}
The solution will yield the optimal amounts of each good to produce:
\[
\text{Output:} \quad \{ x_j \text{ for } j = 1, 2, \ldots, M \}
\]

\end{document}