\documentclass{article}
\usepackage{amsmath}
\begin{document}

\section*{Linear Programming Model for Auto Parts Manufacturer}

\subsection*{Variables}
Let \( b_p \) be the number of batches produced for part \( p \) where \( p = 1, \ldots, P \).

\subsection*{Objective Function}
To maximize profit, we define the objective function as:
\[
\text{Maximize } Z = \sum_{p=1}^{P} (price_p \cdot b_p) - \sum_{m=1}^{M} (cost_m \cdot \sum_{p=1}^{P} (time_{m,p} \cdot b_p))
\]

\subsection*{Constraints}
1. **Time Availability Constraints**:
   \[
   \sum_{p=1}^{P} (time_{m,p} \cdot b_p) \leq available_m \quad \text{for each } m = 1, \ldots, M
   \]
   Note: For machines \( M \) and \( M-1 \), these two machines can share their availability:
   \[
   \sum_{p=1}^{P} (time_{M,p} \cdot b_p) + \sum_{p=1}^{P} (time_{M-1,p} \cdot b_p) \leq available_M + available_{M-1}
   \]

2. **Minimum Production Requirements**:
   \[
   b_p \geq min\_batches_p \quad \text{for each } p = 1, \ldots, P
   \]

3. **Non-Negativity Constraints**:
   \[
   b_p \geq 0 \quad \text{for each } p = 1, \ldots, P
   \]

\subsection*{Output}
The solution will provide:
\begin{itemize}
    \item The number of batches of each part produced: \( batches = [b_1, b_2, \ldots, b_P] \)
    \item The total profit: \( total\_profit = Z \)
\end{itemize}

\end{document}