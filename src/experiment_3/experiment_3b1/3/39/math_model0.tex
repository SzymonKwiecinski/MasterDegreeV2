\documentclass{article}
\usepackage{amsmath}
\begin{document}

\section*{Mathematical Model for Employee Scheduling}

\subsection*{Parameters}
\begin{itemize}
    \item Let \( N \) be the total number of days.
    \item Let \( \text{num}_{n} \) be the desired number of employees on day \( n \) for \( n = 1, \ldots, N \).
    \item Let \( n_{\text{working}} \) be the number of consecutive working days for each employee.
    \item Let \( n_{\text{resting}} \) be the number of consecutive resting days for each employee.
\end{itemize}

\subsection*{Decision Variables}
\begin{itemize}
    \item Let \( \text{total\_number} \) be the total number of employees to hire.
    \item Let \( \text{is\_work}_{n, i} \) be a binary variable, where \( \text{is\_work}_{n, i} = 1 \) if employee \( i \) is working on day \( n \) and \( 0 \) otherwise.
\end{itemize}

\subsection*{Objective Function}
Minimize the total number of employees:
\[
\text{minimize} \quad \text{total\_number}
\]

\subsection*{Constraints}
1. Ensure that the number of employees working on each day meets the requirement:
\[
\sum_{i=1}^{\text{total\_number}} \text{is\_work}_{n, i} \geq \text{num}_{n}, \quad \forall n \in \{1, \ldots, N\}
\]

2. Define the working and resting cycle for each employee:
\[
\text{is\_work}_{n, i} = 1 \quad \text{for } n = k(n_{\text{working}} + n_{\text{resting}}) + j, \quad \text{with } j = 0, \ldots, n_{\text{working}} - 1
\]
\[
\text{is\_work}_{n, i} = 0 \quad \text{for } n = k(n_{\text{working}} + n_{\text{resting}}) + j, \quad \text{with } j = n_{\text{working}}, \ldots, n_{\text{working}} + n_{\text{resting}} - 1
\]
where \( k \) is a non-negative integer determined by the employee's work cycle.

\end{document}