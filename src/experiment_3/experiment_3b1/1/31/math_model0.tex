\documentclass{article}
\usepackage{amsmath}
\begin{document}

\title{Mathematical Model for Power Generation Cost Minimization}
\author{}
\date{}
\maketitle

\section*{Problem Definition}

We aim to minimize the total operational cost of power generation, given the specified demands and constraints of various generating units over a day. 

Let:
- \( T \) be the number of periods in a day.
- \( K \) be the number of types of generating units.

\subsection*{Variables}

Define the following decision variables:
\begin{itemize}
    \item \( numon_{k, t} \): The number of \( k \)-th type generators that are operational at time period \( t \).
    \item \( level_{k, t} \): The amount of power (in MW) generated by \( k \)-th type generators at time period \( t \).
    \item \( start_{k, t} \): A binary variable indicating if the generator of type \( k \) is started at time period \( t \) (1 if started, 0 otherwise).
\end{itemize}

\subsection*{Input Parameters}

The parameters are as follows:
\begin{itemize}
    \item \( demand_t \): The electricity demand at time period \( t \).
    \item \( num_k \): The number of available generators of type \( k \).
    \item \( minlevel_k \): The minimum operational level of the \( k \)-th type generator.
    \item \( maxlevel_k \): The maximum operational level of the \( k \)-th type generator.
    \item \( runcost_k \): The hourly cost of running each \( k \)-th type generator at minimum level.
    \item \( extracost_k \): The extra hourly cost for operating above the minimum level for type \( k \).
    \item \( startcost_k \): The cost incurred to start up the \( k \)-th type generator.
\end{itemize}

\subsection*{Objective Function}

The objective is to minimize the total cost:
\[
\text{Minimize} \quad Z = \sum_{t=1}^{T} \sum_{k=1}^{K} \left( runcost_k \cdot numon_{k, t} + startcost_k \cdot start_{k, t} + \frac{1}{2} \cdot extracost_k \cdot (level_{k, t} - minlevel_k) \right)
\]

\subsection*{Constraints}

The constraints involve maintaining the demand and operational limits:
1. Load balance for each time period:
\[
\sum_{k=1}^{K} level_{k, t} = demand_t \quad \forall t
\]

2. Operational limits for each generator type:
\[
minlevel_k \cdot numon_{k, t} \leq level_{k, t} \leq maxlevel_k \cdot numon_{k, t} \quad \forall k, t
\]

3. Number of units operational:
\[
numon_{k, t} \leq num_k \quad \forall k, t
\]

4. Linking level and number of operational units:
\[
level_{k, t} \leq maxlevel_k \cdot numon_{k, t} \quad \forall k, t
\]

5. Startup condition:
\[
start_{k, t} \leq numon_{k, t} \quad \forall k, t
\]

\subsection*{Output}

The output will represent the number of generators of each type that are operational at each period:
\[
\text{Output:} \quad numon = \left[ \left[numon_{k, t} \; | \; t = 1, \ldots, T \right] \; | \; k = 1, \ldots, K \right]
\]

\end{document}