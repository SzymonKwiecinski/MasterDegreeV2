\documentclass{article}
\usepackage{amsmath}
\begin{document}

\section*{Linear Programming Model for Food Optimization}

\subsection*{Variables}
Let \( x_k \) be the quantity of food \( k \) purchased, where \( k = 1, 2, \ldots, K \).

\subsection*{Parameters}
\begin{itemize}
    \item Let \( \text{price}_k \) be the price per unit of food \( k \).
    \item Let \( \text{demand}_m \) be the required units of nutrient \( m \) per day.
    \item Let \( \text{nutri}_{k,m} \) be the units of nutrient \( m \) contained in one unit of food \( k \).
\end{itemize}

\subsection*{Objective}
The objective is to minimize the total cost of the foods purchased:
\[
\text{Minimize} \quad Z = \sum_{k=1}^{K} \text{price}_k \cdot x_k
\]

\subsection*{Constraints}
To ensure that the nutritional requirements are met, we have the following constraints for each nutrient \( m \):
\[
\sum_{k=1}^{K} \text{nutri}_{k,m} \cdot x_k \geq \text{demand}_m, \quad \text{for } m = 1, 2, \ldots, M
\]

\subsection*{Non-negativity Constraints}
The quantity of each food must be non-negative:
\[
x_k \geq 0, \quad \text{for } k = 1, 2, \ldots, K
\]

\subsection*{Summary}
The problem can be summarized as follows:

\begin{align*}
\text{Minimize} & \quad Z = \sum_{k=1}^{K} \text{price}_k \cdot x_k \\
\text{Subject to} & \quad \sum_{k=1}^{K} \text{nutri}_{k,m} \cdot x_k \geq \text{demand}_m, \quad m = 1, 2, \ldots, M \\
& \quad x_k \geq 0, \quad k = 1, 2, \ldots, K
\end{align*}

\end{document}