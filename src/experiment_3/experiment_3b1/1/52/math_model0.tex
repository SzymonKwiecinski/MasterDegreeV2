\documentclass{article}
\usepackage{amsmath}
\begin{document}

\section*{Linear Programming Model for Electric Utility}

\subsection*{Variables}
Let \( send_{p,c} \) represent the amount of electricity (in million kWh) sent from power plant \( p \) to city \( c \).

\subsection*{Parameters}
\begin{itemize}
    \item \( P \): Number of power plants
    \item \( C \): Number of cities
    \item \( supply_p \): Capacity of power plant \( p \) (in million kWh)
    \item \( demand_c \): Peak demand of city \( c \) (in million kWh)
    \item \( transmission_{p,c} \): Transmission cost from power plant \( p \) to city \( c \) (in dollars per million kWh)
\end{itemize}

\subsection*{Objective Function}
The objective is to minimize the total transmission cost:
\[
\text{Minimize } Z = \sum_{p=1}^{P} \sum_{c=1}^{C} transmission_{p,c} \cdot send_{p,c}
\]

\subsection*{Constraints}
1. Supply Constraints:
\[
\sum_{c=1}^{C} send_{p,c} \leq supply_p \quad \forall p = 1, \ldots, P
\]
2. Demand Constraints:
\[
\sum_{p=1}^{P} send_{p,c} \geq demand_c \quad \forall c = 1, \ldots, C
\]
3. Non-negativity Constraints:
\[
send_{p,c} \geq 0 \quad \forall p = 1, \ldots, P \text{ and } c = 1, \ldots, C
\]

\subsection*{Output}
The output will consist of:
\begin{itemize}
    \item \( send_{p,c} \): The amount of electricity sent from power plant \( p \) to city \( c \) (in million kWh)
    \item \( total\_cost \): The total transmission cost for the utility
\end{itemize}

\end{document}