\documentclass{article}
\usepackage{amsmath}
\begin{document}

\section*{Linear Programming Model for Police Officer Shift Assignment}

\subsection*{Indices}
\begin{itemize}
    \item $s$: Index for shifts, where $s = 1, 2, \ldots, S$.
\end{itemize}

\subsection*{Parameters}
\begin{itemize}
    \item $officers\_needed_{s}$: Number of police officers needed during shift $s$.
    \item $shift\_cost_{s}$: Cost of starting shift $s$.
\end{itemize}

\subsection*{Decision Variables}
\begin{itemize}
    \item $officers\_assigned_{s}$: Number of police officers assigned to shift $s$.
\end{itemize}

\subsection*{Objective Function}
Minimize the total cost:
\[
\text{Minimize } Z = \sum_{s=1}^{S} shift\_cost_{s} \cdot x_s
\]
where \( x_s \) is the binary variable indicating whether shift \( s \) is active (1) or not (0), and is defined as:
\[
x_s = 1 \text{ if } officers\_assigned_{s} > 0 \text{, else } 0
\]

\subsection*{Constraints}
The constraints to satisfy the officer requirements are given by:
\[
officers\_assigned_{s} + officers\_assigned_{s-1} \geq officers\_needed_{s}, \quad \forall s = 2, 3, \ldots, S
\]
(for $s = 1$, we can consider it only involving the second shift due to the nature of the problem)

Also, each officer can only work two consecutive shifts:
\[
officers\_assigned_{s} \geq 0, \quad \forall s = 1, 2, \ldots, S
\]
\[
x_s \in \{0, 1\}
\]

\subsection*{Output Variables}
\begin{itemize}
    \item $officers\_assigned$: The number of officers assigned to each shift.
    \item $total\_cost$: Total cost for the town.
\end{itemize}

\end{document}