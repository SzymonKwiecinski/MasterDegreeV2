\documentclass{article}
\usepackage{amsmath}
\begin{document}

\section*{Linear Programming Model for Spare Automobile Parts Production}

\subsection*{Indices}
\begin{itemize}
    \item $k$: Index for spare parts, where $k = 1, \ldots, K$.
    \item $s$: Index for shops, where $s = 1, \ldots, S$.
\end{itemize}

\subsection*{Parameters}
\begin{itemize}
    \item $time_{k,s}$: Required worker-hours for part $k$ in shop $s$.
    \item $profit_{k}$: Profit of part $k$.
    \item $capacity_{s}$: Capacity of shop $s$ in working hours.
\end{itemize}

\subsection*{Decision Variables}
\begin{itemize}
    \item $quantity_{k}$: Quantity of spare part $k$ to be produced.
\end{itemize}

\subsection*{Objective Function}
Maximize the total profit:
\[
\text{Maximize } Z = \sum_{k=1}^{K} profit_{k} \cdot quantity_{k}
\]

\subsection*{Constraints}
The shop capacity constraints can be formulated as follows:
\[
\sum_{k=1}^{K} time_{k,s} \cdot quantity_{k} \leq capacity_{s}, \quad \forall s = 1, \ldots, S
\]

\subsection*{Non-negativity Constraints}
\[
quantity_{k} \geq 0, \quad \forall k = 1, \ldots, K
\]

\subsection*{Complete Model}
The complete linear programming formulation can be summarized as follows:

\begin{align*}
\text{Maximize } & Z = \sum_{k=1}^{K} profit_{k} \cdot quantity_{k} \\
\text{Subject to:} \quad & \sum_{k=1}^{K} time_{k,s} \cdot quantity_{k} \leq capacity_{s}, \quad \forall s = 1, \ldots, S \\
& quantity_{k} \geq 0, \quad \forall k = 1, \ldots, K
\end{align*}

\end{document}