\documentclass{article}
\usepackage{amsmath}
\begin{document}

\section*{Linear Programming Model for Alloy Production}

\subsection*{Parameters}
\begin{itemize}
    \item Let \( A \) be the total quantity of the alloy to be produced: \( A = \text{alloy\_quant} \)
    \item Let \( M \) be the number of metals: \( M = \text{len(target)} \)
    \item Let \( K \) be the number of available alloys: \( K = \text{len(price)} \)
    \item Let \( t_m \) be the target quantity of metal \( m \): \( t_m = \text{target}[m], \; m = 1, \ldots, M \)
    \item Let \( r_{k, m} \) be the ratio of metal \( m \) in alloy \( k \): \( r_{k, m} = \text{ratio}[k][m], \; k = 1, \ldots, K, \; m = 1, \ldots, M \)
    \item Let \( p_k \) be the price of alloy \( k \): \( p_k = \text{price}[k], \; k = 1, \ldots, K \)
    \item Let \( x_k \) be the amount of alloy \( k \) to be purchased: \( x_k = \text{amount}[k], \; k = 1, \ldots, K \)
\end{itemize}

\subsection*{Objective Function}
Minimize the total cost of alloys purchased:
\[
\text{Minimize} \quad Z = \sum_{k=1}^{K} p_k x_k
\]

\subsection*{Constraints}
1. The total weight of the alloys must equal \( A \):
\[
\sum_{k=1}^{K} x_k = A
\]

2. The amount of each metal \( m \) in the alloys must meet the target requirement:
\[
\sum_{k=1}^{K} r_{k, m} x_k = t_m, \quad m = 1, \ldots, M
\]

3. Non-negativity constraints:
\[
x_k \geq 0, \quad k = 1, \ldots, K
\]

\subsection*{Output}
The output will be the amounts of each alloy to be purchased:
\[
\text{amount} = [x_1, x_2, \ldots, x_K]
\]

\end{document}