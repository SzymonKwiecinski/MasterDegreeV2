\documentclass{article}
\usepackage{amsmath}
\begin{document}

\title{MILP Model for Night Shift Nurse Scheduling}
\author{}
\date{}
\maketitle

\section*{Problem Formulation}

\subsection*{Parameters}
\begin{itemize}
    \item Let $N$ be the total number of nurses hired.
    \item Let $d_j$ be the demand for nurses on day $j$, for $j = 1, \ldots, 7$.
    \item Let $period$ be the number of consecutive days a nurse works the night shift.
\end{itemize}

\subsection*{Decision Variables}
\begin{itemize}
    \item Let $start_j$ be the number of nurses that start their shift on day $j$, for $j = 1, \ldots, 7$.
\end{itemize}

\subsection*{Objective Function}
The goal is to minimize the total number of nurses hired:

\[
\text{Minimize } N = \sum_{j=1}^{7} start_j
\]

\subsection*{Constraints}
For each day $j$, the total number of nurses working the night shift must meet the demand:

\[
\sum_{i=0}^{period - 1} start_{(j - i + 7) \mod 7} \geq d_j, \quad \forall j = 1, \ldots, 7
\]

Where $start_{(j - i + 7) \mod 7}$ represents the number of nurses starting their shift on the day that satisfies the period of work.

We also have the non-negativity constraint:

\[
start_j \geq 0 \quad \forall j = 1, \ldots, 7
\]

\subsection*{Output}
The output will be as follows:

\begin{verbatim}
{
    "start": [start_1, start_2, start_3, start_4, start_5, start_6, start_7],
    "total": N
}
\end{verbatim}

\end{document}