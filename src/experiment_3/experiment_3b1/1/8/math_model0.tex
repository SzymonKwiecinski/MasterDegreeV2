\documentclass{article}
\usepackage{amsmath}
\begin{document}

\section*{Linear Programming Model for Maximizing Profit in Spare Parts Production}

\subsection*{Parameters}
\begin{itemize}
    \item Let \( K \) be the number of spare parts.
    \item Let \( S \) be the number of shops.
    \item Let \( time_{k,s} \) be the required worker-hours for part \( k \) in shop \( s \).
    \item Let \( profit_k \) be the profit for part \( k \).
    \item Let \( capacity_s \) be the capacity of shop \( s \) in working hours.
\end{itemize}

\subsection*{Decision Variables}
Let \( quantity_k \) be the quantity of part \( k \) to be produced, for \( k = 1, \ldots, K \).

\subsection*{Objective Function}
The objective is to maximize total profit:
\[
\text{Maximize} \quad Z = \sum_{k=1}^{K} profit_k \cdot quantity_k
\]

\subsection*{Constraints}
The production in each shop must not exceed its capacity. For each shop \( s \):
\[
\sum_{k=1}^{K} time_{k,s} \cdot quantity_k \leq capacity_s, \quad \text{for } s = 1, \ldots, S
\]

Additionally, the quantities produced must be non-negative:
\[
quantity_k \geq 0, \quad \text{for } k = 1, \ldots, K
\]

\subsection*{Final Model}
Thus, the complete linear programming model can be summarized as follows:
\begin{align*}
\text{Maximize} & \quad Z = \sum_{k=1}^{K} profit_k \cdot quantity_k \\
\text{subject to} & \quad \sum_{k=1}^{K} time_{k,s} \cdot quantity_k \leq capacity_s, \quad \forall s = 1, \ldots, S \\
& \quad quantity_k \geq 0, \quad \forall k = 1, \ldots, K
\end{align*}

\end{document}