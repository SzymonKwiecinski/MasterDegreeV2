\documentclass{article}
\usepackage{amsmath}
\begin{document}

\section*{Problem Formulation}

We need to find the Chebyshev center of a set \(P\) defined by the following constraints:

\[
P = \{ \mathbf{x} \in \mathbb{R}^N \mid \mathbf{a}_i^T \mathbf{x} \leq b_i, \, i = 1, \ldots, m \}
\]

where \(\mathbf{a}_i \in \mathbb{R}^N\) and \(b_i \in \mathbb{R}\).

The Chebyshev center is a point \(\mathbf{y} \in \mathbb{R}^N\) such that a ball of radius \(r\) centered at \(\mathbf{y}\) is entirely contained within the set \(P\). The radius \(r\) of the largest such ball can be expressed mathematically as:

\[
r = \min_{i=1, \ldots, m} \frac{b_i - \mathbf{a}_i^T \mathbf{y}}{\|\mathbf{a}_i\|_2}
\]

This ensures that the distance from the center \(\mathbf{y}\) to the boundary defined by each constraint \(\mathbf{a}_i^T \mathbf{x} \leq b_i\) is maximized.

\subsection*{Input Format}

The input is given as follows:

\[
\text{input} = \{
    \mathbf{A} = \begin{bmatrix}
    a_{1,1} & a_{1,2} & \ldots & a_{1,N} \\
    a_{2,1} & a_{2,2} & \ldots & a_{2,N} \\
    \vdots & \vdots & \ddots & \vdots \\
    a_{m,1} & a_{m,2} & \ldots & a_{m,N}
    \end{bmatrix},
    \mathbf{b} = \begin{bmatrix}
    b_1 \\
    b_2 \\
    \vdots \\
    b_m
    \end{bmatrix}
\}
\]

\subsection*{Output Format}

The output should include the center and radius of the largest ball contained in \(P\):

\[
\text{output} = \{
    \text{center} = [y_1, y_2, \ldots, y_N],
    \text{radius} = r
\}
\]

\end{document}