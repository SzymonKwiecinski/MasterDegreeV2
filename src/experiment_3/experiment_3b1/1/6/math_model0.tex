\documentclass{article}
\usepackage{amsmath}
\begin{document}

\section*{Rocket Motion Optimization Model}

\subsection*{Variables}
Let:
\begin{itemize}
    \item $x_t$: Position of the rocket at time $t$
    \item $v_t$: Velocity of the rocket at time $t$
    \item $a_t$: Acceleration of the rocket at time $t$
\end{itemize}

\subsection*{Initial Conditions}
The initial conditions of the rocket are given by:
\begin{align*}
    x_0 & \text{ (initial position)} \\
    v_0 & \text{ (initial velocity)} 
\end{align*}

\subsection*{Target Conditions}
The target conditions to be achieved at time $T$ are:
\begin{align*}
    x_T & \text{ (target position)} \\
    v_T & \text{ (target velocity)} 
\end{align*}

\subsection*{Discrete-Time Model}
The rocket's motion can be described by the following recursive equations:
\begin{align*}
    x_{t+1} &= x_t + v_t \\
    v_{t+1} &= v_t + a_t
\end{align*}

\subsection*{Fuel Consumption}
The total fuel spent is represented as:
\[
\text{Total Fuel} = \sum_{t=0}^{T-1} |a_t|
\]

\subsection*{Objective}
The objective is to minimize the total fuel spent:
\[
\min \sum_{t=0}^{T-1} |a_t|
\]

\subsection*{Constraints}
The model is subject to the following constraints:
\begin{align*}
    x_0 & = \text{initial position} \\
    v_0 & = \text{initial velocity} \\
    x_T & = \text{target position} \\
    v_T & = \text{target velocity}
\end{align*}

\subsection*{Output}
The desired output format is as follows:
\begin{itemize}
    \item $x = [x_i \text{ for } i = 1, \ldots, T]$
    \item $v = [v_i \text{ for } i = 1, \ldots, T]$
    \item $a = [a_i \text{ for } i = 1, \ldots, T]$
    \item $\text{fuel\_spend} = \text{total fuel spent}$
\end{itemize}

\end{document}