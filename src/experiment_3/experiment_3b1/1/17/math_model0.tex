\documentclass{article}
\usepackage{amsmath}
\begin{document}

\section*{Linear Programming Model for Stock Selling Problem}

\subsection*{Indices}
Let \( i \) denote the index for stocks where \( i = 1, 2, \ldots, N \).

\subsection*{Parameters}
\begin{itemize}
    \item \( \text{bought}_i \): Number of shares bought of stock \( i \)
    \item \( \text{buyPrice}_i \): Purchase price of stock \( i \)
    \item \( \text{currentPrice}_i \): Current price of stock \( i \)
    \item \( \text{futurePrice}_i \): Expected future price of stock \( i \)
    \item \( \text{transactionRate} \): Transaction cost percentage (as a fraction)
    \item \( \text{taxRate} \): Tax percentage on capital gains (as a fraction)
    \item \( K \): Net amount of money the investor wants to raise
\end{itemize}

\subsection*{Decision Variables}
Let \( \text{sell}_i \) represent the number of shares of stock \( i \) that the investor sells.

\subsection*{Objective Function}
We aim to maximize the expected value of the portfolio next year, considering the future price of the stocks, which can be expressed as:

\[
\text{Maximize } Z = \sum_{i=1}^{N} \left( \text{futurePrice}_i \cdot \left( \text{bought}_i - \text{sell}_i \right) \right)
\]

\subsection*{Constraints}
1. The amount raised from selling stocks must cover the desired amount \( K \) net of transaction costs and taxes:

\[
\sum_{i=1}^{N} \left( \text{currentPrice}_i \cdot \text{sell}_i \cdot (1 - \text{transactionRate}) - \text{taxRate} \cdot \left( \text{currentPrice}_i \cdot \text{sell}_i - \text{buyPrice}_i \cdot \text{sell}_i \right) \right) \geq K
\]

2. The number of shares sold must not exceed the number of shares bought:

\[
\text{sell}_i \leq \text{bought}_i \quad \forall i
\]

3. Non-negativity constraints:

\[
\text{sell}_i \geq 0 \quad \forall i
\]

\subsection*{Output}
The output will be the optimal number of shares to sell for each stock \( i \):

\[
\text{Output: } \{\text{sell} = [\text{sell}_1, \text{sell}_2, \ldots, \text{sell}_N]\}
\]

\end{document}