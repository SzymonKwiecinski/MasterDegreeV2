\documentclass{article}
\usepackage{amsmath}
\begin{document}

\section*{Mathematical Model for Translator Selection Problem}

Let:
\begin{itemize}
    \item \( N \) be the number of translators available.
    \item \( M \) be the number of required languages.
    \item \( c_i \) be the cost of hiring translator \( i \).
    \item \( S_i \) be a binary decision variable that equals 1 if translator \( i \) is selected, and 0 otherwise.
    \item \( L_i \) be the set of languages that translator \( i \) can translate.
    \item \( R \) be the set of required languages.
\end{itemize}

The objective is to minimize the total cost of hiring the translators while ensuring that all required languages can be translated.

\subsection*{Objective Function}
\[
\text{Minimize } Z = \sum_{i=1}^{N} c_i S_i
\]

\subsection*{Constraints}
\[
\sum_{i: language_m \in L_i} S_i \geq 1, \quad \forall language_m \in R
\]
This constraint ensures that at least one translator is selected for each required language.

\subsection*{Decision Variables}
\[
S_i \in \{0, 1\}, \quad \forall i = 1, 2, \ldots, N
\]

\end{document}