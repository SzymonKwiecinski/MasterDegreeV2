\documentclass{article}
\usepackage{amsmath}
\begin{document}

\section*{Linear Programming Model for Nutritional Food Allocation}

\subsection*{Sets}
\begin{itemize}
    \item Let \( K \) be the set of foods, indexed by \( k \).
    \item Let \( M \) be the set of nutrients, indexed by \( m \).
\end{itemize}

\subsection*{Parameters}
\begin{itemize}
    \item \( \text{price}_{k} \): the price per unit of food \( k \).
    \item \( \text{demand}_{m} \): the minimum required units of nutrient \( m \).
    \item \( \text{nutri}_{k, m} \): the units of nutrient \( m \) contained in one unit of food \( k \).
\end{itemize}

\subsection*{Decision Variables}
\begin{itemize}
    \item \( \text{quantity}_{k} \): the quantity of food \( k \) to purchase, for \( k \in K \).
\end{itemize}

\subsection*{Objective}
Minimize the total cost of the foods purchased:
\[
\text{Minimize } Z = \sum_{k \in K} \text{price}_{k} \cdot \text{quantity}_{k}
\]

\subsection*{Constraints}
The total amount of each nutrient provided must meet or exceed the demand:
\[
\sum_{k \in K} \text{nutri}_{k, m} \cdot \text{quantity}_{k} \geq \text{demand}_{m}, \quad \forall m \in M
\]

\subsection*{Non-negativity Constraints}
\[
\text{quantity}_{k} \geq 0, \quad \forall k \in K
\]

\end{document}