\documentclass{article}
\usepackage{amsmath}
\begin{document}

\section*{Transportation Problem Model}

\textbf{Parameters:}
\begin{itemize}
    \item Let \( n \) be the number of cities.
    \item Let \( A \) be the set of all routes.
    \item Transport cost per metric ton along route \( (i,j) \) is denoted as \( C_{i,j} \).
    \item Supply at terminal city \( k \) is denoted as \( supply_k \).
    \item Demand at destination city \( l \) is denoted as \( demand_l \).
\end{itemize}

\textbf{Variables:}
\begin{itemize}
    \item Let \( amount_{i,j} \) be the amount of soybeans shipped from city \( i \) to city \( j \).
\end{itemize}

\textbf{Objective Function:}
\[
\text{Minimize } Z = \sum_{(i,j) \in A} C_{i,j} \cdot amount_{i,j}
\]

\textbf{Constraints:}

1. Supply Constraints:
\[
\sum_{j} amount_{k,j} \leq supply_k \quad \forall k
\]

2. Demand Constraints:
\[
\sum_{i} amount_{i,l} \geq demand_l \quad \forall l
\]

3. Non-negativity Constraints:
\[
amount_{i,j} \geq 0 \quad \forall (i,j) \in A
\]

\textbf{Output:}
\begin{itemize}
    \item The distribution of amounts is given by:
    \[
    \text{distribution} = \{ (i, j, amount_{i,j}) \,|\, (i,j) \in A \}
    \]
    \item The total transportation cost is given by:
    \[
    \text{total\_cost} = Z
    \]
\end{itemize}

\end{document}