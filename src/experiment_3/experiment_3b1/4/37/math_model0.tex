\documentclass{article}
\usepackage{amsmath}
\begin{document}

\title{Linear Programming Model for Spare Automobile Parts}
\author{}
\date{}
\maketitle

\section*{Problem Formulation}

Let:
\begin{itemize}
    \item \( K \): Number of different spare parts.
    \item \( S \): Number of different shops.
    \item \( time_{k,s} \): Worker-hours required for part \( k \) in shop \( s \).
    \item \( profit_{k} \): Profit earned from producing part \( k \).
    \item \( capacity_{s} \): Capacity (in worker-hours) of shop \( s \).
    \item \( quantity_{k} \): Amount of spare part \( k \) to be produced.
\end{itemize}

\subsection*{Objective Function}
The objective is to maximize the total profit from producing all spare parts:
\[
\text{Maximize } Z = \sum_{k=1}^{K} profit_{k} \cdot quantity_{k}
\]

\subsection*{Constraints}
The constraints are based on the capacity of each shop. For each shop \( s \), the total worker-hours used should not exceed the capacity:
\[
\sum_{k=1}^{K} time_{k,s} \cdot quantity_{k} \leq capacity_{s}, \quad \forall s \in \{1, 2, \ldots, S\}
\]

Additionally, we cannot produce a negative quantity of any spare part:
\[
quantity_{k} \geq 0, \quad \forall k \in \{1, 2, \ldots, K\}
\]

\subsection*{Final Model}
Putting it all together, the linear programming model can be summarized as follows:

\begin{align*}
\text{Maximize } & Z = \sum_{k=1}^{K} profit_{k} \cdot quantity_{k} \\
\text{Subject to } & \sum_{k=1}^{K} time_{k,s} \cdot quantity_{k} \leq capacity_{s}, \quad \forall s \in \{1, 2, \ldots, S\} \\
& quantity_{k} \geq 0, \quad \forall k \in \{1, 2, \ldots, K\}
\end{align*}

\end{document}