\documentclass{article}
\usepackage{amsmath}
\begin{document}

\title{Linear Programming Model for Alloy Production}
\author{}
\date{}
\maketitle

\section*{Problem Definition}
A manufacturer wishes to produce a total of \( \text{alloy\_quant} \) lb of an alloy that consists of various metals. The desired alloy contains a target quantity of metal \( \text{metal}_m \) by weight, denoted as \( \text{target}_{m} \). Several alloys, each with a specific ratio of metals and price, are available for purchase.

\section*{Parameters}
\begin{itemize}
    \item \( \text{alloy\_quant} \): Total weight of the desired alloy (in lb)
    \item \( \text{target}_{m} \): Target weight of metal \( m \) in the alloy
    \item \( \text{ratio}_{k, m} \): Ratio of metal \( m \) in alloy \( k \)
    \item \( \text{price}_{k} \): Price of alloy \( k \)
\end{itemize}

\section*{Decision Variables}
Let \( \text{amount}_{k} \) be the quantity of alloy \( k \) to purchase.

\section*{Objective Function}
The objective is to minimize the total cost of the alloys purchased:
\[
\text{Minimize} \quad Z = \sum_{k=1}^{K} \text{price}_{k} \cdot \text{amount}_{k}
\]

\section*{Constraints}
1. The total weight of the alloys purchased must equal the target weight:
\[
\sum_{k=1}^{K} \text{amount}_{k} = \text{alloy\_quant}
\]

2. The total amount of each metal \( m \) from the purchased alloys must meet the target for that metal:
\[
\sum_{k=1}^{K} \text{ratio}_{k, m} \cdot \text{amount}_{k} = \text{target}_{m} \quad \forall m = 1, \ldots, M
\]

3. Non-negativity constraints:
\[
\text{amount}_{k} \geq 0 \quad \forall k = 1, \ldots, K
\]

\section*{Output}
The expected output is the quantities of each alloy to purchase:
\[
\text{amount} = [\text{amount}_{k} \quad \text{for } k = 1, \ldots, K]
\]

\end{document}