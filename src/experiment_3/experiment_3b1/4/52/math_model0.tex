\documentclass{article}
\usepackage{amsmath}
\begin{document}

\section*{Linear Programming Model for Electric Utility}

\subsection*{Variables}
Let \( send_{p,c} \) be the amount of electricity sent from power plant \( p \) to city \( c \) (in million kWh).

\subsection*{Parameters}
\begin{itemize}
    \item \( P \): Number of power plants
    \item \( C \): Number of cities
    \item \( supply_{p} \): Capacity of power plant \( p \) (in million kWh)
    \item \( demand_{c} \): Peak demand of city \( c \) (in million kWh)
    \item \( transmission_{p,c} \): Transmission cost from power plant \( p \) to city \( c \) (in $/million kWh)
\end{itemize}

\subsection*{Objective Function}
Minimize the total transmission cost:
\[
\text{Total Cost} = \sum_{p=1}^{P} \sum_{c=1}^{C} transmission_{p,c} \cdot send_{p,c}
\]

\subsection*{Constraints}
1. **Supply Constraints**: The total electricity sent from each power plant cannot exceed its capacity:
   \[
   \sum_{c=1}^{C} send_{p,c} \leq supply_{p}, \quad \forall p = 1, \ldots, P
   \]

2. **Demand Constraints**: The total electricity received by each city must meet its demand:
   \[
   \sum_{p=1}^{P} send_{p,c} \geq demand_{c}, \quad \forall c = 1, \ldots, C
   \]

3. **Non-negativity Constraints**: The amount of electricity sent must be non-negative:
   \[
   send_{p,c} \geq 0, \quad \forall p = 1, \ldots, P, \quad c = 1, \ldots, C
   \]

\end{document}