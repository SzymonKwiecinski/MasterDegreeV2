\documentclass{article}
\usepackage{amsmath}
\begin{document}

\section*{Mathematical Model for Nurse Scheduling Problem}

\subsection*{Parameters}
\begin{itemize}
    \item $period$: The number of consecutive days a nurse works night shifts.
    \item $d_j$: The demand for nurses on night shift for day $j$, where $j = 1, 2, \ldots, 7$.
\end{itemize}

\subsection*{Decision Variables}
\begin{itemize}
    \item $start_j$: Number of nurses that start their working period on day $j$, where $j = 1, 2, \ldots, 7$.
    \item $total$: Total number of nurses hired.
\end{itemize}

\subsection*{Objective Function}
Minimize the total number of nurses hired:
\[
\text{Minimize } total = \sum_{j=1}^{7} start_j
\]

\subsection*{Constraints}
For each day $j$, the total number of nurses available must meet or exceed the demand:
\[
\sum_{i=0}^{period-1} start_{(j-i \mod 7)} \geq d_j \quad \forall j = 1, 2, \ldots, 7
\]

Where $start_{(j-i \mod 7)}$ represents the nurses starting on day $j-i$, adjusted for the weekly cycle.

\subsection*{Non-negativity Constraints}
\[
start_j \geq 0 \quad \forall j = 1, 2, \ldots, 7
\]
\[
total \geq 0
\]

\end{document}