\documentclass{article}
\usepackage{amsmath}
\begin{document}

\section*{Linear Programming Model for Police Officer Shifts}

\subsection*{Parameters}
\begin{itemize}
    \item Let \( S \) be the total number of shifts.
    \item Let \( \text{officers\_needed}_{s} \) be the number of police officers needed during shift \( s \) for \( s = 1, 2, \ldots, S \).
    \item Let \( \text{shift\_cost}_{s} \) be the cost of assigning officers to shift \( s \).
\end{itemize}

\subsection*{Decision Variables}
Let \( x_{s} \) be the number of police officers assigned to shift \( s \).

\subsection*{Objective Function}
The objective is to minimize the total cost:
\[
\text{Minimize} \quad Z = \sum_{s=1}^{S} \text{shift\_cost}_{s} \cdot x_{s}
\]

\subsection*{Constraints}
Each officer works for two consecutive shifts. Therefore, we have the following constraints:

\begin{itemize}
    \item For each shift \( s \):
    \[
    x_{s} + x_{s-1} \geq \text{officers\_needed}_{s} \quad \text{for } s = 2, 3, \ldots, S
    \]
    \item For the first shift:
    \[
    x_{1} \geq \text{officers\_needed}_{1}
    \]
    \item Each variable \( x_s \) must be non-negative:
    \[
    x_{s} \geq 0 \quad \text{for } s = 1, 2, \ldots, S
    \]
\end{itemize}

\subsection*{Output}
The outputs will be:
\begin{itemize}
    \item The number of officers assigned to each shift \( \text{officers\_assigned}_{s} = x_{s} \) for \( s = 1, 2, \ldots, S \).
    \item The total cost \( \text{total\_cost} = Z \).
\end{itemize}

\end{document}