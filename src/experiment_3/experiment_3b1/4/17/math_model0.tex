\documentclass{article}
\usepackage{amsmath}
\begin{document}

\section*{Linear Programming Model for Stock Selling Optimization}

\textbf{Parameters:}
\begin{itemize}
    \item $N$: Number of different stocks.
    \item $bought_i$: Number of shares of stock $i$ bought.
    \item $buyPrice_i$: Price at which stock $i$ was bought.
    \item $currentPrice_i$: Current price of stock $i$.
    \item $futurePrice_i$: Expected future price of stock $i$ in one year.
    \item $transactionRate$: Transaction cost rate (as a percentage).
    \item $taxRate$: Tax rate on capital gains (as a percentage).
    \item $K$: Amount of money to be raised net of capital gains and transaction costs.
\end{itemize}

\textbf{Decision Variables:}
\begin{itemize}
    \item $sell_i$: Number of shares of stock $i$ to sell, for $i = 1, \ldots, N$.
\end{itemize}

\textbf{Objective Function:}
We want to maximize the expected value of the portfolio next year, which can be expressed as:
\[
\text{Maximize } E(V) = \sum_{i=1}^{N} futurePrice_i \cdot (bought_i - sell_i) 
\]

\textbf{Constraints:}
1. The total amount raised after considering transaction costs and taxes should be at least $K$:
\[
\sum_{i=1}^{N} \left( currentPrice_i \cdot sell_i \cdot (1 - \frac{transactionRate}{100}) - \left( (currentPrice_i - buyPrice_i) \cdot sell_i \cdot \frac{taxRate}{100} \right) \right) \geq K
\]
2. The number of shares sold cannot exceed the number of shares bought:
\[
0 \leq sell_i \leq bought_i, \quad \forall i = 1, \ldots, N
\]

\textbf{Output:}
The solution provides the optimal number of shares to sell for each stock:
\[
\text{Output: } \{ sell : [sell_i, \text{ for } i = 1, \ldots, N] \}
\]

\end{document}