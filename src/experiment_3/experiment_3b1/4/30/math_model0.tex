\documentclass{article}
\usepackage{amsmath}
\begin{document}

\section*{Quadratic Curve Fitting Problem}

We aim to fit the best quadratic curve of the form 
\[
y = c \cdot x^2 + b \cdot x + a
\]
to a set of data points \((x_k, y_k)\) for \(k = 1, \ldots, K\). The objective is to minimize the sum of absolute deviations of each observed value of \(y\) from the predicted value of \(y\) given by the quadratic function.

\subsection*{Inputs}
The values of \(y\) and \(x\) are provided as follows:
\begin{align*}
y & = [y_{1}, y_{2}, \ldots, y_{K}] \\
x & = [x_{1}, x_{2}, \ldots, x_{K}]
\end{align*}

\subsection*{Objective Function}
We define the absolute deviations for each data point as follows:
\[
d_k = |y_k - (c \cdot x_k^2 + b \cdot x_k + a)|, \quad \text{for } k = 1, 2, \ldots, K
\]
The objective is to minimize the sum of these absolute deviations:
\[
\text{Minimize} \quad \sum_{k=1}^{K} d_k
\]

\subsection*{Variables}
The coefficients \(c\), \(b\), and \(a\) are the variables we need to determine:
\begin{align*}
\text{Let } & c \text{ be the coefficient of the quadratic term,} \\
& b \text{ be the coefficient of the linear term,} \\
& a \text{ be the constant term.}
\end{align*}

\subsection*{Output}
The output will specify the values of the coefficients:
\begin{verbatim}
{
    "quadratic": c,
    "linear": b,
    "constant": a
}
\end{verbatim}

\end{document}