\documentclass{article}
\usepackage{amsmath}
\begin{document}

\section*{Linear Programming Model for Maximizing Profit}

\subsection*{Parameters}
\begin{itemize}
    \item Let \( K \) be the number of spare parts.
    \item Let \( S \) be the number of shops.
    \item \( time_{k,s} \) = required worker-hours for part \( k \) in shop \( s \).
    \item \( profit_{k} \) = profit from part \( k \).
    \item \( capacity_{s} \) = capacity of shop \( s \) in working hours.
\end{itemize}

\subsection*{Decision Variables}
Let \( quantity_{k} \) be the quantity of spare part \( k \) to be produced.

\subsection*{Objective Function}
Maximize total profit:
\[
\text{Maximize } Z = \sum_{k=1}^{K} profit_{k} \cdot quantity_{k}
\]

\subsection*{Constraints}
The time constraints for each shop \( s \) are given by:
\[
\sum_{k=1}^{K} time_{k,s} \cdot quantity_{k} \leq capacity_{s}, \quad \forall s = 1, 2, \ldots, S
\]

Additionally, we impose non-negativity constraints:
\[
quantity_{k} \geq 0, \quad \forall k = 1, 2, \ldots, K
\]

\subsection*{Model Formulation}
Putting it all together, the linear programming model can be defined as:

\[
\begin{align*}
\text{Maximize} & \quad Z = \sum_{k=1}^{K} profit_{k} \cdot quantity_{k} \\
\text{subject to} & \quad \sum_{k=1}^{K} time_{k,s} \cdot quantity_{k} \leq capacity_{s}, \quad \forall s \in \{1, \ldots, S\} \\
& \quad quantity_{k} \geq 0, \quad \forall k \in \{1, \ldots, K\}
\end{align*}
\]

\end{document}