\documentclass{article}
\usepackage{amsmath}
\begin{document}

\title{Linear Programming Formulation for Currency Exchange Problem}
\author{}
\date{}
\maketitle

\section*{Problem Formulation}

Let \( N \) be the number of available currencies, indexed by \( i \in \{1, \ldots, N\} \).

Define:
\begin{itemize}
    \item \( \text{start}_i \): the initial amount of currency \( i \).
    \item \( \text{limit}_i \): the maximum amount of currency \( i \) that can be exchanged.
    \item \( \text{rate}_{i,j} \): the exchange rate from currency \( i \) to currency \( j \), where \( \text{rate}_{i,j} > 0 \).
    \item \( x_{i,j} \): the amount of currency \( i \) exchanged for currency \( j \).
\end{itemize}

The objective is to maximize the total amount of currency \( N \) at the end of the day. This can be expressed mathematically as:

\[
\text{Maximize} \quad x_{N,j} \quad \text{for } j \in \{1, \ldots, N\}
\]

subject to the following constraints:

1. **Exchange Limit Constraints**: The total exchanged amount for each currency cannot exceed its limit:

\[
\sum_{j=1}^N x_{i,j} \leq \text{limit}_i \quad \forall i \in \{1, \ldots, N\}
\]

2. **Initial Amount Constraints**: The amount exchanged must not exceed the starting amount minus any amount taken back:

\[
x_{i,j} \leq \text{start}_i \quad \forall i,j \in \{1, \ldots, N\}
\]

3. **Non-negativity Constraints**:

\[
x_{i,j} \geq 0 \quad \forall i,j \in \{1, \ldots, N\}
\]

4. **Wealth Conservation**: The amount of currency \( j \) that we end up with after all exchanges must be accounted for:

\[
\text{end}_i = \text{start}_i - \sum_{j=1}^N x_{i,j} + \sum_{k=1}^N x_{j,k} \cdot \text{rate}_{j,k} \quad \forall i \in \{1, \ldots, N\}
\]

where \( \text{end}_i \) denotes the ending amount of currency \( i \).

\section*{Conclusion}

This linear programming formulation can be solved using standard LP solvers to determine the optimal exchange strategy to maximize the final amount of currency \( N \). 

\end{document}