\documentclass{article}
\usepackage{amsmath}
\begin{document}

\section*{Linear Programming Model for Line Fitting}

Given a set of observations \( (x_k, y_k) \) for \( k = 1, \ldots, K \), we aim to fit a line described by the equation:

\[
y = bx + a
\]

where \( b \) is the slope, and \( a \) is the intercept. The objective is to minimize the maximum deviation of the observed values \( y_k \) from the predicted values \( \hat{y}_k = bx_k + a \).

Define the deviations as:

\[
d_k = y_k - (bx_k + a) \quad \text{for } k = 1, \ldots, K
\]

To minimize the maximum deviation, we introduce a new variable \( M \) to represent this maximum deviation. Thus, we have the following constraints:

\[
d_k \leq M \quad \text{for } k = 1, \ldots, K
\]
\[
-d_k \leq M \quad \text{for } k = 1, \ldots, K
\]

We want to minimize \( M \) subject to the above constraints. Therefore, the Linear Programming model can be formulated as follows:

\begin{align*}
\text{Minimize} \quad & M \\
\text{subject to} \quad & y_k - (bx_k + a) \leq M, \quad k = 1, \ldots, K \\
& -(y_k - (bx_k + a)) \leq M, \quad k = 1, \ldots, K \\
& b \in \mathbb{R}, \; a \in \mathbb{R}
\end{align*}

The output of the model will provide the intercept and slope of the fitted line.

\begin{align*}
\text{Output:} \quad & \{ \text{"intercept"}: a, \; \text{"slope"}: b \}
\end{align*}

\end{document}