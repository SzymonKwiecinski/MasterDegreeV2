\documentclass{article}
\usepackage{amsmath}
\begin{document}

\title{Linear Programming Model for School Assignment Problem}
\author{}
\date{}
\maketitle

\section*{Problem Description}

Consider a school district with \( N \) neighborhoods, \( S \) schools, and \( G \) grades at each school. Each school \( s \) has a capacity of \( \text{capacity}_{s,g} \) for grade \( g \). In each neighborhood \( n \), the student population of grade \( g \) is \( \text{population}_{n,g} \). Finally, the distance of school \( s \) from neighborhood \( n \) is \( d_{n,s} \).

\section*{Parameters}

\begin{itemize}
    \item \( \text{capacity}_{s,g} \): Capacity of school \( s \) for grade \( g \).
    \item \( \text{population}_{n,g} \): Population of grade \( g \) in neighborhood \( n \).
    \item \( d_{n,s} \): Distance from neighborhood \( n \) to school \( s \).
\end{itemize}

\section*{Decision Variables}

Let \( x_{n,s,g} \) be the number of students of grade \( g \) from neighborhood \( n \) assigned to school \( s \).

\section*{Objective Function}

We want to minimize the total distance traveled by all students, which can be expressed as:

\[
\text{minimize} \quad Z = \sum_{n=1}^{N} \sum_{s=1}^{S} \sum_{g=1}^{G} d_{n,s} \cdot x_{n,s,g}
\]

\section*{Constraints}

1. Capacity constraints for each school:
\[
\sum_{n=1}^{N} \sum_{g=1}^{G} x_{n,s,g} \leq \text{capacity}_{s,g}, \quad \forall s, g
\]

2. Population constraints for each neighborhood:
\[
\sum_{s=1}^{S} x_{n,s,g} = \text{population}_{n,g}, \quad \forall n, g
\]

3. Non-negativity constraints:
\[
x_{n,s,g} \geq 0, \quad \forall n, s, g
\]

\section*{Output Information}

The output should include:

1. The assignment \( x_{n,s,g} \) for all \( n \), \( s \), \( g \).
2. The total distance traveled by all students \( \text{total\_distance} \).

\end{document}