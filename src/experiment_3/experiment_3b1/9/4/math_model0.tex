\documentclass{article}
\usepackage{amsmath}
\begin{document}

\section*{Mathematical Model}

Given the problem of scheduling nurses for the night shift, we define the following variables and parameters:

\begin{itemize}
    \item Let \( n \) be the total number of nurses hired.
    \item Let \( d_j \) be the demand for nurses on day \( j \) where \( j = 1, \ldots, 7 \).
    \item Let \( \text{period} \) be the number of consecutive days a nurse works.
    \item Let \( start_j \) be the number of nurses who start their period on day \( j \).
\end{itemize}

We need to satisfy the demand for nurses for each day while minimizing the total number of nurses hired.

\subsection*{Objective Function}

The objective is to minimize the total number of nurses hired:

\[
\text{minimize } n = \sum_{j=1}^{7} start_j
\]

\subsection*{Constraints}

For each day \( j \), the total number of nurses available must meet or exceed the demand \( d_j \):

\[
\sum_{i=0}^{\text{period}-1} start_{j-i} \geq d_j \quad \forall j = 1, \ldots, 7
\]

Where \( start_{j-i} \) refers to nurses starting their shifts on days prior to \( j \), up to the length of the period.

Additionally, since the period of work is followed by days off, we also have:

\[
start_j = 0 \quad \text{for } j > 7 - \text{period}
\]

This models the scenario that no new nurses can start their period if the schedule would push their shifts beyond the 7-day week.

\subsection*{Output Information}

The solution will provide:

\begin{itemize}
    \item \( start = [start_1, start_2, start_3, start_4, start_5, start_6, start_7] \)
    \item total number of nurses \( n \)
\end{itemize}

\end{document}