\documentclass{article}
\usepackage{amsmath}
\begin{document}

\title{Linear Programming Model for Alloy Production}
\author{}
\date{}
\maketitle

\section*{Problem Definition}

A manufacturer wishes to produce \( \text{alloy\_quant} \) lb of an alloy that consists of several kinds of metals. The goal is to meet a target quantity \( \text{target}_{m} \) of metal \( \text{metal}_{m} \). Various alloys are available for purchase, each having different compositions of metals and prices.

\section*{Parameters}

\begin{itemize}
    \item \( \text{alloy\_quant} \): Total weight of the alloy to be produced (in lb).
    \item \( \text{target}_{m} \): Target weight of metal \( \text{metal}_{m} \).
    \item \( \text{ratio}_{k, m} \): Ratio of metal \( \text{metal}_{m} \) in alloy \( k \).
    \item \( \text{price}_{k} \): Selling price of alloy \( k \).
\end{itemize}

\section*{Decision Variables}

Let \( \text{amount}_{k} \) be the amount of alloy \( k \) to purchase (in lb).

\section*{Objective Function}

The manufacturer aims to minimize the total cost of the alloys purchased:

\[
\text{Minimize } Z = \sum_{k=1}^{K} \text{price}_{k} \cdot \text{amount}_{k}
\]

\section*{Constraints}

1. **Alloy Quantity Constraint**:
   The total amount of alloys purchased must equal the required alloy quantity:
   \[
   \sum_{k=1}^{K} \text{amount}_{k} = \text{alloy\_quant}
   \]

2. **Metal Quantity Constraint**:
   The total amount of each metal \( m \) must meet the target quantity:
   \[
   \sum_{k=1}^{K} \text{ratio}_{k, m} \cdot \text{amount}_{k} = \text{target}_{m}, \quad \forall m=1,\ldots,M
   \]

3. **Non-negativity Constraints**:
   \[
   \text{amount}_{k} \geq 0, \quad \forall k=1,\ldots,K
   \]

\section*{Output Information}

The output will be the quantities of each alloy to purchase, represented as:
\[
\text{amount} = [\text{amount}_{k} \text{ for } k = 1, \ldots, K]
\]

\end{document}