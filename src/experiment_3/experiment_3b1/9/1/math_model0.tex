\documentclass{article}
\usepackage{amsmath}
\begin{document}

\section*{Linear Programming Model}

Let \( M \) be the number of different goods produced and \( N \) be the number of different raw materials available. We denote:

\begin{itemize}
    \item \( \text{available}_{i} \): the amount of raw material \( i \) available for \( i = 1, \ldots, N \).
    \item \( \text{req}_{i,j} \): the amount of raw material \( i \) required to produce one unit of good \( j \) for \( j = 1, \ldots, M \).
    \item \( \text{price}_{j} \): the revenue obtained from producing one unit of good \( j \).
    \item \( \text{amount}_{j} \): the number of units of good \( j \) produced.
\end{itemize}

\subsection*{Objective Function}
The objective is to maximize the total revenue, given by:

\[
\text{Maximize } Z = \sum_{j=1}^{M} \text{price}_{j} \cdot \text{amount}_{j}
\]

\subsection*{Constraints}
The production of goods must not exceed the available raw materials. For each raw material \( i \), we have the constraint:

\[
\sum_{j=1}^{M} \text{req}_{i,j} \cdot \text{amount}_{j} \leq \text{available}_{i} \quad \text{for } i = 1, \ldots, N
\]

Additionally, the production amounts must be non-negative:

\[
\text{amount}_{j} \geq 0 \quad \text{for } j = 1, \ldots, M
\]

\subsection*{Formulation}
Putting it all together, the Linear Program can be formulated as:

\[
\begin{align*}
\text{Maximize } & Z = \sum_{j=1}^{M} \text{price}_{j} \cdot \text{amount}_{j} \\
\text{subject to } & \sum_{j=1}^{M} \text{req}_{i,j} \cdot \text{amount}_{j} \leq \text{available}_{i}, \quad i = 1, \ldots, N \\
& \text{amount}_{j} \geq 0, \quad j = 1, \ldots, M
\end{align*}
\]

\end{document}