\documentclass{article}
\usepackage{amsmath}
\begin{document}

\section*{Mathematical Model for Cafeteria Staffing Problem}

\subsection*{Parameters}
\begin{itemize}
    \item $N$: Total number of days.
    \item $num_n$: Number of desired employees on day $n$, where $n \in \{1, 2, \ldots, N\}$.
    \item $n_{\text{working\_days}}$: Number of consecutive days an employee works.
    \item $n_{\text{resting\_days}}$: Number of consecutive days an employee rests.
\end{itemize}

\subsection*{Decision Variables}
\begin{itemize}
    \item $total\_number$: Total number of employees to hire.
    \item $is\_work_{n,i}$: Binary variable indicating if employee $i$ is working on day $n$ ($is\_work_{n,i} = 1$) or resting ($is\_work_{n,i} = 0$), where $i \in \{1, 2, \ldots, total\_number\}$ and $n \in \{1, 2, \ldots, N\}$.
\end{itemize}

\subsection*{Objective Function}
Minimize the number of employees hired:
\[
\text{Minimize } total\_number
\]

\subsection*{Constraints}
\begin{itemize}
    \item For each day $n$, the total number of employees working must be at least $num_n$:
    \[
    \sum_{i=1}^{total\_number} is\_work_{n,i} \geq num_n \quad \forall n \in \{1, 2, \ldots, N\}
    \]
    
    \item Each employee works for $n_{\text{working\_days}}$ days and then rests for $n_{\text{resting\_days}}$ days. For an employee $i$ working on day $n$, the following condition holds for their cycle:
    \[
    is\_work_{k,i} = 
    \begin{cases} 
      1 & \text{if } k \text{ mod } (n_{\text{working\_days}} + n_{\text{resting\_days}}) < n_{\text{working\_days}} \\
      0 & \text{otherwise}
    \end{cases}
    \quad \forall i \in \{1, 2, \ldots, total\_number\}, \, \forall k \in \{1, 2, \ldots, N\}
    \]
\end{itemize}

\end{document}