\documentclass{article}
\usepackage{amsmath}
\begin{document}

\title{Linear Programming Model for Nutritional Optimization}
\author{}
\date{}
\maketitle

\section*{Problem Description}
We aim to minimize the total cost of purchasing food while satisfying nutritional demands.

\subsection*{Inputs}
Let:
\begin{itemize}
    \item $K$ = number of different food types
    \item $M$ = number of nutritional ingredients
    \item $\text{price}_k$ = price per unit of food $k$ for $k = 1, \ldots, K$
    \item $\text{demand}_m$ = minimum required units of nutrient $m$ for $m = 1, \ldots, M$
    \item $\text{nutri}_{k,m}$ = units of nutrient $m$ contained in food $k$
\end{itemize}

\subsection*{Decision Variables}
Define $x_k$ as the quantity of food $k$ to purchase for $k = 1, \ldots, K$.

\subsection*{Objective Function}
The objective is to minimize the total cost:
\[
\text{Minimize } Z = \sum_{k=1}^{K} \text{price}_k \cdot x_k
\]

\subsection*{Constraints}
To ensure that each nutrient demand is met, we have the following constraints:
\[
\sum_{k=1}^{K} \text{nutri}_{k,m} \cdot x_k \geq \text{demand}_m \quad \forall m = 1, \ldots, M
\]

Additionally, we include non-negativity constraints on the decision variables:
\[
x_k \geq 0 \quad \forall k = 1, \ldots, K
\]

\subsection*{Output}
\begin{itemize}
    \item The solution will provide the quantities of each food to purchase, represented as:
    \[
    \text{quantity} = [x_1, x_2, \ldots, x_K]
    \]
\end{itemize}

\end{document}