\documentclass{article}
\usepackage{amsmath}
\begin{document}

\section*{Linear Programming Formulation}

Let \( N \) be the number of available currencies. We denote the currencies as \( C_1, C_2, \ldots, C_N \).

Define:
\begin{itemize}
    \item \( start_i \): the initial amount of currency \( C_i \).
    \item \( limit_i \): the limit on the amount of currency \( C_i \) that can be exchanged.
    \item \( rate_{i,j} \): the exchange rate from currency \( C_i \) to currency \( C_j \).
\end{itemize}

Let \( x_{i,j} \) be the amount of currency \( C_i \) that is exchanged for currency \( C_j \).

\subsection*{Objective Function}
The objective is to maximize the amount of currency \( C_N \) at the end of the day:
\[
\text{Maximize } z = x_{N,N} + \sum_{j=1}^{N} x_{N,j}
\]

\subsection*{Constraints}
1. **Exchange Limits**:
   For each currency \( C_i \):
   \[
   \sum_{j=1}^{N} x_{i,j} \leq limit_i \quad \forall i \in \{1, 2, \ldots, N\}
   \]

2. **Initial Constraints**:
   For each currency \( C_i \):
   \[
   x_{i,i} + \sum_{j=1}^{N} x_{j,i} \leq start_i \quad \forall i \in \{1, 2, \ldots, N\}
   \]

3. **Exchange Rate Constraints**:
   The amount of currency exchanged must adhere to the exchange rates:
   \[
   x_{i,j} \leq rate_{i,j} \cdot \sum_{k=1}^{N} x_{k,i} \quad \forall i,j \in \{1, 2, \ldots, N\}
   \]

4. **Non-negativity Constraints**:
   All exchanges must be non-negative:
   \[
   x_{i,j} \geq 0 \quad \forall i,j \in \{1, 2, \ldots, N\}
   \]

\subsection*{Conclusion}
This linear programming formulation captures the constraints and objectives involved in maximizing the amount of currency \( C_N \) at the end of the day while respecting the limits and exchange rates.

\end{document}