\documentclass{article}
\usepackage{amsmath}
\begin{document}

\section*{Linear Programming Model for Maximizing Profit in Spare Parts Production}

\subsection*{Variables}
Let:
\begin{itemize}
    \item \( K \) be the number of spare parts.
    \item \( S \) be the number of shops.
    \item \( quantity_k \) be the quantity of spare part \( k \) produced.
\end{itemize}

\subsection*{Parameters}
\begin{itemize}
    \item \( time_{k,s} \): required worker-hours for part \( k \) in shop \( s \).
    \item \( profit_k \): profit from part \( k \).
    \item \( capacity_s \): working hours available in shop \( s \).
\end{itemize}

\subsection*{Objective Function}
We want to maximize the total profit, which is given by:
\[
\text{Maximize } Z = \sum_{k=1}^{K} profit_k \cdot quantity_k
\]

\subsection*{Constraints}
The production of spare parts is limited by the capacity of each shop. Thus, the following constraints must hold for each shop \( s \):
\[
\sum_{k=1}^{K} time_{k,s} \cdot quantity_k \leq capacity_s \quad \text{for } s = 1, \ldots, S
\]

Additionally, the quantity of each spare part must be non-negative:
\[
quantity_k \geq 0 \quad \text{for } k = 1, \ldots, K
\]

\subsection*{Summary of the Model}
The linear programming model can be summarized as follows:

\begin{align*}
\text{Maximize } & Z = \sum_{k=1}^{K} profit_k \cdot quantity_k \\
\text{subject to } & \sum_{k=1}^{K} time_{k,s} \cdot quantity_k \leq capacity_s, \quad s = 1, \ldots, S \\
& quantity_k \geq 0, \quad k = 1, \ldots, K
\end{align*}

\end{document}