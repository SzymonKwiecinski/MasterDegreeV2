\documentclass{article}
\usepackage{amsmath}
\begin{document}

\section*{Mathematical Model for Quadratic Curve Fitting}

Given a set of corresponding values \( x_k \) and \( y_k \) for \( k = 1, \ldots, K \), we aim to fit the quadratic curve defined by the equation:

\[
y = c \cdot x^2 + b \cdot x + a
\]

Our objective is to minimize the sum of absolute deviations between the observed values \( y_k \) and the values predicted by the quadratic model. This can be expressed mathematically as:

\[
\text{Minimize } Z = \sum_{k=1}^{K} |y_k - (c \cdot x_k^2 + b \cdot x_k + a)|
\]

Where:
- \( c \) is the coefficient of the quadratic term,
- \( b \) is the coefficient of the linear term,
- \( a \) is the constant term.

We can define the absolute deviations with auxiliary variables \( d_k \) for \( k = 1, \ldots, K \):

\[
d_k \geq y_k - (c \cdot x_k^2 + b \cdot x_k + a)
\]

\[
d_k \geq -(y_k - (c \cdot x_k^2 + b \cdot x_k + a))
\]

Our linear program thus becomes:

\[
\text{Minimize } Z = \sum_{k=1}^{K} d_k
\]

Subject to:

\[
d_k \geq y_k - (c \cdot x_k^2 + b \cdot x_k + a), \quad \forall k
\]

\[
d_k \geq -(y_k - (c \cdot x_k^2 + b \cdot x_k + a)), \quad \forall k
\]

\begin{itemize}
    \item The output of the model will provide us with the coefficients:
    \begin{itemize}
        \item \( \text{quadratic} = c \)
        \item \( \text{linear} = b \)
        \item \( \text{constant} = a \)
    \end{itemize}
\end{itemize}

\end{document}