\documentclass{article}
\usepackage{amsmath}
\begin{document}

\section*{Linear Programming Model}

\subsection*{Parameters}
\begin{itemize}
    \item \( N \): Number of segments of the road.
    \item \( M \): Number of lamps.
    \item \( \text{coeff}_{i,j} \): Coefficient indicating the contribution of lamp \( j \) to the illumination of segment \( i \).
    \item \( \text{desired}_i \): Desired illumination for segment \( i \).
\end{itemize}

\subsection*{Decision Variables}
\begin{itemize}
    \item \( \text{power}_j \): Power of the \( j \)-th lamp.
    \item \( \text{ill}_i \): Illumination of the \( i \)-th segment.
\end{itemize}

\subsection*{Objective Function}
The objective is to minimize the total absolute error between the actual illumination and the desired illumination:

\[
\text{Minimize } Z = \sum_{i=1}^{N} | \text{ill}_i - \text{desired}_i |
\]

\subsection*{Constraints}
The illumination for each segment is calculated as follows:

\[
\text{ill}_i = \sum_{j=1}^{M} \text{coeff}_{i,j} \cdot \text{power}_j, \quad \forall i = 1, \ldots, N
\]

Additionally, we need to ensure that the errors are defined. To handle the absolute values in the objective function, we introduce auxiliary variables:

Let \( e_i \) be the error for each segment \( i \):

\[
e_i \geq \text{ill}_i - \text{desired}_i, \quad \forall i = 1, \ldots, N
\]
\[
e_i \geq \text{desired}_i - \text{ill}_i, \quad \forall i = 1, \ldots, N
\]

The objective can then be rewritten in terms of \( e_i \):

\[
\text{Minimize } Z = \sum_{i=1}^{N} e_i
\]

\subsection*{Final Model}
\begin{align*}
\text{Minimize } & \sum_{i=1}^{N} e_i \\
\text{Subject to } & \text{ill}_i = \sum_{j=1}^{M} \text{coeff}_{i,j} \cdot \text{power}_j, \quad \forall i = 1, \ldots, N \\
& e_i \geq \text{ill}_i - \text{desired}_i, \quad \forall i = 1, \ldots, N \\
& e_i \geq \text{desired}_i - \text{ill}_i, \quad \forall i = 1, \ldots, N \\
& \text{power}_j \geq 0, \quad \forall j = 1, \ldots, M
\end{align*}

\end{document}