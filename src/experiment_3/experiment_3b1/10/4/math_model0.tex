\documentclass{article}
\usepackage{amsmath}
\begin{document}

\section*{Mathematical Model for Nurse Scheduling}

\textbf{Parameters:}
\begin{itemize}
    \item $period$: Number of days a nurse works in a row
    \item $d_j$: Demand for nurses on day $j$, for $j = 1, \ldots, 7$
\end{itemize}

\textbf{Decision Variables:}
\begin{itemize}
    \item $start_j$: Number of nurses that start their period on day $j$, for $j = 1, \ldots, 7$
    \item $total$: Total number of nurses hired
\end{itemize}

\textbf{Objective:} 
Minimize the total number of nurses hired:
\[
\text{Minimize } total = \sum_{j=1}^{7} start_j
\]

\textbf{Constraints:}
For each day $j$:
\[
\sum_{i=0}^{period-1} start_{(j-i) \mod 7} \geq d_j \quad \forall j = 1, \ldots, 7
\]
This constraint ensures that the demand for nurses on each day $j$ is met by considering the nurses who started their shifts up to the last $period$ days.

\textbf{Non-negativity:}
\[
start_j \geq 0 \quad \forall j = 1, \ldots, 7
\]

\textbf{Integer Constraints:}
\[
start_j \in \mathbb{Z}^+ \quad \forall j = 1, \ldots, 7
\]

\end{document}