\documentclass{article}
\usepackage{amsmath}
\begin{document}

\title{Linear Programming Model for Best Fit Line}
\author{}
\date{}
\maketitle

\section*{Problem Description}
We want to fit a 'best' straight line of the form \( y = bx + a \) to a set of observed data points \( (x_k, y_k) \) for \( k = 1, \ldots, K \). The goal is to minimize the sum of absolute deviations of each observed value of \( y_k \) from the predicted value given by the line.

\section*{Modeling the Problem}

\subsection*{Variables}
\begin{itemize}
    \item Let \( a \) be the intercept of the fitted line.
    \item Let \( b \) be the slope of the fitted line.
    \item Let \( d_k \) be the absolute deviation for the \( k \)-th observation, defined as \( d_k = |y_k - (bx_k + a)| \) for \( k = 1, \ldots, K \).
\end{itemize}

\subsection*{Objective Function}
The objective is to minimize the total absolute deviations:
\[
\text{Minimize } Z = \sum_{k=1}^{K} d_k
\]

\subsection*{Constraints}
The absolute deviations can be expressed using auxiliary variables \( d_k \) as follows:
\[
d_k \geq y_k - (bx_k + a) \quad \forall k
\]
\[
d_k \geq -(y_k - (bx_k + a)) \quad \forall k
\]

This ensures that \( d_k \) correctly represents the absolute deviation.

\subsection*{Linear Programming Formulation}
The complete linear programming model can be formulated as:

\[
\begin{align*}
\text{Minimize} & \quad Z = \sum_{k=1}^{K} d_k \\
\text{subject to} & \quad d_k \geq y_k - (bx_k + a) & \forall k \\
& \quad d_k \geq -(y_k - (bx_k + a)) & \forall k \\
& \quad d_k \geq 0 & \forall k \\
& \quad a \text{ and } b \text{ are unrestricted}
\end{align*}
\]

\section*{Output}
After solving the above linear programming model, the output will provide the values of the intercept and slope:
\[
\begin{align*}
\text{intercept} & : a \\
\text{slope} & : b
\end{align*}
\]

\end{document}