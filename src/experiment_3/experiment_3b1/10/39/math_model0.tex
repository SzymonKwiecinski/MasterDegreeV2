\documentclass{article}
\usepackage{amsmath}
\begin{document}

\section*{Mathematical Model for Cafeteria Staffing Problem}

\textbf{Parameters:}
\begin{itemize}
    \item \( N \): Total number of days
    \item \( num_n \): Number of desired employees on day \( n \) for \( n = 1, \ldots, N \)
    \item \( n_{\text{working}} \): Number of working days for each employee
    \item \( n_{\text{resting}} \): Number of resting days for each employee
    \item \( T = n_{\text{working}} + n_{\text{resting}} \): Total cycle of working and resting days
\end{itemize}

\textbf{Decision Variables:}
\begin{itemize}
    \item \( total\_number \): Total number of employees to hire
    \item \( is\_work_{n,i} \): Binary variable which is 1 if employee \( i \) is working on day \( n \), and 0 otherwise.
\end{itemize}

\textbf{Objective Function:}
\[
\text{Minimize } total\_number
\]

\textbf{Constraints:}
\begin{align}
    \sum_{i=1}^{total\_number} is\_work_{n,i} & \geq num_n, \quad \forall n = 1, \ldots, N \label{constraint1} \\
    is\_work_{n,i} & = 1 \text{ for } n = k(T) + j \text{ where } j = 0, \ldots, n_{\text{working}} - 1 \text{ (Employee works)} \\
    is\_work_{n,i} & = 0 \text{ for } n = k(T) + j \text{ where } j = n_{\text{working}}, \ldots, T - 1 \text{ (Employee rests)}
\end{align}
where \( k \) is any integer such that \( n \) stays within the range of days.

\textbf{Output:}
\begin{itemize}
    \item \( total\_number \): Total number of employees to employ
    \item \( is\_work \): A matrix of size \( N \times total\_number \) showing the work schedule of each employee.
\end{itemize}

\end{document}