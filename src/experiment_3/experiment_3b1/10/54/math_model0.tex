\documentclass{article}
\usepackage{amsmath}
\begin{document}

\section*{Linear Programming Model for Auto Parts Manufacturer}

\subsection*{Decision Variables}
Let \( b_p \) represent the number of batches of part \( p \) produced, where \( p = 1, 2, \ldots, P \).

\subsection*{Objective Function}
The objective is to maximize the total profit:
\[
\text{Maximize } Z = \sum_{p=1}^{P} (price_p \cdot b_p) - \sum_{m=1}^{M} (cost_m \cdot \text{time}_{m,p} \cdot b_p)
\]
where \( \text{time}_{m,p} \) is the time required on machine \( m \) for part \( p \).

\subsection*{Constraints}

1. **Machine Availability Constraint**: 
   For each machine \( m \) (where \( m = 1, 2, \ldots, M \)):
   \[
   \sum_{p=1}^{P} \text{time}_{m,p} \cdot b_p \leq available_m
   \]

2. **Minimum Batches Requirement**:
   For each part \( p \):
   \[
   b_p \geq min\_batches_p
   \]

3. **Labor Cost Constraints on Machine 1**:
   For machine 1, the labor cost structure applies:
   \[
   \text{time}_{1,p} \cdot b_p \leq \text{max\_hours}
   \]
   Where \(\text{max\_hours}\) is the number of hours available, which may include overtime hours.

4. **Additional Overtime Constraints**:
   If the total required hours on machine 1 exceed available hours, then:
   \[
   \text{time}_{1,p} \cdot b_p \leq available_1 + overtime\_hour \quad \text{(within standard cost)}
   \]
   and for hours exceeding this limit, the overtime cost applies.

\subsection*{Non-negativity Constraints}
\[
b_p \geq 0 \quad \text{for all } p
\]

\subsection*{Output}
The solution will yield:
\[
\text{Output: } \{ \text{batches} = [b_p \text{ for } p = 1, \ldots, P], \text{ total\_profit} = Z \} 
\]

\end{document}