\documentclass{article}
\usepackage{amsmath}
\begin{document}

\title{Linear Programming Model for Alloy Production}
\author{}
\date{}
\maketitle

\section*{Problem Definition}
A manufacturer wishes to produce a total of \( \text{alloy\_quant} \) lb of an alloy that contains targeted quantities of various metals. The alloy consists of several kinds of metals, and the quantity of metal \( m \) in the final alloy should meet the target weight \( \text{target}_{m} \). 

The available alloys have varying compositions and prices:

\begin{itemize}
    \item Alloy \( k \) has a ratio \( \text{ratio}_{k,m} \) of metal \( m \) and is sold at a price \( \text{price}_k \).
    \item The ratio \( \text{ratio}_{k,m} \) of each alloy is within the range [0, 1].
\end{itemize}

\section*{Decision Variables}
Let \( \text{amount}_k \) represent the quantity of alloy \( k \) purchased.

\section*{Objective Function}
The objective of the manufacturer is to minimize the total cost of the alloys purchased:

\[
\text{Minimize } Z = \sum_{k=1}^{K} \text{price}_k \cdot \text{amount}_k
\]

\section*{Constraints}
1. The total amount of the produced alloy must equal the desired alloy quantity:

\[
\sum_{k=1}^{K} \text{amount}_k = \text{alloy\_quant}
\]

2. The amount of each metal \( m \) in the produced alloy must meet the target:

\[
\sum_{k=1}^{K} \text{ratio}_{k,m} \cdot \text{amount}_k = \text{target}_{m}, \quad \forall m = 1, \ldots, M
\]

3. Non-negativity constraints on the amount of each alloy:

\[
\text{amount}_k \geq 0, \quad \forall k = 1, \ldots, K
\]

\section*{Output}
The solution will provide the quantities of each alloy to be purchased:

\[
\text{Output: } \text{"amount": } [ \text{amount}_k \text{ for } k = 1, \ldots, K ]
\]

\end{document}