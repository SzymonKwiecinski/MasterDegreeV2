\documentclass{article}
\usepackage{amsmath}
\begin{document}

\section*{Mathematical Model for Electricity Capacity Expansion}

\subsection*{Definitions}
\begin{itemize}
    \item \( T \): Total number of years
    \item \( d_t \): Demand for electricity in year \( t \)
    \item \( oil_t \): Existing oil-fired capacity in year \( t \)
    \item \( coal\_cost \): Capital cost per megawatt for coal-fired capacity
    \item \( nuke\_cost \): Capital cost per megawatt for nuclear capacity
    \item \( max\_nuke \): Maximum percentage of total capacity that can be nuclear
    \item \( coal\_life \): Lifespan of coal plants in years
    \item \( nuke\_life \): Lifespan of nuclear plants in years
    \item \( coal_t \): Capacity added in year \( t \) for coal
    \item \( nuke_t \): Capacity added in year \( t \) for nuclear
\end{itemize}

\subsection*{Variables}
Let \( C_t \) be the total capacity in year \( t \):
\[
C_t = oil_t + \sum_{j=1}^{t} coal_j + \sum_{k=1}^{t} nuke_k
\]

\subsection*{Constraints}
1. Capacity must meet demand:
\[
C_t \geq d_t \quad \forall t \in \{1, \ldots, T\}
\]
2. Nuclear capacity constraint:
\[
nuke_t \leq \frac{max\_nuke}{100} \cdot C_t \quad \forall t \in \{1, \ldots, T\}
\]
3. Lifetime constraints for coal and nuclear:
\[
coal_t = 0, \forall t < coal\_life
\]
\[
nuke_t = 0, \forall t < nuke\_life
\]

\subsection*{Objective Function}
Minimize total cost:
\[
\text{Total Cost} = \sum_{t=1}^{T} (coal\_cost \cdot coal_t + nuke\_cost \cdot nuke_t)
\]

\subsection*{Output Format}
The output should be structured as follows:
\begin{itemize}
    \item \texttt{"coal\_cap\_added"}: List of coal capacities added in each year
    \item \texttt{"nuke\_cap\_added"}: List of nuclear capacities added in each year
    \item \texttt{"total\_cost"}: Total cost of the capacity expansion plan
\end{itemize}

\end{document}