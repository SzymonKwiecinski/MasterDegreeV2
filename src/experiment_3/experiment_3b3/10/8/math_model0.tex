\documentclass{article}
\usepackage{amsmath}
\begin{document}

\title{Linear Programming Model for Maximizing Profit in Spare Automobile Parts Production}
\author{}
\date{}
\maketitle

\section*{Variables}
Let \( x_k \) be the quantity of spare part \( k \) produced, where \( k = 1, 2, \ldots, K \).

\section*{Parameters}
\begin{itemize}
    \item \( time_{k, s} \): The required worker-hours for part \( k \) in shop \( s \).
    \item \( profit_k \): The profit from part \( k \).
    \item \( capacity_s \): The capacity of shop \( s \) in working hours.
\end{itemize}

\section*{Objective Function}
The objective is to maximize the total profit from producing the spare parts:
\[
\text{Maximize } Z = \sum_{k=1}^{K} profit_k \cdot x_k
\]

\section*{Constraints}
The production of each part must adhere to the capacity constraints of each shop. For each shop \( s \):
\[
\sum_{k=1}^{K} time_{k, s} \cdot x_k \leq capacity_s \quad \text{for } s = 1, 2, \ldots, S
\]

Additionally, we need to ensure that the quantities of parts produced are non-negative:
\[
x_k \geq 0 \quad \text{for } k = 1, 2, \ldots, K
\]

\section*{Model Formulation}
The linear programming model can be summarized as follows:

\begin{align*}
\text{Maximize } & Z = \sum_{k=1}^{K} profit_k \cdot x_k \\
\text{subject to } & \sum_{k=1}^{K} time_{k, s} \cdot x_k \leq capacity_s \quad \forall s = 1, 2, \ldots, S \\
& x_k \geq 0 \quad \forall k = 1, 2, \ldots, K
\end{align*}

\end{document}