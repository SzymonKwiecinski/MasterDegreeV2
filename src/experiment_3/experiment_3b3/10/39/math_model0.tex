\documentclass{article}
\usepackage{amsmath}
\begin{document}

\section*{Mathematical Model for Cafeteria Staffing Problem}

\subsection*{Parameters}
\begin{itemize}
    \item \( N \): Total number of days
    \item \( \text{num}_n \): Number of employees required on day \( n \), for \( n = 1, 2, \ldots, N \)
    \item \( d_w \): Number of working days for each employee (i.e., \( n_{\text{working\_days}} \))
    \item \( d_r \): Number of resting days for each employee (i.e., \( n_{\text{resting\_days}} \))
\end{itemize}

\subsection*{Decision Variables}
Let:
\begin{itemize}
    \item \( \text{total\_number} \): Total number of employees to hire
    \item \( x_{n,i} \): A binary variable that is 1 if employee \( i \) works on day \( n \), and 0 otherwise, where \( i = 1, 2, \ldots, \text{total\_number} \)
\end{itemize}

\subsection*{Objective Function}
The objective is to minimize the number of employees hired:
\[
\text{Minimize } \text{total\_number}
\]

\subsection*{Constraints}
1. Ensure the required number of employees for each day \( n \):
\[
\sum_{i=1}^{\text{total\_number}} x_{n,i} \geq \text{num}_n, \quad \forall n = 1, 2, \ldots, N
\]

2. Each employee works for \( d_w \) days and then rests for \( d_r \) days. Define a cycle of \( d_c = d_w + d_r \). The days an employee works influence the days they can take off:
\[
x_{n,i} + x_{n+1,i} + \ldots + x_{n+d_w-1,i} = d_w, \quad \forall n \text{ such that } n + d_w - 1 \leq N
\]
\[
x_{n+d_w,i} + x_{n+d_w+1,i} + \ldots + x_{n+d_c-1,i} = 0, \quad \forall n \text{ such that } n + d_c - 1 \leq N
\]

\subsection*{Output Variables}
The output consists of:
\begin{itemize}
    \item \( \text{total\_number} \): Total number of employees hired.
    \item \( \text{is\_work}_{n,i} \): A 2D array where \( \text{is\_work}_{n,i} \) is 1 if employee \( i \) works on day \( n \), and 0 otherwise.
\end{itemize}

\end{document}