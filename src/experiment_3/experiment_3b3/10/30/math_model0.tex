\documentclass{article}
\usepackage{amsmath}
\begin{document}

\section*{Quadratic Curve Fitting Problem}

Given a set of corresponding values for \( x \) and \( y \):

\[
y = \{ y_k \,|\, k = 1, 2, \ldots, K \}
\]
\[
x = \{ x_k \,|\, k = 1, 2, \ldots, K \}
\]

We aim to fit a quadratic curve of the form:

\[
y = c \cdot x^2 + b \cdot x + a
\]

where:
- \( c \) is the coefficient of the quadratic term,
- \( b \) is the coefficient of the linear term,
- \( a \) is the constant term.

The goal is to minimize the sum of absolute deviations of each observed value of \( y \) from the predicted value given by the quadratic relationship. This can be mathematically expressed as:

\[
\text{Minimize} \quad \sum_{k=1}^{K} |y_k - (c \cdot x_k^2 + b \cdot x_k + a)|
\]

Subject to:

\[
y_k - (c \cdot x_k^2 + b \cdot x_k + a) \leq e_k, \quad \forall k
\]
\[
-(y_k - (c \cdot x_k^2 + b \cdot x_k + a)) \leq e_k, \quad \forall k
\]

where \( e_k \geq 0 \) are the new variables representing the absolute deviations.

This forms a linear programming problem where our output will be:

\[
\text{Output Format:} \quad \{ \text{quadratic}: c, \text{linear}: b, \text{constant}: a \}
\]

\end{document}