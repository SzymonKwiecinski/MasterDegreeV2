\documentclass{article}
\usepackage{amsmath}
\begin{document}

\section*{Linear Programming Model for Alloy Production}

\subsection*{Parameters}
\begin{itemize}
    \item \( alloy\_quant \): Total quantity of alloy to produce (in lb)
    \item \( target_{m} \): Target quantity of metal \( m \) (for \( m = 1, \ldots, M \))
    \item \( ratio_{k,m} \): Ratio of metal \( m \) in alloy \( k \) (for \( k = 1, \ldots, K \) and \( m = 1, \ldots, M \))
    \item \( price_{k} \): Price of alloy \( k \) (for \( k = 1, \ldots, K \))
\end{itemize}

\subsection*{Decision Variables}
Let \( amount_{k} \) represent the amount of alloy \( k \) to purchase (for \( k = 1, \ldots, K \)).

\subsection*{Objective Function}
Minimize the total cost of the alloys:
\[
\text{Minimize } Z = \sum_{k=1}^{K} price_{k} \cdot amount_{k}
\]

\subsection*{Constraints}
1. Total quantity constraint:
\[
\sum_{k=1}^{K} amount_{k} = alloy\_quant
\]

2. Metal composition constraints:
\[
\sum_{k=1}^{K} ratio_{k,m} \cdot amount_{k} = target_{m} \quad \text{for } m = 1, \ldots, M
\]

3. Non-negativity constraints:
\[
amount_{k} \geq 0 \quad \text{for } k = 1, \ldots, K
\]

\subsection*{Output}
The output will provide the amounts of each alloy needed:
\[
\text{Output: } amount = [amount_{1}, amount_{2}, \ldots, amount_{K}]
\]

\end{document}