\documentclass{article}
\usepackage{amsmath}
\begin{document}

\section*{Linear Programming Model for Fitting a Line}

Given data points \((x_k, y_k)\) for \(k = 1, \ldots, K\), we aim to find the best fitting line of the form:

\[
y = bx + a
\]

where \(b\) is the slope and \(a\) is the intercept. The objective is to minimize the sum of the absolute deviations between the observed values \(y_k\) and the predicted values \((bx_k + a)\).

The objective function to minimize can be expressed as:

\[
\min \sum_{k=1}^{K} |y_k - (bx_k + a)|
\]

To convert this into a standard linear programming form, we introduce auxiliary variables \(d_k\) to represent the absolute deviations:

\[
d_k \geq y_k - (bx_k + a) \quad \forall k
\]
\[
d_k \geq -(y_k - (bx_k + a)) \quad \forall k
\]

This implies:

\[
d_k \geq |y_k - (bx_k + a)| \quad \forall k
\]

Thus, the complete formulation is:

\[
\text{Minimize } \sum_{k=1}^{K} d_k
\]

subject to:

\[
d_k \geq y_k - (bx_k + a) \quad \forall k
\]
\[
d_k \geq -(y_k - (bx_k + a)) \quad \forall k
\]

The output of this linear programming problem will yield the optimal values of \(a\) (intercept) and \(b\) (slope), defined as:

\begin{itemize}
    \item \textbf{Intercept}: \(a\)
    \item \textbf{Slope}: \(b\)
\end{itemize}

The final results can be formatted as:

\[
\{\text{"intercept"}: a, \text{"slope"}: b\}
\]

\end{document}