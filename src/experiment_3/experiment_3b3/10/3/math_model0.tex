\documentclass{article}
\usepackage{amsmath}
\begin{document}

\section*{Mathematical Model for Electricity Capacity Expansion}

\subsection*{Parameters}
\begin{align*}
T & : \text{Number of years} \\
demand_t & : \text{Demand for electricity in year } t \\
oil_t & : \text{Existing oil-fired capacity in year } t \\
coal\_cost & : \text{Cost per megawatt of coal-fired capacity} \\
nuke\_cost & : \text{Cost per megawatt of nuclear power capacity} \\
max\_nuke & : \text{Maximum percentage of total capacity that can be nuclear} \\
coal\_life & : \text{Lifetime of coal plants (in years)} \\
nuke\_life & : \text{Lifetime of nuclear plants (in years)} \\
\end{align*}

\subsection*{Decision Variables}
Let:
\begin{align*}
coal_t & : \text{Capacity added in year } t \text{ (coal)} \\
nuke_t & : \text{Capacity added in year } t \text{ (nuclear)} \\
\end{align*}

\subsection*{Objective Function}
Minimize the total cost:
\begin{align*}
\text{Minimize } Z = \sum_{t=1}^{T} (coal\_cost \cdot coal_t + nuke\_cost \cdot nuke_t)
\end{align*}

\subsection*{Constraints}

1. **Demand Constraint:**
   The total capacity must meet or exceed the demand:
   \begin{align*}
   \sum_{i=\max(1, t - coal\_life + 1)}^{t} coal_i + \sum_{j=\max(1, t - nuke\_life + 1)}^{t} nuke_j + oil_t \geq demand_t \quad \forall t = 1, \ldots, T
   \end{align*}

2. **Nuclear Capacity Constraint:**
   The percentage of nuclear capacity must not exceed the maximum allowed:
   \begin{align*}
   \frac{\sum_{j=1}^{t} nuke_j}{\sum_{i=1}^{t} (coal_i + nuke_i + oil_t)} \leq \frac{max\_nuke}{100} \quad \forall t = 1, \ldots, T
   \end{align*}

3. **Capacity Limit Constraints:**
   The capacities added in each year must be non-negative:
   \begin{align*}
   coal_t \geq 0 \\
   nuke_t \geq 0 \quad \forall t = 1, \ldots, T
   \end{align*}

\end{document}