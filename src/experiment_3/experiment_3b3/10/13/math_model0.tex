\documentclass{article}
\usepackage{amsmath}
\begin{document}

\section*{Problem Formulation}

We define the feasible region \( P \) as the set of points \( x \in \mathbb{R}^N \) that satisfy the linear inequality constraints:

\[
P = \{ x \in \mathbb{R}^N \mid a_i^T x \leq b_i, \; i = 1, \ldots, m \}
\]

where \( a_i \in \mathbb{R}^N \) and \( b_i \in \mathbb{R} \).

A ball with center \( y \) in \( \mathbb{R}^N \) and radius \( r \) is defined as:

\[
B(y, r) = \{ x \in \mathbb{R}^N \mid \| x - y \|_2 \leq r \}
\]

Our objective is to find a ball \( B(y, r) \) that is entirely contained within the set \( P \) while maximizing the radius \( r \). The center of such a ball is referred to as the Chebyshev center of the set \( P \).

\subsection*{Input}

The input is provided in the following format:

\begin{verbatim}
{
    "A": [[a_i_j for j in 1, ..., n] for i in 1, ..., m],
    "b": [b_i for i in 1, ..., m]
}
\end{verbatim}

\subsection*{Output}

The output should include the center of the ball and the radius, formatted as follows:

\begin{verbatim}
{
    "center": [y_j for j in 1, ..., n],
    "radius": r
}
\end{verbatim}

\end{document}