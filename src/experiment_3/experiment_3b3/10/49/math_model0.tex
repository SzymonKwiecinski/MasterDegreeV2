\documentclass{article}
\usepackage{amsmath}
\begin{document}

\section*{Linear Programming Model for Auto Parts Manufacturing}

\subsection*{Variables}
Let \(b_p\) be the number of batches produced of part \(p\), where \(p = 1, 2, \ldots, P\).  
Let \(T\) be the total profit.

\subsection*{Parameters}
\begin{itemize}
    \item \(time_{m,p}\): time required on machine \(m\) to produce one batch of part \(p\) (in hours).
    \item \(cost_{m}\): cost per hour of using machine \(m\).
    \item \(available_{m}\): availability of machine \(m\) (in hours) per month.
    \item \(price_{p}\): selling price per batch of part \(p\).
    \item \(min\_batches_{p}\): minimum number of batches that must be produced for part \(p\) to fulfill a contract.
\end{itemize}  

\subsection*{Objective Function}
The objective is to maximize the total profit \(T\), which can be expressed as:
\[
T = \sum_{p=1}^{P} (price_{p} \cdot b_{p} - \sum_{m=1}^{M} cost_{m} \cdot \frac{time_{m,p}}{100} \cdot b_{p})
\]

\subsection*{Constraints}
1. **Production Constraints**: Each part must meet the minimum production requirement:
   \[
   b_{p} \geq min\_batches_{p}, \quad \forall p
   \]

2. **Machine Availability Constraints**: The total time used on all machines for all produced parts must not exceed the available time:
   \[
   \sum_{p=1}^{P} \sum_{m=1}^{M} time_{m,p} \cdot b_{p} \leq \sum_{m=1}^{M} available_{m}
   \]

3. **Non-negativity Constraints**: The number of batches produced must be non-negative:
   \[
   b_{p} \geq 0, \quad \forall p
   \]

\subsection*{Output}
The solution will yield the batch quantities:
\[
\text{Output: } \{batches = [b_{p} \text{ for } p = 1, \ldots, P], \, total\_profit = T \}
\]

\end{document}