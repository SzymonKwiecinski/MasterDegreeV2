\documentclass{article}
\usepackage{amsmath}
\begin{document}

\section*{Linear Programming Model for Lamp Power Optimization}

\subsection*{Problem Definition}
Consider a road divided into \( N \) segments that is illuminated by \( M \) lamps. Let \( \text{power}_j \) be the power of the \( j \)-th lamp. The illumination \( \text{ill}_i \) of the \( i \)-th segment is given by:

\[
\text{ill}_i = \sum_{j=1}^{M} \text{coeff}_{i,j} \cdot \text{power}_j
\]

where \( \text{coeff}_{i,j} \) are known coefficients. Let \( \text{desired}_i \) be the desired illumination of road segment \( i \).

\subsection*{Objective}
We aim to choose the lamp powers \( \text{power}_j \) so that the illuminations \( \text{ill}_i \) are close to the desired illuminations \( \text{desired}_i \) by minimizing the absolute error.

\subsection*{Mathematical Formulation}
Define the absolute error for each segment as:

\[
\text{error}_i = |\text{ill}_i - \text{desired}_i|
\]

The objective is to minimize the total absolute error:

\[
\min \sum_{i=1}^{N} \text{error}_i
\]

Taking into account the relationship between illuminations and powers, the optimization problem can be expressed as:

\[
\min \sum_{i=1}^{N} \left| \sum_{j=1}^{M} \text{coeff}_{i,j} \cdot \text{power}_j - \text{desired}_i \right|
\]

\subsection*{Constraints}
Assuming that the power of each lamp must be non-negative, we have the following constraints:

\[
\text{power}_j \geq 0 \quad \text{for } j = 1, \ldots, M
\]

\subsection*{Output}
The expected output of the optimization will include:

\begin{enumerate}
    \item \( \text{power}_j \) - the optimal power of the \( j \)-th lamp.
    \item \( \text{error} \) - the total absolute error between the illuminations \( \text{ill}_i \) and the desired illuminations \( \text{desired}_i \).
\end{enumerate}

The output format will be:

\[
\{
    "power": [\text{power}_j \text{ for } j = 1, \ldots, M],
    "error": \text{error}
\}
\]

\end{document}