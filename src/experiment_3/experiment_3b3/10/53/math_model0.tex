\documentclass{article}
\usepackage{amsmath}
\begin{document}

\section*{Transportation Problem Formulation}

We define the following parameters and sets:

\begin{itemize}
    \item Let \( n \) be the number of cities.
    \item Let \( k \) be the terminal city.
    \item Let \( p \) be the port city.
    \item Let \( l \) be the destination city.
    \item Let \( A \) be the set of all routes.
    \item Let \( C_{i,j} \) be the transportation cost per metric ton along route from city \( i \) to city \( j \).
    \item Let \( supply_k \) be the total supply of soybeans at terminal \( k \).
    \item Let \( demand_l \) be the total demand of soybeans at destination \( l \).
\end{itemize}

\textbf{Decision Variables:}
\begin{itemize}
    \item Let \( amount_{i,j} \) be the amount of soybeans shipped from city \( i \) to city \( j \).
\end{itemize}

\textbf{Objective Function:}
We want to minimize the total transportation cost:
\[
\text{Minimize } Z = \sum_{(i,j) \in A} C_{i,j} \cdot amount_{i,j}
\]

\textbf{Subject to:}
1. Supply constraints:
\[
\sum_{j \in \text{outgoing}(k)} amount_{k,j} \leq supply_k \quad \forall k
\]
2. Demand constraints:
\[
\sum_{i \in \text{incoming}(l)} amount_{i,l} \geq demand_l \quad \forall l
\]
3. Non-negativity constraints:
\[
amount_{i,j} \geq 0 \quad \forall (i,j) \in A
\]

\textbf{Output:}
The expected output format is:
\begin{verbatim}
{
    "distribution": [
        {"from": i,
         "to": j,
         "amount": amount_{i,j}
        }
        for id in 1, ..., m
    ],
    "total_cost": total_cost
}
\end{verbatim}

\end{document}