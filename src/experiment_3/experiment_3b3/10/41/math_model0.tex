\documentclass{article}
\usepackage{amsmath}
\begin{document}

\section*{Mathematical Model for Floppy Disk Backup Problem}

\subsection*{Parameters}
\begin{itemize}
    \item \( C \): Capacity of each floppy disk (in GB)
    \item \( N \): Total number of files
    \item \( size_j \): Size of file \( j \) (in GB) for \( j = 1, 2, \ldots, N \)
\end{itemize}

\subsection*{Decision Variables}
\begin{itemize}
    \item \( n \): Total number of floppy disks used
    \item \( x_{ij} \): Binary variable that is 1 if file \( j \) is placed on disk \( i \), and 0 otherwise
\end{itemize}

\subsection*{Objective Function}
Minimize the total number of floppy disks used:
\[
\text{Minimize } n
\]

\subsection*{Constraints}
1. Each file must be assigned to exactly one disk:
\[
\sum_{i=1}^{n} x_{ij} = 1, \quad \forall j \in \{1, 2, \ldots, N\}
\]

2. The total size of files on each disk cannot exceed its capacity:
\[
\sum_{j=1}^{N} size_j \cdot x_{ij} \leq C, \quad \forall i \in \{1, 2, \ldots, n\}
\]

3. The number of disks used must equal \( n \):
\[
\sum_{i=1}^{n} y_i = n
\]
where \( y_i \) is a binary variable that is 1 if disk \( i \) is used, and 0 otherwise.

4. Link between \( x_{ij} \) and \( y_i \):
\[
x_{ij} \leq y_i, \quad \forall i \in \{1, 2, \ldots, n\}, \quad \forall j \in \{1, 2, \ldots, N\}
\]

\subsection*{Output}
The final output will include:
\begin{itemize}
    \item \( n \): Total number of floppy disks used
    \item \( whichdisk_j \): An array indicating the disk on which each file \( j \) is stored, for \( j = 1, 2, \ldots, N \)
\end{itemize}

\end{document}