\documentclass{article}
\usepackage{amsmath}
\begin{document}

\section*{Linear Programming Model for Police Officer Shift Assignment}

\textbf{Indices:}
\begin{itemize}
    \item $s$: index for shifts, where $s = 1, 2, \ldots, S$
\end{itemize}

\textbf{Parameters:}
\begin{itemize}
    \item $officers\_needed_s$: number of police officers needed during shift $s$
    \item $shift\_cost_s$: cost incurred for starting shift $s$
\end{itemize}

\textbf{Decision Variables:}
\begin{itemize}
    \item $officers\_assigned_s$: number of police officers assigned to shift $s$
\end{itemize}

\textbf{Objective Function:}
\begin{equation}
\text{Minimize } Z = \sum_{s=1}^{S} shift\_cost_s \cdot x_s
\end{equation}

\textbf{Constraints:}
\begin{itemize}
    \item For each shift, the number of officers assigned must meet the requirement for that shift:
    \begin{equation}
    officers\_assigned_s \geq officers\_needed_s, \quad \forall s \in \{1, 2, \ldots, S\}
    \end{equation}
    
    \item An officer works for two consecutive shifts, thus the assignment for each shift must be consistent with the officer's duty:
    \begin{equation}
    officers\_assigned_s = officers\_assigned_{s-1} \text{ for } s=2,3,\ldots,S
    \end{equation}
    
    \item Additionally, we can state that the officers assigned for the first shift must cover the need:
    \begin{equation}
    officers\_assigned_1 \geq officers\_needed_1
    \end{equation}
\end{itemize}

\textbf{Expected Output:}
\begin{itemize}
    \item $officers\_assigned = \{officers\_assigned_1, officers\_assigned_2, \ldots, officers\_assigned_S\}$
    \item $total\_cost = Z$
\end{itemize}

\end{document}