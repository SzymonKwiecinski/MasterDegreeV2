\documentclass{article}
\usepackage{amsmath}
\begin{document}

\section*{Linear Programming Model for Linear Regression}

Given a set of observed values \( y_k \) and corresponding values \( x_k \) for \( k = 1, \ldots, K \), we aim to fit the best straight line of the form \( y = bx + a \). The objective is to minimize the sum of absolute deviations between the observed values and the predicted values.

Let us define the parameters and variables involved in our linear programming model:

\begin{itemize}
    \item Let \( a \) be the intercept of the fitted line.
    \item Let \( b \) be the slope of the fitted line.
    \item Let \( d_k \) represent the absolute deviation for each observation, defined as \( d_k = |y_k - (bx_k + a)| \) for \( k = 1, 2, \ldots, K \).
\end{itemize}

\subsection*{Objective Function}

We want to minimize the total absolute deviation:

\[
\text{Minimize } Z = \sum_{k=1}^{K} d_k
\]

\subsection*{Constraints}

To incorporate the absolute values into our linear programming formulation, we can use auxiliary variables. We introduce non-negative variables \( u_k \) and \( v_k \) such that:

\[
d_k = u_k + v_k \quad \text{for each } k
\]

with the following constraints for each \( k \):

\[
y_k - (bx_k + a) \leq u_k \quad \text{(upper bound)}
\]
\[
-(y_k - (bx_k + a)) \leq v_k \quad \text{(lower bound)}
\]

These constraints ensure that the absolute deviation \( d_k \) is represented correctly.

\subsection*{Final Model}

The complete linear programming model can be summarized as follows:

\begin{align*}
\text{Minimize } & Z = \sum_{k=1}^{K} (u_k + v_k) \\
\text{Subject to } & \\
& y_k - (bx_k + a) \leq u_k & \forall k = 1, \ldots, K \\
& -(y_k - (bx_k + a)) \leq v_k & \forall k = 1, \ldots, K \\
& u_k \geq 0, v_k \geq 0 & \forall k = 1, \ldots, K
\end{align*}

\subsection*{Output Variables}

After solving the linear programming model, we obtain:

\begin{itemize}
    \item The value of \( a \) which represents the intercept of the fitted line.
    \item The value of \( b \) which represents the slope of the fitted line.
\end{itemize}

The output format is:

\[
\begin{align*}
\text{Output:} \quad \{ & \\
& \text{"intercept": } a, \\
& \text{"slope": } b \\
& \}
\end{align*}
\]

\end{document}