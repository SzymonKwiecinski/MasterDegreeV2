\documentclass{article}
\usepackage{amsmath}
\begin{document}

\section*{Mathematical Model for Nurse Scheduling Problem}

\subsection*{Parameters}
\begin{itemize}
    \item $period$: The number of consecutive days a nurse works the night shift.
    \item $d_j$: The demand for nurses on day $j$, where $j = 1, \ldots, 7$.
\end{itemize}

\subsection*{Decision Variables}
\begin{itemize}
    \item $x_j$: The number of nurses that start their working period on day $j$, for $j = 1, \ldots, 7$.
    \item $N$: The total number of nurses hired.
\end{itemize}

\subsection*{Objective Function}
Minimize the total number of nurses hired:
\[
\text{Minimize } N = \sum_{j=1}^{7} x_j
\]

\subsection*{Constraints}
For each day $j$, the total number of nurses working must meet the demand:
\[
\sum_{i=0}^{period-1} x_{(j-i) \bmod 7} \geq d_j, \quad \text{for } j = 1, \ldots, 7
\]

Where $(j-i) \bmod 7$ ensures that the index wraps around the week.

\subsection*{Non-negativity Constraints}
\[
x_j \geq 0 \quad \text{and integer, for } j = 1, \ldots, 7
\]

\end{document}