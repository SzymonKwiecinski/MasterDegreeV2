\documentclass{article}
\usepackage{amsmath}
\begin{document}

\section*{Linear Programming Model}

\subsection*{Sets}
\begin{itemize}
    \item Let \( K \) be the set of foods, indexed by \( k \) where \( k = 1, \ldots, K \).
    \item Let \( M \) be the set of nutritional ingredients, indexed by \( m \) where \( m = 1, \ldots, M \).
\end{itemize}

\subsection*{Parameters}
\begin{itemize}
    \item \( \text{price}_{k} \): Price per unit of food \( k \).
    \item \( \text{demand}_{m} \): Minimum required units of nutrient \( m \) per day.
    \item \( \text{nutri}_{k, m} \): Units of nutrient \( m \) contained in one unit of food \( k \).
\end{itemize}

\subsection*{Decision Variables}
\begin{itemize}
    \item \( \text{quantity}_{k} \): Amount of food \( k \) to purchase.
\end{itemize}

\subsection*{Objective Function}
Minimize the total cost of food:
\[
\text{Minimize} \quad Z = \sum_{k=1}^{K} \text{price}_{k} \cdot \text{quantity}_{k}
\]

\subsection*{Constraints}
To ensure that the nutritional demands are met, we have the following constraints for each nutrient \( m \):
\[
\sum_{k=1}^{K} \text{nutri}_{k, m} \cdot \text{quantity}_{k} \geq \text{demand}_{m}, \quad \forall m \in M
\]

\subsection*{Non-negativity Constraints}
\[
\text{quantity}_{k} \geq 0, \quad \forall k \in K
\]

\end{document}