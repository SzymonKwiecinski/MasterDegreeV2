\documentclass{article}
\usepackage{amsmath}
\begin{document}

\section*{Mathematical Model for Communication Network Optimization}

\subsection*{Notation}
\begin{itemize}
    \item Let \( n \) be the number of nodes in the communication network.
    \item Let \( A \) be the set of all links in the network, where a link is represented as an ordered pair \( (i, j) \).
    \item Let \( U_{i,j} \) be the maximum capacity (in bits per second) of the link \( (i, j) \).
    \item Let \( C_{i,j} \) be the cost per bit transmitted along the link \( (i, j) \).
    \item Let \( B_{k,l} \) be the data generation rate (in bits per second) from node \( k \) to destination node \( l \).
\end{itemize}

\subsection*{Decision Variables}
Let \( x_{i,j} \) be the flow of data (in bits per second) transmitted along the link \( (i, j) \).

\subsection*{Objective Function}
We aim to minimize the total cost of transmission:
\[
\text{Minimize} \quad Z = \sum_{(i,j) \in A} C_{i,j} \cdot x_{i,j}
\]

\subsection*{Constraints}
1. **Flow Capacity Constraint**: The flow along each link cannot exceed its capacity:
   \[
   x_{i,j} \leq U_{i,j}, \quad \forall (i,j) \in A
   \]

2. **Flow Conservation Constraint**: For each node \( k \) (except for source and destination nodes), the flow in must equal the flow out:
   \[
   \sum_{j: (k,j) \in A} x_{k,j} - \sum_{i: (i,k) \in A} x_{i,k} = 0, \quad \forall k \text{ where } k \text{ is not a source or destination node}
   \]

3. **Supply Constraint**: The flow originating from a source node to its destination must equal the rate of data generation:
   \[
   \sum_{j: (k,j) \in A} x_{k,j} = B_{k,l}, \quad \forall k, l \text{ where } k \text{ is a source node}
   \]

4. **Demand Constraint**: The flow into a destination node must match the total data rate required to be delivered:
   \[
   \sum_{i: (i,l) \in A} x_{i,l} = \sum_{k} B_{k,l}, \quad \forall l \text{ where } l \text{ is a destination node}
   \]

5. **Non-negativity Constraint**: The flow on each link must be non-negative:
   \[
   x_{i,j} \geq 0, \quad \forall (i,j) \in A
   \]

\subsection*{Output Variables}
\begin{itemize}
    \item \( \text{total\_cost} \): The total cost of all the paths.
    \item \( \text{path\_cost} \): The cost associated with each specific path.
    \item \( \text{path\_flow} \): The flow going through each corresponding path.
\end{itemize}

\subsection*{Output Format}
The output should be structured as follows:
\begin{verbatim}
{
    "optimized_paths": {
        "paths": [
            {
                "source": k,
                "destination": l,
                "route": [k, i_1, i_2, ..., l],
                "path_flow": path_flow,
                "path_cost": path_cost
            },
            for id in 1, ..., m 
        ],
    "total_cost": "total_cost"
}
\end{verbatim}

\end{document}