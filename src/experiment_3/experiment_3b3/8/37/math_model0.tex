\documentclass{article}
\usepackage{amsmath}
\begin{document}

\section*{Linear Programming Model for Spare Automobile Parts Production}

\subsection*{Indices}
\begin{itemize}
    \item \( k \): index for spare parts, where \( k = 1, 2, \ldots, K \)
    \item \( s \): index for shops, where \( s = 1, 2, \ldots, S \)
\end{itemize}

\subsection*{Parameters}
\begin{itemize}
    \item \( time_{k,s} \): worker-hours required to produce part \( k \) in shop \( s \)
    \item \( profit_k \): profit from producing one unit of part \( k \)
    \item \( capacity_s \): total working hours available in shop \( s \)
\end{itemize}

\subsection*{Decision Variables}
\begin{itemize}
    \item \( quantity_k \): quantity of spare part \( k \) to be produced
\end{itemize}

\subsection*{Objective Function}
The objective is to maximize the total profit:
\[
\text{Maximize } Z = \sum_{k=1}^{K} profit_k \cdot quantity_k
\]

\subsection*{Constraints}
The production of each part must not exceed the capacity in any shop. Thus, for each shop \( s \):
\[
\sum_{k=1}^{K} time_{k,s} \cdot quantity_k \leq capacity_s \quad \forall s = 1, 2, \ldots, S
\]

Additionally, the production quantities must be non-negative:
\[
quantity_k \geq 0 \quad \forall k = 1, 2, \ldots, K
\]

\subsection*{Complete Model}
Combining the objective function and the constraints, the complete linear programming model can be stated as:

\begin{align*}
\text{Maximize } & Z = \sum_{k=1}^{K} profit_k \cdot quantity_k \\
\text{subject to } & \sum_{k=1}^{K} time_{k,s} \cdot quantity_k \leq capacity_s, \quad s = 1, 2, \ldots, S \\
& quantity_k \geq 0, \quad k = 1, 2, \ldots, K
\end{align*}

\end{document}