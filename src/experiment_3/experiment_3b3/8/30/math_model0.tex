\documentclass{article}
\usepackage{amsmath}
\begin{document}

\section*{Quadratic Curve Fitting}

Given a set of observations for quantities \(x\) and \(y\):

\[
\begin{align*}
y & = [y_{1}, y_{2}, \ldots, y_{K}] \\
x & = [x_{1}, x_{2}, \ldots, x_{K}]
\end{align*}
\]

We aim to fit the best quadratic curve of the form:

\[
y = c \cdot x^2 + b \cdot x + a
\]

where:
- \(c\) is the coefficient of the quadratic term,
- \(b\) is the coefficient of the linear term,
- \(a\) is the coefficient of the constant term.

The objective is to minimize the sum of absolute deviations between the observed \(y_k\) values and the predicted values \(y_{predicted,k}\):

\[
\text{Minimize} \quad \sum_{k=1}^{K} |y_{k} - (c \cdot x_{k}^2 + b \cdot x_{k} + a)|
\]

\textbf{Output:} The coefficients of the quadratic, linear, and constant terms can be represented as:

\[
\begin{align*}
\text{quadratic} & \rightarrow c \\
\text{linear} & \rightarrow b \\
\text{constant} & \rightarrow a \\
\end{align*}
\]

\textbf{Output Format:}

\[
\{
    \text{"quadratic"}: c,
    \text{"linear"}: b,
    \text{"constant"}: a
\}
\]

\end{document}