\documentclass{article}
\usepackage{amsmath}
\begin{document}

\section*{Linear Programming Model for Lamp Power Optimization}

\subsection*{Problem Definition}
Consider a road divided into \(N\) segments illuminated by \(M\) lamps. Let \(power_j\) represent the power of the \(j\)-th lamp. The illumination of the \(i\)-th segment is given by:

\[
ill_i = \sum_{j=1}^{M} coeff_{i,j} \cdot power_j
\]

where \(coeff_{i,j}\) are known coefficients. Let \(desired_i\) be the desired illumination for the road segment \(i\). Our goal is to select the lamp powers \(power_j\) such that the illuminations \(ill_i\) are close to the desired illuminations \(desired_i\).

\subsection*{Input}
The input is structured as follows:

\begin{itemize}
    \item \(coeff\): A matrix of coefficients where \(coeff[i][j]\) corresponds to the coefficient of the \(j\)-th lamp for the \(i\)-th segment.
    \item \(desired\): A vector of desired illuminations for each segment, where \(desired[i]\) is the desired illumination for the \(i\)-th segment.
\end{itemize}

\subsection*{Objective}
We aim to minimize the absolute error between the actual illuminations and the desired illuminations, defined as:

\[
\text{error} = \sum_{i=1}^{N} |ill_i - desired_i|
\]

\subsection*{Optimization Problem}
The optimization problem can be formulated as follows:

\begin{align*}
\text{Minimize} & \quad \sum_{i=1}^{N} |ill_i - desired_i| \\
\text{subject to} & \quad ill_i = \sum_{j=1}^{M} coeff_{i,j} \cdot power_j \quad \text{for } i = 1, \ldots, N \\
& \quad power_j \geq 0 \quad \text{for } j = 1, \ldots, M
\end{align*}

\subsection*{Output}
The output of the optimization process will include:

\begin{itemize}
    \item \(power\): The optimal power settings for the \(M\) lamps, represented as a vector where \(power[j]\) is the power of the \(j\)-th lamp.
    \item \(error\): The absolute error calculated between the actual and desired illuminations.
\end{itemize}

\end{document}