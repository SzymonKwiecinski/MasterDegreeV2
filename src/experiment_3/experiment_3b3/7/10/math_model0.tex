\documentclass{article}
\usepackage{amsmath}
\begin{document}

\section*{Linear Programming Model for School Assignment Problem}

\subsection*{Sets}
\begin{itemize}
    \item \( N \): Set of neighborhoods
    \item \( S \): Set of schools
    \item \( G \): Set of grades
\end{itemize}

\subsection*{Parameters}
\begin{itemize}
    \item \( \text{capacity}_{s,g} \): Capacity of school \( s \) for grade \( g \)
    \item \( \text{population}_{n,g} \): Student population of grade \( g \) in neighborhood \( n \)
    \item \( d_{n,s} \): Distance from neighborhood \( n \) to school \( s \)
\end{itemize}

\subsection*{Decision Variables}
\begin{itemize}
    \item \( x_{n,s,g} \): Number of students of grade \( g \) in neighborhood \( n \) assigned to school \( s \)
\end{itemize}

\subsection*{Objective Function}
Minimize the total distance traveled by all students:
\[
\text{Minimize } Z = \sum_{n \in N} \sum_{s \in S} \sum_{g \in G} d_{n,s} \cdot x_{n,s,g}
\]

\subsection*{Constraints}
1. Capacity Constraints for each school:
\[
\sum_{n \in N} \sum_{g \in G} x_{n,s,g} \leq \text{capacity}_{s,g} \quad \forall s \in S, \forall g \in G
\]

2. Demand Constraints for each neighborhood:
\[
\sum_{s \in S} x_{n,s,g} = \text{population}_{n,g} \quad \forall n \in N, \forall g \in G
\]

3. Non-negativity Constraints:
\[
x_{n,s,g} \geq 0 \quad \forall n \in N, \forall s \in S, \forall g \in G
\]

\subsection*{Output}
The expected output format is:
\begin{verbatim}
{
    "assignment": [[[x_{n,s,g} for g in 1, ..., G] for s in 1, ..., S] for n in 1, ..., N],
    "total_distance": total_distance
}
\end{verbatim}

\end{document}