\documentclass{article}
\usepackage{amsmath}
\begin{document}

\section*{Transportation Problem Model}

Given a transportation network with the following parameters:

\begin{itemize}
    \item \( n \) cities
    \item A set of routes \( A \) that connect different cities
    \item Transportation cost \( C_{i,j} \) per metric ton along the route from city \( i \) to city \( j \)
    \item Supply \( supply_k \) at terminal city \( k \)
    \item Demand \( demand_l \) at destination city \( l \)
\end{itemize}

Define the variables:
\begin{itemize}
    \item \( amount_{i,j} \): the amount of soybeans shipped from city \( i \) to city \( j \)
\end{itemize}

\subsection*{Objective Function}
Minimize the total transportation cost:
\[
\text{Minimize} \quad Z = \sum_{(i,j) \in A} C_{i,j} \cdot amount_{i,j}
\]

\subsection*{Constraints}

1. Supply constraints for each terminal city \( k \):
\[
\sum_{j} amount_{k,j} \leq supply_k, \quad \forall k
\]

2. Demand constraints for each destination city \( l \):
\[
\sum_{i} amount_{i,l} \geq demand_l, \quad \forall l
\]

3. Non-negativity constraints:
\[
amount_{i,j} \geq 0, \quad \forall (i,j) \in A
\]

\subsection*{Output}
The solution will provide:
\begin{itemize}
    \item The quantity of soybeans shipped from each source to destination: \( amount_{i,j} \)
    \item The total transportation cost: \( total\_cost \)
\end{itemize}

The output format will be:
\begin{verbatim}
{
    "distribution": [
        {"from": i,
         "to": j,
         "amount": amount_{i,j}
        }
        for id in 1, ..., m
    ],
    "total_cost": total_cost
}
\end{verbatim}

\end{document}