\documentclass{article}
\usepackage{amsmath}
\begin{document}

\section*{Linear Programming Model for Nutritional Optimization}

\subsection*{Parameters}
\begin{itemize}
    \item Let \( K \) be the number of different foods available in the market.
    \item Let \( M \) be the number of different nutritional ingredients.
    \item \( \text{price}_{k} \): Price per unit of food \( k \) for \( k = 1, 2, \ldots, K \).
    \item \( \text{demand}_{m} \): Minimum units of nutrient \( m \) required per day for \( m = 1, 2, \ldots, M \).
    \item \( \text{nutri}_{k, m} \): Amount of nutrient \( m \) in one unit of food \( k \).
\end{itemize}

\subsection*{Decision Variables}
\begin{itemize}
    \item Let \( \text{quantity}_{k} \) be the quantity of food \( k \) to purchase, for \( k = 1, 2, \ldots, K \).
\end{itemize}

\subsection*{Objective Function}
We aim to minimize the total cost of purchased foods:
\[
\text{Minimize: } Z = \sum_{k=1}^{K} \text{price}_{k} \cdot \text{quantity}_{k}
\]

\subsection*{Constraints}
Each nutrient must meet its daily demand:
\[
\sum_{k=1}^{K} \text{nutri}_{k, m} \cdot \text{quantity}_{k} \geq \text{demand}_{m}, \quad \forall m = 1, 2, \ldots, M
\]

Additionally, the quantities purchased must be non-negative:
\[
\text{quantity}_{k} \geq 0, \quad \forall k = 1, 2, \ldots, K
\]

\subsection*{Final Model}
The complete linear programming model can be summarized as follows:

\begin{align*}
\text{Minimize:} & \quad Z = \sum_{k=1}^{K} \text{price}_{k} \cdot \text{quantity}_{k} \\
\text{Subject to:} & \quad \sum_{k=1}^{K} \text{nutri}_{k, m} \cdot \text{quantity}_{k} \geq \text{demand}_{m}, \quad \forall m = 1, 2, \ldots, M \\
& \quad \text{quantity}_{k} \geq 0, \quad \forall k = 1, 2, \ldots, K
\end{align*}

\end{document}