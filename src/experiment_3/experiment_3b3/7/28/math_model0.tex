\documentclass{article}
\usepackage{amsmath}
\begin{document}

\section*{Linear Programming Model for Fitting a Line}

Let \( y_k \) be the \( k \)-th observed value of \( y \) and \( x_k \) be the \( k \)-th observed value of \( x \) where \( k = 1, 2, \ldots, K \).

The relationship to fit is given by the straight line equation:
\[
y = bx + a
\]
where \( b \) is the slope and \( a \) is the intercept. 

We aim to minimize the sum of absolute deviations of the observed values from the predicted values. The objective function can be expressed as:
\[
\text{Minimize } \sum_{k=1}^{K} |y_k - (bx_k + a)|
\]

To reformulate this objective function into a linear programming model, we introduce auxiliary variables \( d_k \) to represent the absolute deviations:
\[
d_k \geq y_k - (bx_k + a) \quad \forall k = 1, 2, \ldots, K
\]
\[
d_k \geq -(y_k - (bx_k + a)) \quad \forall k = 1, 2, \ldots, K
\]

Thus, we can rewrite the objective function as:
\[
\text{Minimize } \sum_{k=1}^{K} d_k
\]

Subject to the constraints:
\[
d_k \geq y_k - (bx_k + a) \quad \forall k
\]
\[
d_k \geq -(y_k - (bx_k + a)) \quad \forall k
\]

The complete linear programming model is as follows:

\textbf{Objective:}
\[
\text{Minimize } Z = \sum_{k=1}^{K} d_k
\]

\textbf{Subject to:}
\[
d_k \geq y_k - (bx_k + a) \quad \forall k
\]
\[
d_k \geq -(y_k - (bx_k + a)) \quad \forall k
\]

\textbf{Variables:}
\[
b \text{ (slope)}, \quad a \text{ (intercept)}, \quad d_k \text{ (absolute deviations)}, \quad k = 1, 2, \ldots, K
\]

The output will provide the values of:
\begin{itemize}
    \item \textit{intercept}: \( a \)
    \item \textit{slope}: \( b \)
\end{itemize}

\end{document}