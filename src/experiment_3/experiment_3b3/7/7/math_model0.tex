\documentclass{article}
\usepackage{amsmath}
\begin{document}

\section*{Mathematical Model for Rocket Motion}

Let:
\begin{itemize}
    \item \( x_t \) be the position of the rocket at time \( t \).
    \item \( v_t \) be the velocity of the rocket at time \( t \).
    \item \( a_t \) be the acceleration of the rocket at time \( t \).
    \item \( x_0 \) be the initial position of the rocket.
    \item \( v_0 \) be the initial velocity of the rocket.
    \item \( x_T \) be the target position at time \( T \).
    \item \( v_T \) be the target velocity at time \( T \).
    \item \( T \) be the total time of travel.
\end{itemize}

The discrete-time dynamics of the rocket can be described by the following equations:
\[
x_{t+1} = x_t + v_t
\]
\[
v_{t+1} = v_t + a_t
\]

The objective is to minimize the maximum thrust required, which is represented as:
\[
\min \max_{t} |a_t|
\]

Subject to the following constraints:

1. Initial conditions:
    \[
    x_0 = \text{given initial position}
    \]
    \[
    v_0 = \text{given initial velocity}
    \]

2. Final conditions:
    \[
    x_T = \text{target position}
    \]
    \[
    v_T = \text{target velocity}
    \]

3. For each time step \( t \) from 0 to \( T-1 \):
    \[
    x_{t+1} = x_t + v_t
    \]
    \[
    v_{t+1} = v_t + a_t
    \]

4. The fuel consumption is proportional to the absolute value of acceleration:
   \[
   \text{Fuel Consumption} = \sum_{t=0}^{T-1} |a_t|
   \]

The output format should include:
\begin{itemize}
    \item \( x = [x_i \text{ for } i \text{ in } 0, \ldots, T] \)
    \item \( v = [v_i \text{ for } i \text{ in } 0, \ldots, T] \)
    \item \( a = [a_i \text{ for } i \text{ in } 0, \ldots, T] \)
    \item \( \text{fuel\_spend} = \text{total fuel spent} \)
\end{itemize}

\end{document}