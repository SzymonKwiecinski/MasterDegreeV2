\documentclass{article}
\usepackage{amsmath}
\begin{document}

\title{MILP Model for Nurse Scheduling}
\author{}
\date{}
\maketitle

\section{Problem Definition}
A hospital needs to create a weekly night shift schedule for nurses. The demand for nurses for the night shift on day \( j \) (where \( j = 1, \ldots, 7 \)) is given as an integer \( d_j \). Each nurse works for \( \text{period} \) days in a row and then has \( 7 - \text{period} \) days off.

\section{Variables}
Let:
\begin{itemize}
    \item \( x_j \): Number of nurses that start their period on day \( j \) for \( j = 1, \ldots, 7 \).
    \item \( N \): Total number of nurses hired.
\end{itemize}

\section{Objective Function}
Minimize the total number of nurses hired:
\[
\text{Minimize } N = \sum_{j=1}^{7} x_j
\]

\section{Constraints}
For each day \( j \), the total number of nurses available must meet or exceed the demand:
\[
\sum_{k=0}^{\text{period}-1} x_{(j-k \mod 7)} \geq d_j \quad \forall j = 1, \ldots, 7
\]

\section{Input Format}
The model requires the following inputs:
\begin{itemize}
    \item \( \text{period} \): Number of consecutive days a nurse works.
    \item \( d_j \): Array of integer demands for each day \( j \).
\end{itemize}

\section{Output Format}
The output should provide:
\begin{itemize}
    \item \( \text{start} \): Array containing \( x_j \) for \( j = 1, \ldots, 7 \).
    \item \( \text{total} \): Total number of nurses hired, \( N \).
\end{itemize}

\end{document}