\documentclass{article}
\usepackage{amsmath}
\begin{document}

\section*{Linear Programming Problem Description}

Consider a set \( P \) described by linear inequality constraints, defined as:

\[
P = \{ x \in \mathbb{R}^N \mid a_i^T x \leq b_i , \; i = 1, \ldots, m \}
\]

where \( a_i \in \mathbb{R}^N \) and \( b_i \in \mathbb{R} \).

A ball with center \( y \) in \( \mathbb{R}^N \) and radius \( r \) is defined as the set of all points within (Euclidean) distance \( r \) from \( y \). Mathematically, this is represented as:

\[
B(y, r) = \{ x \in \mathbb{R}^N \mid \| x - y \|_2 \leq r \}
\]

The objective is to find a ball with the largest possible radius, which is entirely contained within the set \( P \). The center of such a ball is called the Chebyshev center of \( P \).

\subsection*{Input}

The input is provided in the following format:

\begin{verbatim}
{
    "A": [[a_i_j for j in 1, ..., n] for i in 1, ..., m],
    "b": [b_i for i in 1, ..., m]
}
\end{verbatim}

\subsection*{Output}

The output will contain:

\begin{verbatim}
{
    "center": [y_j for j in 1, ..., n],
    "radius": r
}
\end{verbatim}

where:
\begin{itemize}
    \item \texttt{center} is a list of floats of length \( N \) representing the center of the ball.
    \item \texttt{radius} is a float representing the radius of the ball.
\end{itemize}

\end{document}