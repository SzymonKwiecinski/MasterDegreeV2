\documentclass{article}
\usepackage{amsmath}
\begin{document}

\section*{Quadratic Curve Fitting using Linear Programming}

We want to fit a quadratic curve of the form 

\[
y = c \cdot x^2 + b \cdot x + a
\]

to a given set of data points \((x_k, y_k)\) for \(k = 1, \ldots, K\). The goal is to minimize the sum of absolute deviations between the observed values \(y_k\) and the predicted values \(y_k' = c \cdot x_k^2 + b \cdot x_k + a\).

\subsection*{Variables}

Let \(y_k'\) be the predicted value of \(y\) for each \(k\):

\[
y_k' = c \cdot x_k^2 + b \cdot x_k + a
\]

We introduce non-negative slack variables \(d_k\) for each deviation:

\[
d_k \geq 0 \quad \text{for } k = 1, \ldots, K
\]

The absolute deviation for each point can be expressed as:

\[
|y_k - y_k'| \leq d_k
\]

This leads to two inequalities:

\[
y_k - y_k' \leq d_k \quad \text{and} \quad y_k' - y_k \leq d_k
\]

Combining these, we can express the absolute deviation as follows:

\[
d_k \geq y_k - (c \cdot x_k^2 + b \cdot x_k + a) \quad \text{and} \quad d_k \geq -(y_k - (c \cdot x_k^2 + b \cdot x_k + a))
\]

\subsection*{Objective Function}

The objective is to minimize the total absolute deviation:

\[
\text{Minimize } Z = \sum_{k=1}^{K} d_k
\]

\subsection*{Linear Programming Formulation}

We can summarize our linear programming problem as follows:

Minimize:

\[
Z = \sum_{k=1}^{K} d_k
\]

Subject to:

\[
d_k \geq y_k - (c \cdot x_k^2 + b \cdot x_k + a) \quad \forall k = 1, \ldots, K
\]
\[
d_k \geq -(y_k - (c \cdot x_k^2 + b \cdot x_k + a)) \quad \forall k = 1, \ldots, K
\]
\[
d_k \geq 0 \quad \forall k = 1, \ldots, K
\]

Where \(a\), \(b\), and \(c\) are the coefficients that we need to determine.

\subsection*{Output}

Once the optimization problem is solved, the values of the coefficients will be reported as follows:

\[
\text{Output: } \{ \text{"quadratic": } c, \text{"linear": } b, \text{"constant": } a \}
\]

\end{document}