\documentclass{article}
\usepackage{amsmath}
\begin{document}

\section*{Linear Programming Model for Delivery Path Optimization}

\subsection*{Problem Definition}
A delivery person starts at the intersection of the 1st Avenue and 1st Street and needs to reach the intersection of the $W$th Avenue and $N$th Street on an $N-1 \times W-1$ grid. The time to move between intersections varies depending on pedestrian traffic. The objective is to minimize the total walking time while moving only north or west.

\subsection*{Variables}
Let:
\begin{itemize}
    \item $x_{n,w}$ = time to walk from intersection $(w,n)$ to $(w+1,n)$ (westward movement)
    \item $y_{n,w}$ = time to walk from intersection $(w,n)$ to $(w,n+1)$ (northward movement)
\end{itemize}

\subsection*{Objective Function}
Minimize the total walking time:
\[
\text{Minimize } Z = \sum_{n=1}^{N-1} \sum_{w=1}^{W-1} \left( x_{n,w} + y_{n,w} \right)
\]

\subsection*{Constraints}
The following constraints must hold:
\begin{itemize}
    \item Starting point: At $(1,1)$, the delivery person begins with a time of 0.
    \item Non-negativity: 
    \[
    x_{n,w} \geq 0, \quad y_{n,w} \geq 0
    \]
    \item Flow conservation: For each intersection $(w,n)$, the flow in must equal the flow out:
    \[
    \sum_{w=1}^{W-1} x_{n,w} + \sum_{n=1}^{N-1} y_{n,w} = \text{Total time taken}
    \]
\end{itemize}

\subsection*{Input Format}
The input data is given in the following structure:
\begin{verbatim}
{
    "west_time": [[west_{n,w} for w in 1, ..., W-1] for n in 1, ..., N],
    "north_time": [[north_{n,w} for w in 1, ..., W] for n in 1, ..., N-1],
}
\end{verbatim}

\subsection*{Output Format}
The output consists of:
\begin{verbatim}
{
    "paths": [(street_{n}, avenue_{w}) for id in 1, ..., m],
    "total_time": total_travel_time
}
\end{verbatim}

\end{document}