\documentclass{article}
\usepackage{amsmath}
\begin{document}

\section*{Electricity Capacity Expansion Model}

\textbf{Variables:}

Let:
\begin{itemize}
    \item \( C_t \) = capacity of coal-fired power plants added in year \( t \) (in megawatts)
    \item \( N_t \) = capacity of nuclear power plants added in year \( t \) (in megawatts)
    \item \( T \) = total number of years
\end{itemize}

\textbf{Parameters:}

\begin{itemize}
    \item \( \text{demand}_t \) = demand for electricity during year \( t \) (in megawatts)
    \item \( \text{oil}_t \) = existing oil capacity available during year \( t \) (in megawatts)
    \item \( \text{coal\_cost} \) = capital cost per megawatt for coal-fired plants
    \item \( \text{nuke\_cost} \) = capital cost per megawatt for nuclear plants
    \item \( \text{max\_nuke} \) = maximum percentage of total capacity that can be nuclear
    \item \( \text{coal\_life} \) = lifespan of coal plants (in years)
    \item \( \text{nuke\_life} \) = lifespan of nuclear plants (in years)
\end{itemize}

\textbf{Objective Function:}

Minimize the total cost:
\[
\text{Total Cost} = \sum_{t=1}^{T} (\text{coal\_cost} \cdot C_t + \text{nuke\_cost} \cdot N_t)
\]

\textbf{Constraints:}

1. Capacity Constraint:
\[
\text{oil}_t + \sum_{i=1}^{\min(t, \text{coal\_life})} C_{t-i} + \sum_{j=1}^{\min(t, \text{nuke\_life})} N_{t-j} \geq \text{demand}_t \quad \forall t \in \{1, \ldots, T\}
\]

2. Nuclear Capacity Limit:
\[
\sum_{j=1}^{\min(t, \text{nuke\_life})} N_{t-j} \leq \frac{\text{max\_nuke}}{100} \left( \text{oil}_t + \sum_{i=1}^{\min(t, \text{coal\_life})} C_{t-i} + \sum_{j=1}^{\min(t, \text{nuke\_life})} N_{t-j} \right) \quad \forall t \in \{1, \ldots, T\}
\]

3. Non-negativity:
\[
C_t \geq 0, \quad N_t \geq 0 \quad \forall t \in \{1, \ldots, T\}
\]

\textbf{Output:}
The output should be represented as follows:
\begin{verbatim}
{
    "coal_cap_added": [C_1, C_2, ..., C_T],
    "nuke_cap_added": [N_1, N_2, ..., N_T],
    "total_cost": total_cost
}
\end{verbatim}

\end{document}