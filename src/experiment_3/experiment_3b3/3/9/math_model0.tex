\documentclass{article}
\usepackage{amsmath}
\begin{document}

\section*{Linear Programming Model for Road Illumination}

\subsection*{Decision Variables}
Let \( power_j \) be the power of the \( j \)-th lamp, where \( j = 1, 2, \ldots, M \).

\subsection*{Parameters}
\begin{itemize}
    \item \( N \): Number of segments of the road
    \item \( M \): Number of lamps
    \item \( coeff_{i,j} \): Coefficient representing the contribution of lamp \( j \) to the illumination of segment \( i \)
    \item \( desired_i \): Desired illumination for segment \( i \)
\end{itemize}

\subsection*{Illumination Model}
The illumination \( ill_i \) of the \( i \)-th segment is given by:
\[
ill_i = \sum_{j=1}^{M} coeff_{i,j} \cdot power_j
\]

\subsection*{Objective Function}
The objective is to minimize the absolute error between the actual and desired illuminations, formulated as:
\[
\text{Minimize} \quad Z = \sum_{i=1}^{N} |ill_i - desired_i|
\]

\subsection*{Constraints}
There are no specific constraints given for the powers of the lamps in the problem statement. However, it is generally assumed that:
\[
power_j \geq 0 \quad \text{for } j = 1, 2, \ldots, M
\]

\subsection*{Output}
The output will provide:
\begin{itemize}
    \item \( power \): Optimal powers of the lamps, \( [power_j \text{ for } j = 1, \ldots, M] \)
    \item \( error \): The minimum absolute error, \( error \)
\end{itemize}

\subsection*{Final Model}
The final linear programming model can be summarized as follows:
\[
\begin{align*}
\text{Minimize} \quad & Z = \sum_{i=1}^{N} | \sum_{j=1}^{M} coeff_{i,j} \cdot power_j - desired_i | \\
\text{subject to} \quad & power_j \geq 0, \quad j = 1, 2, \ldots, M
\end{align*}
\]

\end{document}