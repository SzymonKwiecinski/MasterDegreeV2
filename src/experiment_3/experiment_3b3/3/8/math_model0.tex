\documentclass{article}
\usepackage{amsmath}
\begin{document}

\section*{Linear Programming Model for Maximizing Profit in a Spare Parts Firm}

\subsection*{Parameters}
\begin{itemize}
    \item Let \( K \) be the number of spare parts.
    \item Let \( S \) be the number of shops.
    \item Let \( time_{k, s} \) be the required worker-hours for part \( k \) in shop \( s \).
    \item Let \( profit_{k} \) be the profit of part \( k \).
    \item Let \( capacity_{s} \) be the capacity of shop \( s \) in working hours.
\end{itemize}

\subsection*{Decision Variables}
\begin{itemize}
    \item Let \( quantity_{k} \) be the quantity of spare part \( k \) to be produced.
\end{itemize}

\subsection*{Objective Function}
Maximize the total profit:
\[
\text{Maximize } Z = \sum_{k=1}^{K} profit_{k} \cdot quantity_{k}
\]

\subsection*{Constraints}
We need to ensure that the total worker-hours in each shop do not exceed their respective capacities. This can be expressed as:
\[
\sum_{k=1}^{K} time_{k, s} \cdot quantity_{k} \leq capacity_{s} \quad \forall s = 1, \ldots, S
\]

\subsection*{Non-negativity Constraints}
The quantity of each spare part produced must be non-negative:
\[
quantity_{k} \geq 0 \quad \forall k = 1, \ldots, K
\]

\subsection*{Complete Model}
Putting it all together, the linear programming model can be summarized as follows:

\[
\begin{align*}
\text{Maximize} & \quad Z = \sum_{k=1}^{K} profit_{k} \cdot quantity_{k} \\
\text{subject to} & \quad \sum_{k=1}^{K} time_{k, s} \cdot quantity_{k} \leq capacity_{s}, \quad \forall s = 1, \ldots, S \\
& \quad quantity_{k} \geq 0, \quad \forall k = 1, \ldots, K
\end{align*}
\]

\end{document}