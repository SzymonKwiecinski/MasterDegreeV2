\documentclass{article}
\usepackage{amsmath}
\begin{document}

\section*{Linear Programming Model for Oil Refinery Production}

\subsection*{Parameters}
\begin{itemize}
    \item Let \( O \) be the number of crude oil types.
    \item Let \( P \) be the number of products.
    \item Let \( L \) be the number of production processes.
    \item Let \( \text{allocated}_i \) be the allocated barrels of crude oil type \( i \) for \( i = 1, \ldots, O \).
    \item Let \( \text{price}_p \) be the selling price per barrel of product \( p \) for \( p = 1, \ldots, P \).
    \item Let \( \text{input}_{l,i} \) be the input required from crude \( i \) for process \( l \) for \( l = 1, \ldots, L \) and \( i = 1, \ldots, O \).
    \item Let \( \text{output}_{l,p} \) be the output of product \( p \) from process \( l \) for \( l = 1, \ldots, L \) and \( p = 1, \ldots, P \).
    \item Let \( \text{cost}_l \) be the cost per barrel for process \( l \) for \( l = 1, \ldots, L \).
\end{itemize}

\subsection*{Decision Variables}
\begin{itemize}
    \item Let \( \text{execute}_l \) be the number of times process \( l \) is executed for \( l = 1, \ldots, L \).
\end{itemize}

\subsection*{Objective Function}
Maximize the total revenue:
\[
\text{maximize} \quad \text{revenue} = \sum_{p=1}^{P} \text{price}_p \sum_{l=1}^{L} \text{output}_{l,p} \cdot \text{execute}_l
\]

\subsection*{Constraints}
1. Crude oil supply constraints:
\[
\sum_{l=1}^{L} \text{input}_{l,i} \cdot \text{execute}_l \leq \text{allocated}_i \quad \text{for } i = 1, \ldots, O 
\]

2. Non-negativity constraints:
\[
\text{execute}_l \geq 0 \quad \text{for } l = 1, \ldots, L
\]

\subsection*{Output}
The output of the model will provide:
\begin{itemize}
    \item The total revenue \( \text{revenue} \).
    \item The optimal number of executions for each process \( \text{execute}_l \) for \( l = 1, \ldots, L \).
\end{itemize}

\end{document}