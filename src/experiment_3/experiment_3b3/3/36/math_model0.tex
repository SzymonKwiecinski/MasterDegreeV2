\documentclass{article}
\usepackage{amsmath}
\begin{document}

\section*{Linear Programming Model for Alloy Production}

\subsection*{Parameters}
\begin{itemize}
    \item Let \( alloy\_quant \) be the total weight of the alloy to be produced (in pounds).
    \item Let \( target_{m} \) be the target quantity of metal \( m \) in the alloy for \( m = 1, \ldots, M \).
    \item Let \( ratio_{k, m} \) be the ratio of metal \( m \) in alloy \( k \) for \( k = 1, \ldots, K \).
    \item Let \( price_{k} \) be the price of alloy \( k \).
\end{itemize}

\subsection*{Decision Variables}
\begin{itemize}
    \item Let \( amount_{k} \) be the amount of alloy \( k \) to purchase for \( k = 1, \ldots, K \).
\end{itemize}

\subsection*{Objective Function}
Minimize the total cost:
\[
\text{Minimize } Z = \sum_{k=1}^{K} price_{k} \cdot amount_{k}
\]

\subsection*{Constraints}
1. Total weight constraint:
\[
\sum_{k=1}^{K} amount_{k} = alloy\_quant
\]

2. Target metal weight constraints for each metal \( m \):
\[
\sum_{k=1}^{K} ratio_{k, m} \cdot amount_{k} = target_{m}, \quad \forall m = 1, \ldots, M
\]

3. Non-negativity constraints:
\[
amount_{k} \geq 0, \quad \forall k = 1, \ldots, K
\]

\subsection*{Output}
The solution will provide the quantities of each alloy needed:
\[
\text{Output: } amount = [amount_{k} \text{ for } k = 1,\ldots,K]
\]

\end{document}