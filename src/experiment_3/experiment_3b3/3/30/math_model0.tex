\documentclass{article}
\usepackage{amsmath}
\begin{document}

\section*{Quadratic Curve Fitting Problem}

Let \( y_k \) represent the \( k \)-th observed value of \( y \) and \( x_k \) represent the \( k \)-th observed value of \( x \) for \( k = 1, \ldots, K \).

We want to fit the quadratic curve defined by the equation:

\[
y = c \cdot x^2 + b \cdot x + a
\]

Our goal is to minimize the sum of absolute deviations between the observed values \( y_k \) and the predicted values from the quadratic equation for all data points:

\[
\text{Minimize} \quad \sum_{k=1}^{K} |y_k - (c \cdot x_k^2 + b \cdot x_k + a)|
\]

To formulate this as a linear programming problem, we introduce auxiliary variables \( z_k \) to represent the absolute deviations for each data point:

\[
z_k \geq y_k - (c \cdot x_k^2 + b \cdot x_k + a)
\]
\[
z_k \geq -(y_k - (c \cdot x_k^2 + b \cdot x_k + a))
\]

Thus, we can rewrite our objective function to minimize the sum of \( z_k \):

\[
\text{Minimize} \quad \sum_{k=1}^{K} z_k
\]

Subject to the following constraints:

\[
z_k \geq y_k - (c \cdot x_k^2 + b \cdot x_k + a) \quad \forall k = 1, \ldots, K
\]
\[
z_k \geq -(y_k - (c \cdot x_k^2 + b \cdot x_k + a)) \quad \forall k = 1, \ldots, K
\]

The solution will yield the values of \( c \), \( b \), and \( a \):

\[
\text{Output:} \quad 
\begin{align*}
\text{quadratic} & : c \\
\text{linear} & : b \\
\text{constant} & : a 
\end{align*}
\]

\end{document}