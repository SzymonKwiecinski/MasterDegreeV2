\documentclass{article}
\usepackage{amsmath}
\begin{document}

\section*{Linear Programming Model}

\textbf{Variables:} \\
Let \( x_j \) be the amount of good \( j \) produced, where \( j = 1, 2, \ldots, M \).

\textbf{Parameters:} \\
Let \( \text{available}_i \) be the amount of raw material \( i \) available, where \( i = 1, 2, \ldots, N \). \\
Let \( \text{req}_{i,j} \) be the units of material \( i \) required to produce one unit of good \( j \). \\
Let \( \text{price}_j \) be the revenue generated from producing one unit of good \( j \).

\textbf{Objective:} \\
Maximize total revenue:
\[
\text{Maximize } Z = \sum_{j=1}^{M} \text{price}_j \cdot x_j
\]

\textbf{Constraints:} \\
For each raw material \( i \):
\[
\sum_{j=1}^{M} \text{req}_{i,j} \cdot x_j \leq \text{available}_i, \quad \forall i = 1, 2, \ldots, N
\]

\textbf{Non-negativity:} \\
\[
x_j \geq 0, \quad \forall j = 1, 2, \ldots, M
\]

The final output should provide the optimal amounts of each good to be produced:
\[
\text{Output: } \{ \text{amount} : [x_j \text{ for } j = 1, \ldots, M] \}
\]

\end{document}