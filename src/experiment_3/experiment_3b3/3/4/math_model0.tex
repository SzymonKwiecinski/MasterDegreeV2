\documentclass{article}
\usepackage{amsmath}
\begin{document}

\section*{Mathematical Model for Nurse Scheduling Problem}

\subsection*{Parameters}
\begin{itemize}
    \item Let \( \text{period} \) be the number of days a nurse works consecutively.
    \item Let \( d_j \) be the demand for nurses on day \( j \) for \( j = 1, \ldots, 7 \).
\end{itemize}

\subsection*{Decision Variables}
\begin{itemize}
    \item Let \( x_j \) be the number of nurses that start their period on day \( j \) for \( j = 1, \ldots, 7 \).
\end{itemize}

\subsection*{Objective Function}
Minimize the total number of nurses hired:

\[
\text{Minimize } Z = \sum_{j=1}^{7} x_j
\]

\subsection*{Constraints}
For each day \( j \), the total number of nurses available must meet the demand:

\[
\sum_{i=0}^{\text{period}-1} x_{j-i} \geq d_j \quad \forall j \in \{1, \ldots, 7\}
\]

where \( x_k = 0 \) for \( k < 1 \) or \( k > 7 \) (nurses cannot start outside the defined days).

\subsection*{Output}
The output will include:
\begin{itemize}
    \item The number of nurses that start their period on each day \( j \): \( \text{start} = [x_1, x_2, \ldots, x_7] \)
    \item The total number of nurses hired: \( \text{total} = Z \)
\end{itemize}

\end{document}