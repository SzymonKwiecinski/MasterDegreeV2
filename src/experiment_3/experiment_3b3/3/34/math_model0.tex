\documentclass{article}
\usepackage{amsmath}
\begin{document}

\section*{Linear Programming Model}

\subsection*{Parameters}
\begin{itemize}
    \item Let \( K \) be the number of different foods.
    \item Let \( M \) be the number of nutritional ingredients.
    \item Let \( price_k \) be the price per unit of food \( k \) where \( k = 1, 2, \ldots, K \).
    \item Let \( demand_m \) be the minimum required units of nutrient \( m \) where \( m = 1, 2, \ldots, M \).
    \item Let \( nutri_{k, m} \) be the units of nutrient \( m \) contained in one unit of food \( k \).
\end{itemize}

\subsection*{Decision Variables}
Let \( quantity_k \) be the quantity of food \( k \) to purchase, where \( k = 1, 2, \ldots, K \).

\subsection*{Objective Function}
The objective is to minimize the total cost of the foods purchased:
\[
\text{Minimize} \quad Z = \sum_{k=1}^{K} price_k \cdot quantity_k
\]

\subsection*{Constraints}
The nutritional constraints are defined by the requirements for each nutrient \( m \):
\[
\sum_{k=1}^{K} nutri_{k, m} \cdot quantity_k \geq demand_m \quad \forall m = 1, 2, \ldots, M
\]

\subsection*{Non-negativity Constraints}
The quantities of food purchased must be non-negative:
\[
quantity_k \geq 0 \quad \forall k = 1, 2, \ldots, K
\]

\subsection*{Output Format}
The result will include the quantities of each food to purchase:
\[
\{ "quantity": [quantity_k \text{ for } k = 1, \ldots, K] \}
\]

\end{document}