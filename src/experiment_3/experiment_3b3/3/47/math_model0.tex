\documentclass{article}
\usepackage{amsmath}
\begin{document}

\section*{Linear Programming Model for Police Officer Shift Assignment}

\subsection*{Parameters}
\begin{itemize}
    \item Let \( S \) be the number of shifts.
    \item \( \text{officers\_needed}_s \): Number of police officers needed during shift \( s \), for \( s = 1, 2, \ldots, S \).
    \item \( \text{shift\_cost}_s \): Cost incurred if a shift \( s \) is started, for \( s = 1, 2, \ldots, S \).
\end{itemize}

\subsection*{Decision Variables}
\begin{itemize}
    \item \( x_s \): Number of police officers assigned to shift \( s \), for \( s = 1, 2, \ldots, S \).
\end{itemize}

\subsection*{Objective Function}
The objective is to minimize the total cost:
\[
\text{Minimize } Z = \sum_{s=1}^{S} \text{shift\_cost}_s \cdot y_s
\]
where \( y_s \) is a binary variable indicating whether shift \( s \) is started (1 if started, 0 otherwise). 

To relate \( x_s \) to \( y_s \), we can express it as:
\[
y_s = 
\begin{cases}
1 & \text{if } x_s > 0 \\
0 & \text{if } x_s = 0
\end{cases}
\]

\subsection*{Constraints}
The model must satisfy the officer requirements:
\[
x_s \geq \text{officers\_needed}_s \cdot y_s, \quad \forall s = 1, 2, \ldots, S
\]

Each police officer works for two consecutive shifts:
\[
x_{s} = x_{s+1}, \quad \forall s = 1, 2, \ldots, S-1
\]

\subsection*{Non-negativity Constraints}
\[
x_s \geq 0, \quad \forall s = 1, 2, \ldots, S
\]

\subsection*{Binary Constraint}
\[
y_s \in \{0, 1\}, \quad \forall s = 1, 2, \ldots, S
\]

\subsection*{Output}
The output should include:
\begin{itemize}
    \item \( \text{officers\_assigned}_s \): The number of officers assigned to each shift \( s \).
    \item \( \text{total\_cost} \): The total cost incurred by the town.
\end{itemize}

\end{document}