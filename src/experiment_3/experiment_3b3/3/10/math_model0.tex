\documentclass{article}
\usepackage{amsmath}
\begin{document}

\section*{Mathematical Model}

\subsection*{Sets}
\begin{itemize}
    \item \(N\): Set of neighborhoods
    \item \(S\): Set of schools
    \item \(G\): Set of grades
\end{itemize}

\subsection*{Parameters}
\begin{itemize}
    \item \(capacity_{s,g}\): Capacity of school \(s\) for grade \(g\)
    \item \(population_{n,g}\): Student population of grade \(g\) in neighborhood \(n\)
    \item \(d_{n,s}\): Distance from neighborhood \(n\) to school \(s\)
\end{itemize}

\subsection*{Decision Variables}
\begin{itemize}
    \item \(x_{n,s,g}\): Number of students of grade \(g\) in neighborhood \(n\) assigned to school \(s\)
\end{itemize}

\subsection*{Objective Function}
Minimize the total distance traveled by all students:
\[
\text{Minimize } Z = \sum_{n \in N} \sum_{s \in S} \sum_{g \in G} d_{n,s} \cdot x_{n,s,g}
\]

\subsection*{Constraints}
1. Capacity constraints for each school and grade:
\[
\sum_{n \in N} x_{n,s,g} \leq capacity_{s,g}, \quad \forall s \in S, \forall g \in G
\]

2. Demand constraints for each neighborhood and grade:
\[
\sum_{s \in S} x_{n,s,g} = population_{n,g}, \quad \forall n \in N, \forall g \in G
\]

3. Non-negativity constraints:
\[
x_{n,s,g} \geq 0, \quad \forall n \in N, \forall s \in S, \forall g \in G
\]

\end{document}