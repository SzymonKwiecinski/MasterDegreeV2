\documentclass{article}
\usepackage{amsmath}
\begin{document}

\section*{Mathematical Model for Maximizing Profit in Auto Parts Production}

\subsection*{Decision Variables}
\begin{itemize}
    \item Let \( batches_{p} \) be the number of batches produced for part \( p \) where \( p = 1, \ldots, P \).
    \item Let \( setup\_flags_{p} \) be a binary variable that indicates whether part \( p \) is set up (1) or not (0).
\end{itemize}

\subsection*{Parameters}
\begin{itemize}
    \item \( time_{m,p} \): time (in hours) required on machine \( m \) to produce one batch of part \( p \).
    \item \( cost_{m} \): cost (in $/hour) for using machine \( m \).
    \item \( available_{m} \): available hours per month for machine \( m \).
    \item \( price_{p} \): selling price per batch of part \( p \).
    \item \( setup\_time_{p} \): setup time (in hours) required on machine 1 for part \( p \).
\end{itemize}

\subsection*{Objective Function}
Maximize the total profit:

\[
\text{Maximize } Z = \sum_{p=1}^{P} \left( price_{p} \cdot batches_{p} - \sum_{m=1}^{M} cost_{m} \cdot time_{m,p} \cdot batches_{p} \right)
\]

\subsection*{Constraints}

1. **Machine Availability Constraints**:
   For each machine \( m \):
   \[
   \sum_{p=1}^{P} (time_{m,p} \cdot batches_{p}) \leq available_{m} \quad \forall m = 1, \ldots, M
   \]

2. **Setup Time Constraint for Machine 1**:
   For parts that require setup, if a part is produced, it includes setup time:
   \[
   (setup\_time_{p} \cdot setup\_flags_{p}) + \sum_{p=1}^{P} (time_{1,p} \cdot batches_{p}) \leq available_{1} \quad \forall p
   \]

3. **Binary Setup Flags**:
   \[
   setup\_flags_{p} \in \{0, 1\} \quad \forall p = 1, \ldots, P
   \]

4. **Non-negativity Constraint**:
   \[
   batches_{p} \geq 0 \quad \forall p = 1, \ldots, P
   \]

\subsection*{Output Variables}
The output will include:
\begin{itemize}
    \item \( batches \): the number of batches produced for each part, i.e., \( [batches_{p} \text{ for } p = 1, \ldots, P] \).
    \item \( setup\_flags \): the setup flags for parts, i.e., \( [setup\_flags_{p} \text{ for } p = 1, \ldots, P] \).
    \item \( total\_profit \): the total profit obtained for the month.
\end{itemize}

\end{document}