\documentclass{article}
\usepackage{amsmath}
\begin{document}

\section*{Linear Programming Model for Alloy Production}

\subsection*{Parameters}
\begin{itemize}
    \item \( alloy\_quant \): Total quantity of the alloy to be produced (in lb).
    \item \( target_{m} \): Target quantity of metal \( m \) (for \( m = 1, \ldots, M \)).
    \item \( ratio_{k, m} \): Ratio of metal \( m \) in alloy \( k \) (for \( k = 1, \ldots, K \) and \( m = 1, \ldots, M \)).
    \item \( price_k \): Price of alloy \( k \) (for \( k = 1, \ldots, K \)).
\end{itemize}

\subsection*{Variables}
\begin{itemize}
    \item \( amount_k \): Amount of alloy \( k \) to be purchased (for \( k = 1, \ldots, K \)).
\end{itemize}

\subsection*{Objective Function}
Minimize the total cost:
\[
\text{Minimize} \quad Z = \sum_{k=1}^{K} price_k \cdot amount_k
\]

\subsection*{Constraints}
1. Total weight of the alloys must equal the desired alloy weight:
\[
\sum_{k=1}^{K} amount_k = alloy\_quant
\]

2. Each metal \( m \) must meet its target requirement:
\[
\sum_{k=1}^{K} ratio_{k, m} \cdot amount_k = target_{m} \quad \text{for } m = 1, \ldots, M
\]

3. Non-negativity constraints:
\[
amount_k \geq 0 \quad \text{for } k = 1, \ldots, K
\]

\subsection*{Output}
The output should provide the quantities of each alloy to be purchased:
\[
\text{Output: } amount = [amount_k \text{ for } k = 1, \ldots, K]
\]

\end{document}