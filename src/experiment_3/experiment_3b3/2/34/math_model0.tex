\documentclass{article}
\usepackage{amsmath}
\begin{document}

\section*{Linear Programming Model for Dietary Optimization}

\subsection*{Variables}
Let:
\begin{itemize}
    \item \( x_k \) be the quantity of food \( k \) purchased, where \( k = 1, 2, \ldots, K \).
\end{itemize}

\subsection*{Parameters}
Given:
\begin{itemize}
    \item \( \text{price}_k \) is the price per unit of food \( k \).
    \item \( \text{demand}_m \) is the minimum required units of nutrient \( m \) per day.
    \item \( \text{nutri}_{k,m} \) is the amount of nutrient \( m \) in one unit of food \( k \).
\end{itemize}

\subsection*{Objective Function}
We want to minimize the total cost of food purchases:
\[
\text{Minimize } Z = \sum_{k=1}^{K} \text{price}_k \cdot x_k
\]

\subsection*{Constraints}
The dietary requirements for each nutrient \( m \) must be met:
\[
\sum_{k=1}^{K} \text{nutri}_{k,m} \cdot x_k \geq \text{demand}_m \quad \forall m = 1, 2, \ldots, M
\]

Additionally, we impose non-negativity constraints on the quantities:
\[
x_k \geq 0 \quad \forall k = 1, 2, \ldots, K
\]

\subsection*{Model Summary}
The complete linear programming model can be summarized as follows:

\begin{align*}
\text{Minimize} & \quad Z = \sum_{k=1}^{K} \text{price}_k \cdot x_k \\
\text{Subject to} & \quad \sum_{k=1}^{K} \text{nutri}_{k,m} \cdot x_k \geq \text{demand}_m \quad \forall m = 1, 2, \ldots, M \\
& \quad x_k \geq 0 \quad \forall k = 1, 2, \ldots, K
\end{align*}

\end{document}