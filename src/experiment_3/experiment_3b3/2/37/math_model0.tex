\documentclass{article}
\usepackage{amsmath}
\begin{document}

\section*{Linear Programming Model for Maximizing Profit in Spare Part Production}

\subsection*{Indices}
\begin{itemize}
    \item \( k \): Index for spare parts, where \( k = 1, \ldots, K \)
    \item \( s \): Index for shops, where \( s = 1, \ldots, S \)
\end{itemize}

\subsection*{Parameters}
\begin{itemize}
    \item \( time_{k,s} \): Worker-hours required for part \( k \) in shop \( s \)
    \item \( profit_{k} \): Profit obtained from producing part \( k \)
    \item \( capacity_{s} \): Capacity of shop \( s \) in terms of working hours
\end{itemize}

\subsection*{Decision Variables}
\begin{itemize}
    \item \( quantity_{k} \): Quantity of spare part \( k \) to be produced
\end{itemize}

\subsection*{Objective Function}
Maximize the total profit:
\[
\text{Maximize} \quad Z = \sum_{k=1}^{K} profit_{k} \cdot quantity_{k}
\]

\subsection*{Constraints}
The total worker-hours required in each shop must not exceed its capacity:
\[
\sum_{k=1}^{K} time_{k,s} \cdot quantity_{k} \leq capacity_{s}, \quad \forall s \in \{1, \ldots, S\}
\]

Additionally, the quantity of each spare part must be non-negative:
\[
quantity_{k} \geq 0, \quad \forall k \in \{1, \ldots, K\}
\]

\subsection*{Output Format}
The output will provide the optimal quantities of each spare part to be produced:
\[
\text{Output:} \quad quantity = [quantity_{1}, quantity_{2}, \ldots, quantity_{K}]
\]

\end{document}