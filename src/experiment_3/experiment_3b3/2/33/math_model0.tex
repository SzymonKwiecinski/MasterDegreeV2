\documentclass{article}
\usepackage{amsmath}
\begin{document}

\title{Mathematical Model for the Knapsack Problem}
\author{}
\date{}
\maketitle

\section{Problem Definition}
We are given a set of items, each with a respective value and size, and a container with a maximum capacity. The goal is to select a subset of items that maximizes the total value while ensuring that the total size does not exceed the capacity of the container.

\section{Parameters}
\begin{itemize}
    \item \( C \): Maximum capacity of the container.
    \item \( K \): Total number of items available.
    \item \( \text{value}_k \): Value of item \( k \) for \( k = 1, \ldots, K \).
    \item \( \text{size}_k \): Size of item \( k \) for \( k = 1, \ldots, K \).
\end{itemize}

\section{Variables}
\begin{itemize}
    \item \( \text{isincluded}_k \): Binary variable indicating whether item \( k \) is included in the knapsack (1 if included, 0 otherwise) for \( k = 1, \ldots, K \).
\end{itemize}

\section{Objective Function}
The objective is to maximize the total value of the items included in the knapsack:
\[
\text{Maximize } Z = \sum_{k=1}^{K} \text{value}_k \cdot \text{isincluded}_k
\]

\section{Constraints}
The total size of the selected items must not exceed the maximum capacity of the knapsack:
\[
\sum_{k=1}^{K} \text{size}_k \cdot \text{isincluded}_k \leq C
\]

\section{Model Formulation}
The complete model can be formulated as follows:

\begin{align*}
\text{Maximize } & Z = \sum_{k=1}^{K} \text{value}_k \cdot \text{isincluded}_k \\
\text{subject to } & \sum_{k=1}^{K} \text{size}_k \cdot \text{isincluded}_k \leq C \\
& \text{isincluded}_k \in \{0, 1\} \quad \text{for } k = 1, \ldots, K
\end{align*}

\end{document}