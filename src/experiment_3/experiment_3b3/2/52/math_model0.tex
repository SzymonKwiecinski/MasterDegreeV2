\documentclass{article}
\usepackage{amsmath}
\begin{document}

\section*{Linear Programming Model for Electric Utility}

\subsection*{Parameters}
\begin{itemize}
    \item \( P \): Number of power plants
    \item \( C \): Number of cities
    \item \( supply_p \): Capacity of power plant \( p \) in million kWh
    \item \( demand_c \): Peak demand of city \( c \) in million kWh
    \item \( transmission_{p,c} \): Transmission cost from power plant \( p \) to city \( c \) in \$ per million kWh
\end{itemize}

\subsection*{Decision Variables}
\begin{itemize}
    \item \( send_{p,c} \): Amount of electricity sent from power plant \( p \) to city \( c \) in million kWh
\end{itemize}

\subsection*{Objective Function}
Minimize the total transmission cost:
\[
\text{Total Cost} = \sum_{p=1}^{P} \sum_{c=1}^{C} transmission_{p,c} \cdot send_{p,c}
\]

\subsection*{Constraints}
1. Supply Constraints:
\[
\sum_{c=1}^{C} send_{p,c} \leq supply_p \quad \forall p = 1, \ldots, P
\]

2. Demand Constraints:
\[
\sum_{p=1}^{P} send_{p,c} \geq demand_c \quad \forall c = 1, \ldots, C
\]

3. Non-negativity Constraints:
\[
send_{p,c} \geq 0 \quad \forall p = 1, \ldots, P \quad \forall c = 1, \ldots, C
\]

\subsection*{Output}
The output will provide the amount of electricity that each power plant sends to each city as follows:
\[
\text{send} = \left[ \left[ send_{p,c} \text{ for } c = 1, \ldots, C \right] \text{ for } p = 1, \ldots, P \right]
\]
and the total cost:
\[
\text{total\_cost} = \text{Total Cost}
\]

\end{document}