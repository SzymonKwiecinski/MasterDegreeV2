\documentclass{article}
\usepackage{amsmath}
\begin{document}

\section*{Linear Programming Formulation}

Let \( N \) be the number of available currencies. We define the following variables:

\begin{itemize}
    \item Let \( x_{i,j} \) be the amount of currency \( i \) exchanged for currency \( j \).
    \item Let \( final_i \) be the final amount of currency \( i \) at the end of the day.
\end{itemize}

We are given the following parameters:

\begin{itemize}
    \item \( start_i \): the initial amount of currency \( i \).
    \item \( limit_i \): the maximum amount of currency \( i \) that can be exchanged.
    \item \( rate_{i,j} \): the rate of exchange from currency \( i \) to currency \( j \).
\end{itemize}

\subsection*{Objective Function}

We aim to maximize the final amount of currency \( N \):

\[
\text{Maximize } final_N
\]

\subsection*{Constraints}

1. **Initial Amounts**:
   The final amount of each currency after transactions can be expressed as:
   \[
   final_i = start_i + \sum_{j} x_{j,i} - \sum_{j} x_{i,j}, \quad \forall i = 1, \ldots, N
   \]

2. **Exchange Limits**:
   The total amount of currency that can be exchanged must not exceed the limits:
   \[
   \sum_{j} x_{i,j} \leq limit_i, \quad \forall i = 1, \ldots, N
   \]
   \[
   \sum_{j} x_{j,i} \leq limit_i, \quad \forall i = 1, \ldots, N
   \]

3. **Rate Constraints**:
   The amount of currency exchanged must respect the exchange rates:
   \[
   x_{i,j} \cdot rate_{i,j} \leq final_j, \quad \forall i,j = 1, \ldots, N
   \]

4. **Non-negativity Constraints**:
   All variables must be non-negative:
   \[
   x_{i,j} \geq 0, \quad \forall i,j = 1, \ldots, N
   \]

\subsection*{Conclusion}

This formulation allows us to determine the optimal transactions to maximize the amount of currency \( N \) at the end of the day while adhering to the initial amounts, limits on exchanges, and exchange rates.

\end{document}