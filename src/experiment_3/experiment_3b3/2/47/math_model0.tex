\documentclass{article}
\usepackage{amsmath}
\begin{document}

\title{Linear Programming Model for Police Shift Assignment}
\author{}
\date{}
\maketitle

\section*{Problem Definition}

Let:
\begin{itemize}
    \item \( S \) be the total number of shifts.
    \item \( officers\_needed_{s} \) be the number of police officers needed for shift \( s \).
    \item \( shift\_cost_{s} \) be the cost incurred by assigning a shift \( s \).
    \item \( officers\_assigned_{s} \) be the number of police officers assigned to shift \( s \).
\end{itemize}

\section*{Objective}

Minimize the total cost of assigning police officers:

\[
\text{Minimize } Z = \sum_{s=1}^{S} shift\_cost_{s} \cdot x_s
\]

where \( x_s \) is the number of officers assigned to shift \( s \).

\section*{Constraints}

1. Each police officer works for two consecutive shifts. Therefore, the number of officers assigned must meet the requirement for both shifts \( s \) and \( s+1 \):

\[
x_s + x_{s+1} \geq officers\_needed_{s}, \quad \forall s = 1, 2, \ldots, S-1
\]

2. The number of officers assigned must be non-negative:

\[
x_s \geq 0, \quad \forall s = 1, 2, \ldots, S
\]

3. For the last shift, since it has no succeeding shift, we can define it simply to meet its own requirement:

\[
x_S \geq officers\_needed_{S}
\]

\section*{Variables}

\[
x_s = \text{officers\_assigned}_{s}, \quad \forall s = 1, 2, \ldots, S
\]

\section*{Output}

The required output is:

\begin{itemize}
    \item \( officers\_assigned \): the number of officers assigned to each shift.
    \item \( total\_cost \): the total cost for the town.
\end{itemize}

\end{document}