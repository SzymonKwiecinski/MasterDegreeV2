\documentclass{article}
\usepackage{amsmath}
\begin{document}

\section*{Linear Programming Model for Police Officer Shift Assignment}

\subsection*{Parameters}
\begin{itemize}
    \item Let \( S \) be the number of different shifts.
    \item Let \( \text{officers\_needed}_{s} \) be the number of police officers needed during shift \( s \) for \( s = 1, \ldots, S \).
    \item Let \( \text{shift\_cost}_{s} \) be the cost incurred by the town when starting a shift at \( s \) for \( s = 1, \ldots, S \).
\end{itemize}

\subsection*{Decision Variables}
\begin{itemize}
    \item Let \( \text{officers\_assigned}_{s} \) be the number of police officers assigned to shift \( s \) for \( s = 1, \ldots, S \).
\end{itemize}

\subsection*{Objective Function}
The objective is to minimize the total cost:
\[
\text{Minimize } Z = \sum_{s=1}^{S} \text{shift\_cost}_{s} \cdot \text{officers\_assigned}_{s}
\]

\subsection*{Constraints}
The following constraints must be satisfied:
\begin{itemize}
    \item Each officer works for two consecutive shifts:
    \[
    \text{officers\_assigned}_{s} + \text{officers\_assigned}_{s+1} \geq \text{officers\_needed}_{s} \quad \forall s \in \{1, \ldots, S-1\}
    \]
    \item For the last shift:
    \[
    \text{officers\_assigned}_{S-1} + \text{officers\_assigned}_{S} \geq \text{officers\_needed}_{S-1}
    \]
    \item Non-negativity constraints:
    \[
    \text{officers\_assigned}_{s} \geq 0 \quad \forall s \in \{1, \ldots, S\}
    \end{itemize}
\]

\subsection*{Output Variables}
\begin{itemize}
    \item The number of officers assigned to each shift: \( \text{officers\_assigned}_{s} \)
    \item The total cost incurred by the town: \( Z \)
\end{itemize}

\end{document}