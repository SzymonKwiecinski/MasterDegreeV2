\documentclass{article}
\usepackage{amsmath}
\begin{document}

\section*{Mathematical Model for the Roll Cutting Problem}

\textbf{Indices:}
\begin{itemize}
    \item \( j \): Index for the different roll widths, \( j = 1, 2, \ldots, M \) 
    \item \( i \): Index for the different cutting patterns, \( i = 1, 2, \ldots, N \)
\end{itemize}

\textbf{Parameters:}
\begin{itemize}
    \item \( L \): Width of the large roll (\textit{large\_roll\_width})
    \item \( d_j \): Demand for rolls of width \( j \) (\textit{demand\_j})
    \item \( p_{i,j} \): Number of rolls of width \( j \) produced by pattern \( i \) (\textit{pattern\_{i,j}})
\end{itemize}

\textbf{Variables:}
\begin{itemize}
    \item \( x_i \): Number of times pattern \( i \) is used
\end{itemize}

\textbf{Objective Function:}
\[
\text{Minimize} \quad Z = \sum_{i=1}^{N} x_i
\]
(The total number of large rolls used)

\textbf{Constraints:}
\[
\sum_{i=1}^{N} p_{i,j} x_i \geq d_j, \quad \forall j \in \{1, 2, \ldots, M\}
\]
(The demand constraint for each roll width)

\[
x_i \geq 0, \quad \forall i \in \{1, 2, \ldots, N\}
\]
(The non-negativity constraint for the number of patterns used)

\end{document}