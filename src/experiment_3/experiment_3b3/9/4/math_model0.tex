\documentclass{article}
\usepackage{amsmath}
\begin{document}

\section*{Mathematical Model for Nurse Scheduling Problem}

\subsection*{Parameters}
\begin{itemize}
    \item Let \( \text{period} \) be the number of consecutive days a nurse works.
    \item Let \( d_j \) be the demand for nurses on day \( j \) for \( j = 1, \ldots, 7 \).
\end{itemize}

\subsection*{Variables}
\begin{itemize}
    \item Let \( x_j \) be the number of nurses that start their period on day \( j \).
    \item Let \( N \) be the total number of nurses hired.
\end{itemize}

\subsection*{Objective Function}
Minimize the total number of nurses hired:
\[
\text{Minimize } N = \sum_{j=1}^{7} x_j
\]

\subsection*{Constraints}
1. The demand satisfaction constraint for each day \( j \):
\[
\sum_{i=0}^{\text{period}-1} x_{(j-i \mod 7)} \geq d_j, \quad \forall j = 1, \ldots, 7
\]
   This constraint ensures that the demand \( d_j \) for each day \( j \) is met, considering the nurses who started their shifts in the previous days of their working period.

2. Non-negativity constraint for the number of nurses:
\[
x_j \geq 0 \quad \forall j = 1, \ldots, 7
\]

3. Integer constraint for the decision variables:
\[
x_j \in \mathbb{Z}^+ \quad \forall j = 1, \ldots, 7
\]

\subsection*{Summary}
The decision variables \( x_j \) will indicate how many nurses start their working period on each day \( j \), while the objective is to minimize the total number of nurses hired \( N \) subject to the demand for nursing shifts across the week.

\end{document}