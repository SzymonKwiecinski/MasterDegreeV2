\documentclass{article}
\usepackage{amsmath}
\begin{document}

\title{Linear Programming Model for Alloy Production}
\author{}
\date{}
\maketitle

\section*{Problem Definition}
A manufacturer wishes to produce a total of \( \text{alloy\_quant} \) lb of an alloy that consists of several kinds of metals. The target composition by weight for metal \( m \) is represented as \( \text{target}_m \). 

Several different alloys \( k \) are available, each containing a specific ratio of metal \( m \) given by \( \text{ratio}_{k,m} \), and each alloy is sold at a price of \( \text{price}_k \).

\section*{Variables}
Let \( x_k \) represent the amount of alloy \( k \) to be purchased, where \( k = 1, \ldots, K \).

\section*{Objective Function}
The objective is to minimize the total cost of purchasing the alloys, which can be expressed as:
\[
\text{Minimize } Z = \sum_{k=1}^{K} \text{price}_k \cdot x_k
\]

\section*{Constraints}
The following constraints must be satisfied:

1. The total weight of the alloys must equal the target alloy weight:
\[
\sum_{k=1}^{K} x_k = \text{alloy\_quant}
\]

2. The target composition of metal \( m \) must be met:
\[
\sum_{k=1}^{K} \text{ratio}_{k,m} \cdot x_k = \text{target}_m, \quad \text{for } m = 1, \ldots, M
\]

3. All amounts of alloys must be non-negative:
\[
x_k \geq 0, \quad \text{for } k = 1, \ldots, K
\]

\section*{Output}
The outputs of the LP model will indicate the amounts of each alloy to be purchased:
\[
\text{amount} = [x_k \text{ for } k = 1, \ldots, K]
\]

\end{document}