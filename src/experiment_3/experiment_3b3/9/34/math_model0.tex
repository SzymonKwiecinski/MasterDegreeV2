\documentclass{article}
\usepackage{amsmath}
\begin{document}

\section*{Linear Programming Model for Food Purchase}

\subsection*{Parameters}
\begin{itemize}
    \item Let \( K \) be the number of different foods available in the market.
    \item Let \( M \) be the number of nutritional ingredients.
    \item Let \( price_k \) be the price per unit of food \( k \) for \( k = 1, \ldots, K \).
    \item Let \( demand_m \) be the minimum required units of nutrient \( m \) per day for \( m = 1, \ldots, M \).
    \item Let \( nutri_{k,m} \) be the units of nutrient \( m \) contained in one unit of food \( k \).
\end{itemize}

\subsection*{Decision Variables}
\begin{itemize}
    \item Let \( quantity_k \) be the quantity of food \( k \) to purchase for \( k = 1, \ldots, K \).
\end{itemize}

\subsection*{Objective Function}
We want to minimize the total cost of the foods purchased:
\[
\text{Minimize } Z = \sum_{k=1}^{K} price_k \cdot quantity_k
\]

\subsection*{Constraints}
To ensure that the nutritional demands are met, we have the following constraints:
\[
\sum_{k=1}^{K} nutri_{k,m} \cdot quantity_k \geq demand_m \quad \forall m = 1, \ldots, M
\]

\subsection*{Non-negativity Constraints}
All quantities of food must be non-negative:
\[
quantity_k \geq 0 \quad \forall k = 1, \ldots, K
\]

\subsection*{Summary of the Model}
The complete Linear Programming model can be summarized as follows:

\begin{align*}
\text{Minimize } & Z = \sum_{k=1}^{K} price_k \cdot quantity_k \\
\text{subject to } & \sum_{k=1}^{K} nutri_{k,m} \cdot quantity_k \geq demand_m, \quad \forall m = 1, \ldots, M \\
& quantity_k \geq 0, \quad \forall k = 1, \ldots, K
\end{align*}

\end{document}