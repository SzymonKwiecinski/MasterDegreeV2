\documentclass{article}
\usepackage{amsmath}
\begin{document}

\section*{Mathematical Model for Paper Cutting Problem}

\subsection*{Parameters}
\begin{itemize}
    \item Let \( W \) be the width of the large roll (in inches).
    \item Let \( M \) be the number of different smaller roll types.
    \item Let \( d_j \) be the demand for rolls of width \( j \) (units), where \( j = 1, \ldots, M \).
    \item Let \( N \) be the number of available cutting patterns.
    \item Let \( p_{i,j} \) be the number of rolls of width \( j \) produced by cutting pattern \( i \), where \( i = 1, \ldots, N \) and \( j = 1, \ldots, M \).
\end{itemize}

\subsection*{Decision Variables}
\begin{itemize}
    \item Let \( x_i \) be the number of times cutting pattern \( i \) is used (non-negative integer), where \( i = 1, \ldots, N \).
\end{itemize}

\subsection*{Objective Function}
Minimize the total number of large rolls used:
\[
\text{Minimize } Z = \sum_{i=1}^{N} x_i
\]

\subsection*{Constraints}
1. Demand Satisfaction Constraints:
\[
\sum_{i=1}^{N} p_{i,j} x_i \geq d_j \quad \forall j = 1, \ldots, M
\]
2. Non-negativity Constraints:
\[
x_i \geq 0 \quad \forall i = 1, \ldots, N
\]

\subsection*{Output Format}
The output will provide the patterns used and the total number of large rolls utilized, structured as follows:
\begin{verbatim}
{
    "patterns": [
        {
         "pattern": [p_{i,j} for j in 1, ..., M],
         "amount": x_i
        }
        for i in 1, ..., N
    ],
    "total_large_rolls_used": Z
}
\end{verbatim}

\end{document}