\documentclass{article}
\usepackage{amsmath}
\begin{document}

\section*{Linear Programming Model for Best Fit Line}

Let \( y_k \) be the observed value for the \( k \)-th data point and \( x_k \) be the corresponding value for the \( k \)-th data point, where \( k = 1, \ldots, K \). We aim to fit a linear equation of the form:

\[
y = bx + a
\]

where \( a \) is the intercept and \( b \) is the slope of the line. We want to minimize the sum of absolute deviations between the observed values \( y_k \) and the predicted values \( bx_k + a \).

\subsection*{Objective Function}

We need to minimize the total absolute deviation:

\[
\text{Minimize} \quad \sum_{k=1}^{K} |y_k - (bx_k + a)|
\]

To handle the absolute values, we introduce auxiliary variables \( e_k \) such that:

\[
e_k \geq y_k - (bx_k + a) \quad \text{and} \quad e_k \geq -(y_k - (bx_k + a))
\]

This translates our objective into a linear programming format as follows:

\[
\text{Minimize} \quad \sum_{k=1}^{K} e_k
\]

\subsection*{Constraints}

The model includes the following constraints:

\[
e_k \geq y_k - (bx_k + a), \quad \forall k = 1, \ldots, K
\]

\[
e_k \geq -(y_k - (bx_k + a)), \quad \forall k = 1, \ldots, K
\]

\subsection*{Variables}

The decision variables are:

- \( a \): intercept of the fitted line
- \( b \): slope of the fitted line
- \( e_k \): absolute deviations for each data point

\subsection*{Output}

After solving the linear programming problem, the outputs are:

\begin{itemize}
    \item "intercept": \( a \)
    \item "slope": \( b \)
\end{itemize}

\end{document}