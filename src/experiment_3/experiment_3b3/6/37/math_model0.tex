\documentclass{article}
\usepackage{amsmath}
\begin{document}

\section*{Linear Programming Formulation}

\textbf{Objective:} Maximize the total profit from producing spare parts.

Let:
\begin{itemize}
    \item \( K \) = number of different spare parts
    \item \( S \) = number of shops
    \item \( \text{time}_{k,s} \) = worker-hours required for part \( k \) in shop \( s \)
    \item \( \text{profit}_k \) = profit from part \( k \)
    \item \( \text{capacity}_s \) = capacity of shop \( s \) in working hours
    \item \( q_k \) = quantity of part \( k \) to be produced
\end{itemize}

\textbf{Objective Function:}
\[
\text{Maximize } Z = \sum_{k=1}^{K} \text{profit}_k \cdot q_k
\]

\textbf{Subject to:}
For each shop \( s \):
\[
\sum_{k=1}^{K} \text{time}_{k,s} \cdot q_k \leq \text{capacity}_s \quad \forall s = 1, 2, \ldots, S
\]

\textbf{Non-negativity Constraints:}
\[
q_k \geq 0 \quad \forall k = 1, 2, \ldots, K
\]

The output format for the quantities of each spare part to be produced is:
\[
\text{quantity} = [q_k \text{ for } k = 1, \ldots, K]
\]

\end{document}