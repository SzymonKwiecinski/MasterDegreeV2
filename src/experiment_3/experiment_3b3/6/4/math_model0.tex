\documentclass{article}
\usepackage{amsmath}
\begin{document}

\section*{Mathematical Model for Nurse Scheduling Problem}

\subsection*{Parameters}
\begin{itemize}
    \item $d_j$: Demand for nurses on day $j$ where $j = 1, \ldots, 7$.
    \item $p$: Number of consecutive days a nurse works (the period).
\end{itemize}

\subsection*{Variables}
\begin{itemize}
    \item $x_j$: Number of nurses starting their period on day $j$ where $j = 1, \ldots, 7$.
    \item $N$: Total number of nurses hired.
\end{itemize}

\subsection*{Objective Function}
Minimize the total number of nurses hired:
\[
\text{minimize } N = \sum_{j=1}^{7} x_j
\]

\subsection*{Constraints}
The number of nurses available to meet the demand on day $j$ must satisfy:
\[
\sum_{i=0}^{p-1} x_{(j-i) \mod 7} \geq d_j \quad \forall j = 1, \ldots, 7
\]
Where the index is taken modulo 7 to account for the weekly cycle.

\subsection*{Non-negativity Constraints}
\[
x_j \geq 0 \quad \forall j = 1, \ldots, 7
\]

\section*{Solution Format}
The output should specify:
\[
\begin{align*}
\text{start} &: [x_1, x_2, x_3, x_4, x_5, x_6, x_7] \\
\text{total} &: N
\end{align*}
\]

\end{document}