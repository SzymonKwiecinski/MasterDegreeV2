\documentclass{article}
\usepackage{amsmath}
\begin{document}

\section*{Mathematical Model for Police Officer Shift Assignment}

\textbf{Parameters:}
\begin{itemize}
    \item $S$: Number of different shifts
    \item $officers\_needed_s$: Number of officers needed during shift $s$, for $s = 1, \ldots, S$
    \item $shift\_cost_s$: Cost incurred for starting shift $s$, for $s = 1, \ldots, S$
\end{itemize}

\textbf{Decision Variables:}
\begin{itemize}
    \item $officers\_assigned_s$: Number of officers assigned to shift $s$, for $s = 1, \ldots, S$
\end{itemize}

\textbf{Objective Function:}
\[
\text{Minimize } Z = \sum_{s=1}^{S} shift\_cost_s \cdot \left\lceil \frac{officers\_needed_s}{2} \right\rceil
\]

\textbf{Constraints:}
\begin{itemize}
    \item Each police officer works for two consecutive shifts. Therefore, the number of officers assigned to each shift must suffice for the following constraints:
    \[
    officers\_assigned_s + officers\_assigned_{s-1} \geq officers\_needed_s, \quad \text{for } s = 2, \ldots, S
    \]
    \item For the first shift:
    \[
    officers\_assigned_1 \geq officers\_needed_1
    \]
    \item Non-negativity constraints:
    \[
    officers\_assigned_s \geq 0, \quad \text{for } s = 1, \ldots, S
    \]
\end{itemize}

\textbf{Output:}
\begin{itemize}
    \item $officers\_assigned_s$: The number of officers assigned to each shift $s$, for $s = 1, \ldots, S$
    \item $total\_cost$: The total cost incurred by the town, $Z$.
\end{itemize}

\end{document}