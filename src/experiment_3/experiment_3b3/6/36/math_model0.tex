\documentclass{article}
\usepackage{amsmath}
\begin{document}

\section*{Linear Programming Model for Alloy Production}

\textbf{Variables:}
\begin{itemize}
    \item Let \( a_k \) represent the amount of alloy \( k \) to be purchased (in lb), for \( k = 1, 2, \ldots, K \).
\end{itemize}

\textbf{Parameters:}
\begin{itemize}
    \item \( \text{alloy\_quant} \): Total quantity of alloy to be produced (in lb).
    \item \( \text{target}_m \): Target quantity of metal \( m \) in the final alloy (in lb), for \( m = 1, 2, \ldots, M \).
    \item \( \text{ratio}_{k,m} \): Ratio of metal \( m \) in alloy \( k \).
    \item \( \text{price}_k \): Price per lb of alloy \( k \).
\end{itemize}

\textbf{Objective:} Minimize the total cost of the alloys used.
\[
\text{Minimize } Z = \sum_{k=1}^{K} \text{price}_k \cdot a_k
\]

\textbf{Constraints:}
1. The total weight of the alloys used must equal the target alloy weight:
\[
\sum_{k=1}^{K} a_k = \text{alloy\_quant}
\]

2. The total amount of each metal \( m \) must meet the target:
\[
\sum_{k=1}^{K} \text{ratio}_{k,m} \cdot a_k = \text{target}_m, \quad \text{for } m = 1, 2, \ldots, M
\]

3. Non-negativity constraints:
\[
a_k \geq 0, \quad \text{for } k = 1, 2, \ldots, K
\]

\textbf{Output Variables:}
The output will consist of the amounts of each alloy purchased:
\[
\text{amount} = [a_k \text{ for } k = 1, \ldots, K]
\]

\end{document}