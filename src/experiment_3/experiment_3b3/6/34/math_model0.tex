\documentclass{article}
\usepackage{amsmath}
\begin{document}

\section*{Linear Programming Model}

\subsection*{Variables}
Let \( x_k \) be the quantity of food \( k \) purchased, where \( k = 1, \ldots, K \).

\subsection*{Parameters}
\begin{itemize}
    \item \( \text{price}_k \): Price per unit of food \( k \)
    \item \( \text{demand}_m \): Minimum required units of nutrient \( m \) per day
    \item \( \text{nutri}_{k,m} \): Units of nutrient \( m \) provided by one unit of food \( k \)
\end{itemize}

\subsection*{Objective Function}
We want to minimize the total cost of the food purchased:
\[
\text{Minimize} \quad Z = \sum_{k=1}^{K} \text{price}_k \cdot x_k
\]

\subsection*{Constraints}
We need to ensure that the nutritional requirements are met for each nutrient \( m \):
\[
\sum_{k=1}^{K} \text{nutri}_{k,m} \cdot x_k \geq \text{demand}_m, \quad \forall m = 1, \ldots, M
\]

\subsection*{Non-negativity Constraints}
The quantities of food purchased must be non-negative:
\[
x_k \geq 0, \quad \forall k = 1, \ldots, K
\]

\subsection*{Output}
The output will be the vector of quantities to purchase:
\[
\text{quantity} = \left[ x_1, x_2, \ldots, x_K \right]
\]

\end{document}