\documentclass{article}
\usepackage{amsmath}
\begin{document}

\section*{Linear Programming Model for Chebyshev Center}

Consider a set \( P \) defined by the constraints:
\[
P = \{ \mathbf{x} \in \mathbb{R}^N \mid \mathbf{a_i}^T \mathbf{x} \leq b_i, \; i = 1, \ldots, m \}
\]
where \( \mathbf{a_i} \in \mathbb{R}^N \) and \( b_i \in \mathbb{R} \).

The goal is to find a ball with the largest possible radius \( r \) centered at \( \mathbf{y} \in \mathbb{R}^N \) such that:
\[
\|\mathbf{x} - \mathbf{y}\| \leq r \quad \forall \mathbf{x} \in P
\]

This can be formulated as a linear programming problem with the following constraints:
\[
\mathbf{a_i}^T \mathbf{y} + r \|\mathbf{a_i}\| \leq b_i \quad \forall i = 1, \ldots, m
\]

\subsection*{Decision Variables}
\begin{itemize}
    \item \( \mathbf{y} \) - the center of the ball in \( \mathbb{R}^N \)
    \item \( r \) - the radius of the ball
\end{itemize}

\subsection*{Objective Function}
Maximize the radius \( r \):
\[
\text{maximize} \quad r
\]

\subsection*{Subject to}
\[
\mathbf{a_i}^T \mathbf{y} + r \|\mathbf{a_i}\| \leq b_i, \quad i = 1, 2, \ldots, m
\]
\[
r \geq 0
\]

The output of the model should include:
\begin{itemize}
    \item \textbf{center} (list): the center of the ball as a list of floats of length \( N \)
    \item \textbf{radius} (float): the radius of the ball as a float
\end{itemize}

The resulting format will be:
\[
\{ 
    \text{center}: [y_j \text{ for } j = 1, \ldots, N], 
    \text{radius}: r 
\}
\]

\end{document}