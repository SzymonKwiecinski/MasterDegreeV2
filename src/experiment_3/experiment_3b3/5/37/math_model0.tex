\documentclass{article}
\usepackage{amsmath}
\begin{document}

\section*{Linear Programming Formulation for Spare Automobile Parts Production}

\subsection*{Variables}
Let \( x_k \) be the quantity of spare part \( k \) to be produced, where \( k = 1, 2, \ldots, K \).

\subsection*{Parameters}
\begin{itemize}
    \item \( time_{k, s} \): the required worker-hours for part \( k \) in shop \( s \).
    \item \( profit_k \): the profit obtained from producing part \( k \).
    \item \( capacity_s \): the capacity of shop \( s \) in working hours.
\end{itemize}

\subsection*{Objective Function}
The objective is to maximize the total profit from the production of all spare parts:

\[
\text{Maximize } Z = \sum_{k=1}^{K} profit_k \cdot x_k
\]

\subsection*{Constraints}
Each shop must not exceed its capacity in terms of worker-hours. Therefore, for each shop \( s \):

\[
\sum_{k=1}^{K} time_{k, s} \cdot x_k \leq capacity_s, \quad \forall s = 1, 2, \ldots, S
\]

Additionally, we need to ensure that the quantities produced are non-negative:

\[
x_k \geq 0, \quad \forall k = 1, 2, \ldots, K
\]

\subsection*{Formulation Summary}
The complete linear programming model is given by:

\begin{align*}
\text{Maximize} & \quad Z = \sum_{k=1}^{K} profit_k \cdot x_k \\
\text{Subject to} & \quad \sum_{k=1}^{K} time_{k, s} \cdot x_k \leq capacity_s, \quad \forall s = 1, 2, \ldots, S \\
                  & \quad x_k \geq 0, \quad \forall k = 1, 2, \ldots, K
\end{align*}

\end{document}