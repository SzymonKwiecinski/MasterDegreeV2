\documentclass{article}
\usepackage{amsmath}
\begin{document}

\section*{Linear Programming Model for Alloy Production}

\textbf{Given:}
\begin{itemize}
    \item \( \text{alloy\_quant} \): Total weight of the alloy to be produced (in lbs).
    \item \( \text{target}_{m} \): Target quantity of metal \( m \) (in lbs).
    \item \( \text{ratio}_{k,m} \): Ratio of metal \( m \) in alloy \( k \).
    \item \( \text{price}_{k} \): Price per lb of alloy \( k \).
\end{itemize}

\textbf{Indices:}
\begin{itemize}
    \item \( k = 1, \ldots, K \): Alloys available for purchase.
    \item \( m = 1, \ldots, M \): Metals in the alloys.
\end{itemize}

\textbf{Variables:}
\begin{itemize}
    \item \( x_k \): Amount of alloy \( k \) to be purchased (in lbs).
\end{itemize}

\textbf{Objective Function:}
Minimize the total cost of the alloys:
\[
\text{Minimize} \quad Z = \sum_{k=1}^{K} \text{price}_{k} \cdot x_k
\]

\textbf{Constraints:}
\begin{itemize}
    \item Total weight of the alloys must equal the desired alloy weight:
    \[
    \sum_{k=1}^{K} x_k = \text{alloy\_quant}
    \]
    
    \item For each metal \( m \), the total weight of the metal from the alloys must equal the target quantity:
    \[
    \sum_{k=1}^{K} \text{ratio}_{k,m} \cdot x_k = \text{target}_{m}, \quad \forall m \in \{1, \ldots, M\}
    \]
    
    \item Non-negativity constraints on the amounts of alloys:
    \[
    x_k \geq 0, \quad \forall k \in \{1, \ldots, K\}
    \]
\end{itemize}

\textbf{Output:}
The solution provides the amounts of each alloy to purchase:
\[
\text{amount} = [x_{1}, x_{2}, \ldots, x_{K}]
\]

\end{document}