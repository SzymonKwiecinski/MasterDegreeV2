\documentclass{article}
\usepackage{amsmath}
\begin{document}

\section*{Quadratic Curve Fitting Problem}

Given a set of observed data points \( (x_k, y_k) \) for \( k = 1, \ldots, K \), we aim to fit the best quadratic curve of the form:

\[
y = c \cdot x^2 + b \cdot x + a
\]

The goal is to minimize the sum of absolute deviations of each observed value of \( y \) from the predicted value:

\[
\text{Objective:} \quad \min \sum_{k=1}^{K} |y_k - (c \cdot x_k^2 + b \cdot x_k + a)|
\]

We introduce auxiliary variables to handle the absolute values:

\[
d_k = y_k - (c \cdot x_k^2 + b \cdot x_k + a), \quad \text{for} \quad k = 1, \ldots, K
\]

Thus, we transform the absolute deviation into:

\[
\min \sum_{k=1}^{K} (u_k + v_k)
\]

subject to:

\[
d_k = u_k - v_k, \quad u_k \geq 0, \quad v_k \geq 0 \quad \text{for} \quad k = 1, \ldots, K
\]

The optimization problem can then be formulated as:

\[
\begin{aligned}
\text{Minimize} \quad & \sum_{k=1}^{K} (u_k + v_k) \\
\text{Subject to} \quad & u_k - v_k = y_k - (c \cdot x_k^2 + b \cdot x_k + a), \quad \forall k = 1, \ldots, K \\
& u_k \geq 0, \quad v_k \geq 0, \quad \forall k = 1, \ldots, K
\end{aligned}
\]

\textbf{Output:}
After solving this linear programming problem, we will obtain the coefficients:

\[
\text{Output Format:} \quad \{ "quadratic": c, "linear": b, "constant": a \}
\]

\end{document}