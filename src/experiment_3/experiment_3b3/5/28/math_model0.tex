\documentclass{article}
\usepackage{amsmath}
\begin{document}

\section*{Linear Programming Model for Best Fit Line}

Given a set of observed values \( y_k \) and corresponding values \( x_k \) for \( k = 1, \ldots, K \), we want to fit the best straight line of the form:

\[
y = bx + a
\]

To minimize the sum of absolute deviations, we define the objective function as follows:

\[
\text{Minimize } Z = \sum_{k=1}^{K} |y_k - (bx_k + a)|
\]

To handle the absolute values in the objective function, we introduce auxiliary variables \( u_k \) for each observation:

\[
u_k \geq y_k - (bx_k + a) \quad \text{for } k = 1, \ldots, K
\]

\[
u_k \geq -(y_k - (bx_k + a)) \quad \text{for } k = 1, \ldots, K
\]

This transforms our minimization problem into a linear programming format:

\[
\text{Minimize } Z = \sum_{k=1}^{K} u_k
\]

Subject to the constraints:

\[
u_k \geq y_k - (bx_k + a) \quad \text{for } k = 1, \ldots, K
\]

\[
u_k \geq -(y_k - (bx_k + a)) \quad \text{for } k = 1, \ldots, K
\]

\end{document}