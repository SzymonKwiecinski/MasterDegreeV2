\documentclass{article}
\usepackage{amsmath}
\begin{document}

\section*{Linear Programming Model for Line Fitting}

Given a set of observed values of \( y \) and corresponding values of \( x \), we aim to find the best fitting line of the form:

\[
y = bx + a
\]

where \( a \) is the intercept and \( b \) is the slope. 

Let the observed values be defined as follows:

\[
\begin{align*}
y_k & \text{ for } k = 1, \ldots, K \\
x_k & \text{ for } k = 1, \ldots, K
\end{align*}
\]

We define the deviations from the predicted values as:

\[
d_k = y_k - (bx_k + a) \quad \text{for } k = 1, \ldots, K
\]

Our objective is to minimize the maximum deviation, which can be formulated as:

\[
\min \quad t
\]

subject to the constraints:

\[
\begin{align*}
y_k - (bx_k + a) & \leq t \quad \text{for } k = 1, \ldots, K \\
-(y_k - (bx_k + a)) & \leq t \quad \text{for } k = 1, \ldots, K \\
t & \geq 0
\end{align*}
\]

The above constraints ensure that \( t \) is at least as large as the maximum positive and negative deviations.

Thus, the complete formulation in a linear programming format is:

\[
\begin{align*}
\text{Minimize} \quad & t \\
\text{subject to} \quad & y_k - bx_k - a \leq t, \quad k = 1, \ldots, K \\
& -y_k + bx_k + a \leq t, \quad k = 1, \ldots, K \\
& t \geq 0
\end{align*}
\]

\textbf{Output:} The optimal values of \( a \) (intercept) and \( b \) (slope) will give us the best fitting line that minimizes the maximum deviation from the observed values.

\end{document}