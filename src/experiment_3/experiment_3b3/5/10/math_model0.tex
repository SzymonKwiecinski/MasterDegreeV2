\documentclass{article}
\usepackage{amsmath}
\begin{document}

\section*{Linear Programming Model for School Assignment Problem}

\subsection*{Sets}
\begin{itemize}
    \item Let \( N \) represent the number of neighborhoods.
    \item Let \( S \) represent the number of schools.
    \item Let \( G \) represent the number of grades.
\end{itemize}

\subsection*{Parameters}
\begin{itemize}
    \item \( capacity_{s,g} \): Capacity of school \( s \) for grade \( g \).
    \item \( population_{n,g} \): Student population of grade \( g \) in neighborhood \( n \).
    \item \( d_{n,s} \): Distance from neighborhood \( n \) to school \( s \).
\end{itemize}

\subsection*{Decision Variables}
Define \( x_{n,s,g} \) as the number of students of grade \( g \) from neighborhood \( n \) assigned to school \( s \).

\subsection*{Objective Function}
The objective is to minimize the total distance traveled by all students:
\[
\text{Minimize } Z = \sum_{n=1}^{N} \sum_{s=1}^{S} \sum_{g=1}^{G} d_{n,s} \cdot x_{n,s,g}
\]

\subsection*{Constraints}

1. \textbf{Capacity Constraints:}
   Each school must not exceed its capacity for each grade:
   \[
   \sum_{n=1}^{N} x_{n,s,g} \leq capacity_{s,g}, \quad \forall s \in \{1, \ldots, S\}, \forall g \in \{1, \ldots, G\}
   \]

2. \textbf{Demand Constraints:}
   The number of students assigned to schools from each neighborhood must meet the student population:
   \[
   \sum_{s=1}^{S} x_{n,s,g} = population_{n,g}, \quad \forall n \in \{1, \ldots, N\}, \forall g \in \{1, \ldots, G\}
   \]

3. \textbf{Non-negativity Constraints:}
   The number of students assigned must be non-negative:
   \[
   x_{n,s,g} \geq 0, \quad \forall n \in \{1, \ldots, N\}, \forall s \in \{1, \ldots, S\}, \forall g \in \{1, \ldots, G\}
   \]

\subsection*{Output}
The output will be:
\begin{itemize}
    \item The optimal assignments \( x_{n,s,g} \) for all \( n, s, g \).
    \item The total distance traveled by all students, \( total\_distance \).
\end{itemize}

\end{document}