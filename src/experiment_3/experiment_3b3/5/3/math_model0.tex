\documentclass{article}
\usepackage{amsmath}
\begin{document}

\section*{Mathematical Model for Electricity Capacity Planning}

\subsection*{Indices}
\begin{itemize}
    \item \( t \): Year, where \( t = 1, 2, \ldots, T \)
\end{itemize}

\subsection*{Parameters}
\begin{itemize}
    \item \( d_t \): Demand for electricity in year \( t \) (megawatts)
    \item \( oil_t \): Existing oil capacity available in year \( t \) (megawatts)
    \item \( coal\_cost \): Capital cost per megawatt of coal-fired capacity
    \item \( nuke\_cost \): Capital cost per megawatt of nuclear power capacity
    \item \( max\_nuke \): Maximum percentage of total capacity that can be nuclear
    \item \( coal\_life \): Lifespan of coal-fired plants (years)
    \item \( nuke\_life \): Lifespan of nuclear plants (years)
\end{itemize}

\subsection*{Decision Variables}
\begin{itemize}
    \item \( coal_t \): Coal capacity added in year \( t \) (megawatts)
    \item \( nuke_t \): Nuclear capacity added in year \( t \) (megawatts)
\end{itemize}

\subsection*{Objective Function}
Minimize the total cost:
\[
\text{Minimize } Z = \sum_{t=1}^{T} (coal\_cost \cdot coal_t + nuke\_cost \cdot nuke_t)
\]

\subsection*{Constraints}

1. **Demand Satisfaction**:
\[
oil_t + \sum_{i=1}^{\min(t, coal\_life)} coal_{t-i} + \sum_{j=1}^{\min(t, nuke\_life)} nuke_{t-j} \geq d_t \quad \forall t
\]

2. **Total Capacity Constraint**:
\[
nuke_{cap\_total} \leq \frac{max\_nuke}{100} \cdot (oil_t + \sum_{i=1}^{T} coal_i + \sum_{j=1}^{T} nuke_j) \quad \forall t
\]
where \( nuke_{cap\_total} = \sum_{j=1}^{t} nuke_j \)

3. **Non-negativity**:
\[
coal_t \geq 0, \quad nuke_t \geq 0 \quad \forall t
\]

\subsection*{Output}
The output will be a dictionary of the form:
\[
\text{Output} = \{ 
    \text{"coal\_cap\_added"}: [coal_t \text{ for } t \in 1, \ldots, T], 
    \text{"nuke\_cap\_added"}: [nuke_t \text{ for } t \in 1, \ldots, T], 
    \text{"total\_cost"}: \text{total\_cost} 
\}
\]

\end{document}