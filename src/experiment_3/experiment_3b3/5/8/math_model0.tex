\documentclass{article}
\usepackage{amsmath}
\begin{document}

\section*{Linear Programming Model for Spare Automobile Parts Production}

\subsection*{Definitions}
\begin{itemize}
    \item Let \( K \) be the total number of spare parts.
    \item Let \( S \) be the total number of shops.
    \item Let \( time_{k,s} \) be the required worker-hours for part \( k \) in shop \( s \).
    \item Let \( profit_{k} \) be the profit of part \( k \).
    \item Let \( capacity_{s} \) be the capacity of shop \( s \) in working hours.
    \item Let \( quantity_{k} \) be the quantity of part \( k \) to be produced.
\end{itemize}

\subsection*{Objective Function}
The objective is to maximize the total profit from producing spare parts:
\[
\text{Maximize } Z = \sum_{k=1}^{K} profit_{k} \cdot quantity_{k}
\]

\subsection*{Constraints}
The production of spare parts must satisfy the capacity constraints of each shop:
\[
\sum_{k=1}^{K} time_{k,s} \cdot quantity_{k} \leq capacity_{s}, \quad \forall s = 1, \ldots, S
\]
Also, we must ensure that the quantity produced for each part is non-negative:
\[
quantity_{k} \geq 0, \quad \forall k = 1, \ldots, K
\]

\subsection*{Complete Mathematical Model}
The complete linear programming model can be summarized as follows:

\begin{align*}
\text{Maximize} \quad & Z = \sum_{k=1}^{K} profit_{k} \cdot quantity_{k} \\
\text{Subject to} \quad & \sum_{k=1}^{K} time_{k,s} \cdot quantity_{k} \leq capacity_{s}, \quad \forall s = 1, \ldots, S \\
& quantity_{k} \geq 0, \quad \forall k = 1, \ldots, K
\end{align*}

\end{document}