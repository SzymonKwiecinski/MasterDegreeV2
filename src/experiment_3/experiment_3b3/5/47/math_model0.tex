\documentclass{article}
\usepackage{amsmath}
\begin{document}

\section*{Mathematical Model for Police Officer Scheduling}

\textbf{Parameters:}
\begin{itemize}
    \item $S$: Total number of shifts
    \item $officers\_needed_{s}$: Number of officers needed for shift $s$, for $s = 1, \ldots, S$
    \item $shift\_cost_{s}$: Cost incurred when starting a shift $s$, for $s = 1, \ldots, S$
\end{itemize}

\textbf{Decision Variables:}
\begin{itemize}
    \item $officers\_assigned_{s}$: Number of officers assigned to shift $s$, for $s = 1, \ldots, S$
\end{itemize}

\textbf{Objective Function:}
Minimize the total cost:
\[
\text{Total Cost} = \sum_{s=1}^{S} shift\_cost_{s} \cdot \left\lceil \frac{officers\_needed_{s}}{2} \right\rceil
\]

\textbf{Constraints:}
\begin{itemize}
    \item Each police officer works for two consecutive shifts. Therefore, for each shift $s$:
    \[
    officers\_assigned_{s} + officers\_assigned_{s-1} \geq officers\_needed_{s}
    \]
    where $officers\_assigned_{0} = 0$ for the first shift, and $officers\_assigned_{S+1} = 0$ for the last shift.
\end{itemize}

\textbf{Output:}
\begin{itemize}
    \item $officers\_assigned$: Vector containing the number of officers assigned to each shift
    \item $total\_cost$: Total cost incurred by the town
\end{itemize}

\end{document}