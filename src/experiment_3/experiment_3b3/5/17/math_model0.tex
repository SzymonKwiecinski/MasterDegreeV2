\documentclass{article}
\usepackage{amsmath}
\begin{document}

\title{Linear Programming Model for Stock Selling Problem}
\author{}
\date{}
\maketitle

\section*{Problem Formulation}

Let:
\begin{itemize}
    \item \( N \) be the number of different stocks.
    \item \( bought_i \) be the number of shares bought for stock \( i \) for \( i = 1, \ldots, N \).
    \item \( buyPrice_i \) be the price at which stock \( i \) was bought.
    \item \( currentPrice_i \) be the current price of stock \( i \).
    \item \( futurePrice_i \) be the expected future price of stock \( i \).
    \item \( transactionRate \) be the transaction cost rate (in percentage).
    \item \( taxRate \) be the tax rate on capital gains (in percentage).
    \item \( K \) be the net amount of money the investor wants to raise.
    \item \( sell_i \) be the number of shares of stock \( i \) that the investor should sell.
\end{itemize}

\subsection*{Objective Function}

Maximize the expected value of the portfolio next year:
\[
\max \sum_{i=1}^{N} futurePrice_i \cdot (bought_i - sell_i)
\]

\subsection*{Constraints}

1. The investor needs to meet the net amount requirement after considering transaction costs and taxes:
\[
\sum_{i=1}^{N} \left( currentPrice_i \cdot sell_i \cdot (1 - \frac{transactionRate}{100}) - (currentPrice_i - buyPrice_i) \cdot sell_i \cdot \frac{taxRate}{100} \right) \geq K
\]

2. The number of shares sold should not exceed the number of shares bought:
\[
0 \leq sell_i \leq bought_i, \quad \forall i = 1, \ldots, N
\]

3. Non-negativity:
\[
sell_i \geq 0, \quad \forall i = 1, \ldots, N
\]

\subsection*{Output Format}

The solution to the optimization problem will provide the amounts to sell for each stock:
\[
\text{Output: } \{ sell: [sell_i, \text{ for } i = 1, \ldots, N] \}
\]

\end{document}