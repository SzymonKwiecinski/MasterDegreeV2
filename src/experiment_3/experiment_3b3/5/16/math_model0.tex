\documentclass{article}
\usepackage{amsmath}
\begin{document}

\section*{Linear Programming Model for Oil Refinery Production}

Let \( O \) be the number of crude oil types, \( P \) be the number of products, and \( L \) be the number of production processes.

Define the following variables:
\begin{itemize}
    \item \( \text{allocated}_i \): the allocated million barrels of crude oil type \( i \).
    \item \( \text{price}_p \): the selling price per barrel of product \( p \).
    \item \( \text{input}_{l,i} \): the number of barrels of crude \( i \) required for process \( l \).
    \item \( \text{output}_{l,p} \): the number of barrels of product \( p \) produced by process \( l \).
    \item \( \text{cost}_l \): the cost per barrel of product produced by process \( l \).
    \item \( x_l \): the number of times process \( l \) is executed.
    \item \( \text{revenue} \): total revenue for the month.
\end{itemize}

\subsection*{Objective Function}
Maximize total revenue:
\[
\text{revenue} = \sum_{p=1}^{P} \text{price}_p \cdot \sum_{l=1}^{L} \text{output}_{l,p} \cdot x_l
\]

\subsection*{Constraints}
\begin{itemize}
    \item Resource Constraints:
    \[
    \sum_{l=1}^{L} \text{input}_{l,i} \cdot x_l \leq \text{allocated}_i \quad \forall i = 1, \ldots, O
    \]
    
    \item Non-negativity Constraints:
    \[
    x_l \geq 0 \quad \forall l = 1, \ldots, L
    \end{itemize}

\subsection*{Summary of the Model}
The objective is to maximize the revenue subject to the constraints presented above, ensuring that the number of times each process is executed is non-negative and does not exceed the allocated resources of each crude oil type.

\end{document}