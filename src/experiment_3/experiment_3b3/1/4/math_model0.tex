\documentclass{article}
\usepackage{amsmath}
\begin{document}

\section*{Mathematical Model of the Nurse Scheduling Problem}

\subsection*{Parameters}
\begin{itemize}
    \item $period$: Number of days each nurse works in a row on the night shift.
    \item $d_j$: Demand for nurses on day $j$, where $j = 1, \ldots, 7$.
\end{itemize}

\subsection*{Decision Variables}
\begin{itemize}
    \item $start_j$: Number of nurses that start their period on day $j$, where $j = 1, \ldots, 7$.
    \item $N$: Total number of nurses hired.
\end{itemize}

\subsection*{Objective Function}
Minimize the total number of nurses hired:
\[
\text{Minimize } N = \sum_{j=1}^{7} start_j
\]

\subsection*{Constraints}
For each day $j$:
\[
\sum_{i=0}^{period-1} start_{(j-i) \mod 7} \geq d_j \quad \forall j \in \{1, 2, \ldots, 7\}
\]

Where $start_k = 0$ for $k \leq 0$ to account for the circular nature of the week.

\subsection*{Output}
The output will consist of:
\begin{itemize}
    \item A list $start = [start_1, start_2, \ldots, start_7]$ which denotes the number of nurses starting on each day.
    \item The total number of nurses hired $total = N$.
\end{itemize}

\end{document}