\documentclass{article}
\usepackage{amsmath}
\begin{document}

\title{Linear Programming Model for Police Shift Assignment}
\author{}
\date{}
\maketitle

\section*{Problem Definition}
Let:
\begin{itemize}
    \item \( S \) be the total number of shifts.
    \item \( officers\_needed_{s} \) be the number of police officers needed for shift \( s \), where \( s \in \{1, 2, \ldots, S\} \).
    \item \( shift\_cost_{s} \) be the cost incurred when starting shift \( s \).
    \item \( officers\_assigned_{s} \) be the number of police officers assigned to shift \( s \).
\end{itemize}

\section*{Objective Function}
We aim to minimize the total cost for assigning police officers, represented mathematically as:

\[
\text{Minimize } Z = \sum_{s=1}^{S} shift\_cost_{s} \cdot x_{s}
\]

where \( x_{s} \) is the number of officers assigned to the shift \( s \).

\section*{Constraints}
Each officer works for two consecutive shifts, leading to the following constraints:

1. For each shift \( s \):
   \[
   x_{s-1} + x_{s} \geq officers\_needed_{s} \quad \text{for } s = 2, 3, \ldots, S
   \]
   \[
   x_{1} \geq officers\_needed_{1}
   \]

2. Non-negativity:
   \[
   x_{s} \geq 0 \quad \text{for all } s
   \]

\section*{Model Summary}
The mathematical model can be summarized as follows:

\begin{align*}
\text{Minimize } & Z = \sum_{s=1}^{S} shift\_cost_{s} \cdot x_{s} \\
\text{Subject to: } & x_{1} \geq officers\_needed_{1} \\
& x_{s-1} + x_{s} \geq officers\_needed_{s}, \quad s = 2, 3, \ldots, S \\
& x_{s} \geq 0, \quad s = 1, 2, \ldots, S
\end{align*}

\section*{Output}
The solution will provide:
\begin{itemize}
    \item \( officers\_assigned \): the number of officers assigned to each shift \( s \).
    \item \( total\_cost \): the total cost incurred by the town.
\end{itemize}

\end{document}