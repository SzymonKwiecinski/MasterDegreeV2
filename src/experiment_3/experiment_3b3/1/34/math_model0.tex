\documentclass{article}
\usepackage{amsmath}
\begin{document}

\section*{Linear Programming Model for Food Purchasing}

\subsection*{Variables}
Let \( x_k \) be the quantity of food \( k \) to purchase, where \( k = 1, 2, \ldots, K \).

\subsection*{Parameters}
\begin{itemize}
    \item \( \text{price}_k \)  = price of food \( k \)
    \item \( \text{demand}_m \) = minimum required units of nutrient \( m \)
    \item \( \text{nutri}_{k, m} \) = units of nutrient \( m \) in food \( k \)
    \item \( M \) = number of nutrients
    \item \( K \) = number of different foods
\end{itemize}

\subsection*{Objective Function}
The objective is to minimize the total cost of food purchases:
\[
\text{Minimize } Z = \sum_{k=1}^{K} \text{price}_k \cdot x_k
\]

\subsection*{Constraints}
To ensure that the nutrient demands are met, the following constraints must be satisfied for each nutrient \( m \):
\[
\sum_{k=1}^{K} \text{nutri}_{k, m} \cdot x_k \geq \text{demand}_m \quad \text{for } m = 1, 2, \ldots, M
\]

Additionally, quantities must be non-negative:
\[
x_k \geq 0 \quad \text{for } k = 1, 2, \ldots, K
\]

\subsection*{Output}
The output will be:
\begin{itemize}
    \item \( x_k \): quantity of food \( k \) to purchase.
\end{itemize}

\end{document}