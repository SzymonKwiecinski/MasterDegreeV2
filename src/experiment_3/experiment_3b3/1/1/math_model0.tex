\documentclass{article}
\usepackage{amsmath}
\begin{document}

\section*{Linear Programming Model for Maximizing Revenue}

\subsection*{Sets and Parameters}
\begin{itemize}
    \item Let \( M \) be the number of different goods.
    \item Let \( N \) be the number of different raw materials.
    \item Let \( \text{available}_i \) be the available amount of raw material \( i \) for \( i = 1, \ldots, N \).
    \item Let \( \text{req}_{i,j} \) be the requirement of raw material \( i \) for good \( j \) for \( j = 1, \ldots, M \).
    \item Let \( \text{price}_j \) be the revenue per unit produced of good \( j \).
\end{itemize}

\subsection*{Decision Variable}
Let \( \text{amount}_j \) be the amount of good \( j \) produced for \( j = 1, \ldots, M \).

\subsection*{Objective Function}
The objective is to maximize total revenue, given by:
\[
\text{Maximize} \quad Z = \sum_{j=1}^{M} \text{price}_j \cdot \text{amount}_j
\]

\subsection*{Constraints}
1. Material constraints for each raw material \( i \):
\[
\sum_{j=1}^{M} \text{req}_{i,j} \cdot \text{amount}_j \leq \text{available}_i \quad \text{for } i = 1, \ldots, N
\]

2. Non-negativity constraints:
\[
\text{amount}_j \geq 0 \quad \text{for } j = 1, \ldots, M
\]

\subsection*{Summary of the Model}
The complete linear programming model can be summarized as follows:

\begin{align*}
\text{Maximize} \quad & Z = \sum_{j=1}^{M} \text{price}_j \cdot \text{amount}_j \\
\text{subject to} \quad & \sum_{j=1}^{M} \text{req}_{i,j} \cdot \text{amount}_j \leq \text{available}_i, \quad i = 1, \ldots, N \\
& \text{amount}_j \geq 0, \quad j = 1, \ldots, M
\end{align*}

\end{document}