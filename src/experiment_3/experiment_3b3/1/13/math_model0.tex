\documentclass{article}
\usepackage{amsmath}
\begin{document}

\section*{Linear Programming Model for the Chebyshev Center}

Given a set \( P \) defined by the linear inequality constraints:
\[
P = \{ x \in \mathbb{R}^N \mid a_i^T x \leq b_i, \, i = 1, \ldots, m \}
\]
where \( a_i \in \mathbb{R}^N \) and \( b_i \in \mathbb{R} \), we aim to find a ball (or sphere) centered at \( y \in \mathbb{R}^N \) with radius \( r \) such that the ball is entirely contained within the set \( P \).

The ball is defined as:
\[
B(y, r) = \{ x \in \mathbb{R}^N \mid \| x - y \| \leq r \}
\]

To formulate the problem, we need to maximize the radius \( r \) under the constraints imposed by the set \( P \). The Chebyshev center of \( P \) is the center of the largest ball that fits within \( P \).

### Optimization Problem:

The optimization problem can be formulated as follows:

\[
\begin{align*}
\text{Maximize} \quad & r \\
\text{subject to} \quad & \| a_i^T y - b_i \| \leq r, \quad \forall i = 1, \ldots, m \\
& \text{for the Euclidean distance constraint: } \\
& \sqrt{(y_j - \frac{b_i - a_i^T y}{\|a_i\|_2^2} a_{ij})^2} \leq r, \quad \forall j = 1, \ldots, N, \, i = 1, \ldots, m
\end{align*}
\]

This can be rewritten in a more explicit form suitable for linear programming by introducing auxiliary variables as needed, and the constraints would ensure that the distance from the center \( y \) to the boundary defined by the inequalities does not exceed the radius \( r \).

### Output Format:

The output of this optimization problem will yield:
\[
\text{Output} =
\begin{cases}
\text{center: } [y_1, y_2, \ldots, y_N] \\
\text{radius: } r
\end{cases}
\]

where \( y_j \) are the coordinates of the center of the ball in \( \mathbb{R}^N \) and \( r \) is the maximum radius of the ball.

\end{document}