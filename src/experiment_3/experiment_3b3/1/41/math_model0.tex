\documentclass{article}
\usepackage{amsmath}
\begin{document}

\title{Mathematical Model for Floppy Disk Backup Problem}
\author{}
\date{}
\maketitle

\section*{Problem Definition}

We want to minimize the number of floppy disks used to back up important files, given the size of each file and the capacity of the disks.

\subsection*{Parameters}
\begin{itemize}
    \item $C$: Capacity of each floppy disk (in GB)
    \item $N$: Number of files
    \item $size_j$: Size of file $j$ (for $j = 1, \ldots, N$)
\end{itemize}

\subsection*{Decision Variables}
\begin{itemize}
    \item $d_k$: Binary variable, $d_k = 1$ if disk $k$ is used, $0$ otherwise
    \item $x_{jk}$: Binary variable, $x_{jk} = 1$ if file $j$ is placed on disk $k$, $0$ otherwise
\end{itemize}

\subsection*{Objective Function}
Minimize the total number of disks used:
\[
\text{Minimize } n_{disks} = \sum_{k=1}^{M} d_k
\]
where $M$ is the maximum number of disks we can potentially use.

\subsection*{Constraints}
1. Each file must be assigned to exactly one disk:
\[
\sum_{k=1}^{M} x_{jk} = 1 \quad \forall j \in \{1, \ldots, N\}
\]

2. The total size of the files on each disk must not exceed its capacity:
\[
\sum_{j=1}^{N} size_j \cdot x_{jk} \leq C \cdot d_k \quad \forall k \in \{1, \ldots, M\}
\]

3. A disk can only be marked as used if at least one file is assigned to it:
\[
d_k \geq x_{jk} \quad \forall j \in \{1, \ldots, N\}, \forall k \in \{1, \ldots, M\}
\]

4. Binary constraints:
\[
x_{jk} \in \{0, 1\}, \quad d_k \in \{0, 1\} \quad \forall j, k
\]

\subsection*{Outputs}
The outputs of the model will be:
\begin{itemize}
    \item $n_{disks}$: Total number of floppy disks used
    \item $whichdisk$: An array indicating on which disk each file is distributed
\end{itemize}

\end{document}