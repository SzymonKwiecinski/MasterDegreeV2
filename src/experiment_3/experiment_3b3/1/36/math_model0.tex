\documentclass{article}
\usepackage{amsmath}
\begin{document}

\title{Linear Programming Model for Alloy Production}
\author{}
\date{}
\maketitle

\section*{Problem Statement}
A manufacturer wishes to produce a total of \( \text{alloy\_quant} \) lb of an alloy that consists of several kinds of metals, specifically targeting \( \text{target}_{m} \) lb of metal \( \text{metal}_{m} \). The manufacturer has \( K \) different alloys available for purchase.

\subsection*{Parameters}
\begin{itemize}
    \item \( \text{alloy\_quant} \): Total weight of the desired alloy (in lb).
    \item \( \text{target}_{m} \): Target weight of metal \( m \) (for \( m = 1, \ldots, M \)).
    \item \( \text{ratio}_{k,m} \): Ratio of metal \( m \) in alloy \( k \) (for \( k = 1, \ldots, K \) and \( m = 1, \ldots, M \)).
    \item \( \text{price}_{k} \): Price of alloy \( k \) (for \( k = 1, \ldots, K \)).
\end{itemize}

\subsection*{Decision Variables}
Let \( \text{amount}_{k} \) be the quantity of alloy \( k \) purchased (for \( k = 1, \ldots, K \)).

\subsection*{Objective Function}
The objective is to minimize the total cost of the alloys purchased:
\[
\text{Minimize } Z = \sum_{k=1}^{K} \text{price}_{k} \cdot \text{amount}_{k}
\]

\subsection*{Constraints}
1. The total weight of the alloy produced must equal the desired weight:
   \[
   \sum_{k=1}^{K} \text{amount}_{k} = \text{alloy\_quant}
   \]

2. The quantity of each metal in the alloy must satisfy the target requirement:
   \[
   \sum_{k=1}^{K} \text{ratio}_{k,m} \cdot \text{amount}_{k} = \text{target}_{m}, \quad \text{for } m = 1, \ldots, M
   \]

3. Non-negativity constraint:
   \[
   \text{amount}_{k} \geq 0, \quad \text{for } k = 1, \ldots, K
   \]

\subsection*{Output}
The solution will provide the amounts of each alloy to purchase:
\[
\text{amount} = [\text{amount}_{1}, \text{amount}_{2}, \ldots, \text{amount}_{K}]
\]

\end{document}