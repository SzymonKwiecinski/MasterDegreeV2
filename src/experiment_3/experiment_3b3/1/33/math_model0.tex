\documentclass{article}
\usepackage{amsmath}
\begin{document}

\title{Mathematical Model for the Knapsack Problem}
\author{}
\date{}
\maketitle

\section{Introduction}
In the knapsack problem, we aim to select a subset of items that maximizes the total value while staying within a specified capacity. Each item \( k \) has an associated value \( \text{value}_{k} \) and size \( \text{size}_{k} \). The capacity of the knapsack is denoted as \( C \).

\section{Mathematical Model}

\subsection{Indices}
\begin{itemize}
    \item \( k \): Index for items, where \( k = 1, \ldots, K \)
\end{itemize}

\subsection{Parameters}
\begin{itemize}
    \item \( C \): Maximum capacity of the knapsack
    \item \( \text{value}_{k} \): Value of item \( k \)
    \item \( \text{size}_{k} \): Size of item \( k \)
\end{itemize}

\subsection{Variables}
\begin{itemize}
    \item \( \text{isincluded}_{k} \): Binary variable which is 1 if item \( k \) is included in the knapsack, and 0 otherwise
\end{itemize}

\subsection{Objective Function}
We want to maximize the total value of the packed items:
\[
\text{Maximize} \quad Z = \sum_{k=1}^{K} \text{value}_{k} \cdot \text{isincluded}_{k}
\]

\subsection{Constraints}
The total size of the selected items must not exceed the capacity of the knapsack:
\[
\sum_{k=1}^{K} \text{size}_{k} \cdot \text{isincluded}_{k} \leq C
\]
Additionally, the decision variable must be binary:
\[
\text{isincluded}_{k} \in \{0, 1\}, \quad \forall k \in \{1, \ldots, K\}
\]

\section{Output Format}
The output will indicate which items are included in the knapsack:
\[
\text{output} = \{ \text{isincluded} = [\text{isincluded}_{k} \text{ for } k \text{ in } 1,\ldots,K] \}
\]

\end{document}