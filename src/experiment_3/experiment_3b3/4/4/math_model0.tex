\documentclass{article}
\usepackage{amsmath}
\begin{document}

\section*{Mathematical Model for Nurse Scheduling Problem}

\subsection*{Parameters}
\begin{itemize}
    \item $period$: The number of consecutive days a nurse works the night shift.
    \item $d_j$: The demand for nurses on night shift for day $j$, where $j = 1, \ldots, 7$.
\end{itemize}

\subsection*{Variables}
\begin{itemize}
    \item $start_j$: The number of nurses that start their shift on day $j$, where $j = 1, \ldots, 7$.
    \item $total$: The total number of nurses hired.
\end{itemize}

\subsection*{Objective}
Minimize the total number of nurses hired:
\[
\text{Minimize } total = \sum_{j=1}^{7} start_j
\]

\subsection*{Constraints}
The demand for nurses on each day must be satisfied, which can be expressed as:
\[
\sum_{i=0}^{period-1} start_{(j-i) \mod 7} \geq d_j, \quad \forall j = 1, \ldots, 7
\]
where $start_{(j-i) \mod 7}$ is the number of nurses starting on day $j-i$.

\subsection*{Non-negativity Constraints}
All variables must be non-negative integers:
\[
start_j \geq 0 \quad \forall j = 1, \ldots, 7
\]
\[
total \geq 0
\]

\end{document}