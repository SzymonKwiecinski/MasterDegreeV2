\documentclass{article}
\usepackage{amsmath}
\begin{document}

\section*{Linear Programming Model for Alloy Production}

\subsection*{Parameters}

\begin{itemize}
    \item Let \( alloy\_quant \) be the total weight of the alloy to be produced (in lb).
    \item Let \( target_m \) be the target quantity of metal \( m \) in the alloy (for \( m=1, \ldots, M \)).
    \item Let \( ratio_{k, m} \) be the ratio of metal \( m \) in alloy \( k \) (for \( k=1, \ldots, K \) and \( m=1, \ldots, M \)).
    \item Let \( price_k \) be the price of alloy \( k \).
\end{itemize}

\subsection*{Decision Variables}

Let \( amount_k \) represent the quantity of alloy \( k \) that we buy (for \( k=1, \ldots, K \)).

\subsection*{Objective Function}

The objective is to minimize the total cost of the alloys used:

\[
\text{Minimize } Z = \sum_{k=1}^{K} price_k \cdot amount_k
\]

\subsection*{Constraints}

1. The total weight of the alloys used must equal the desired weight of the alloy:
\[
\sum_{k=1}^{K} amount_k = alloy\_quant
\]

2. For each metal \( m \), the total amount of that metal contributed by the alloys must equal the desired target quantity:
\[
\sum_{k=1}^{K} ratio_{k, m} \cdot amount_k = target_m \quad \text{for } m = 1, 2, \ldots, M
\]

3. Non-negativity constraints:
\[
amount_k \geq 0 \quad \text{for } k = 1, 2, \ldots, K
\]

\subsection*{Output}

The output will be the amounts of each alloy to purchase:

\[
\text{Output: } amount = [amount_k \text{ for } k = 1, \ldots, K]
\]

\end{document}