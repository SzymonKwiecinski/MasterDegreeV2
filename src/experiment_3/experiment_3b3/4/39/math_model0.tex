\documentclass{article}
\usepackage{amsmath}
\begin{document}

\section*{Mathematical Model for the Cafeteria Staffing Problem}

\subsection*{Parameters}
\begin{itemize}
    \item Let \( N \) be the total number of days.
    \item Let \( num_n \) be the desired number of employees on day \( n \) where \( n \in \{1, 2, \ldots, N\} \).
    \item \( n_{working\_days} \) is the number of consecutive working days for each employee.
    \item \( n_{resting\_days} \) is the number of consecutive resting days for each employee.
\end{itemize}

\subsection*{Variables}
\begin{itemize}
    \item Let \( total\_number \) be the total number of employees to hire.
    \item Define the binary variable \( is\_work_{n,i} \) where:
    \[
    is\_work_{n,i} =
    \begin{cases}
    1 & \text{if employee } i \text{ is working on day } n \\
    0 & \text{otherwise}
    \end{cases}
    \]
    for \( i \in \{1, 2, \ldots, total\_number\} \) and \( n \in \{1, 2, \ldots, N\} \).
\end{itemize}

\subsection*{Objective Function}
Minimize the total number of employees:
\[
\text{Minimize } total\_number
\]

\subsection*{Constraints}
1. Ensure that the number of employees working on each day meets the required staff:
\[
\sum_{i=1}^{total\_number} is\_work_{n,i} \geq num_n \quad \forall n \in \{1, 2, \ldots, N\}
\]

2. Define the working and resting days for each employee:
   Each employee works for \( n_{working\_days} \) days and then rests for \( n_{resting\_days} \) days. Thus, for each employee \( i \):

\[
is\_work_{n,i} = 
\begin{cases}
1 & \text{if } n \mod (n_{working\_days} + n_{resting\_days}) \in \{1, 2, \ldots, n_{working\_days}\} \\
0 & \text{otherwise}
\end{cases}
\]

for \( n \in \{1, 2, \ldots, N\} \).

\subsection*{Output}
The output of the problem is given by:
\[
\{ 
    "total\_number": total\_number, 
    "is\_work": [[is\_work_{n,i} \text{ for } n \in \{1,\ldots,N\}] \text{ for } i \in \{1,\ldots,total\_number\}]
\}
\]

\end{document}