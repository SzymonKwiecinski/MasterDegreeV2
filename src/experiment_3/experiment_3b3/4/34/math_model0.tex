\documentclass{article}
\usepackage{amsmath}
\begin{document}

\section*{Linear Programming Model}

\subsection*{Parameters}
\begin{itemize}
    \item Let \( K \) be the number of different foods available.
    \item Let \( M \) be the number of nutritional ingredients.
    \item Let \( \text{price}_k \) be the price per unit of food \( k \) for \( k = 1, \ldots, K \).
    \item Let \( \text{demand}_m \) be the minimum required units of nutrient \( m \) for \( m = 1, \ldots, M \).
    \item Let \( \text{nutri}_{k,m} \) be the units of nutrient \( m \) contained in food \( k \).
\end{itemize}

\subsection*{Decision Variables}
\begin{itemize}
    \item Let \( \text{quantity}_k \) be the quantity of food \( k \) to purchase for \( k = 1, \ldots, K \).
\end{itemize}

\subsection*{Objective Function}
We want to minimize the total price of the foods purchased:
\[
\text{Minimize} \quad Z = \sum_{k=1}^{K} \text{price}_k \cdot \text{quantity}_k
\]

\subsection*{Constraints}
The following constraints must be satisfied to meet the nutritional demands:
\[
\sum_{k=1}^{K} \text{nutri}_{k,m} \cdot \text{quantity}_k \geq \text{demand}_m \quad \text{for } m = 1, \ldots, M
\]

\subsection*{Non-negativity Constraints}
\[
\text{quantity}_k \geq 0 \quad \text{for } k = 1, \ldots, K
\]

\subsection*{Summary}
The complete linear programming model can be stated as follows:
\begin{align*}
    \text{Minimize} & \quad Z = \sum_{k=1}^{K} \text{price}_k \cdot \text{quantity}_k \\
    \text{subject to} & \quad \sum_{k=1}^{K} \text{nutri}_{k,m} \cdot \text{quantity}_k \geq \text{demand}_m, \quad m = 1, \ldots, M \\
    & \quad \text{quantity}_k \geq 0, \quad k = 1, \ldots, K 
\end{align*}

\end{document}