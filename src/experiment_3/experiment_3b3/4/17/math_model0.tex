\documentclass{article}
\usepackage{amsmath}
\begin{document}

\section*{Linear Programming Model for Investor's Portfolio}

\subsection*{Variables}
Let:
\begin{itemize}
    \item \( N \) = number of different stocks
    \item \( bought_i \) = number of shares bought of stock \( i \)
    \item \( buyPrice_i \) = price at which the investor bought stock \( i \)
    \item \( currentPrice_i \) = current price of stock \( i \)
    \item \( futurePrice_i \) = expected future price of stock \( i \)
    \item \( transactionRate \) = transaction cost rate (in percentage)
    \item \( taxRate \) = tax rate on capital gains (in percentage)
    \item \( K \) = amount of money the investor wants to raise
    \item \( sell_i \) = number of shares of stock \( i \) that the investor should sell
\end{itemize}

\subsection*{Objective Function}
The objective is to maximize the expected value of the portfolio next year after selling shares. Therefore, the objective function can be expressed as:

\[
\text{Maximize } Z = \sum_{i=1}^{N} \left( futurePrice_i \cdot (bought_i - sell_i) \right)
\]

\subsection*{Constraints}
1. The net amount raised from selling the shares must be at least \( K \):

\[
\sum_{i=1}^{N} \left( \left( currentPrice_i \cdot sell_i \right) \cdot \left( 1 - \frac{transactionRate}{100} \right) - \left( \left( currentPrice_i - buyPrice_i \right) \cdot sell_i \right) \cdot \frac{taxRate}{100} \right) \geq K
\]

2. The number of shares sold cannot exceed the number of shares bought:

\[
0 \leq sell_i \leq bought_i \quad \forall i
\]

3. Non-negativity constraints:

\[
sell_i \geq 0 \quad \forall i
\]

\subsection*{Input Format}
The input is given by:
\begin{verbatim}
{
    "bought": [bought_i, for i in 1, ..., N],
    "buyPrice": [buyPrice_i, for i in 1, ..., N],
    "currentPrice": [currentPrice_i, for i in 1, ..., N],
    "futurePrice": [futurePrice_i, for i in 1, ..., N],
    "transactionRate": transactionRate,
    "taxRate": taxRate,
    "K": K
}
\end{verbatim}

\subsection*{Output Format}
The output will provide the shares to sell as follows:
\begin{verbatim}
{
    "sell": [sell_i, for i in 1, ..., N],
}
\end{verbatim}

\end{document}