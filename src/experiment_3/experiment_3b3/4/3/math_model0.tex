\documentclass{article}
\usepackage{amsmath}
\begin{document}

\section*{Mathematical Model for Electricity Capacity Expansion}

\subsection*{Parameters}
\begin{align*}
T & : \text{number of years} \\
demand_t & : \text{megawatts of electricity demand in year } t \\
oil_t & : \text{existing oil-fired capacity in year } t \\
coal\_cost & : \text{capital cost per megawatt of coal-fired capacity} \\
nuke\_cost & : \text{capital cost per megawatt of nuclear power capacity} \\
max\_nuke & : \text{maximum allowable percentage of nuclear capacity} \\
coal\_life & : \text{lifetime of coal plants in years} \\
nuke\_life & : \text{lifetime of nuclear plants in years} \\
\end{align*}

\subsection*{Decision Variables}
\begin{align*}
coal_t & : \text{amount of coal-fired capacity added in year } t \\
nuke_t & : \text{amount of nuclear capacity added in year } t \\
\end{align*}

\subsection*{Objective Function}
Minimize the total cost of capacity expansion:
\[
\text{Minimize } Z = \sum_{t=1}^{T} \left( coal\_cost \cdot coal_t + nuke\_cost \cdot nuke_t \right)
\]

\subsection*{Constraints}
1. **Capacity Constraint**:
   The total capacity available in year \( t \) must meet the demand:
   \[
   oil_t + \sum_{j=0}^{\min(t-1, coal\_life-1)} coal_{t-j} + \sum_{k=0}^{\min(t-1, nuke\_life-1)} nuke_{t-k} \geq demand_t \quad \forall t \in \{1, \ldots, T\}
   \]

2. **Nuclear Capacity Limit**:
   The nuclear capacity must not exceed the specified percentage of total capacity:
   \[
   \sum_{k=0}^{\min(t-1, nuke\_life-1)} nuke_{t-k} \leq \frac{max\_nuke}{100} \left( oil_t + \sum_{j=0}^{\min(t-1, coal\_life-1)} coal_{t-j} + \sum_{k=0}^{\min(t-1, nuke\_life-1)} nuke_{t-k} \right) \quad \forall t \in \{1, \ldots, T\}
   \]

3. **Non-negativity Constraints**:
   The capacities added must be non-negative:
   \[
   coal_t \geq 0, \quad nuke_t \geq 0 \quad \forall t \in \{1, \ldots, T\}
   \]

\end{document}