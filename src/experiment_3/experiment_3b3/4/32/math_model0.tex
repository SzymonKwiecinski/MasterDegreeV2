\documentclass{article}
\usepackage{amsmath}
\begin{document}

\section*{Linear Programming Model for Production Optimization}

\subsection*{Definitions}
\begin{itemize}
    \item Let \( K \) be the number of products.
    \item Let \( S \) be the number of stages in production.
    \item Let \( \text{produce\_time}_{k,s} \) be the time required to produce 1 unit of product \( k \) at stage \( s \).
    \item Let \( \text{time}_{s} \) be the total available working hours for stage \( s \).
    \item Let \( \text{profit}_{k} \) be the profit earned from producing 1 unit of product \( k \).
    \item Let \( \text{quantity}_{k} \) be the quantity of product \( k \) to produce.
\end{itemize}

\subsection*{Objective Function}
The objective is to maximize the total profit:

\[
\text{Maximize} \quad Z = \sum_{k=1}^{K} \text{profit}_{k} \cdot \text{quantity}_{k}
\]

\subsection*{Constraints}
The production must respect the available time for each stage. For each stage \( s \):

\[
\sum_{k=1}^{K} \text{produce\_time}_{k,s} \cdot \text{quantity}_{k} \leq \text{time}_{s} \quad \forall s = 1, \ldots, S
\]

\subsection*{Non-negativity Constraints}
The quantity produced for each product must be non-negative:

\[
\text{quantity}_{k} \geq 0 \quad \forall k = 1, \ldots, K
\]

\subsection*{Summary of the LP Model}
The complete linear programming model can be summarized as follows:

\begin{align*}
    \text{Maximize} \quad & Z = \sum_{k=1}^{K} \text{profit}_{k} \cdot \text{quantity}_{k} \\
    \text{subject to} \quad & \sum_{k=1}^{K} \text{produce\_time}_{k,s} \cdot \text{quantity}_{k} \leq \text{time}_{s}, \quad \forall s = 1, \ldots, S \\
    & \text{quantity}_{k} \geq 0, \quad \forall k = 1, \ldots, K
\end{align*}

\end{document}