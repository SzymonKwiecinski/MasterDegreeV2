\documentclass{article}
\usepackage{amsmath}
\begin{document}

\section*{Linear Programming Model for Fitting a Straight Line}

Given a set of data points where the \( k \)-th \( y \) value is denoted as \( y_k \) and the \( k \)-th \( x \) value as \( x_k \) for \( k = 1, \ldots, K \), we want to fit a straight line of the form:

\[
y = bx + a
\]

The objective is to minimize the sum of absolute deviations between the observed values of \( y \) and the values predicted by the linear relationship. This can be formulated as follows:

\subsection*{Objective Function}
Minimize:

\[
Z = \sum_{k=1}^{K} |y_k - (bx_k + a)|
\]

\subsection*{Decision Variables}
Let:
\begin{itemize}
    \item \( b \): slope of the fitted line
    \item \( a \): intercept of the fitted line
    \item \( e_k \): deviation for each point, where \( e_k = y_k - (bx_k + a) \)
\end{itemize}

We can express the absolute value as two constraints:

\[
e_k \geq y_k - (bx_k + a) \quad \text{for } k = 1, \ldots, K
\]
\[
-e_k \geq -(y_k - (bx_k + a)) \quad \text{for } k = 1, \ldots, K
\]

Thus, we can rewrite the objective function as:

\[
Z = \sum_{k=1}^{K} e_k
\]

\subsection*{Linear Constraints}
The constraints can be formulated as:

\[
y_k - (bx_k + a) \leq e_k \quad \text{for } k = 1, \ldots, K
\]
\[
-(y_k - (bx_k + a)) \leq e_k \quad \text{for } k = 1, \ldots, K
\]

\subsection*{Final Model}
Putting this all together, the linear programming model can be summarized as follows:

\begin{align*}
\text{Minimize} \quad & Z = \sum_{k=1}^{K} e_k \\
\text{subject to} \quad & y_k - (bx_k + a) \leq e_k, \quad k = 1, \ldots, K \\
& -(y_k - (bx_k + a)) \leq e_k, \quad k = 1, \ldots, K \\
& e_k \geq 0, \quad k = 1, \ldots, K \\
& b, a \text{ are real numbers.}
\end{align*}

\subsection*{Output}
The output will define the fitted line parameters as:

\[
\text{Output:} \quad \{ \text{"intercept": } a, \text{"slope": } b \}
\]

\end{document}