\documentclass{article}
\usepackage{amsmath}
\begin{document}

\section*{Linear Programming Model for Spare Automobile Parts Production}

\textbf{Variables:} \\
Let \( x_k \) represent the quantity of spare part \( k \) produced, where \( k = 1, 2, \ldots, K \).

\textbf{Objective Function:} \\
The objective is to maximize the total profit from the production of spare parts:
\[
\text{Maximize } Z = \sum_{k=1}^{K} \text{profit}_k \cdot x_k
\]

\textbf{Constraints:} \\
The production process is constrained by the worker-hours available in each shop. For each shop \( s \), the total worker-hours used must not exceed the capacity of that shop:
\[
\sum_{k=1}^{K} \text{time}_{k,s} \cdot x_k \leq \text{capacity}_{s}, \quad \forall s = 1, 2, \ldots, S
\]

\textbf{Non-negativity Constraints:} \\
The quantities produced must be non-negative:
\[
x_k \geq 0, \quad \forall k = 1, 2, \ldots, K
\]

\end{document}