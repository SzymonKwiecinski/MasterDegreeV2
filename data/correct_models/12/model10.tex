\documentclass{article}
\usepackage{amsmath}
\begin{document}

\section*{Mathematical Model for Alloy Production}

\subsection*{Objective}
Minimize the total cost of the alloys used in the production:
\[
\text{Minimize} \quad \sum_{k=1}^{K} \text{Price}_k \cdot x_k
\]

\subsection*{Constraints}
1. The total quantity of alloys produced is exactly \(\text{AlloyQuantity}\):
\[
\sum_{k=1}^{K} x_k = \text{AlloyQuantity}
\]

2. The quantity of each target component in the alloy must be met or exceeded:
\[
\sum_{k=1}^{K} \text{Ratio}_{k,m} \cdot x_k \geq \text{Target}_m \quad \forall m \in \{1, 2, \ldots, M\}
\]

3. The quantity of component \(k\) in alloy \(m\) must adhere to the specified \(\text{Ratio}\):
\[
\text{Ratio}_{k,m} \cdot x_k \quad \text{(implicitly included in the above constraint)}
\]

4. Each alloy's quantity is non-negative:
\[
x_k \geq 0 \quad \forall k \in \{1, 2, \ldots, K\}
\]

\subsection*{Parameters}
\begin{itemize}
    \item \(\text{AlloyQuantity}\): Total quantity of alloy to produce (constant)
    \item \(\text{Target}_m\): Quantity of target components in the alloy for each \(m \in \{1, 2, \ldots, M\}\)
    \item \(\text{Ratio}_{k,m}\): Ratio of each component \(k\) in the alloy \(m\) for each \(k \in \{1, 2, \ldots, K\}\) and \(m \in \{1, 2, \ldots, M\}\)
    \item \(\text{Price}_k\): Price of each alloy \(k\) for each \(k \in \{1, 2, \ldots, K\}\)
\end{itemize}

\subsection*{Decision Variables}
\[
x_k \quad \text{(quantity of alloy } k \text{ to produce)}
\]

\end{document}