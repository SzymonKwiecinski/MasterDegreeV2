\documentclass{article}
\usepackage{amsmath}
\begin{document}

\section*{Mathematical Model for Food Selection Problem}

\subsection*{Parameters}
\begin{itemize}
    \item $K$: Number of different types of food
    \item $M$: Number of nutrients to consider
    \item $Price_k$: Price of food $k$ for $k = 1, 2, \ldots, K$
    \item $Demand_m$: Demand for nutrient $m$ for $m = 1, 2, \ldots, M$
    \item $Nutrition_{k,m}$: Amount of nutrient $m$ in food $k$ for $k = 1, 2, \ldots, K$ and $m = 1, 2, \ldots, M$
\end{itemize}

\subsection*{Decision Variables}
\begin{itemize}
    \item $x_k$: Number of units purchased from food type $k$ for $k = 1, 2, \ldots, K$
\end{itemize}

\subsection*{Objective Function}
Minimize the total cost of the foods purchased:
\[
\text{Minimize} \quad \sum_{k=1}^{K} Price_k \cdot x_k
\]

\subsection*{Constraints}
\begin{enumerate}
    \item The total amount of each nutrient from all food types must meet or exceed the specific demand for that nutrient:
    \[
    \sum_{k=1}^{K} Nutrition_{k,m} \cdot x_k \geq Demand_m \quad \text{for } m = 1, 2, \ldots, M
    \]
    \item The number of units purchased from each food type is non-negative:
    \[
    x_k \geq 0 \quad \text{for } k = 1, 2, \ldots, K
    \]
\end{enumerate}

\end{document}